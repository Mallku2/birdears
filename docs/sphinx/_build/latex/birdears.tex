%% Generated by Sphinx.
\def\sphinxdocclass{report}
\documentclass[letterpaper,10pt,english]{sphinxmanual}
\ifdefined\pdfpxdimen
   \let\sphinxpxdimen\pdfpxdimen\else\newdimen\sphinxpxdimen
\fi \sphinxpxdimen=.75bp\relax

\usepackage[utf8]{inputenc}
\ifdefined\DeclareUnicodeCharacter
 \ifdefined\DeclareUnicodeCharacterAsOptional
  \DeclareUnicodeCharacter{"00A0}{\nobreakspace}
  \DeclareUnicodeCharacter{"2500}{\sphinxunichar{2500}}
  \DeclareUnicodeCharacter{"2502}{\sphinxunichar{2502}}
  \DeclareUnicodeCharacter{"2514}{\sphinxunichar{2514}}
  \DeclareUnicodeCharacter{"251C}{\sphinxunichar{251C}}
  \DeclareUnicodeCharacter{"2572}{\textbackslash}
 \else
  \DeclareUnicodeCharacter{00A0}{\nobreakspace}
  \DeclareUnicodeCharacter{2500}{\sphinxunichar{2500}}
  \DeclareUnicodeCharacter{2502}{\sphinxunichar{2502}}
  \DeclareUnicodeCharacter{2514}{\sphinxunichar{2514}}
  \DeclareUnicodeCharacter{251C}{\sphinxunichar{251C}}
  \DeclareUnicodeCharacter{2572}{\textbackslash}
 \fi
\fi
\usepackage{cmap}
\usepackage[T1]{fontenc}
\usepackage{amsmath,amssymb,amstext}
\usepackage{babel}
\usepackage{times}
\usepackage[Bjarne]{fncychap}
\usepackage[dontkeepoldnames]{sphinx}

\usepackage{geometry}

% Include hyperref last.
\usepackage{hyperref}
% Fix anchor placement for figures with captions.
\usepackage{hypcap}% it must be loaded after hyperref.
% Set up styles of URL: it should be placed after hyperref.
\urlstyle{same}

\addto\captionsenglish{\renewcommand{\figurename}{Fig.}}
\addto\captionsenglish{\renewcommand{\tablename}{Table}}
\addto\captionsenglish{\renewcommand{\literalblockname}{Listing}}

\addto\captionsenglish{\renewcommand{\literalblockcontinuedname}{continued from previous page}}
\addto\captionsenglish{\renewcommand{\literalblockcontinuesname}{continues on next page}}

\addto\extrasenglish{\def\pageautorefname{page}}

\setcounter{tocdepth}{2}



\title{birdears Documentation}
\date{Dec 20, 2017}
\release{0.1.4}
\author{Iacchus Mercurius}
\newcommand{\sphinxlogo}{\vbox{}}
\renewcommand{\releasename}{Release}
\makeindex

\begin{document}

\maketitle
\sphinxtableofcontents
\phantomsection\label{\detokenize{index::doc}}


Welcome to birdears documentation.

\sphinxcode{birdears} is a software written in Python 3 for ear training for
musicians (musical intelligence, transcribing music, composing). It is a
clone of the method used by \sphinxhref{https://play.google.com/store/apps/details?id=com.kaizen9.fet.android}{Funcitional Ear
Trainer}
app for Android.

It comes with four modes, or four kind of exercises, which are:
\sphinxcode{melodic}, \sphinxcode{harmonic}, \sphinxcode{dictation} and \sphinxcode{instrumental}.

In resume, with the \sphinxstyleemphasis{melodic} mode two notes are played one after the
other and you have to guess the interval; with the \sphinxstyleemphasis{harmonic} mode,
two notes are played simoutaneously (harmonically) and you should guess
the interval.

With the \sphinxstyleemphasis{dictation} mode, more than 2 notes are played (\sphinxstyleemphasis{ie}., a
melodic dictation) and you should tell what are the intervals between
them.

With the \sphinxstyleemphasis{instrumental} mode, it is a like the \sphinxstyleemphasis{dictation}, but you will
be expected to play the notes on your instrument, \sphinxstyleemphasis{ie}., birdears will
not wait for a typed reply and you should prectice with your own
judgement. The melody can be repeat any times and you can have as much
time as you want to try it out.

Project at \sphinxhref{https://github.com/iacchus/birdears}{GitHub}.

Download the PDF version of this book. Clicking \sphinxhref{https://github.com/iacchus/birdears/raw/master/docs/sphinx/\_build/latex/birdears.pdf}{here}.




\chapter{Support}
\label{\detokenize{community:birdears-documentation}}\label{\detokenize{community:support}}\label{\detokenize{community::doc}}
If you need help you can get in touch via IRC or file an issue on any matter regarding birdears at Github.


\begin{savenotes}\sphinxattablestart
\centering
\begin{tabulary}{\linewidth}[t]{|T|T|}
\hline
\sphinxstylethead{\sphinxstyletheadfamily 
Media
\unskip}\relax &\sphinxstylethead{\sphinxstyletheadfamily 
Channel
\unskip}\relax \\
\hline
IRC
&
\sphinxhref{https://webchat.freenode.net/?randomnick=1\&channels=\%23birdears\&uio=MTY9dHJ1ZSYxMT0yNDY57}{\#birdears} at irc.freenode.org/6697 -ssl
\\
\hline
GitHub
&
\sphinxurl{https://github.com/iacchus/birdears}
\\
\hline
GH issues
&
\sphinxurl{https://github.com/iacchus/birdears/issues}
\\
\hline
ReadTheDocs
&
\sphinxurl{https://birdears.readthedocs.io}
\\
\hline
PyPI
&
\sphinxurl{https://pypi.python.org/pypi/birdears}
\\
\hline
TravisCI
&
\sphinxurl{https://travis-ci.org/iacchus/birdears}
\\
\hline
Coveralls
&
\sphinxurl{https://coveralls.io/github/iacchus/birdears}
\\
\hline
\end{tabulary}
\par
\sphinxattableend\end{savenotes}


\chapter{Features}
\label{\detokenize{features:features}}\label{\detokenize{features::doc}}\begin{itemize}
\item {} 
questions

\item {} 
pretty much configurable

\item {} 
load from config file

\item {} 
you can make your own presets

\item {} 
can be used interactively \sphinxstyleemphasis{(docs needed)}

\item {} 
can be used as a library \sphinxstyleemphasis{(docs needed)}

\end{itemize}


\chapter{Installing birdears}
\label{\detokenize{installing::doc}}\label{\detokenize{installing:installing-birdears}}

\section{Installing the dependencies}
\label{\detokenize{installing:installing-the-dependencies}}

\subsection{Arch Linux}
\label{\detokenize{installing:arch-linux}}
\begin{sphinxVerbatim}[commandchars=\\\{\}]
sudo pacman \PYGZhy{}Syu sox python python\PYGZhy{}pip
\end{sphinxVerbatim}


\section{Installing birdears}
\label{\detokenize{installing:id1}}
To install,simple do this command with pip3

\begin{sphinxVerbatim}[commandchars=\\\{\}]
pip3 install \PYGZhy{}\PYGZhy{}user \PYGZhy{}\PYGZhy{}upgrade \PYGZhy{}\PYGZhy{}no\PYGZhy{}cache\PYGZhy{}dir birdears
\end{sphinxVerbatim}


\subsection{In-depth installation}
\label{\detokenize{installing:in-depth-installation}}
You can choose to use a virtualenv to use birdears; this should give you
an idea on how to setup one virtualenv.

You should first install virtualenv (for python3) using your
distribution’s package (supposing you’re on linux), then issue on terminal:

\begin{sphinxVerbatim}[commandchars=\\\{\}]
virtualenv \PYGZhy{}p python3 \PYGZti{}/.venv \PYGZsh{} use the directory \PYGZti{}/.venv/ for the virtualenv

source \PYGZti{}/.venv/bin/activate   \PYGZsh{} activate the virtualenv; this should be done
                              \PYGZsh{} every time you may want to run the software
                              \PYGZsh{} installed here.

pip3 install birdears         \PYGZsh{} this will install the software

birdears \PYGZhy{}\PYGZhy{}help               \PYGZsh{} and this will run it
\end{sphinxVerbatim}


\chapter{Using birdears}
\label{\detokenize{using:using-birdears}}\label{\detokenize{using::doc}}

\section{What is Functional Ear Training}
\label{\detokenize{using:what-is-functional-ear-training}}
\sphinxstyleemphasis{write me!}


\section{The method}
\label{\detokenize{using:the-method}}
We can use abc language to notate music withing the documentation, ok

\begin{sphinxVerbatim}[commandchars=\\\{\}]
X: 1
T: Banish Misfortune
R: jig
M: 6/8
L: 1/8
K: Dmix
fed cAG\textbar{} A2d cAG\textbar{} F2D DED\textbar{} FEF GFG\textbar{}
AGA cAG\textbar{} AGA cde\textbar{}fed cAG\textbar{} Ad\PYGZca{}c d3:\textbar{}
f2d d\PYGZca{}cd\textbar{} f2g agf\textbar{} e2c cBc\textbar{}e2f gfe\textbar{}
f2g agf\textbar{} e2f gfe\textbar{}fed cAG\textbar{}Ad\PYGZca{}c d3:\textbar{}
f2g e2f\textbar{} d2e c2d\textbar{}ABA GAG\textbar{} F2F GED\textbar{}
c3 cAG\textbar{} AGA cde\textbar{} fed cAG\textbar{} Ad\PYGZca{}c d3:\textbar{}
\end{sphinxVerbatim}


\section{birdears modes and basic usage}
\label{\detokenize{using:birdears-modes-and-basic-usage}}
birdears actually has four modes:
\begin{itemize}
\item {} 
melodic interval question

\item {} 
harmonic interval question

\item {} 
melodic dictation question

\item {} 
instrumental dictation question

\end{itemize}

To see the commands avaliable just invoke the command without any arguments:

\begin{sphinxVerbatim}[commandchars=\\\{\}]
birdears
\end{sphinxVerbatim}

\begin{sphinxVerbatim}[commandchars=\\\{\}]
Usage: birdears  \PYGZlt{}command\PYGZgt{} [options]

  birdears ─ Functional Ear Training for Musicians!

Options:
  \PYGZhy{}\PYGZhy{}debug / \PYGZhy{}\PYGZhy{}no\PYGZhy{}debug  Turns on debugging; instead you can set DEBUG=1.
  \PYGZhy{}h, \PYGZhy{}\PYGZhy{}help            Show this message and exit.

Commands:
  dictation     Melodic dictation
  harmonic      Harmonic interval recognition
  instrumental  Instrumental melodic time\PYGZhy{}based dictation
  load          Loads exercise from .toml config file...
  melodic       Melodic interval recognition

  You can use \PYGZsq{}birdears \PYGZlt{}command\PYGZgt{} \PYGZhy{}\PYGZhy{}help\PYGZsq{} to show options for a specific
  command.

  More info at https://github.com/iacchus/birdears
\end{sphinxVerbatim}

\begin{sphinxVerbatim}[commandchars=\\\{\}]
birdears \PYGZlt{}command\PYGZgt{} \PYGZhy{}\PYGZhy{}help
\end{sphinxVerbatim}


\subsection{melodic}
\label{\detokenize{using:melodic}}
In this exercise birdears will play two notes, the tonic and the interval
melodically, ie., one after the other and you should reply which is the
correct distance between the two.

\begin{sphinxVerbatim}[commandchars=\\\{\}]
birdears melodic \PYGZhy{}\PYGZhy{}help
\end{sphinxVerbatim}

\begin{sphinxVerbatim}[commandchars=\\\{\}]
Usage: birdears melodic [options]

  Melodic interval recognition

Options:
  \PYGZhy{}m, \PYGZhy{}\PYGZhy{}mode \PYGZlt{}mode\PYGZgt{}               Mode of the question.
  \PYGZhy{}t, \PYGZhy{}\PYGZhy{}tonic \PYGZlt{}tonic\PYGZgt{}             Tonic of the question.
  \PYGZhy{}o, \PYGZhy{}\PYGZhy{}octave \PYGZlt{}octave\PYGZgt{}           Octave of the question.
  \PYGZhy{}d, \PYGZhy{}\PYGZhy{}descending                Whether the question interval is descending.
  \PYGZhy{}c, \PYGZhy{}\PYGZhy{}chromatic                 If chosen, question has chromatic notes.
  \PYGZhy{}n, \PYGZhy{}\PYGZhy{}n\PYGZus{}octaves \PYGZlt{}n max\PYGZgt{}         Maximum number of octaves.
  \PYGZhy{}v, \PYGZhy{}\PYGZhy{}valid\PYGZus{}intervals \PYGZlt{}1,2,..\PYGZgt{}  A comma\PYGZhy{}separated list without spaces
                                  of valid scale degrees to be chosen for the
                                  question.
  \PYGZhy{}q, \PYGZhy{}\PYGZhy{}user\PYGZus{}durations \PYGZlt{}1,0.5,n..\PYGZgt{}
                                  A comma\PYGZhy{}separated list without
                                  spaces with PRECISLY 9 floating values. Or
                                  \PYGZsq{}n\PYGZsq{} for default              duration.
  \PYGZhy{}p, \PYGZhy{}\PYGZhy{}prequestion\PYGZus{}method \PYGZlt{}prequestion\PYGZus{}method\PYGZgt{}
                                  The name of a pre\PYGZhy{}question method.
  \PYGZhy{}r, \PYGZhy{}\PYGZhy{}resolution\PYGZus{}method \PYGZlt{}resolution\PYGZus{}method\PYGZgt{}
                                  The name of a resolution method.
  \PYGZhy{}h, \PYGZhy{}\PYGZhy{}help                      Show this message and exit.

  In this exercise birdears will play two notes, the tonic and the interval
  melodically, ie., one after the other and you should reply which is the
  correct distance between the two.

  Valid values are as follows:

  \PYGZhy{}m \PYGZlt{}mode\PYGZgt{} is one of: major, dorian, phrygian, lydian, mixolydian, minor,
  locrian

  \PYGZhy{}t \PYGZlt{}tonic\PYGZgt{} is one of: A, A\PYGZsh{}, Ab, B, Bb, C, C\PYGZsh{}, D, D\PYGZsh{}, Db, E, Eb, F, F\PYGZsh{}, G,
  G\PYGZsh{}, Gb

  \PYGZhy{}p \PYGZlt{}prequestion\PYGZus{}method\PYGZgt{} is one of: none, tonic\PYGZus{}only, progression\PYGZus{}i\PYGZus{}iv\PYGZus{}v\PYGZus{}i

  \PYGZhy{}r \PYGZlt{}resolution\PYGZus{}method\PYGZgt{} is one of: nearest\PYGZus{}tonic, repeat\PYGZus{}only
\end{sphinxVerbatim}


\subsection{harmonic}
\label{\detokenize{using:harmonic}}
In this exercise birdears will play two notes, the tonic and the interval
harmonically, ie., both on the same time and you should reply which is the
correct distance between the two.

\begin{sphinxVerbatim}[commandchars=\\\{\}]
birdears harmonic \PYGZhy{}\PYGZhy{}help
\end{sphinxVerbatim}

\begin{sphinxVerbatim}[commandchars=\\\{\}]
Usage: birdears harmonic [options]

  Harmonic interval recognition

Options:
  \PYGZhy{}m, \PYGZhy{}\PYGZhy{}mode \PYGZlt{}mode\PYGZgt{}               Mode of the question.
  \PYGZhy{}t, \PYGZhy{}\PYGZhy{}tonic \PYGZlt{}note\PYGZgt{}              Tonic of the question.
  \PYGZhy{}o, \PYGZhy{}\PYGZhy{}octave \PYGZlt{}octave\PYGZgt{}           Octave of the question.
  \PYGZhy{}d, \PYGZhy{}\PYGZhy{}descending                Whether the question interval is descending.
  \PYGZhy{}c, \PYGZhy{}\PYGZhy{}chromatic                 If chosen, question has chromatic notes.
  \PYGZhy{}n, \PYGZhy{}\PYGZhy{}n\PYGZus{}octaves \PYGZlt{}n max\PYGZgt{}         Maximum number of octaves.
  \PYGZhy{}v, \PYGZhy{}\PYGZhy{}valid\PYGZus{}intervals \PYGZlt{}1,2,..\PYGZgt{}  A comma\PYGZhy{}separated list without spaces
                                  of valid scale degrees to be chosen for the
                                  question.
  \PYGZhy{}q, \PYGZhy{}\PYGZhy{}user\PYGZus{}durations \PYGZlt{}1,0.5,n..\PYGZgt{}
                                  A comma\PYGZhy{}separated list without
                                  spaces with PRECISLY 9 floating values. Or
                                  \PYGZsq{}n\PYGZsq{} for default              duration.
  \PYGZhy{}p, \PYGZhy{}\PYGZhy{}prequestion\PYGZus{}method \PYGZlt{}prequestion\PYGZus{}method\PYGZgt{}
                                  The name of a pre\PYGZhy{}question method.
  \PYGZhy{}r, \PYGZhy{}\PYGZhy{}resolution\PYGZus{}method \PYGZlt{}resolution\PYGZus{}method\PYGZgt{}
                                  The name of a resolution method.
  \PYGZhy{}h, \PYGZhy{}\PYGZhy{}help                      Show this message and exit.

  In this exercise birdears will play two notes, the tonic and the interval
  harmonically, ie., both on the same time and you should reply which is the
  correct distance between the two.

  Valid values are as follows:

  \PYGZhy{}m \PYGZlt{}mode\PYGZgt{} is one of: major, dorian, phrygian, lydian, mixolydian, minor,
  locrian

  \PYGZhy{}t \PYGZlt{}tonic\PYGZgt{} is one of: A, A\PYGZsh{}, Ab, B, Bb, C, C\PYGZsh{}, D, D\PYGZsh{}, Db, E, Eb, F, F\PYGZsh{}, G,
  G\PYGZsh{}, Gb

  \PYGZhy{}p \PYGZlt{}prequestion\PYGZus{}method\PYGZgt{} is one of: none, tonic\PYGZus{}only, progression\PYGZus{}i\PYGZus{}iv\PYGZus{}v\PYGZus{}i

  \PYGZhy{}r \PYGZlt{}resolution\PYGZus{}method\PYGZgt{} is one of: nearest\PYGZus{}tonic, repeat\PYGZus{}only
\end{sphinxVerbatim}


\subsection{dictation}
\label{\detokenize{using:dictation}}
In this exercise birdears will choose some random intervals and create a
melodic dictation with them. You should reply the correct intervals of the
melodic dictation.

\begin{sphinxVerbatim}[commandchars=\\\{\}]
birdears dictation \PYGZhy{}\PYGZhy{}help
\end{sphinxVerbatim}

\begin{sphinxVerbatim}[commandchars=\\\{\}]
Usage: birdears dictation [options]

  Melodic dictation

Options:
  \PYGZhy{}m, \PYGZhy{}\PYGZhy{}mode \PYGZlt{}mode\PYGZgt{}               Mode of the question.
  \PYGZhy{}i, \PYGZhy{}\PYGZhy{}max\PYGZus{}intervals \PYGZlt{}n max\PYGZgt{}     Max random intervals for the dictation.
  \PYGZhy{}x, \PYGZhy{}\PYGZhy{}n\PYGZus{}notes \PYGZlt{}n notes\PYGZgt{}         Number of notes for the dictation.
  \PYGZhy{}t, \PYGZhy{}\PYGZhy{}tonic \PYGZlt{}note\PYGZgt{}              Tonic of the question.
  \PYGZhy{}o, \PYGZhy{}\PYGZhy{}octave \PYGZlt{}octave\PYGZgt{}           Octave of the question.
  \PYGZhy{}d, \PYGZhy{}\PYGZhy{}descending                Wether the question interval is descending.
  \PYGZhy{}c, \PYGZhy{}\PYGZhy{}chromatic                 If chosen, question has chromatic notes.
  \PYGZhy{}n, \PYGZhy{}\PYGZhy{}n\PYGZus{}octaves \PYGZlt{}n max\PYGZgt{}         Maximum number of octaves.
  \PYGZhy{}v, \PYGZhy{}\PYGZhy{}valid\PYGZus{}intervals \PYGZlt{}1,2,..\PYGZgt{}  A comma\PYGZhy{}separated list without spaces
                                  of valid scale degrees to be chosen for the
                                  question.
  \PYGZhy{}q, \PYGZhy{}\PYGZhy{}user\PYGZus{}durations \PYGZlt{}1,0.5,n..\PYGZgt{}
                                  A comma\PYGZhy{}separated list without
                                  spaces with PRECISLY 9 floating values. Or
                                  \PYGZsq{}n\PYGZsq{} for default              duration.
  \PYGZhy{}p, \PYGZhy{}\PYGZhy{}prequestion\PYGZus{}method \PYGZlt{}prequestion\PYGZus{}method\PYGZgt{}
                                  The name of a pre\PYGZhy{}question method.
  \PYGZhy{}r, \PYGZhy{}\PYGZhy{}resolution\PYGZus{}method \PYGZlt{}resolution\PYGZus{}method\PYGZgt{}
                                  The name of a resolution method.
  \PYGZhy{}h, \PYGZhy{}\PYGZhy{}help                      Show this message and exit.

  In this exercise birdears will choose some random intervals and create a
  melodic dictation with them. You should reply the correct intervals of the
  melodic dictation.

  Valid values are as follows:

  \PYGZhy{}m \PYGZlt{}mode\PYGZgt{} is one of: major, dorian, phrygian, lydian, mixolydian, minor,
  locrian

  \PYGZhy{}t \PYGZlt{}tonic\PYGZgt{} is one of: A, A\PYGZsh{}, Ab, B, Bb, C, C\PYGZsh{}, D, D\PYGZsh{}, Db, E, Eb, F, F\PYGZsh{}, G,
  G\PYGZsh{}, Gb

  \PYGZhy{}p \PYGZlt{}prequestion\PYGZus{}method\PYGZgt{} is one of: none, tonic\PYGZus{}only, progression\PYGZus{}i\PYGZus{}iv\PYGZus{}v\PYGZus{}i

  \PYGZhy{}r \PYGZlt{}resolution\PYGZus{}method\PYGZgt{} is one of: nearest\PYGZus{}tonic, repeat\PYGZus{}only
\end{sphinxVerbatim}


\subsection{instrumental}
\label{\detokenize{using:instrumental}}
In this exercise birdears will choose some random intervals and create a
melodic dictation with them. You should play the correct melody in you
musical instrument.

\begin{sphinxVerbatim}[commandchars=\\\{\}]
birdears instrumental \PYGZhy{}\PYGZhy{}help
\end{sphinxVerbatim}

\begin{sphinxVerbatim}[commandchars=\\\{\}]
Usage: birdears instrumental [options]

  Instrumental melodic time\PYGZhy{}based dictation

Options:
  \PYGZhy{}m, \PYGZhy{}\PYGZhy{}mode \PYGZlt{}mode\PYGZgt{}               Mode of the question.
  \PYGZhy{}w, \PYGZhy{}\PYGZhy{}wait\PYGZus{}time \PYGZlt{}seconds\PYGZgt{}       Time in seconds for next question/repeat.
  \PYGZhy{}u, \PYGZhy{}\PYGZhy{}n\PYGZus{}repeats \PYGZlt{}times\PYGZgt{}         Times to repeat question.
  \PYGZhy{}i, \PYGZhy{}\PYGZhy{}max\PYGZus{}intervals \PYGZlt{}n max\PYGZgt{}     Max random intervals for the dictation.
  \PYGZhy{}x, \PYGZhy{}\PYGZhy{}n\PYGZus{}notes \PYGZlt{}n notes\PYGZgt{}         Number of notes for the dictation.
  \PYGZhy{}t, \PYGZhy{}\PYGZhy{}tonic \PYGZlt{}note\PYGZgt{}              Tonic of the question.
  \PYGZhy{}o, \PYGZhy{}\PYGZhy{}octave \PYGZlt{}octave\PYGZgt{}           Octave of the question.
  \PYGZhy{}d, \PYGZhy{}\PYGZhy{}descending                Wether the question interval is descending.
  \PYGZhy{}c, \PYGZhy{}\PYGZhy{}chromatic                 If chosen, question has chromatic notes.
  \PYGZhy{}n, \PYGZhy{}\PYGZhy{}n\PYGZus{}octaves \PYGZlt{}n max\PYGZgt{}         Maximum number of octaves.
  \PYGZhy{}v, \PYGZhy{}\PYGZhy{}valid\PYGZus{}intervals \PYGZlt{}1,2,..\PYGZgt{}  A comma\PYGZhy{}separated list without spaces
                                  of valid scale degrees to be chosen for the
                                  question.
  \PYGZhy{}q, \PYGZhy{}\PYGZhy{}user\PYGZus{}durations \PYGZlt{}1,0.5,n..\PYGZgt{}
                                  A comma\PYGZhy{}separated list without
                                  spaces with PRECISLY 9 floating values. Or
                                  \PYGZsq{}n\PYGZsq{} for default              duration.
  \PYGZhy{}p, \PYGZhy{}\PYGZhy{}prequestion\PYGZus{}method \PYGZlt{}prequestion\PYGZus{}method\PYGZgt{}
                                  The name of a pre\PYGZhy{}question method.
  \PYGZhy{}r, \PYGZhy{}\PYGZhy{}resolution\PYGZus{}method \PYGZlt{}resolution\PYGZus{}method\PYGZgt{}
                                  The name of a resolution method.
  \PYGZhy{}h, \PYGZhy{}\PYGZhy{}help                      Show this message and exit.

  In this exercise birdears will choose some random intervals and create a
  melodic dictation with them. You should play the correct melody in you
  musical instrument.

  Valid values are as follows:

  \PYGZhy{}m \PYGZlt{}mode\PYGZgt{} is one of: major, dorian, phrygian, lydian, mixolydian, minor,
  locrian

  \PYGZhy{}t \PYGZlt{}tonic\PYGZgt{} is one of: A, A\PYGZsh{}, Ab, B, Bb, C, C\PYGZsh{}, D, D\PYGZsh{}, Db, E, Eb, F, F\PYGZsh{}, G,
  G\PYGZsh{}, Gb

  \PYGZhy{}p \PYGZlt{}prequestion\PYGZus{}method\PYGZgt{} is one of: none, tonic\PYGZus{}only, progression\PYGZus{}i\PYGZus{}iv\PYGZus{}v\PYGZus{}i

  \PYGZhy{}r \PYGZlt{}resolution\PYGZus{}method\PYGZgt{} is one of: nearest\PYGZus{}tonic, repeat\PYGZus{}only
\end{sphinxVerbatim}


\section{Loading from config/preset files}
\label{\detokenize{using:loading-from-config-preset-files}}

\subsection{Pre-made presets}
\label{\detokenize{using:pre-made-presets}}
\sphinxcode{birdears} cointains some pre-made presets in it’s \sphinxcode{presets/}
subdirectory.

The study for beginners is recommended by following the numeric order of
those files (000, 001, then 002 etc.)


\subsubsection{Pre-made presets description}
\label{\detokenize{using:pre-made-presets-description}}
\sphinxstyleemphasis{write me}


\subsection{Creating new preset files}
\label{\detokenize{using:creating-new-preset-files}}
You can open the files cointained in birdears premade \sphinxcode{presets/}
folder to have an ideia on how config files are made; it is simply the
command line options written in a form \sphinxcode{toml} standard.


\section{Keybindings}
\label{\detokenize{using:keybindings}}

\subsection{On the keybindings}
\label{\detokenize{using:on-the-keybindings}}
The following keyboard diagrams should give you an idea on how the
keybindings work. Please note how the keys on the line from \sphinxcode{z}
(\sphinxstyleemphasis{unison}) to \sphinxcode{,} (comma, \sphinxstyleemphasis{octave}) represent the notes that are
\sphinxstyleemphasis{natural} to the mode, and the line above represent the chromatics.

Also, for exercises with two octaves, the \sphinxstylestrong{uppercased keys represent
the second octave}. For example, \sphinxcode{z} is \sphinxstyleemphasis{unison}, \sphinxcode{,} is the
\sphinxstyleemphasis{octave}, \sphinxcode{Z} (uppercased) is the \sphinxstyleemphasis{double octave}. The same for all the other
intervals.


\subsection{Major (Ionian)}
\label{\detokenize{using:major-ionian}}
\begin{figure}[htbp]
\centering
\capstart

\noindent\sphinxincludegraphics[scale=1.0]{{ionian}.png}
\caption{Keyboard diagram for the \sphinxcode{-{-}mode major} (default).}\label{\detokenize{using:id1}}\end{figure}


\subsection{Dorian}
\label{\detokenize{using:dorian}}
\begin{figure}[htbp]
\centering
\capstart

\noindent\sphinxincludegraphics[scale=1.0]{{dorian}.png}
\caption{Keyboard diagram for the \sphinxcode{-{-}mode dorian}.}\label{\detokenize{using:id2}}\end{figure}


\subsection{Phrygian}
\label{\detokenize{using:phrygian}}
\begin{figure}[htbp]
\centering
\capstart

\noindent\sphinxincludegraphics[scale=1.0]{{phrygian}.png}
\caption{Keyboard diagram for the \sphinxcode{-{-}mode phrygian}.}\label{\detokenize{using:id3}}\end{figure}


\subsection{Lydian}
\label{\detokenize{using:lydian}}
\begin{figure}[htbp]
\centering
\capstart

\noindent\sphinxincludegraphics[scale=1.0]{{lydian}.png}
\caption{Keyboard diagram for the \sphinxcode{-{-}mode lydian}.}\label{\detokenize{using:id4}}\end{figure}


\subsection{Mixolydian}
\label{\detokenize{using:mixolydian}}
\begin{figure}[htbp]
\centering
\capstart

\noindent\sphinxincludegraphics[scale=1.0]{{mixolydian}.png}
\caption{Keyboard diagram for the \sphinxcode{-{-}mode mixolydian}.}\label{\detokenize{using:id5}}\end{figure}


\subsection{Minor (Aeolian)}
\label{\detokenize{using:minor-aeolian}}
\begin{figure}[htbp]
\centering
\capstart

\noindent\sphinxincludegraphics[scale=1.0]{{minor}.png}
\caption{Keyboard diagram for the \sphinxcode{-{-}mode minor}.}\label{\detokenize{using:id6}}\end{figure}


\subsection{Locrian}
\label{\detokenize{using:locrian}}
\begin{figure}[htbp]
\centering
\capstart

\noindent\sphinxincludegraphics[scale=1.0]{{locrian}.png}
\caption{Keyboard diagram for the \sphinxcode{-{-}mode locrian}.}\label{\detokenize{using:id7}}\end{figure}


\chapter{birdears package}
\label{\detokenize{birdears:birdears-package}}\label{\detokenize{birdears:module-birdears}}\label{\detokenize{birdears::doc}}\index{birdears (module)}
birdears provides facilities to building musical ear training exercises.
\index{CHROMATIC\_FLAT (in module birdears)}

\begin{fulllineitems}
\phantomsection\label{\detokenize{birdears:birdears.CHROMATIC_FLAT}}\pysigline{\sphinxcode{birdears.}\sphinxbfcode{CHROMATIC\_FLAT}\sphinxbfcode{ = ('C', 'Db', 'D', 'Eb', 'E', 'F', 'Gb', 'G', 'Ab', 'A', 'Bb', 'B')}}
\sphinxstyleemphasis{tuple} \textendash{} Chromatic notes names using flats.

A mapping of the chromatic note names using flats.

\end{fulllineitems}

\index{CHROMATIC\_SHARP (in module birdears)}

\begin{fulllineitems}
\phantomsection\label{\detokenize{birdears:birdears.CHROMATIC_SHARP}}\pysigline{\sphinxcode{birdears.}\sphinxbfcode{CHROMATIC\_SHARP}\sphinxbfcode{ = ('C', 'C\#', 'D', 'D\#', 'E', 'F', 'F\#', 'G', 'G\#', 'A', 'A\#', 'B')}}
\sphinxstyleemphasis{tuple} \textendash{} Chromatic notes names using sharps.

A mapping of the chromatic note namesu sing sharps

\end{fulllineitems}

\index{CHROMATIC\_TYPE (in module birdears)}

\begin{fulllineitems}
\phantomsection\label{\detokenize{birdears:birdears.CHROMATIC_TYPE}}\pysigline{\sphinxcode{birdears.}\sphinxbfcode{CHROMATIC\_TYPE}\sphinxbfcode{ = (0, 1, 2, 3, 4, 5, 6, 7, 8, 9, 10, 11, 12)}}
\sphinxstyleemphasis{tuple} \textendash{} A map of the chromatic chromatic scale.

A map of the the semitones which compound the chromatic scale.

\end{fulllineitems}

\index{CIRCLE\_OF\_FIFTHS (in module birdears)}

\begin{fulllineitems}
\phantomsection\label{\detokenize{birdears:birdears.CIRCLE_OF_FIFTHS}}\pysigline{\sphinxcode{birdears.}\sphinxbfcode{CIRCLE\_OF\_FIFTHS}\sphinxbfcode{ = {[}('C', 'G', 'D', 'A', 'E', 'B', 'Gb', 'Db', 'Ab', 'Eb', 'Bb', 'F'), ('C', 'F', 'Bb', 'Eb', 'Ab', 'C\#', 'F\#', 'B', 'E', 'A', 'D', 'G'){]}}}
\sphinxstyleemphasis{list of tuples} \textendash{} Circle of fifths.

These are the circle of fifth in both directions.

\end{fulllineitems}

\index{DIATONIC\_MODES (in module birdears)}

\begin{fulllineitems}
\phantomsection\label{\detokenize{birdears:birdears.DIATONIC_MODES}}\pysigline{\sphinxcode{birdears.}\sphinxbfcode{DIATONIC\_MODES}\sphinxbfcode{ = \{'major': (0, 2, 4, 5, 7, 9, 11, 12), 'dorian': (0, 2, 3, 5, 7, 9, 10, 12), 'phrygian': (0, 1, 3, 5, 7, 8, 10, 12), 'lydian': (0, 2, 4, 6, 7, 9, 11, 12), 'mixolydian': (0, 2, 4, 5, 7, 9, 10, 12), 'minor': (0, 2, 3, 5, 7, 8, 10, 12), 'locrian': (0, 1, 3, 5, 6, 8, 10, 12)\}}}
\sphinxstyleemphasis{dict of tuples} \textendash{} A map of the diatonic scale.

A mapping of the semitones which compound each of the greek modes.

\end{fulllineitems}

\index{INTERVALS (in module birdears)}

\begin{fulllineitems}
\phantomsection\label{\detokenize{birdears:birdears.INTERVALS}}\pysigline{\sphinxcode{birdears.}\sphinxbfcode{INTERVALS}\sphinxbfcode{ = ((0, 'P1', 'Perfect Unison'), (1, 'm2', 'Minor Second'), (2, 'M2', 'Major Second'), (3, 'm3', 'Minor Third'), (4, 'M3', 'Major Third'), (5, 'P4', 'Perfect Fourth'), (6, 'A4', 'Augmented Fourth'), (7, 'P5', 'Perfect Fifth'), (8, 'm6', 'Minor Sixth'), (9, 'M6', 'Major Sixth'), (10, 'm7', 'Minor Seventh'), (11, 'M7', 'Major Seventh'), (12, 'P8', 'Perfect Octave'), (13, 'A8', 'Minor Ninth'), (14, 'M9', 'Major Ninth'), (15, 'm10', 'Minor Tenth'), (16, 'M10', 'Major Tenth'), (17, 'P11', 'Perfect Eleventh'), (18, 'A11', 'Augmented Eleventh'), (19, 'P12', 'Perfect Twelfth'), (20, 'm13', 'Minor Thirteenth'), (21, 'M13', 'Major Thirteenth'), (22, 'm14', 'Minor Fourteenth'), (23, 'M14', 'Major Fourteenth'), (24, 'P15', 'Perfect Double-octave'), (25, 'A15', 'Minor Sixteenth'), (26, 'M16', 'Major Sixteenth'), (27, 'm17', 'Minor Seventeenth'), (28, 'M17', 'Major Seventeenth'), (29, 'P18', 'Perfect Eighteenth'), (30, 'A18', 'Augmented Eighteenth'), (31, 'P19', 'Perfect Nineteenth'), (32, 'm20', 'Minor Twentieth'), (33, 'M20', 'Major Twentieth'), (34, 'm21', 'Minor Twenty-first'), (35, 'M21', 'Major Twenty-first'), (36, 'P22', 'Perfect Triple-octave'))}}
\sphinxstyleemphasis{tuple of tuples} \textendash{} Data representing intervals.

A tuple of tuples representing data for the intervals with format
(semitones, short name, full name).

\end{fulllineitems}

\index{INTERVAL\_INDEX (in module birdears)}

\begin{fulllineitems}
\phantomsection\label{\detokenize{birdears:birdears.INTERVAL_INDEX}}\pysigline{\sphinxcode{birdears.}\sphinxbfcode{INTERVAL\_INDEX}\sphinxbfcode{ = \{1: {[}0{]}, 2: {[}1, 2{]}, 3: {[}3, 4{]}, 4: {[}5, 6{]}, 5: {[}6, 7{]}, 6: {[}8, 9{]}, 7: {[}10, 11{]}, 8: {[}12{]}\}}}
\sphinxstyleemphasis{dict of lists} \textendash{} A mapping of semitones of each interval.

A mapping of semitones which index to each interval name, major/minor,
perfect, augmented/diminished

\end{fulllineitems}

\index{KEYS (in module birdears)}

\begin{fulllineitems}
\phantomsection\label{\detokenize{birdears:birdears.KEYS}}\pysigline{\sphinxcode{birdears.}\sphinxbfcode{KEYS}\sphinxbfcode{ = ('C', 'C\#', 'Db', 'D', 'D\#', 'Eb', 'E', 'F', 'F\#', 'Gb', 'G', 'G\#', 'Ab', 'A', 'A\#', 'Bb', 'B')}}
\sphinxstyleemphasis{tuple} \textendash{} Allowed keys

These are the allowed keys for exercise as comprehended by birdears.

\end{fulllineitems}



\section{Subpackages}
\label{\detokenize{birdears:subpackages}}

\subsection{birdears.interfaces package}
\label{\detokenize{birdears.interfaces:birdears-interfaces-package}}\label{\detokenize{birdears.interfaces::doc}}\phantomsection\label{\detokenize{birdears.interfaces:module-birdears.interfaces}}\index{birdears.interfaces (module)}

\subsubsection{Submodules}
\label{\detokenize{birdears.interfaces:submodules}}

\subsubsection{birdears.interfaces.commandline module}
\label{\detokenize{birdears.interfaces:birdears-interfaces-commandline-module}}\label{\detokenize{birdears.interfaces:module-birdears.interfaces.commandline}}\index{birdears.interfaces.commandline (module)}\index{CommandLine() (in module birdears.interfaces.commandline)}

\begin{fulllineitems}
\phantomsection\label{\detokenize{birdears.interfaces:birdears.interfaces.commandline.CommandLine}}\pysiglinewithargsret{\sphinxcode{birdears.interfaces.commandline.}\sphinxbfcode{CommandLine}}{\emph{exercise}, \emph{**kwargs}}{}
This function implements the birdears loop for command line.
\begin{quote}\begin{description}
\item[{Parameters}] \leavevmode\begin{itemize}
\item {} 
\sphinxstyleliteralstrong{exercise} (\sphinxstyleliteralemphasis{str}) \textendash{} The question name.

\item {} 
\sphinxstyleliteralstrong{**kwargs} (\sphinxstyleliteralemphasis{kwargs}) \textendash{} FIXME: The kwargs can contain options for specific
questions.

\end{itemize}

\end{description}\end{quote}

\end{fulllineitems}

\index{center\_text() (in module birdears.interfaces.commandline)}

\begin{fulllineitems}
\phantomsection\label{\detokenize{birdears.interfaces:birdears.interfaces.commandline.center_text}}\pysiglinewithargsret{\sphinxcode{birdears.interfaces.commandline.}\sphinxbfcode{center\_text}}{\emph{text}, \emph{sep=True}, \emph{nl=0}}{}
This function returns input text centered according to terminal columns.
\begin{quote}\begin{description}
\item[{Parameters}] \leavevmode\begin{itemize}
\item {} 
\sphinxstyleliteralstrong{text} (\sphinxstyleliteralemphasis{str}) \textendash{} The string to be centered, it can have multiple lines.

\item {} 
\sphinxstyleliteralstrong{sep} (\sphinxstyleliteralemphasis{bool}) \textendash{} Add line separator after centered text (True) or
not (False).

\item {} 
\sphinxstyleliteralstrong{nl} (\sphinxstyleliteralemphasis{int}) \textendash{} How many new lines to add after text.

\end{itemize}

\end{description}\end{quote}

\end{fulllineitems}

\index{make\_input\_str() (in module birdears.interfaces.commandline)}

\begin{fulllineitems}
\phantomsection\label{\detokenize{birdears.interfaces:birdears.interfaces.commandline.make_input_str}}\pysiglinewithargsret{\sphinxcode{birdears.interfaces.commandline.}\sphinxbfcode{make\_input\_str}}{\emph{user\_input}, \emph{keyboard\_index}}{}
Makes a string representing intervals entered by the user.

This function is to be used by questions which takes more than one interval
input as MelodicDictation, and formats the intervals already entered.
\begin{quote}\begin{description}
\item[{Parameters}] \leavevmode\begin{itemize}
\item {} 
\sphinxstyleliteralstrong{user\_input} (\sphinxstyleliteralemphasis{array\_type}) \textendash{} The list of keyboard keys entered by user.

\item {} 
\sphinxstyleliteralstrong{keyboard\_index} (\sphinxstyleliteralemphasis{array\_type}) \textendash{} The keyboard mapping used by question.

\end{itemize}

\end{description}\end{quote}

\end{fulllineitems}

\index{print\_instrumental() (in module birdears.interfaces.commandline)}

\begin{fulllineitems}
\phantomsection\label{\detokenize{birdears.interfaces:birdears.interfaces.commandline.print_instrumental}}\pysiglinewithargsret{\sphinxcode{birdears.interfaces.commandline.}\sphinxbfcode{print\_instrumental}}{\emph{response}}{}
Prints the formatted response for ‘instrumental’ exercise.
\begin{quote}\begin{description}
\item[{Parameters}] \leavevmode
\sphinxstyleliteralstrong{response} (\sphinxstyleliteralemphasis{dict}) \textendash{} A response returned by question’s check\_question()

\end{description}\end{quote}

\end{fulllineitems}

\index{print\_question() (in module birdears.interfaces.commandline)}

\begin{fulllineitems}
\phantomsection\label{\detokenize{birdears.interfaces:birdears.interfaces.commandline.print_question}}\pysiglinewithargsret{\sphinxcode{birdears.interfaces.commandline.}\sphinxbfcode{print\_question}}{\emph{question}}{}
Prints the question to the user.
\begin{quote}\begin{description}
\item[{Parameters}] \leavevmode
\sphinxstyleliteralstrong{question} (\sphinxstyleliteralemphasis{obj}) \textendash{} A Question class with the question to be printed.

\end{description}\end{quote}

\end{fulllineitems}

\index{print\_response() (in module birdears.interfaces.commandline)}

\begin{fulllineitems}
\phantomsection\label{\detokenize{birdears.interfaces:birdears.interfaces.commandline.print_response}}\pysiglinewithargsret{\sphinxcode{birdears.interfaces.commandline.}\sphinxbfcode{print\_response}}{\emph{response}}{}
Prints the formatted response.
\begin{quote}\begin{description}
\item[{Parameters}] \leavevmode
\sphinxstyleliteralstrong{response} (\sphinxstyleliteralemphasis{dict}) \textendash{} A response returned by question’s check\_question()

\end{description}\end{quote}

\end{fulllineitems}



\subsection{birdears.questions package}
\label{\detokenize{birdears.questions:birdears-questions-package}}\label{\detokenize{birdears.questions::doc}}\phantomsection\label{\detokenize{birdears.questions:module-birdears.questions}}\index{birdears.questions (module)}

\subsubsection{Submodules}
\label{\detokenize{birdears.questions:submodules}}

\subsubsection{birdears.questions.harmonicinterval module}
\label{\detokenize{birdears.questions:birdears-questions-harmonicinterval-module}}\label{\detokenize{birdears.questions:module-birdears.questions.harmonicinterval}}\index{birdears.questions.harmonicinterval (module)}\index{HarmonicIntervalQuestion (class in birdears.questions.harmonicinterval)}

\begin{fulllineitems}
\phantomsection\label{\detokenize{birdears.questions:birdears.questions.harmonicinterval.HarmonicIntervalQuestion}}\pysiglinewithargsret{\sphinxbfcode{class }\sphinxcode{birdears.questions.harmonicinterval.}\sphinxbfcode{HarmonicIntervalQuestion}}{\emph{mode='major'}, \emph{tonic=None}, \emph{octave=None}, \emph{descending=None}, \emph{chromatic=None}, \emph{n\_octaves=None}, \emph{valid\_intervals=None}, \emph{user\_durations=None}, \emph{prequestion\_method='none'}, \emph{resolution\_method='nearest\_tonic'}, \emph{*args}, \emph{**kwargs}}{}
Bases: {\hyperref[\detokenize{index:birdears.questionbase.QuestionBase}]{\sphinxcrossref{\sphinxcode{birdears.questionbase.QuestionBase}}}}

Implements a Harmonic Interval test.
\index{\_\_init\_\_() (birdears.questions.harmonicinterval.HarmonicIntervalQuestion method)}

\begin{fulllineitems}
\phantomsection\label{\detokenize{birdears.questions:birdears.questions.harmonicinterval.HarmonicIntervalQuestion.__init__}}\pysiglinewithargsret{\sphinxbfcode{\_\_init\_\_}}{\emph{mode='major'}, \emph{tonic=None}, \emph{octave=None}, \emph{descending=None}, \emph{chromatic=None}, \emph{n\_octaves=None}, \emph{valid\_intervals=None}, \emph{user\_durations=None}, \emph{prequestion\_method='none'}, \emph{resolution\_method='nearest\_tonic'}, \emph{*args}, \emph{**kwargs}}{}
Inits the class.
\begin{quote}\begin{description}
\item[{Parameters}] \leavevmode\begin{itemize}
\item {} 
\sphinxstyleliteralstrong{mode} (\sphinxstyleliteralemphasis{str}) \textendash{} A string representing the mode of the question.
Eg., ‘major’ or ‘minor’

\item {} 
\sphinxstyleliteralstrong{tonic} (\sphinxstyleliteralemphasis{str}) \textendash{} A string representing the tonic of the question,
eg.: ‘C’; if omitted, it will be selected randomly.

\item {} 
\sphinxstyleliteralstrong{octave} (\sphinxstyleliteralemphasis{int}) \textendash{} A scienfic octave notation, for example, 4 for ‘C4’;
if not present, it will be randomly chosen.

\item {} 
\sphinxstyleliteralstrong{descending} (\sphinxstyleliteralemphasis{bool}) \textendash{} Is the question direction in descending, ie.,
intervals have lower pitch than the tonic.

\item {} 
\sphinxstyleliteralstrong{chromatic} (\sphinxstyleliteralemphasis{bool}) \textendash{} If the question can have (True) or not (False)
chromatic intervals, ie., intervals not in the diatonic scale
of tonic/mode.

\item {} 
\sphinxstyleliteralstrong{n\_octaves} (\sphinxstyleliteralemphasis{int}) \textendash{} Maximum number of octaves of the question.

\item {} 
\sphinxstyleliteralstrong{valid\_intervals} (\sphinxstyleliteralemphasis{list}) \textendash{} A list with intervals (int) valid for
random choice, 1 is 1st, 2 is second etc. Eg. {[}1, 4, 5{]} to
allow only tonics, fourths and fifths.

\item {} 
\sphinxstyleliteralstrong{user\_durations} (\sphinxstyleliteralemphasis{str}) \textendash{} 
A string with 9 comma-separated \sphinxtitleref{int} or
\sphinxtitleref{float{}`s to set the default duration for the notes played. The
values are respectively for: pre-question duration (1st),
pre-question delay (2nd), and pre-question pos-delay (3rd);
question duration (4th), question delay (5th), and question
pos-delay (6th); resolution duration (7th), resolution
delay (8th), and resolution pos-delay (9th).
duration is the duration in of the note in seconds; delay is
the time to wait before playing the next note, and pos\_delay is
the time to wait after all the notes of the respective sequence
have been played. If any of the user durations is {}`n}, the
default duration for the type of question will be used instead.
Example:

\begin{sphinxVerbatim}[commandchars=\\\{\}]
\PYGZdq{}2,0.5,1,2,n,0,2.5,n,1\PYGZdq{}
\end{sphinxVerbatim}


\item {} 
\sphinxstyleliteralstrong{prequestion\_method} (\sphinxstyleliteralemphasis{str}) \textendash{} Method of playing a cadence or the
exercise tonic before the question so to affirm the question
musical tonic key to the ear. Valid ones are registered in the
\sphinxtitleref{birdears.prequestion.PREQUESION\_METHODS} global dict.

\item {} 
\sphinxstyleliteralstrong{resolution\_method} (\sphinxstyleliteralemphasis{str}) \textendash{} Method of playing the resolution of an
exercise. Valid ones are registered in the
\sphinxtitleref{birdears.resolution.RESOLUTION\_METHODS} global dict.

\end{itemize}

\end{description}\end{quote}

\end{fulllineitems}

\index{check\_question() (birdears.questions.harmonicinterval.HarmonicIntervalQuestion method)}

\begin{fulllineitems}
\phantomsection\label{\detokenize{birdears.questions:birdears.questions.harmonicinterval.HarmonicIntervalQuestion.check_question}}\pysiglinewithargsret{\sphinxbfcode{check\_question}}{\emph{user\_input\_char}}{}
Checks whether the given answer is correct.

\end{fulllineitems}

\index{make\_pre\_question() (birdears.questions.harmonicinterval.HarmonicIntervalQuestion method)}

\begin{fulllineitems}
\phantomsection\label{\detokenize{birdears.questions:birdears.questions.harmonicinterval.HarmonicIntervalQuestion.make_pre_question}}\pysiglinewithargsret{\sphinxbfcode{make\_pre\_question}}{\emph{method}}{}
\end{fulllineitems}

\index{make\_question() (birdears.questions.harmonicinterval.HarmonicIntervalQuestion method)}

\begin{fulllineitems}
\phantomsection\label{\detokenize{birdears.questions:birdears.questions.harmonicinterval.HarmonicIntervalQuestion.make_question}}\pysiglinewithargsret{\sphinxbfcode{make\_question}}{}{}
\end{fulllineitems}

\index{make\_resolution() (birdears.questions.harmonicinterval.HarmonicIntervalQuestion method)}

\begin{fulllineitems}
\phantomsection\label{\detokenize{birdears.questions:birdears.questions.harmonicinterval.HarmonicIntervalQuestion.make_resolution}}\pysiglinewithargsret{\sphinxbfcode{make\_resolution}}{\emph{method}}{}
\end{fulllineitems}

\index{play\_question() (birdears.questions.harmonicinterval.HarmonicIntervalQuestion method)}

\begin{fulllineitems}
\phantomsection\label{\detokenize{birdears.questions:birdears.questions.harmonicinterval.HarmonicIntervalQuestion.play_question}}\pysiglinewithargsret{\sphinxbfcode{play\_question}}{}{}
\end{fulllineitems}

\index{play\_resolution() (birdears.questions.harmonicinterval.HarmonicIntervalQuestion method)}

\begin{fulllineitems}
\phantomsection\label{\detokenize{birdears.questions:birdears.questions.harmonicinterval.HarmonicIntervalQuestion.play_resolution}}\pysiglinewithargsret{\sphinxbfcode{play\_resolution}}{}{}
\end{fulllineitems}


\end{fulllineitems}



\subsubsection{birdears.questions.instrumentaldictation module}
\label{\detokenize{birdears.questions:module-birdears.questions.instrumentaldictation}}\label{\detokenize{birdears.questions:birdears-questions-instrumentaldictation-module}}\index{birdears.questions.instrumentaldictation (module)}\index{InstrumentalDictationQuestion (class in birdears.questions.instrumentaldictation)}

\begin{fulllineitems}
\phantomsection\label{\detokenize{birdears.questions:birdears.questions.instrumentaldictation.InstrumentalDictationQuestion}}\pysiglinewithargsret{\sphinxbfcode{class }\sphinxcode{birdears.questions.instrumentaldictation.}\sphinxbfcode{InstrumentalDictationQuestion}}{\emph{mode='major'}, \emph{wait\_time=11}, \emph{n\_repeats=1}, \emph{max\_intervals=3}, \emph{n\_notes=4}, \emph{tonic=None}, \emph{octave=None}, \emph{descending=None}, \emph{chromatic=None}, \emph{n\_octaves=None}, \emph{valid\_intervals=None}, \emph{user\_durations=None}, \emph{prequestion\_method='progression\_i\_iv\_v\_i'}, \emph{resolution\_method='repeat\_only'}, \emph{*args}, \emph{**kwargs}}{}
Bases: {\hyperref[\detokenize{index:birdears.questionbase.QuestionBase}]{\sphinxcrossref{\sphinxcode{birdears.questionbase.QuestionBase}}}}

Implements an instrumental dictation test.
\index{\_\_init\_\_() (birdears.questions.instrumentaldictation.InstrumentalDictationQuestion method)}

\begin{fulllineitems}
\phantomsection\label{\detokenize{birdears.questions:birdears.questions.instrumentaldictation.InstrumentalDictationQuestion.__init__}}\pysiglinewithargsret{\sphinxbfcode{\_\_init\_\_}}{\emph{mode='major'}, \emph{wait\_time=11}, \emph{n\_repeats=1}, \emph{max\_intervals=3}, \emph{n\_notes=4}, \emph{tonic=None}, \emph{octave=None}, \emph{descending=None}, \emph{chromatic=None}, \emph{n\_octaves=None}, \emph{valid\_intervals=None}, \emph{user\_durations=None}, \emph{prequestion\_method='progression\_i\_iv\_v\_i'}, \emph{resolution\_method='repeat\_only'}, \emph{*args}, \emph{**kwargs}}{}
Inits the class.
\begin{quote}\begin{description}
\item[{Parameters}] \leavevmode\begin{itemize}
\item {} 
\sphinxstyleliteralstrong{mode} (\sphinxstyleliteralemphasis{str}) \textendash{} A string representing the mode of the question.
Eg., ‘major’ or ‘minor’.

\item {} 
\sphinxstyleliteralstrong{wait\_time} (\sphinxstyleliteralemphasis{float}) \textendash{} Wait time in seconds for the next question or
repeat.

\item {} 
\sphinxstyleliteralstrong{n\_repeats} (\sphinxstyleliteralemphasis{int}) \textendash{} Number of times the same dictation will be
repeated before the end of the exercise.

\item {} 
\sphinxstyleliteralstrong{max\_intervals} (\sphinxstyleliteralemphasis{int}) \textendash{} The maximum number of random intervals the
question will have.

\item {} 
\sphinxstyleliteralstrong{n\_notes} (\sphinxstyleliteralemphasis{int}) \textendash{} The number of notes the melodic dictation will have.

\item {} 
\sphinxstyleliteralstrong{tonic} (\sphinxstyleliteralemphasis{str}) \textendash{} A string representing the tonic of the question,
eg.: ‘C’; if omitted, it will be selected randomly.

\item {} 
\sphinxstyleliteralstrong{octave} (\sphinxstyleliteralemphasis{int}) \textendash{} A scienfic octave notation, for example, 4 for ‘C4’;
if not present, it will be randomly chosen.

\item {} 
\sphinxstyleliteralstrong{descending} (\sphinxstyleliteralemphasis{bool}) \textendash{} Is the question direction in descending, ie.,
intervals have lower pitch than the tonic.

\item {} 
\sphinxstyleliteralstrong{chromatic} (\sphinxstyleliteralemphasis{bool}) \textendash{} If the question can have (True) or not (False)
chromatic intervals, ie., intervals not in the diatonic scale
of tonic/mode.

\item {} 
\sphinxstyleliteralstrong{n\_octaves} (\sphinxstyleliteralemphasis{int}) \textendash{} Maximum number of octaves of the question.

\item {} 
\sphinxstyleliteralstrong{valid\_intervals} (\sphinxstyleliteralemphasis{list}) \textendash{} A list with intervals (int) valid for
random choice, 1 is 1st, 2 is second etc. Eg. {[}1, 4, 5{]} to
allow only tonics, fourths and fifths.

\item {} 
\sphinxstyleliteralstrong{user\_durations} (\sphinxstyleliteralemphasis{str}) \textendash{} 
A string with 9 comma-separated \sphinxtitleref{int} or
\sphinxtitleref{float{}`s to set the default duration for the notes played. The
values are respectively for: pre-question duration (1st),
pre-question delay (2nd), and pre-question pos-delay (3rd);
question duration (4th), question delay (5th), and question
pos-delay (6th); resolution duration (7th), resolution
delay (8th), and resolution pos-delay (9th).
duration is the duration in of the note in seconds; delay is
the time to wait before playing the next note, and pos\_delay is
the time to wait after all the notes of the respective sequence
have been played. If any of the user durations is {}`n}, the
default duration for the type of question will be used instead.
Example:

\begin{sphinxVerbatim}[commandchars=\\\{\}]
\PYGZdq{}2,0.5,1,2,n,0,2.5,n,1\PYGZdq{}
\end{sphinxVerbatim}


\item {} 
\sphinxstyleliteralstrong{prequestion\_method} (\sphinxstyleliteralemphasis{str}) \textendash{} Method of playing a cadence or the
exercise tonic before the question so to affirm the question
musical tonic key to the ear. Valid ones are registered in the
\sphinxtitleref{birdears.prequestion.PREQUESION\_METHODS} global dict.

\item {} 
\sphinxstyleliteralstrong{resolution\_method} (\sphinxstyleliteralemphasis{str}) \textendash{} Method of playing the resolution of an
exercise. Valid ones are registered in the
\sphinxtitleref{birdears.resolution.RESOLUTION\_METHODS} global dict.

\end{itemize}

\end{description}\end{quote}

\end{fulllineitems}

\index{check\_question() (birdears.questions.instrumentaldictation.InstrumentalDictationQuestion method)}

\begin{fulllineitems}
\phantomsection\label{\detokenize{birdears.questions:birdears.questions.instrumentaldictation.InstrumentalDictationQuestion.check_question}}\pysiglinewithargsret{\sphinxbfcode{check\_question}}{}{}
Checks whether the given answer is correct.

This currently doesn’t applies to instrumental dictation questions.

\end{fulllineitems}

\index{make\_pre\_question() (birdears.questions.instrumentaldictation.InstrumentalDictationQuestion method)}

\begin{fulllineitems}
\phantomsection\label{\detokenize{birdears.questions:birdears.questions.instrumentaldictation.InstrumentalDictationQuestion.make_pre_question}}\pysiglinewithargsret{\sphinxbfcode{make\_pre\_question}}{\emph{method}}{}
\end{fulllineitems}

\index{make\_question() (birdears.questions.instrumentaldictation.InstrumentalDictationQuestion method)}

\begin{fulllineitems}
\phantomsection\label{\detokenize{birdears.questions:birdears.questions.instrumentaldictation.InstrumentalDictationQuestion.make_question}}\pysiglinewithargsret{\sphinxbfcode{make\_question}}{\emph{phrase\_semitones}}{}
\end{fulllineitems}

\index{make\_resolution() (birdears.questions.instrumentaldictation.InstrumentalDictationQuestion method)}

\begin{fulllineitems}
\phantomsection\label{\detokenize{birdears.questions:birdears.questions.instrumentaldictation.InstrumentalDictationQuestion.make_resolution}}\pysiglinewithargsret{\sphinxbfcode{make\_resolution}}{\emph{method}}{}
\end{fulllineitems}

\index{play\_question() (birdears.questions.instrumentaldictation.InstrumentalDictationQuestion method)}

\begin{fulllineitems}
\phantomsection\label{\detokenize{birdears.questions:birdears.questions.instrumentaldictation.InstrumentalDictationQuestion.play_question}}\pysiglinewithargsret{\sphinxbfcode{play\_question}}{}{}
\end{fulllineitems}


\end{fulllineitems}



\subsubsection{birdears.questions.melodicdictation module}
\label{\detokenize{birdears.questions:birdears-questions-melodicdictation-module}}\label{\detokenize{birdears.questions:module-birdears.questions.melodicdictation}}\index{birdears.questions.melodicdictation (module)}\index{MelodicDictationQuestion (class in birdears.questions.melodicdictation)}

\begin{fulllineitems}
\phantomsection\label{\detokenize{birdears.questions:birdears.questions.melodicdictation.MelodicDictationQuestion}}\pysiglinewithargsret{\sphinxbfcode{class }\sphinxcode{birdears.questions.melodicdictation.}\sphinxbfcode{MelodicDictationQuestion}}{\emph{mode='major'}, \emph{max\_intervals=3}, \emph{n\_notes=4}, \emph{tonic=None}, \emph{octave=None}, \emph{descending=None}, \emph{chromatic=None}, \emph{n\_octaves=None}, \emph{valid\_intervals=None}, \emph{user\_durations=None}, \emph{prequestion\_method='progression\_i\_iv\_v\_i'}, \emph{resolution\_method='repeat\_only'}, \emph{*args}, \emph{**kwargs}}{}
Bases: {\hyperref[\detokenize{index:birdears.questionbase.QuestionBase}]{\sphinxcrossref{\sphinxcode{birdears.questionbase.QuestionBase}}}}

Implements a melodic dictation test.
\index{\_\_init\_\_() (birdears.questions.melodicdictation.MelodicDictationQuestion method)}

\begin{fulllineitems}
\phantomsection\label{\detokenize{birdears.questions:birdears.questions.melodicdictation.MelodicDictationQuestion.__init__}}\pysiglinewithargsret{\sphinxbfcode{\_\_init\_\_}}{\emph{mode='major'}, \emph{max\_intervals=3}, \emph{n\_notes=4}, \emph{tonic=None}, \emph{octave=None}, \emph{descending=None}, \emph{chromatic=None}, \emph{n\_octaves=None}, \emph{valid\_intervals=None}, \emph{user\_durations=None}, \emph{prequestion\_method='progression\_i\_iv\_v\_i'}, \emph{resolution\_method='repeat\_only'}, \emph{*args}, \emph{**kwargs}}{}
Inits the class.
\begin{quote}\begin{description}
\item[{Parameters}] \leavevmode\begin{itemize}
\item {} 
\sphinxstyleliteralstrong{mode} (\sphinxstyleliteralemphasis{str}) \textendash{} A string representing the mode of the question.
Eg., ‘major’ or ‘minor’.

\item {} 
\sphinxstyleliteralstrong{max\_intervals} (\sphinxstyleliteralemphasis{int}) \textendash{} The maximum number of random intervals
the question will have.

\item {} 
\sphinxstyleliteralstrong{n\_notes} (\sphinxstyleliteralemphasis{int}) \textendash{} The number of notes the melodic dictation will have.

\item {} 
\sphinxstyleliteralstrong{tonic} (\sphinxstyleliteralemphasis{str}) \textendash{} A string representing the tonic of the question,
eg.: ‘C’; if omitted, it will be selected randomly.

\item {} 
\sphinxstyleliteralstrong{octave} (\sphinxstyleliteralemphasis{int}) \textendash{} A scienfic octave notation, for example, 4 for ‘C4’;
if not present, it will be randomly chosen.

\item {} 
\sphinxstyleliteralstrong{descending} (\sphinxstyleliteralemphasis{bool}) \textendash{} Is the question direction in descending, ie.,
intervals have lower pitch than the tonic.

\item {} 
\sphinxstyleliteralstrong{chromatic} (\sphinxstyleliteralemphasis{bool}) \textendash{} If the question can have (True) or not (False)
chromatic intervals, ie., intervals not in the diatonic scale
of tonic/mode.

\item {} 
\sphinxstyleliteralstrong{n\_octaves} (\sphinxstyleliteralemphasis{int}) \textendash{} Maximum number of octaves of the question.

\item {} 
\sphinxstyleliteralstrong{valid\_intervals} (\sphinxstyleliteralemphasis{list}) \textendash{} A list with intervals (int) valid for
random choice, 1 is 1st, 2 is second etc. Eg. {[}1, 4, 5{]} to
allow only tonics, fourths and fifths.

\item {} 
\sphinxstyleliteralstrong{user\_durations} (\sphinxstyleliteralemphasis{str}) \textendash{} 
A string with 9 comma-separated \sphinxtitleref{int} or
\sphinxtitleref{float{}`s to set the default duration for the notes played. The
values are respectively for: pre-question duration (1st),
pre-question delay (2nd), and pre-question pos-delay (3rd);
question duration (4th), question delay (5th), and question
pos-delay (6th); resolution duration (7th), resolution
delay (8th), and resolution pos-delay (9th).
duration is the duration in of the note in seconds; delay is
the time to wait before playing the next note, and pos\_delay is
the time to wait after all the notes of the respective sequence
have been played. If any of the user durations is {}`n}, the
default duration for the type of question will be used instead.
Example:

\begin{sphinxVerbatim}[commandchars=\\\{\}]
\PYGZdq{}2,0.5,1,2,n,0,2.5,n,1\PYGZdq{}
\end{sphinxVerbatim}


\item {} 
\sphinxstyleliteralstrong{prequestion\_method} (\sphinxstyleliteralemphasis{str}) \textendash{} Method of playing a cadence or the
exercise tonic before the question so to affirm the question
musical tonic key to the ear. Valid ones are registered in the
\sphinxtitleref{birdears.prequestion.PREQUESION\_METHODS} global dict.

\item {} 
\sphinxstyleliteralstrong{resolution\_method} (\sphinxstyleliteralemphasis{str}) \textendash{} Method of playing the resolution of an
exercise. Valid ones are registered in the
\sphinxtitleref{birdears.resolution.RESOLUTION\_METHODS} global dict.

\end{itemize}

\end{description}\end{quote}

\end{fulllineitems}

\index{check\_question() (birdears.questions.melodicdictation.MelodicDictationQuestion method)}

\begin{fulllineitems}
\phantomsection\label{\detokenize{birdears.questions:birdears.questions.melodicdictation.MelodicDictationQuestion.check_question}}\pysiglinewithargsret{\sphinxbfcode{check\_question}}{\emph{user\_input\_keys}}{}
Checks whether the given answer is correct.

\end{fulllineitems}

\index{make\_pre\_question() (birdears.questions.melodicdictation.MelodicDictationQuestion method)}

\begin{fulllineitems}
\phantomsection\label{\detokenize{birdears.questions:birdears.questions.melodicdictation.MelodicDictationQuestion.make_pre_question}}\pysiglinewithargsret{\sphinxbfcode{make\_pre\_question}}{\emph{method}}{}
\end{fulllineitems}

\index{make\_question() (birdears.questions.melodicdictation.MelodicDictationQuestion method)}

\begin{fulllineitems}
\phantomsection\label{\detokenize{birdears.questions:birdears.questions.melodicdictation.MelodicDictationQuestion.make_question}}\pysiglinewithargsret{\sphinxbfcode{make\_question}}{\emph{phrase\_semitones}}{}
\end{fulllineitems}

\index{make\_resolution() (birdears.questions.melodicdictation.MelodicDictationQuestion method)}

\begin{fulllineitems}
\phantomsection\label{\detokenize{birdears.questions:birdears.questions.melodicdictation.MelodicDictationQuestion.make_resolution}}\pysiglinewithargsret{\sphinxbfcode{make\_resolution}}{\emph{method}}{}
\end{fulllineitems}

\index{play\_question() (birdears.questions.melodicdictation.MelodicDictationQuestion method)}

\begin{fulllineitems}
\phantomsection\label{\detokenize{birdears.questions:birdears.questions.melodicdictation.MelodicDictationQuestion.play_question}}\pysiglinewithargsret{\sphinxbfcode{play\_question}}{}{}
\end{fulllineitems}

\index{play\_resolution() (birdears.questions.melodicdictation.MelodicDictationQuestion method)}

\begin{fulllineitems}
\phantomsection\label{\detokenize{birdears.questions:birdears.questions.melodicdictation.MelodicDictationQuestion.play_resolution}}\pysiglinewithargsret{\sphinxbfcode{play\_resolution}}{}{}
\end{fulllineitems}


\end{fulllineitems}



\subsubsection{birdears.questions.melodicinterval module}
\label{\detokenize{birdears.questions:birdears-questions-melodicinterval-module}}\label{\detokenize{birdears.questions:module-birdears.questions.melodicinterval}}\index{birdears.questions.melodicinterval (module)}\index{MelodicIntervalQuestion (class in birdears.questions.melodicinterval)}

\begin{fulllineitems}
\phantomsection\label{\detokenize{birdears.questions:birdears.questions.melodicinterval.MelodicIntervalQuestion}}\pysiglinewithargsret{\sphinxbfcode{class }\sphinxcode{birdears.questions.melodicinterval.}\sphinxbfcode{MelodicIntervalQuestion}}{\emph{mode='major'}, \emph{tonic=None}, \emph{octave=None}, \emph{descending=None}, \emph{chromatic=None}, \emph{n\_octaves=None}, \emph{valid\_intervals=None}, \emph{user\_durations=None}, \emph{prequestion\_method='tonic\_only'}, \emph{resolution\_method='nearest\_tonic'}, \emph{*args}, \emph{**kwargs}}{}
Bases: {\hyperref[\detokenize{index:birdears.questionbase.QuestionBase}]{\sphinxcrossref{\sphinxcode{birdears.questionbase.QuestionBase}}}}

Implements a Melodic Interval test.
\index{check\_question() (birdears.questions.melodicinterval.MelodicIntervalQuestion method)}

\begin{fulllineitems}
\phantomsection\label{\detokenize{birdears.questions:birdears.questions.melodicinterval.MelodicIntervalQuestion.check_question}}\pysiglinewithargsret{\sphinxbfcode{check\_question}}{\emph{user\_input\_char}}{}
Checks whether the given answer is correct.

\end{fulllineitems}

\index{make\_pre\_question() (birdears.questions.melodicinterval.MelodicIntervalQuestion method)}

\begin{fulllineitems}
\phantomsection\label{\detokenize{birdears.questions:birdears.questions.melodicinterval.MelodicIntervalQuestion.make_pre_question}}\pysiglinewithargsret{\sphinxbfcode{make\_pre\_question}}{\emph{method}}{}
\end{fulllineitems}

\index{make\_question() (birdears.questions.melodicinterval.MelodicIntervalQuestion method)}

\begin{fulllineitems}
\phantomsection\label{\detokenize{birdears.questions:birdears.questions.melodicinterval.MelodicIntervalQuestion.make_question}}\pysiglinewithargsret{\sphinxbfcode{make\_question}}{}{}
\end{fulllineitems}

\index{make\_resolution() (birdears.questions.melodicinterval.MelodicIntervalQuestion method)}

\begin{fulllineitems}
\phantomsection\label{\detokenize{birdears.questions:birdears.questions.melodicinterval.MelodicIntervalQuestion.make_resolution}}\pysiglinewithargsret{\sphinxbfcode{make\_resolution}}{\emph{method}}{}
\end{fulllineitems}

\index{play\_question() (birdears.questions.melodicinterval.MelodicIntervalQuestion method)}

\begin{fulllineitems}
\phantomsection\label{\detokenize{birdears.questions:birdears.questions.melodicinterval.MelodicIntervalQuestion.play_question}}\pysiglinewithargsret{\sphinxbfcode{play\_question}}{}{}
\end{fulllineitems}

\index{play\_resolution() (birdears.questions.melodicinterval.MelodicIntervalQuestion method)}

\begin{fulllineitems}
\phantomsection\label{\detokenize{birdears.questions:birdears.questions.melodicinterval.MelodicIntervalQuestion.play_resolution}}\pysiglinewithargsret{\sphinxbfcode{play\_resolution}}{}{}
\end{fulllineitems}


\end{fulllineitems}



\section{Submodules}
\label{\detokenize{birdears:submodules}}

\section{birdears.interval module}
\label{\detokenize{birdears:module-birdears.interval}}\label{\detokenize{birdears:birdears-interval-module}}\index{birdears.interval (module)}\index{ChromaticInterval (class in birdears.interval)}

\begin{fulllineitems}
\phantomsection\label{\detokenize{birdears:birdears.interval.ChromaticInterval}}\pysiglinewithargsret{\sphinxbfcode{class }\sphinxcode{birdears.interval.}\sphinxbfcode{ChromaticInterval}}{\emph{mode}, \emph{tonic}, \emph{octave}, \emph{n\_octaves=None}, \emph{descending=None}, \emph{valid\_intervals=None}}{}
Bases: {\hyperref[\detokenize{index:birdears.interval.IntervalBase}]{\sphinxcrossref{\sphinxcode{birdears.interval.IntervalBase}}}}

Chooses a diatonic interval for the question.
\index{tonic\_octave (birdears.interval.ChromaticInterval attribute)}

\begin{fulllineitems}
\phantomsection\label{\detokenize{birdears:birdears.interval.ChromaticInterval.tonic_octave}}\pysigline{\sphinxbfcode{tonic\_octave}}
\sphinxstyleemphasis{int} \textendash{} Scientific octave for the tonic. For example, if
the tonic is a ‘C4’ then \sphinxtitleref{tonic\_octave} is 4.

\end{fulllineitems}



\begin{fulllineitems}
\pysigline{\sphinxbfcode{interval~octave}}
\sphinxstyleemphasis{int} \textendash{} Scientific octave for the interval. For example,
if the interval is a ‘G5’ then \sphinxtitleref{tonic\_octave} is 5.

\end{fulllineitems}

\index{chromatic\_offset (birdears.interval.ChromaticInterval attribute)}

\begin{fulllineitems}
\phantomsection\label{\detokenize{birdears:birdears.interval.ChromaticInterval.chromatic_offset}}\pysigline{\sphinxbfcode{chromatic\_offset}}
\sphinxstyleemphasis{int} \textendash{} The offset in semitones inside one octave;
maybe it will be deprecated in favour of \sphinxtitleref{distance{[}‘semitones’{]}}
which is the same.

\end{fulllineitems}

\index{note\_and\_octave (birdears.interval.ChromaticInterval attribute)}

\begin{fulllineitems}
\phantomsection\label{\detokenize{birdears:birdears.interval.ChromaticInterval.note_and_octave}}\pysigline{\sphinxbfcode{note\_and\_octave}}
\sphinxstyleemphasis{str} \textendash{} Note and octave of the interval, for example, if
the interval is G5 the note name is ‘G5’.

\end{fulllineitems}

\index{note\_name (birdears.interval.ChromaticInterval attribute)}

\begin{fulllineitems}
\phantomsection\label{\detokenize{birdears:birdears.interval.ChromaticInterval.note_name}}\pysigline{\sphinxbfcode{note\_name}}
\sphinxstyleemphasis{str} \textendash{} The note name of the interval, for example, if the
interval is G5 then the name is ‘G’.

\end{fulllineitems}

\index{semitones (birdears.interval.ChromaticInterval attribute)}

\begin{fulllineitems}
\phantomsection\label{\detokenize{birdears:birdears.interval.ChromaticInterval.semitones}}\pysigline{\sphinxbfcode{semitones}}
\sphinxstyleemphasis{int} \textendash{} Semitones from tonic to octave. If tonic is C4 and
interval is G5 the number of semitones is 19.

\end{fulllineitems}

\index{is\_chromatic (birdears.interval.ChromaticInterval attribute)}

\begin{fulllineitems}
\phantomsection\label{\detokenize{birdears:birdears.interval.ChromaticInterval.is_chromatic}}\pysigline{\sphinxbfcode{is\_chromatic}}
\sphinxstyleemphasis{bool} \textendash{} If the current interval is chromatic (True) or if
it exists in the diatonic scale which key is tonic.

\end{fulllineitems}

\index{is\_descending (birdears.interval.ChromaticInterval attribute)}

\begin{fulllineitems}
\phantomsection\label{\detokenize{birdears:birdears.interval.ChromaticInterval.is_descending}}\pysigline{\sphinxbfcode{is\_descending}}
\sphinxstyleemphasis{bool} \textendash{} If the interval has a descending direction, ie.,
has a lower pitch than the tonic.

\end{fulllineitems}

\index{diatonic\_index (birdears.interval.ChromaticInterval attribute)}

\begin{fulllineitems}
\phantomsection\label{\detokenize{birdears:birdears.interval.ChromaticInterval.diatonic_index}}\pysigline{\sphinxbfcode{diatonic\_index}}
\sphinxstyleemphasis{int} \textendash{} If the interval is chromatic, this will be the
nearest diatonic interval in the direction of the resolution
(closest tonic.) From II to IV degrees, it is the ditonic interval
before; from V to VII it is the diatonic interval after.

\end{fulllineitems}

\index{distance (birdears.interval.ChromaticInterval attribute)}

\begin{fulllineitems}
\phantomsection\label{\detokenize{birdears:birdears.interval.ChromaticInterval.distance}}\pysigline{\sphinxbfcode{distance}}
\sphinxstyleemphasis{dict} \textendash{} A dictionary which the distance from tonic to
interval, for example, if tonic is C4 and interval is G5:

\begin{sphinxVerbatim}[commandchars=\\\{\}]
\PYGZob{}
    \PYGZsq{}octaves\PYGZsq{}: 1,
    \PYGZsq{}semitones\PYGZsq{}: 7
\PYGZcb{}
\end{sphinxVerbatim}

\end{fulllineitems}

\index{data (birdears.interval.ChromaticInterval attribute)}

\begin{fulllineitems}
\phantomsection\label{\detokenize{birdears:birdears.interval.ChromaticInterval.data}}\pysigline{\sphinxbfcode{data}}
\sphinxstyleemphasis{tuple} \textendash{} A tuple representing the interval data in the form of
(semitones, short\_name, long\_name), for example:

\begin{sphinxVerbatim}[commandchars=\\\{\}]
(19, \PYGZsq{}P12\PYGZsq{}, \PYGZsq{}Perfect Twelfth\PYGZsq{})
\end{sphinxVerbatim}

\end{fulllineitems}


\begin{sphinxadmonition}{note}{\label{birdears:index-0}Todo:}\begin{itemize}
\item {} \begin{description}
\item[{Maybe we should refactor some of the attributes with a tuple}] \leavevmode
(note, octave)

\end{description}

\item {} 
Maybe remove \sphinxtitleref{chromatic\_offset} in favor of \sphinxtitleref{distance{[}‘semitones’{]}{}`}

\end{itemize}
\end{sphinxadmonition}
\index{\_\_init\_\_() (birdears.interval.ChromaticInterval method)}

\begin{fulllineitems}
\phantomsection\label{\detokenize{birdears:birdears.interval.ChromaticInterval.__init__}}\pysiglinewithargsret{\sphinxbfcode{\_\_init\_\_}}{\emph{mode}, \emph{tonic}, \emph{octave}, \emph{n\_octaves=None}, \emph{descending=None}, \emph{valid\_intervals=None}}{}
Inits the class and choses a random interval with the given args.
\begin{quote}\begin{description}
\item[{Parameters}] \leavevmode\begin{itemize}
\item {} 
\sphinxstyleliteralstrong{mode} (\sphinxstyleliteralemphasis{str}) \textendash{} Diatonic mode for the interval.
(eg.: ‘major’ or ‘minor’)

\item {} 
\sphinxstyleliteralstrong{tonic} (\sphinxstyleliteralemphasis{str}) \textendash{} Tonic of the scale. (eg.: ‘Bb’)

\item {} 
\sphinxstyleliteralstrong{octave} (\sphinxstyleliteralemphasis{str}) \textendash{} Scientific octave of the scale (eg.: 4)

\item {} 
\sphinxstyleliteralstrong{interval} (\sphinxstyleliteralemphasis{str}) \textendash{} Not implemented. The interval.

\item {} 
\sphinxstyleliteralstrong{chromatic} (\sphinxstyleliteralemphasis{bool}) \textendash{} Can have chromatic notes? (eg.: F\# in a key
of C; default: false)

\item {} 
\sphinxstyleliteralstrong{n\_octaves} (\sphinxstyleliteralemphasis{int}) \textendash{} Maximum number os octaves (eg. 2)

\item {} 
\sphinxstyleliteralstrong{descending} (\sphinxstyleliteralemphasis{bool}) \textendash{} Is the interval descending? (default: false)

\item {} 
\sphinxstyleliteralstrong{valid\_intervals} (\sphinxstyleliteralemphasis{int}) \textendash{} A list with inervals valid for random
choice, 1 is 1st, 2 is second etc.

\end{itemize}

\end{description}\end{quote}

\end{fulllineitems}


\end{fulllineitems}

\index{DiatonicInterval (class in birdears.interval)}

\begin{fulllineitems}
\phantomsection\label{\detokenize{birdears:birdears.interval.DiatonicInterval}}\pysiglinewithargsret{\sphinxbfcode{class }\sphinxcode{birdears.interval.}\sphinxbfcode{DiatonicInterval}}{\emph{mode}, \emph{tonic}, \emph{octave}, \emph{n\_octaves=None}, \emph{descending=None}, \emph{valid\_intervals=None}}{}
Bases: {\hyperref[\detokenize{index:birdears.interval.IntervalBase}]{\sphinxcrossref{\sphinxcode{birdears.interval.IntervalBase}}}}

Chooses a diatonic interval for the question.
\index{tonic\_octave (birdears.interval.DiatonicInterval attribute)}

\begin{fulllineitems}
\phantomsection\label{\detokenize{birdears:birdears.interval.DiatonicInterval.tonic_octave}}\pysigline{\sphinxbfcode{tonic\_octave}}
\sphinxstyleemphasis{int} \textendash{} Scientific octave for the tonic. For example, if
the tonic is a ‘C4’ then \sphinxtitleref{tonic\_octave} is 4.

\end{fulllineitems}



\begin{fulllineitems}
\pysigline{\sphinxbfcode{interval~octave}}
\sphinxstyleemphasis{int} \textendash{} Scientific octave for the interval. For example,
if the interval is a ‘G5’ then \sphinxtitleref{tonic\_octave} is 5.

\end{fulllineitems}

\index{chromatic\_offset (birdears.interval.DiatonicInterval attribute)}

\begin{fulllineitems}
\phantomsection\label{\detokenize{birdears:birdears.interval.DiatonicInterval.chromatic_offset}}\pysigline{\sphinxbfcode{chromatic\_offset}}
\sphinxstyleemphasis{int} \textendash{} The offset in semitones inside one octave.
Relative semitones to tonic.

\end{fulllineitems}

\index{note\_and\_octave (birdears.interval.DiatonicInterval attribute)}

\begin{fulllineitems}
\phantomsection\label{\detokenize{birdears:birdears.interval.DiatonicInterval.note_and_octave}}\pysigline{\sphinxbfcode{note\_and\_octave}}
\sphinxstyleemphasis{str} \textendash{} Note and octave of the interval, for example, if
the interval is G5 the note name is ‘G5’.

\end{fulllineitems}

\index{note\_name (birdears.interval.DiatonicInterval attribute)}

\begin{fulllineitems}
\phantomsection\label{\detokenize{birdears:birdears.interval.DiatonicInterval.note_name}}\pysigline{\sphinxbfcode{note\_name}}
\sphinxstyleemphasis{str} \textendash{} The note name of the interval, for example, if the
interval is G5 then the name is ‘G’.

\end{fulllineitems}

\index{semitones (birdears.interval.DiatonicInterval attribute)}

\begin{fulllineitems}
\phantomsection\label{\detokenize{birdears:birdears.interval.DiatonicInterval.semitones}}\pysigline{\sphinxbfcode{semitones}}
\sphinxstyleemphasis{int} \textendash{} Semitones from tonic to octave. If tonic is C4 and
interval is G5 the number of semitones is 19.

\end{fulllineitems}

\index{is\_chromatic (birdears.interval.DiatonicInterval attribute)}

\begin{fulllineitems}
\phantomsection\label{\detokenize{birdears:birdears.interval.DiatonicInterval.is_chromatic}}\pysigline{\sphinxbfcode{is\_chromatic}}
\sphinxstyleemphasis{bool} \textendash{} If the current interval is chromatic (True) or if
it exists in the diatonic scale which key is tonic.

\end{fulllineitems}

\index{is\_descending (birdears.interval.DiatonicInterval attribute)}

\begin{fulllineitems}
\phantomsection\label{\detokenize{birdears:birdears.interval.DiatonicInterval.is_descending}}\pysigline{\sphinxbfcode{is\_descending}}
\sphinxstyleemphasis{bool} \textendash{} If the interval has a descending direction, ie.,
has a lower pitch than the tonic.

\end{fulllineitems}

\index{diatonic\_index (birdears.interval.DiatonicInterval attribute)}

\begin{fulllineitems}
\phantomsection\label{\detokenize{birdears:birdears.interval.DiatonicInterval.diatonic_index}}\pysigline{\sphinxbfcode{diatonic\_index}}
\sphinxstyleemphasis{int} \textendash{} If the interval is chromatic, this will be the
nearest diatonic interval in the direction of the resolution
(closest tonic.) From II to IV degrees, it is the ditonic interval
before; from V to VII it is the diatonic interval after.

\end{fulllineitems}

\index{distance (birdears.interval.DiatonicInterval attribute)}

\begin{fulllineitems}
\phantomsection\label{\detokenize{birdears:birdears.interval.DiatonicInterval.distance}}\pysigline{\sphinxbfcode{distance}}
\sphinxstyleemphasis{dict} \textendash{} A dictionary which the distance from tonic to
interval, for example, if tonic is C4 and interval is G5:

\begin{sphinxVerbatim}[commandchars=\\\{\}]
\PYGZob{}
    \PYGZsq{}octaves\PYGZsq{}: 1,
    \PYGZsq{}semitones\PYGZsq{}: 7
\PYGZcb{}
\end{sphinxVerbatim}

\end{fulllineitems}

\index{data (birdears.interval.DiatonicInterval attribute)}

\begin{fulllineitems}
\phantomsection\label{\detokenize{birdears:birdears.interval.DiatonicInterval.data}}\pysigline{\sphinxbfcode{data}}
\sphinxstyleemphasis{tuple} \textendash{} A tuple representing the interval data in the form of
(semitones, short\_name, long\_name), for example:

\begin{sphinxVerbatim}[commandchars=\\\{\}]
(19, \PYGZsq{}P12\PYGZsq{}, \PYGZsq{}Perfect Twelfth\PYGZsq{})
\end{sphinxVerbatim}

\end{fulllineitems}

\index{\_\_init\_\_() (birdears.interval.DiatonicInterval method)}

\begin{fulllineitems}
\phantomsection\label{\detokenize{birdears:birdears.interval.DiatonicInterval.__init__}}\pysiglinewithargsret{\sphinxbfcode{\_\_init\_\_}}{\emph{mode}, \emph{tonic}, \emph{octave}, \emph{n\_octaves=None}, \emph{descending=None}, \emph{valid\_intervals=None}}{}
Inits the class and choses a random interval with the given args.
\begin{quote}\begin{description}
\item[{Parameters}] \leavevmode\begin{itemize}
\item {} 
\sphinxstyleliteralstrong{mode} (\sphinxstyleliteralemphasis{str}) \textendash{} Diatonic mode for the interval.
(eg.: ‘major’ or ‘minor’)

\item {} 
\sphinxstyleliteralstrong{tonic} (\sphinxstyleliteralemphasis{str}) \textendash{} Tonic of the scale. (eg.: ‘Bb’)

\item {} 
\sphinxstyleliteralstrong{octave} (\sphinxstyleliteralemphasis{str}) \textendash{} Scientific octave of the scale (eg.: 4)

\item {} 
\sphinxstyleliteralstrong{n\_octaves} (\sphinxstyleliteralemphasis{int}) \textendash{} Maximum number os octaves (eg. 2)

\item {} 
\sphinxstyleliteralstrong{descending} (\sphinxstyleliteralemphasis{bool}) \textendash{} Is the interval descending? (default: false)

\item {} 
\sphinxstyleliteralstrong{valid\_intervals} (\sphinxstyleliteralemphasis{int}) \textendash{} A list with intervals (int) valid for random
choice, 1 is 1st, 2 is second etc.

\end{itemize}

\end{description}\end{quote}

\end{fulllineitems}


\end{fulllineitems}

\index{IntervalBase (class in birdears.interval)}

\begin{fulllineitems}
\phantomsection\label{\detokenize{birdears:birdears.interval.IntervalBase}}\pysigline{\sphinxbfcode{class }\sphinxcode{birdears.interval.}\sphinxbfcode{IntervalBase}}
Bases: \sphinxcode{object}
\index{\_\_init\_\_() (birdears.interval.IntervalBase method)}

\begin{fulllineitems}
\phantomsection\label{\detokenize{birdears:birdears.interval.IntervalBase.__init__}}\pysiglinewithargsret{\sphinxbfcode{\_\_init\_\_}}{}{}
Base class for interval classes.

\end{fulllineitems}

\index{return\_simple() (birdears.interval.IntervalBase method)}

\begin{fulllineitems}
\phantomsection\label{\detokenize{birdears:birdears.interval.IntervalBase.return_simple}}\pysiglinewithargsret{\sphinxbfcode{return\_simple}}{\emph{keys}}{}
This method returns a dict with only the values passed to \sphinxtitleref{keys}.

\end{fulllineitems}


\end{fulllineitems}



\section{birdears.logger module}
\label{\detokenize{birdears:birdears-logger-module}}\label{\detokenize{birdears:module-birdears.logger}}\index{birdears.logger (module)}
This submodule exports \sphinxtitleref{logger} to log events.

Logging messages which are less severe than \sphinxtitleref{lvl} will be ignored:

\begin{sphinxVerbatim}[commandchars=\\\{\}]
Level       Numeric value
\PYGZhy{}\PYGZhy{}\PYGZhy{}\PYGZhy{}\PYGZhy{}       \PYGZhy{}\PYGZhy{}\PYGZhy{}\PYGZhy{}\PYGZhy{}\PYGZhy{}\PYGZhy{}\PYGZhy{}\PYGZhy{}\PYGZhy{}\PYGZhy{}\PYGZhy{}\PYGZhy{}
CRITICAL    50
ERROR       40
WARNING     30
INFO        20
DEBUG       10
NOTSET      0

Level       When it’s used
\PYGZhy{}\PYGZhy{}\PYGZhy{}\PYGZhy{}\PYGZhy{}       \PYGZhy{}\PYGZhy{}\PYGZhy{}\PYGZhy{}\PYGZhy{}\PYGZhy{}\PYGZhy{}\PYGZhy{}\PYGZhy{}\PYGZhy{}\PYGZhy{}\PYGZhy{}\PYGZhy{}\PYGZhy{}
DEBUG       Detailed information, typically of interest only when
                diagnosing problems.
INFO        Confirmation that things are working as expected.
WARNING     An indication that something unexpected happened, or indicative
                of some problem in the near future (e.g. ‘disk space low’).
                The software is still working as expected.
ERROR       Due to a more serious problem, the software has not been able
                to perform some function.
CRITICAL    A serious error, indicating that the program itself may be
                unable to continue running.
\end{sphinxVerbatim}
\index{log\_event() (in module birdears.logger)}

\begin{fulllineitems}
\phantomsection\label{\detokenize{birdears:birdears.logger.log_event}}\pysiglinewithargsret{\sphinxcode{birdears.logger.}\sphinxbfcode{log\_event}}{\emph{f}, \emph{*args}, \emph{**kwargs}}{}
Decorator. Functions and method decorated with this decorator will have
their signature logged when birdears is executed with \sphinxtitleref{\textendash{}debug} mode. Both
function signature with their call values and their return will be logged.

\end{fulllineitems}



\section{birdears.prequestion module}
\label{\detokenize{birdears:module-birdears.prequestion}}\label{\detokenize{birdears:birdears-prequestion-module}}\index{birdears.prequestion (module)}
This module implements pre-questions’ progressions.

Pre questions are chord progressions or notes played before the question is
played, so to affirmate the sound of the question’s key.

For example a common cadence is chords I-IV-V-I from the diatonic scale, which
in a key of \sphinxtitleref{C} is \sphinxtitleref{CM-FM-GM-CM} and in a key of \sphinxtitleref{A} is \sphinxtitleref{AM-DM-EM-AM}.

Pre-question methods should be decorated with \sphinxtitleref{register\_prequestion\_method}
decorator, so that they will be registered as a valid pre-question method.
\index{PreQuestion (class in birdears.prequestion)}

\begin{fulllineitems}
\phantomsection\label{\detokenize{birdears:birdears.prequestion.PreQuestion}}\pysiglinewithargsret{\sphinxbfcode{class }\sphinxcode{birdears.prequestion.}\sphinxbfcode{PreQuestion}}{\emph{method}, \emph{question}}{}
Bases: \sphinxcode{object}
\index{\_\_call\_\_() (birdears.prequestion.PreQuestion method)}

\begin{fulllineitems}
\phantomsection\label{\detokenize{birdears:birdears.prequestion.PreQuestion.__call__}}\pysiglinewithargsret{\sphinxbfcode{\_\_call\_\_}}{\emph{*args}, \emph{**kwargs}}{}
Calls the resolution method and pass arguments to it.

Returns a \sphinxtitleref{birdears.Sequence} object with the pre-question generated by
the method.

\end{fulllineitems}

\index{\_\_init\_\_() (birdears.prequestion.PreQuestion method)}

\begin{fulllineitems}
\phantomsection\label{\detokenize{birdears:birdears.prequestion.PreQuestion.__init__}}\pysiglinewithargsret{\sphinxbfcode{\_\_init\_\_}}{\emph{method}, \emph{question}}{}
This class implements methods for different types of pre-question
progressions.
\begin{quote}\begin{description}
\item[{Parameters}] \leavevmode\begin{itemize}
\item {} 
\sphinxstyleliteralstrong{method} (\sphinxstyleliteralemphasis{str}) \textendash{} The method used in the pre question.

\item {} 
\sphinxstyleliteralstrong{question} (\sphinxstyleliteralemphasis{obj}) \textendash{} Question object from which to generate the

\item {} 
\sphinxstyleliteralstrong{sequence.} (\sphinxstyleliteralemphasis{pre-question}) \textendash{} 

\end{itemize}

\end{description}\end{quote}

\end{fulllineitems}


\end{fulllineitems}

\index{none() (in module birdears.prequestion)}

\begin{fulllineitems}
\phantomsection\label{\detokenize{birdears:birdears.prequestion.none}}\pysiglinewithargsret{\sphinxcode{birdears.prequestion.}\sphinxbfcode{none}}{\emph{question}, \emph{*args}, \emph{**kwargs}}{}
Pre-question method that return an empty sequence with no delay.
:param question: Question object from which to generate the
\begin{quote}

pre-question sequence. (this is provided by the \sphinxtitleref{Resolution} class
when it is {\color{red}\bfseries{}{}`}\_\_call\_\_{}`ed)
\end{quote}
\begin{quote}\begin{description}
\end{description}\end{quote}

\end{fulllineitems}

\index{progression\_i\_iv\_v\_i() (in module birdears.prequestion)}

\begin{fulllineitems}
\phantomsection\label{\detokenize{birdears:birdears.prequestion.progression_i_iv_v_i}}\pysiglinewithargsret{\sphinxcode{birdears.prequestion.}\sphinxbfcode{progression\_i\_iv\_v\_i}}{\emph{question}, \emph{*args}, \emph{**kwargs}}{}
Pre-question method that play’s a chord progression with triad chords built
on the grades I, IV, V the I of the question key.
\begin{quote}\begin{description}
\item[{Parameters}] \leavevmode
\sphinxstyleliteralstrong{question} (\sphinxstyleliteralemphasis{obj}) \textendash{} Question object from which to generate the
pre-question sequence. (this is provided by the \sphinxtitleref{Resolution} class
when it is {\color{red}\bfseries{}{}`}\_\_call\_\_{}`ed)

\end{description}\end{quote}

\end{fulllineitems}

\index{register\_prequestion\_method() (in module birdears.prequestion)}

\begin{fulllineitems}
\phantomsection\label{\detokenize{birdears:birdears.prequestion.register_prequestion_method}}\pysiglinewithargsret{\sphinxcode{birdears.prequestion.}\sphinxbfcode{register\_prequestion\_method}}{\emph{f}, \emph{*args}, \emph{**kwargs}}{}
Decorator for prequestion method functions.

Functions decorated with this decorator will be registered in the
\sphinxtitleref{PREQUESTION\_METHODS} global dict.

\end{fulllineitems}

\index{tonic\_only() (in module birdears.prequestion)}

\begin{fulllineitems}
\phantomsection\label{\detokenize{birdears:birdears.prequestion.tonic_only}}\pysiglinewithargsret{\sphinxcode{birdears.prequestion.}\sphinxbfcode{tonic\_only}}{\emph{question}, \emph{*args}, \emph{**kwargs}}{}
Pre-question method that only play’s the question tonic note before the
question.
\begin{quote}\begin{description}
\item[{Parameters}] \leavevmode
\sphinxstyleliteralstrong{question} (\sphinxstyleliteralemphasis{obj}) \textendash{} Question object from which to generate the
pre-question sequence. (this is provided by the \sphinxtitleref{Resolution} class
when it is {\color{red}\bfseries{}{}`}\_\_call\_\_{}`ed)

\end{description}\end{quote}

\end{fulllineitems}



\section{birdears.questionbase module}
\label{\detokenize{birdears:module-birdears.questionbase}}\label{\detokenize{birdears:birdears-questionbase-module}}\index{birdears.questionbase (module)}\index{QuestionBase (class in birdears.questionbase)}

\begin{fulllineitems}
\phantomsection\label{\detokenize{birdears:birdears.questionbase.QuestionBase}}\pysiglinewithargsret{\sphinxbfcode{class }\sphinxcode{birdears.questionbase.}\sphinxbfcode{QuestionBase}}{\emph{mode='major'}, \emph{tonic=None}, \emph{octave=None}, \emph{descending=None}, \emph{chromatic=None}, \emph{n\_octaves=None}, \emph{valid\_intervals=None}, \emph{user\_durations=None}, \emph{prequestion\_method=None}, \emph{resolution\_method=None}, \emph{default\_durations=None}, \emph{*args}, \emph{**kwargs}}{}
Bases: \sphinxcode{object}

Base Class to be subclassed for Question classes.

This class implements attributes and routines to be used in Question
subclasses.
\index{\_\_init\_\_() (birdears.questionbase.QuestionBase method)}

\begin{fulllineitems}
\phantomsection\label{\detokenize{birdears:birdears.questionbase.QuestionBase.__init__}}\pysiglinewithargsret{\sphinxbfcode{\_\_init\_\_}}{\emph{mode='major'}, \emph{tonic=None}, \emph{octave=None}, \emph{descending=None}, \emph{chromatic=None}, \emph{n\_octaves=None}, \emph{valid\_intervals=None}, \emph{user\_durations=None}, \emph{prequestion\_method=None}, \emph{resolution\_method=None}, \emph{default\_durations=None}, \emph{*args}, \emph{**kwargs}}{}
Inits the class.
\begin{quote}\begin{description}
\item[{Parameters}] \leavevmode\begin{itemize}
\item {} 
\sphinxstyleliteralstrong{mode} (\sphinxstyleliteralemphasis{str}) \textendash{} A string represnting the mode of the question.
Eg., ‘major’ or ‘minor’

\item {} 
\sphinxstyleliteralstrong{tonic} (\sphinxstyleliteralemphasis{str}) \textendash{} A string representing the tonic of the
question, eg.: ‘C’; if omitted, it will be selected
randomly.

\item {} 
\sphinxstyleliteralstrong{octave} (\sphinxstyleliteralemphasis{int}) \textendash{} A scienfic octave notation, for example,
4 for ‘C4’; if not present, it will be randomly chosen.

\item {} 
\sphinxstyleliteralstrong{descending} (\sphinxstyleliteralemphasis{bool}) \textendash{} Is the question direction in descending,
ie., intervals have lower pitch than the tonic.

\item {} 
\sphinxstyleliteralstrong{chromatic} (\sphinxstyleliteralemphasis{bool}) \textendash{} If the question can have (True) or not
(False) chromatic intervals, ie., intervals not in the
diatonic scale of tonic/mode.

\item {} 
\sphinxstyleliteralstrong{n\_octaves} (\sphinxstyleliteralemphasis{int}) \textendash{} Maximum numbr of octaves of the question.

\item {} 
\sphinxstyleliteralstrong{valid\_intervals} (\sphinxstyleliteralemphasis{list}) \textendash{} A list with intervals (int) valid for
random choice, 1 is 1st, 2 is second etc. Eg. {[}1, 4, 5{]} to
allow only tonics, fourths and fifths.

\item {} 
\sphinxstyleliteralstrong{user\_durations} (\sphinxstyleliteralemphasis{dict}) \textendash{} 
A string with 9 comma-separated \sphinxtitleref{int} or
\sphinxtitleref{float{}`s to set the default duration for the notes played. The
values are respectively for: pre-question duration (1st),
pre-question delay (2nd), and pre-question pos-delay (3rd);
question duration (4th), question delay (5th), and question
pos-delay (6th); resolution duration (7th), resolution
delay (8th), and resolution pos-delay (9th).
duration is the duration in of the note in seconds; delay is
the time to wait before playing the next note, and pos\_delay is
the time to wait after all the notes of the respective sequence
have been played. If any of the user durations is {}`n}, the
default duration for the type of question will be used instead.
Example:

\begin{sphinxVerbatim}[commandchars=\\\{\}]
\PYGZdq{}2,0.5,1,2,n,0,2.5,n,1\PYGZdq{}
\end{sphinxVerbatim}


\item {} 
\sphinxstyleliteralstrong{prequestion\_method} (\sphinxstyleliteralemphasis{str}) \textendash{} Method of playing a cadence or the
exercise tonic before the question so to affirm the question
musical tonic key to the ear. Valid ones are registered in the
\sphinxtitleref{birdears.prequestion.PREQUESION\_METHODS} global dict.

\item {} 
\sphinxstyleliteralstrong{resolution\_method} (\sphinxstyleliteralemphasis{str}) \textendash{} Method of playing the resolution of an
exercise Valid ones are registered in the
\sphinxtitleref{birdears.resolution.RESOLUTION\_METHODS} global dict.

\item {} 
\sphinxstyleliteralstrong{user\_durations} \textendash{} Dictionary with the default durations for
each type of sequence. This is provided by the subclasses.

\end{itemize}

\end{description}\end{quote}

\end{fulllineitems}

\index{check\_question() (birdears.questionbase.QuestionBase method)}

\begin{fulllineitems}
\phantomsection\label{\detokenize{birdears:birdears.questionbase.QuestionBase.check_question}}\pysiglinewithargsret{\sphinxbfcode{check\_question}}{}{}
This method should be overwritten by the question subclasses.

\end{fulllineitems}

\index{get\_valid\_semitones() (birdears.questionbase.QuestionBase method)}

\begin{fulllineitems}
\phantomsection\label{\detokenize{birdears:birdears.questionbase.QuestionBase.get_valid_semitones}}\pysiglinewithargsret{\sphinxbfcode{get\_valid\_semitones}}{}{}
Returns a list with valid semitones for question.

\end{fulllineitems}

\index{make\_question() (birdears.questionbase.QuestionBase method)}

\begin{fulllineitems}
\phantomsection\label{\detokenize{birdears:birdears.questionbase.QuestionBase.make_question}}\pysiglinewithargsret{\sphinxbfcode{make\_question}}{}{}
This method should be overwritten by the question subclasses.

\end{fulllineitems}

\index{make\_resolution() (birdears.questionbase.QuestionBase method)}

\begin{fulllineitems}
\phantomsection\label{\detokenize{birdears:birdears.questionbase.QuestionBase.make_resolution}}\pysiglinewithargsret{\sphinxbfcode{make\_resolution}}{}{}
This method should be overwritten by the question subclasses.

\end{fulllineitems}

\index{play\_question() (birdears.questionbase.QuestionBase method)}

\begin{fulllineitems}
\phantomsection\label{\detokenize{birdears:birdears.questionbase.QuestionBase.play_question}}\pysiglinewithargsret{\sphinxbfcode{play\_question}}{}{}
This method should be overwritten by the question subclasses.

\end{fulllineitems}


\end{fulllineitems}

\index{register\_question\_class() (in module birdears.questionbase)}

\begin{fulllineitems}
\phantomsection\label{\detokenize{birdears:birdears.questionbase.register_question_class}}\pysiglinewithargsret{\sphinxcode{birdears.questionbase.}\sphinxbfcode{register\_question\_class}}{\emph{f}, \emph{*args}, \emph{**kwargs}}{}
Decorator for question classes.

Classes decorated with this decorator will be registered in the
\sphinxtitleref{QUESTION\_CLASSES} global.

\end{fulllineitems}



\section{birdears.resolution module}
\label{\detokenize{birdears:birdears-resolution-module}}\label{\detokenize{birdears:module-birdears.resolution}}\index{birdears.resolution (module)}\index{Resolution (class in birdears.resolution)}

\begin{fulllineitems}
\phantomsection\label{\detokenize{birdears:birdears.resolution.Resolution}}\pysiglinewithargsret{\sphinxbfcode{class }\sphinxcode{birdears.resolution.}\sphinxbfcode{Resolution}}{\emph{method}, \emph{question}}{}
Bases: \sphinxcode{object}

This class implements methods for different types of question
resolutions.

A resolution is an answer to a question. It aims to create a mnemonic on
how the inverval resvolver to the tonic.
\index{\_\_call\_\_() (birdears.resolution.Resolution method)}

\begin{fulllineitems}
\phantomsection\label{\detokenize{birdears:birdears.resolution.Resolution.__call__}}\pysiglinewithargsret{\sphinxbfcode{\_\_call\_\_}}{\emph{*args}, \emph{**kwargs}}{}
Calls the resolution method and pass arguments to it.

Returns a \sphinxtitleref{birdears.Sequence} object with the resolution generated by
the.method.

\end{fulllineitems}

\index{\_\_init\_\_() (birdears.resolution.Resolution method)}

\begin{fulllineitems}
\phantomsection\label{\detokenize{birdears:birdears.resolution.Resolution.__init__}}\pysiglinewithargsret{\sphinxbfcode{\_\_init\_\_}}{\emph{method}, \emph{question}}{}
Inits the resolution class.
\begin{quote}\begin{description}
\item[{Parameters}] \leavevmode\begin{itemize}
\item {} 
\sphinxstyleliteralstrong{method} (\sphinxstyleliteralemphasis{str}) \textendash{} The method used in the resolution.

\item {} 
\sphinxstyleliteralstrong{question} (\sphinxstyleliteralemphasis{obj}) \textendash{} Question object from which to generate the

\item {} 
\sphinxstyleliteralstrong{sequence.} ({\hyperref[\detokenize{birdears:module-birdears.resolution}]{\sphinxcrossref{\sphinxstyleliteralemphasis{resolution}}}}) \textendash{} 

\end{itemize}

\end{description}\end{quote}

\end{fulllineitems}


\end{fulllineitems}

\index{nearest\_tonic() (in module birdears.resolution)}

\begin{fulllineitems}
\phantomsection\label{\detokenize{birdears:birdears.resolution.nearest_tonic}}\pysiglinewithargsret{\sphinxcode{birdears.resolution.}\sphinxbfcode{nearest\_tonic}}{\emph{question}}{}
Resolution method that resolve the intervals to their nearest tonics.
\begin{quote}\begin{description}
\item[{Parameters}] \leavevmode
\sphinxstyleliteralstrong{question} (\sphinxstyleliteralemphasis{obj}) \textendash{} Question object from which to generate the
resolution sequence. (this is provided by the \sphinxtitleref{Prequestion} class
when it is {\color{red}\bfseries{}{}`}\_\_call\_\_{}`ed)

\end{description}\end{quote}

\end{fulllineitems}

\index{register\_resolution\_method() (in module birdears.resolution)}

\begin{fulllineitems}
\phantomsection\label{\detokenize{birdears:birdears.resolution.register_resolution_method}}\pysiglinewithargsret{\sphinxcode{birdears.resolution.}\sphinxbfcode{register\_resolution\_method}}{\emph{f}, \emph{*args}, \emph{**kwargs}}{}
Decorator for resolution method functions.

Functions decorated with this decorator will be registered in the
\sphinxtitleref{RESOLUTION\_METHODS} global dict.

\end{fulllineitems}

\index{repeat\_only() (in module birdears.resolution)}

\begin{fulllineitems}
\phantomsection\label{\detokenize{birdears:birdears.resolution.repeat_only}}\pysiglinewithargsret{\sphinxcode{birdears.resolution.}\sphinxbfcode{repeat\_only}}{\emph{question}}{}
Resolution method that only repeats the sequence elements with given
durations.
\begin{quote}\begin{description}
\item[{Parameters}] \leavevmode
\sphinxstyleliteralstrong{question} (\sphinxstyleliteralemphasis{obj}) \textendash{} Question object from which to generate the
resolution sequence. (this is provided by the \sphinxtitleref{Prequestion} class
when it is {\color{red}\bfseries{}{}`}\_\_call\_\_{}`ed)

\end{description}\end{quote}

\end{fulllineitems}



\section{birdears.scale module}
\label{\detokenize{birdears:module-birdears.scale}}\label{\detokenize{birdears:birdears-scale-module}}\index{birdears.scale (module)}\index{ChromaticScale (class in birdears.scale)}

\begin{fulllineitems}
\phantomsection\label{\detokenize{birdears:birdears.scale.ChromaticScale}}\pysiglinewithargsret{\sphinxbfcode{class }\sphinxcode{birdears.scale.}\sphinxbfcode{ChromaticScale}}{\emph{tonic}, \emph{octave=None}, \emph{n\_octaves=None}, \emph{descending=None}, \emph{dont\_repeat\_tonic=None}}{}
Bases: {\hyperref[\detokenize{index:birdears.scale.ScaleBase}]{\sphinxcrossref{\sphinxcode{birdears.scale.ScaleBase}}}}

Builds a musical chromatic scale.
\index{scale (birdears.scale.ChromaticScale attribute)}

\begin{fulllineitems}
\phantomsection\label{\detokenize{birdears:birdears.scale.ChromaticScale.scale}}\pysigline{\sphinxbfcode{scale}}
\sphinxstyleemphasis{array\_type} \textendash{} The array of notes representing the scale.

\end{fulllineitems}

\index{\_\_init\_\_() (birdears.scale.ChromaticScale method)}

\begin{fulllineitems}
\phantomsection\label{\detokenize{birdears:birdears.scale.ChromaticScale.__init__}}\pysiglinewithargsret{\sphinxbfcode{\_\_init\_\_}}{\emph{tonic}, \emph{octave=None}, \emph{n\_octaves=None}, \emph{descending=None}, \emph{dont\_repeat\_tonic=None}}{}
Returns a chromatic scale from tonic.
\begin{quote}\begin{description}
\item[{Parameters}] \leavevmode\begin{itemize}
\item {} 
\sphinxstyleliteralstrong{tonic} (\sphinxstyleliteralemphasis{str}) \textendash{} The note which the scale will be built upon.

\item {} 
\sphinxstyleliteralstrong{octave} (\sphinxstyleliteralemphasis{int}) \textendash{} The scientific octave the scale will be built upon.

\item {} 
\sphinxstyleliteralstrong{n\_octaves} (\sphinxstyleliteralemphasis{int}) \textendash{} The number of octaves the scale will contain.

\item {} 
\sphinxstyleliteralstrong{descending} (\sphinxstyleliteralemphasis{bool}) \textendash{} Whether the scale is descending.

\item {} 
\sphinxstyleliteralstrong{dont\_repeat\_tonic} (\sphinxstyleliteralemphasis{bool}) \textendash{} Whether to skip appending the last
note (octave) to the scale.

\end{itemize}

\end{description}\end{quote}

\end{fulllineitems}

\index{get\_triad() (birdears.scale.ChromaticScale method)}

\begin{fulllineitems}
\phantomsection\label{\detokenize{birdears:birdears.scale.ChromaticScale.get_triad}}\pysiglinewithargsret{\sphinxbfcode{get\_triad}}{\emph{mode}, \emph{index=0}, \emph{degree=None}}{}
Returns an array with notes from a scale’s triad.
\begin{quote}\begin{description}
\item[{Parameters}] \leavevmode\begin{itemize}
\item {} 
\sphinxstyleliteralstrong{mode} (\sphinxstyleliteralemphasis{str}) \textendash{} Mode of the scale (eg. ‘major’ or ‘minor’)

\item {} 
\sphinxstyleliteralstrong{index} (\sphinxstyleliteralemphasis{int}) \textendash{} Triad index (eg.: 0 for 1st degree triad.)

\item {} 
\sphinxstyleliteralstrong{degree} (\sphinxstyleliteralemphasis{int}) \textendash{} Degree of the scale. If provided, overrides the
\sphinxtitleref{index} argument. (eg.: \sphinxtitleref{1} for the 1st degree triad.)

\end{itemize}

\item[{Returns}] \leavevmode
A list with three pitches (str), one for each note of the triad.

\end{description}\end{quote}

\end{fulllineitems}


\end{fulllineitems}

\index{DiatonicScale (class in birdears.scale)}

\begin{fulllineitems}
\phantomsection\label{\detokenize{birdears:birdears.scale.DiatonicScale}}\pysiglinewithargsret{\sphinxbfcode{class }\sphinxcode{birdears.scale.}\sphinxbfcode{DiatonicScale}}{\emph{tonic}, \emph{mode=None}, \emph{octave=None}, \emph{n\_octaves=None}, \emph{descending=None}, \emph{dont\_repeat\_tonic=None}}{}
Bases: {\hyperref[\detokenize{index:birdears.scale.ScaleBase}]{\sphinxcrossref{\sphinxcode{birdears.scale.ScaleBase}}}}

Builds a musical diatonic scale.
\index{scale (birdears.scale.DiatonicScale attribute)}

\begin{fulllineitems}
\phantomsection\label{\detokenize{birdears:birdears.scale.DiatonicScale.scale}}\pysigline{\sphinxbfcode{scale}}
\sphinxstyleemphasis{array\_type} \textendash{} The array of notes representing the scale.

\end{fulllineitems}

\index{\_\_init\_\_() (birdears.scale.DiatonicScale method)}

\begin{fulllineitems}
\phantomsection\label{\detokenize{birdears:birdears.scale.DiatonicScale.__init__}}\pysiglinewithargsret{\sphinxbfcode{\_\_init\_\_}}{\emph{tonic}, \emph{mode=None}, \emph{octave=None}, \emph{n\_octaves=None}, \emph{descending=None}, \emph{dont\_repeat\_tonic=None}}{}
Returns a diatonic scale from tonic and mode.
\begin{quote}\begin{description}
\item[{Parameters}] \leavevmode\begin{itemize}
\item {} 
\sphinxstyleliteralstrong{tonic} (\sphinxstyleliteralemphasis{str}) \textendash{} The note which the scale will be built upon.

\item {} 
\sphinxstyleliteralstrong{mode} (\sphinxstyleliteralemphasis{str}) \textendash{} The mode the scale will be built upon.
(‘major’ or ‘minor’)

\item {} 
\sphinxstyleliteralstrong{octave} (\sphinxstyleliteralemphasis{int}) \textendash{} The scientific octave the scale will be built upon.

\item {} 
\sphinxstyleliteralstrong{n\_octaves} (\sphinxstyleliteralemphasis{int}) \textendash{} The number of octaves the scale will contain.

\item {} 
\sphinxstyleliteralstrong{descending} (\sphinxstyleliteralemphasis{bool}) \textendash{} Whether the scale is descending.

\item {} 
\sphinxstyleliteralstrong{dont\_repeat\_tonic} (\sphinxstyleliteralemphasis{bool}) \textendash{} Whether to skip appending the last
note (octave) to the scale.

\end{itemize}

\end{description}\end{quote}

\end{fulllineitems}

\index{get\_triad() (birdears.scale.DiatonicScale method)}

\begin{fulllineitems}
\phantomsection\label{\detokenize{birdears:birdears.scale.DiatonicScale.get_triad}}\pysiglinewithargsret{\sphinxbfcode{get\_triad}}{\emph{index=0}, \emph{degree=None}}{}
Returns an array with notes from a scale’s triad.
\begin{quote}\begin{description}
\item[{Parameters}] \leavevmode\begin{itemize}
\item {} 
\sphinxstyleliteralstrong{index} (\sphinxstyleliteralemphasis{int}) \textendash{} triad index (eg.: 0 for 1st degree triad.)

\item {} 
\sphinxstyleliteralstrong{degree} (\sphinxstyleliteralemphasis{int}) \textendash{} Degree of the scale. If provided, overrides the
\sphinxtitleref{index} argument. (eg.: \sphinxtitleref{1} for the 1st degree triad.)

\end{itemize}

\item[{Returns}] \leavevmode
An array with three pitches, one for each note of the triad.

\end{description}\end{quote}

\end{fulllineitems}


\end{fulllineitems}

\index{ScaleBase (class in birdears.scale)}

\begin{fulllineitems}
\phantomsection\label{\detokenize{birdears:birdears.scale.ScaleBase}}\pysigline{\sphinxbfcode{class }\sphinxcode{birdears.scale.}\sphinxbfcode{ScaleBase}}
Bases: \sphinxcode{object}

\end{fulllineitems}



\section{birdears.sequence module}
\label{\detokenize{birdears:birdears-sequence-module}}\label{\detokenize{birdears:module-birdears.sequence}}\index{birdears.sequence (module)}\index{Sequence (class in birdears.sequence)}

\begin{fulllineitems}
\phantomsection\label{\detokenize{birdears:birdears.sequence.Sequence}}\pysiglinewithargsret{\sphinxbfcode{class }\sphinxcode{birdears.sequence.}\sphinxbfcode{Sequence}}{\emph{elements={[}{]}}, \emph{duration=2}, \emph{delay=1.5}, \emph{pos\_delay=1}}{}
Bases: \sphinxcode{object}

Register a Sequence of notes and/or chords.
\index{elements (birdears.sequence.Sequence attribute)}

\begin{fulllineitems}
\phantomsection\label{\detokenize{birdears:birdears.sequence.Sequence.elements}}\pysigline{\sphinxbfcode{elements}}
\sphinxstyleemphasis{array\_type} \textendash{} List of notes (strings) ou chords (list of
strings) in this Sequence.

\end{fulllineitems}

\index{append() (birdears.sequence.Sequence method)}

\begin{fulllineitems}
\phantomsection\label{\detokenize{birdears:birdears.sequence.Sequence.append}}\pysiglinewithargsret{\sphinxbfcode{append}}{\emph{elements}}{}
Appends \sphinxtitleref{elements} to Sequence.elements
\begin{quote}\begin{description}
\item[{Parameters}] \leavevmode
\sphinxstyleliteralstrong{elements} (\sphinxstyleliteralemphasis{array\_type}) \textendash{} Elements to be appended to the class.

\end{description}\end{quote}

\end{fulllineitems}

\index{async\_play() (birdears.sequence.Sequence method)}

\begin{fulllineitems}
\phantomsection\label{\detokenize{birdears:birdears.sequence.Sequence.async_play}}\pysiglinewithargsret{\sphinxbfcode{async\_play}}{\emph{callback}, \emph{end\_callback}}{}
Plays the Sequence elements of notes and/or chords and wait for
\sphinxtitleref{Sequence.pos\_delay} seconds.

\end{fulllineitems}

\index{extend() (birdears.sequence.Sequence method)}

\begin{fulllineitems}
\phantomsection\label{\detokenize{birdears:birdears.sequence.Sequence.extend}}\pysiglinewithargsret{\sphinxbfcode{extend}}{\emph{elements}}{}
Extends Sequence.elements with \sphinxtitleref{elements}.
\begin{quote}\begin{description}
\item[{Parameters}] \leavevmode
\sphinxstyleliteralstrong{elements} (\sphinxstyleliteralemphasis{array\_type}) \textendash{} elements extend the class with.

\end{description}\end{quote}

\end{fulllineitems}

\index{make\_chord\_progression() (birdears.sequence.Sequence method)}

\begin{fulllineitems}
\phantomsection\label{\detokenize{birdears:birdears.sequence.Sequence.make_chord_progression}}\pysiglinewithargsret{\sphinxbfcode{make\_chord\_progression}}{\emph{tonic}, \emph{mode}, \emph{degrees}}{}
Appends triad chord(s) to the Sequence.
\begin{quote}\begin{description}
\item[{Parameters}] \leavevmode\begin{itemize}
\item {} 
\sphinxstyleliteralstrong{tonic} (\sphinxstyleliteralemphasis{str}) \textendash{} Tonic note of the scale.

\item {} 
\sphinxstyleliteralstrong{mode} (\sphinxstyleliteralemphasis{str}) \textendash{} Mode of the scale from which build the triads upon.

\item {} 
\sphinxstyleliteralstrong{degrees} (\sphinxstyleliteralemphasis{array\_type}) \textendash{} List with integers represending the degrees
of each triad.

\end{itemize}

\end{description}\end{quote}

\end{fulllineitems}

\index{play() (birdears.sequence.Sequence method)}

\begin{fulllineitems}
\phantomsection\label{\detokenize{birdears:birdears.sequence.Sequence.play}}\pysiglinewithargsret{\sphinxbfcode{play}}{\emph{callback=None}, \emph{end\_callback=None}}{}
\end{fulllineitems}

\index{play\_element() (birdears.sequence.Sequence method)}

\begin{fulllineitems}
\phantomsection\label{\detokenize{birdears:birdears.sequence.Sequence.play_element}}\pysiglinewithargsret{\sphinxbfcode{play\_element}}{\emph{index}}{}
Plays element \sphinxtitleref{sequence.elements{[}index{]}.}

\end{fulllineitems}


\end{fulllineitems}



\chapter{Support}
\label{\detokenize{index:support}}
If you need help you can get in touch via IRC or file an issue on any matter regarding birdears at Github.


\begin{savenotes}\sphinxattablestart
\centering
\begin{tabulary}{\linewidth}[t]{|T|T|}
\hline
\sphinxstylethead{\sphinxstyletheadfamily 
Media
\unskip}\relax &\sphinxstylethead{\sphinxstyletheadfamily 
Channel
\unskip}\relax \\
\hline
IRC
&
\sphinxhref{https://webchat.freenode.net/?randomnick=1\&channels=\%23birdears\&uio=MTY9dHJ1ZSYxMT0yNDY57}{\#birdears} at irc.freenode.org/6697 -ssl
\\
\hline
GitHub
&
\sphinxurl{https://github.com/iacchus/birdears}
\\
\hline
GH issues
&
\sphinxurl{https://github.com/iacchus/birdears/issues}
\\
\hline
ReadTheDocs
&
\sphinxurl{https://birdears.readthedocs.io}
\\
\hline
PyPI
&
\sphinxurl{https://pypi.python.org/pypi/birdears}
\\
\hline
TravisCI
&
\sphinxurl{https://travis-ci.org/iacchus/birdears}
\\
\hline
Coveralls
&
\sphinxurl{https://coveralls.io/github/iacchus/birdears}
\\
\hline
\end{tabulary}
\par
\sphinxattableend\end{savenotes}


\chapter{Features}
\label{\detokenize{index:features}}\begin{itemize}
\item {} 
questions

\item {} 
pretty much configurable

\item {} 
load from config file

\item {} 
you can make your own presets

\item {} 
can be used interactively \sphinxstyleemphasis{(docs needed)}

\item {} 
can be used as a library \sphinxstyleemphasis{(docs needed)}

\end{itemize}


\chapter{Installing birdears}
\label{\detokenize{index:installing-birdears}}

\section{Installing the dependencies}
\label{\detokenize{index:installing-the-dependencies}}

\subsection{Arch Linux}
\label{\detokenize{index:arch-linux}}
\begin{sphinxVerbatim}[commandchars=\\\{\}]
sudo pacman \PYGZhy{}Syu sox python python\PYGZhy{}pip
\end{sphinxVerbatim}


\section{Installing birdears}
\label{\detokenize{index:id1}}
To install,simple do this command with pip3

\begin{sphinxVerbatim}[commandchars=\\\{\}]
pip3 install \PYGZhy{}\PYGZhy{}user \PYGZhy{}\PYGZhy{}upgrade \PYGZhy{}\PYGZhy{}no\PYGZhy{}cache\PYGZhy{}dir birdears
\end{sphinxVerbatim}


\subsection{In-depth installation}
\label{\detokenize{index:in-depth-installation}}
You can choose to use a virtualenv to use birdears; this should give you
an idea on how to setup one virtualenv.

You should first install virtualenv (for python3) using your
distribution’s package (supposing you’re on linux), then issue on terminal:

\begin{sphinxVerbatim}[commandchars=\\\{\}]
virtualenv \PYGZhy{}p python3 \PYGZti{}/.venv \PYGZsh{} use the directory \PYGZti{}/.venv/ for the virtualenv

source \PYGZti{}/.venv/bin/activate   \PYGZsh{} activate the virtualenv; this should be done
                              \PYGZsh{} every time you may want to run the software
                              \PYGZsh{} installed here.

pip3 install birdears         \PYGZsh{} this will install the software

birdears \PYGZhy{}\PYGZhy{}help               \PYGZsh{} and this will run it
\end{sphinxVerbatim}


\chapter{Using birdears}
\label{\detokenize{index:using-birdears}}

\section{What is Functional Ear Training}
\label{\detokenize{index:what-is-functional-ear-training}}
\sphinxstyleemphasis{write me!}


\section{The method}
\label{\detokenize{index:the-method}}
We can use abc language to notate music withing the documentation, ok

\begin{sphinxVerbatim}[commandchars=\\\{\}]
X: 1
T: Banish Misfortune
R: jig
M: 6/8
L: 1/8
K: Dmix
fed cAG\textbar{} A2d cAG\textbar{} F2D DED\textbar{} FEF GFG\textbar{}
AGA cAG\textbar{} AGA cde\textbar{}fed cAG\textbar{} Ad\PYGZca{}c d3:\textbar{}
f2d d\PYGZca{}cd\textbar{} f2g agf\textbar{} e2c cBc\textbar{}e2f gfe\textbar{}
f2g agf\textbar{} e2f gfe\textbar{}fed cAG\textbar{}Ad\PYGZca{}c d3:\textbar{}
f2g e2f\textbar{} d2e c2d\textbar{}ABA GAG\textbar{} F2F GED\textbar{}
c3 cAG\textbar{} AGA cde\textbar{} fed cAG\textbar{} Ad\PYGZca{}c d3:\textbar{}
\end{sphinxVerbatim}


\section{birdears modes and basic usage}
\label{\detokenize{index:birdears-modes-and-basic-usage}}
birdears actually has four modes:
\begin{itemize}
\item {} 
melodic interval question

\item {} 
harmonic interval question

\item {} 
melodic dictation question

\item {} 
instrumental dictation question

\end{itemize}

To see the commands avaliable just invoke the command without any arguments:

\begin{sphinxVerbatim}[commandchars=\\\{\}]
birdears
\end{sphinxVerbatim}

\begin{sphinxVerbatim}[commandchars=\\\{\}]
Usage: birdears  \PYGZlt{}command\PYGZgt{} [options]

  birdears ─ Functional Ear Training for Musicians!

Options:
  \PYGZhy{}\PYGZhy{}debug / \PYGZhy{}\PYGZhy{}no\PYGZhy{}debug  Turns on debugging; instead you can set DEBUG=1.
  \PYGZhy{}h, \PYGZhy{}\PYGZhy{}help            Show this message and exit.

Commands:
  dictation     Melodic dictation
  harmonic      Harmonic interval recognition
  instrumental  Instrumental melodic time\PYGZhy{}based dictation
  load          Loads exercise from .toml config file...
  melodic       Melodic interval recognition

  You can use \PYGZsq{}birdears \PYGZlt{}command\PYGZgt{} \PYGZhy{}\PYGZhy{}help\PYGZsq{} to show options for a specific
  command.

  More info at https://github.com/iacchus/birdears
\end{sphinxVerbatim}

\begin{sphinxVerbatim}[commandchars=\\\{\}]
birdears \PYGZlt{}command\PYGZgt{} \PYGZhy{}\PYGZhy{}help
\end{sphinxVerbatim}


\subsection{melodic}
\label{\detokenize{index:melodic}}
In this exercise birdears will play two notes, the tonic and the interval
melodically, ie., one after the other and you should reply which is the
correct distance between the two.

\begin{sphinxVerbatim}[commandchars=\\\{\}]
birdears melodic \PYGZhy{}\PYGZhy{}help
\end{sphinxVerbatim}

\begin{sphinxVerbatim}[commandchars=\\\{\}]
Usage: birdears melodic [options]

  Melodic interval recognition

Options:
  \PYGZhy{}m, \PYGZhy{}\PYGZhy{}mode \PYGZlt{}mode\PYGZgt{}               Mode of the question.
  \PYGZhy{}t, \PYGZhy{}\PYGZhy{}tonic \PYGZlt{}tonic\PYGZgt{}             Tonic of the question.
  \PYGZhy{}o, \PYGZhy{}\PYGZhy{}octave \PYGZlt{}octave\PYGZgt{}           Octave of the question.
  \PYGZhy{}d, \PYGZhy{}\PYGZhy{}descending                Whether the question interval is descending.
  \PYGZhy{}c, \PYGZhy{}\PYGZhy{}chromatic                 If chosen, question has chromatic notes.
  \PYGZhy{}n, \PYGZhy{}\PYGZhy{}n\PYGZus{}octaves \PYGZlt{}n max\PYGZgt{}         Maximum number of octaves.
  \PYGZhy{}v, \PYGZhy{}\PYGZhy{}valid\PYGZus{}intervals \PYGZlt{}1,2,..\PYGZgt{}  A comma\PYGZhy{}separated list without spaces
                                  of valid scale degrees to be chosen for the
                                  question.
  \PYGZhy{}q, \PYGZhy{}\PYGZhy{}user\PYGZus{}durations \PYGZlt{}1,0.5,n..\PYGZgt{}
                                  A comma\PYGZhy{}separated list without
                                  spaces with PRECISLY 9 floating values. Or
                                  \PYGZsq{}n\PYGZsq{} for default              duration.
  \PYGZhy{}p, \PYGZhy{}\PYGZhy{}prequestion\PYGZus{}method \PYGZlt{}prequestion\PYGZus{}method\PYGZgt{}
                                  The name of a pre\PYGZhy{}question method.
  \PYGZhy{}r, \PYGZhy{}\PYGZhy{}resolution\PYGZus{}method \PYGZlt{}resolution\PYGZus{}method\PYGZgt{}
                                  The name of a resolution method.
  \PYGZhy{}h, \PYGZhy{}\PYGZhy{}help                      Show this message and exit.

  In this exercise birdears will play two notes, the tonic and the interval
  melodically, ie., one after the other and you should reply which is the
  correct distance between the two.

  Valid values are as follows:

  \PYGZhy{}m \PYGZlt{}mode\PYGZgt{} is one of: major, dorian, phrygian, lydian, mixolydian, minor,
  locrian

  \PYGZhy{}t \PYGZlt{}tonic\PYGZgt{} is one of: A, A\PYGZsh{}, Ab, B, Bb, C, C\PYGZsh{}, D, D\PYGZsh{}, Db, E, Eb, F, F\PYGZsh{}, G,
  G\PYGZsh{}, Gb

  \PYGZhy{}p \PYGZlt{}prequestion\PYGZus{}method\PYGZgt{} is one of: none, tonic\PYGZus{}only, progression\PYGZus{}i\PYGZus{}iv\PYGZus{}v\PYGZus{}i

  \PYGZhy{}r \PYGZlt{}resolution\PYGZus{}method\PYGZgt{} is one of: nearest\PYGZus{}tonic, repeat\PYGZus{}only
\end{sphinxVerbatim}


\subsection{harmonic}
\label{\detokenize{index:harmonic}}
In this exercise birdears will play two notes, the tonic and the interval
harmonically, ie., both on the same time and you should reply which is the
correct distance between the two.

\begin{sphinxVerbatim}[commandchars=\\\{\}]
birdears harmonic \PYGZhy{}\PYGZhy{}help
\end{sphinxVerbatim}

\begin{sphinxVerbatim}[commandchars=\\\{\}]
Usage: birdears harmonic [options]

  Harmonic interval recognition

Options:
  \PYGZhy{}m, \PYGZhy{}\PYGZhy{}mode \PYGZlt{}mode\PYGZgt{}               Mode of the question.
  \PYGZhy{}t, \PYGZhy{}\PYGZhy{}tonic \PYGZlt{}note\PYGZgt{}              Tonic of the question.
  \PYGZhy{}o, \PYGZhy{}\PYGZhy{}octave \PYGZlt{}octave\PYGZgt{}           Octave of the question.
  \PYGZhy{}d, \PYGZhy{}\PYGZhy{}descending                Whether the question interval is descending.
  \PYGZhy{}c, \PYGZhy{}\PYGZhy{}chromatic                 If chosen, question has chromatic notes.
  \PYGZhy{}n, \PYGZhy{}\PYGZhy{}n\PYGZus{}octaves \PYGZlt{}n max\PYGZgt{}         Maximum number of octaves.
  \PYGZhy{}v, \PYGZhy{}\PYGZhy{}valid\PYGZus{}intervals \PYGZlt{}1,2,..\PYGZgt{}  A comma\PYGZhy{}separated list without spaces
                                  of valid scale degrees to be chosen for the
                                  question.
  \PYGZhy{}q, \PYGZhy{}\PYGZhy{}user\PYGZus{}durations \PYGZlt{}1,0.5,n..\PYGZgt{}
                                  A comma\PYGZhy{}separated list without
                                  spaces with PRECISLY 9 floating values. Or
                                  \PYGZsq{}n\PYGZsq{} for default              duration.
  \PYGZhy{}p, \PYGZhy{}\PYGZhy{}prequestion\PYGZus{}method \PYGZlt{}prequestion\PYGZus{}method\PYGZgt{}
                                  The name of a pre\PYGZhy{}question method.
  \PYGZhy{}r, \PYGZhy{}\PYGZhy{}resolution\PYGZus{}method \PYGZlt{}resolution\PYGZus{}method\PYGZgt{}
                                  The name of a resolution method.
  \PYGZhy{}h, \PYGZhy{}\PYGZhy{}help                      Show this message and exit.

  In this exercise birdears will play two notes, the tonic and the interval
  harmonically, ie., both on the same time and you should reply which is the
  correct distance between the two.

  Valid values are as follows:

  \PYGZhy{}m \PYGZlt{}mode\PYGZgt{} is one of: major, dorian, phrygian, lydian, mixolydian, minor,
  locrian

  \PYGZhy{}t \PYGZlt{}tonic\PYGZgt{} is one of: A, A\PYGZsh{}, Ab, B, Bb, C, C\PYGZsh{}, D, D\PYGZsh{}, Db, E, Eb, F, F\PYGZsh{}, G,
  G\PYGZsh{}, Gb

  \PYGZhy{}p \PYGZlt{}prequestion\PYGZus{}method\PYGZgt{} is one of: none, tonic\PYGZus{}only, progression\PYGZus{}i\PYGZus{}iv\PYGZus{}v\PYGZus{}i

  \PYGZhy{}r \PYGZlt{}resolution\PYGZus{}method\PYGZgt{} is one of: nearest\PYGZus{}tonic, repeat\PYGZus{}only
\end{sphinxVerbatim}


\subsection{dictation}
\label{\detokenize{index:dictation}}
In this exercise birdears will choose some random intervals and create a
melodic dictation with them. You should reply the correct intervals of the
melodic dictation.

\begin{sphinxVerbatim}[commandchars=\\\{\}]
birdears dictation \PYGZhy{}\PYGZhy{}help
\end{sphinxVerbatim}

\begin{sphinxVerbatim}[commandchars=\\\{\}]
Usage: birdears dictation [options]

  Melodic dictation

Options:
  \PYGZhy{}m, \PYGZhy{}\PYGZhy{}mode \PYGZlt{}mode\PYGZgt{}               Mode of the question.
  \PYGZhy{}i, \PYGZhy{}\PYGZhy{}max\PYGZus{}intervals \PYGZlt{}n max\PYGZgt{}     Max random intervals for the dictation.
  \PYGZhy{}x, \PYGZhy{}\PYGZhy{}n\PYGZus{}notes \PYGZlt{}n notes\PYGZgt{}         Number of notes for the dictation.
  \PYGZhy{}t, \PYGZhy{}\PYGZhy{}tonic \PYGZlt{}note\PYGZgt{}              Tonic of the question.
  \PYGZhy{}o, \PYGZhy{}\PYGZhy{}octave \PYGZlt{}octave\PYGZgt{}           Octave of the question.
  \PYGZhy{}d, \PYGZhy{}\PYGZhy{}descending                Wether the question interval is descending.
  \PYGZhy{}c, \PYGZhy{}\PYGZhy{}chromatic                 If chosen, question has chromatic notes.
  \PYGZhy{}n, \PYGZhy{}\PYGZhy{}n\PYGZus{}octaves \PYGZlt{}n max\PYGZgt{}         Maximum number of octaves.
  \PYGZhy{}v, \PYGZhy{}\PYGZhy{}valid\PYGZus{}intervals \PYGZlt{}1,2,..\PYGZgt{}  A comma\PYGZhy{}separated list without spaces
                                  of valid scale degrees to be chosen for the
                                  question.
  \PYGZhy{}q, \PYGZhy{}\PYGZhy{}user\PYGZus{}durations \PYGZlt{}1,0.5,n..\PYGZgt{}
                                  A comma\PYGZhy{}separated list without
                                  spaces with PRECISLY 9 floating values. Or
                                  \PYGZsq{}n\PYGZsq{} for default              duration.
  \PYGZhy{}p, \PYGZhy{}\PYGZhy{}prequestion\PYGZus{}method \PYGZlt{}prequestion\PYGZus{}method\PYGZgt{}
                                  The name of a pre\PYGZhy{}question method.
  \PYGZhy{}r, \PYGZhy{}\PYGZhy{}resolution\PYGZus{}method \PYGZlt{}resolution\PYGZus{}method\PYGZgt{}
                                  The name of a resolution method.
  \PYGZhy{}h, \PYGZhy{}\PYGZhy{}help                      Show this message and exit.

  In this exercise birdears will choose some random intervals and create a
  melodic dictation with them. You should reply the correct intervals of the
  melodic dictation.

  Valid values are as follows:

  \PYGZhy{}m \PYGZlt{}mode\PYGZgt{} is one of: major, dorian, phrygian, lydian, mixolydian, minor,
  locrian

  \PYGZhy{}t \PYGZlt{}tonic\PYGZgt{} is one of: A, A\PYGZsh{}, Ab, B, Bb, C, C\PYGZsh{}, D, D\PYGZsh{}, Db, E, Eb, F, F\PYGZsh{}, G,
  G\PYGZsh{}, Gb

  \PYGZhy{}p \PYGZlt{}prequestion\PYGZus{}method\PYGZgt{} is one of: none, tonic\PYGZus{}only, progression\PYGZus{}i\PYGZus{}iv\PYGZus{}v\PYGZus{}i

  \PYGZhy{}r \PYGZlt{}resolution\PYGZus{}method\PYGZgt{} is one of: nearest\PYGZus{}tonic, repeat\PYGZus{}only
\end{sphinxVerbatim}


\subsection{instrumental}
\label{\detokenize{index:instrumental}}
In this exercise birdears will choose some random intervals and create a
melodic dictation with them. You should play the correct melody in you
musical instrument.

\begin{sphinxVerbatim}[commandchars=\\\{\}]
birdears instrumental \PYGZhy{}\PYGZhy{}help
\end{sphinxVerbatim}

\begin{sphinxVerbatim}[commandchars=\\\{\}]
Usage: birdears instrumental [options]

  Instrumental melodic time\PYGZhy{}based dictation

Options:
  \PYGZhy{}m, \PYGZhy{}\PYGZhy{}mode \PYGZlt{}mode\PYGZgt{}               Mode of the question.
  \PYGZhy{}w, \PYGZhy{}\PYGZhy{}wait\PYGZus{}time \PYGZlt{}seconds\PYGZgt{}       Time in seconds for next question/repeat.
  \PYGZhy{}u, \PYGZhy{}\PYGZhy{}n\PYGZus{}repeats \PYGZlt{}times\PYGZgt{}         Times to repeat question.
  \PYGZhy{}i, \PYGZhy{}\PYGZhy{}max\PYGZus{}intervals \PYGZlt{}n max\PYGZgt{}     Max random intervals for the dictation.
  \PYGZhy{}x, \PYGZhy{}\PYGZhy{}n\PYGZus{}notes \PYGZlt{}n notes\PYGZgt{}         Number of notes for the dictation.
  \PYGZhy{}t, \PYGZhy{}\PYGZhy{}tonic \PYGZlt{}note\PYGZgt{}              Tonic of the question.
  \PYGZhy{}o, \PYGZhy{}\PYGZhy{}octave \PYGZlt{}octave\PYGZgt{}           Octave of the question.
  \PYGZhy{}d, \PYGZhy{}\PYGZhy{}descending                Wether the question interval is descending.
  \PYGZhy{}c, \PYGZhy{}\PYGZhy{}chromatic                 If chosen, question has chromatic notes.
  \PYGZhy{}n, \PYGZhy{}\PYGZhy{}n\PYGZus{}octaves \PYGZlt{}n max\PYGZgt{}         Maximum number of octaves.
  \PYGZhy{}v, \PYGZhy{}\PYGZhy{}valid\PYGZus{}intervals \PYGZlt{}1,2,..\PYGZgt{}  A comma\PYGZhy{}separated list without spaces
                                  of valid scale degrees to be chosen for the
                                  question.
  \PYGZhy{}q, \PYGZhy{}\PYGZhy{}user\PYGZus{}durations \PYGZlt{}1,0.5,n..\PYGZgt{}
                                  A comma\PYGZhy{}separated list without
                                  spaces with PRECISLY 9 floating values. Or
                                  \PYGZsq{}n\PYGZsq{} for default              duration.
  \PYGZhy{}p, \PYGZhy{}\PYGZhy{}prequestion\PYGZus{}method \PYGZlt{}prequestion\PYGZus{}method\PYGZgt{}
                                  The name of a pre\PYGZhy{}question method.
  \PYGZhy{}r, \PYGZhy{}\PYGZhy{}resolution\PYGZus{}method \PYGZlt{}resolution\PYGZus{}method\PYGZgt{}
                                  The name of a resolution method.
  \PYGZhy{}h, \PYGZhy{}\PYGZhy{}help                      Show this message and exit.

  In this exercise birdears will choose some random intervals and create a
  melodic dictation with them. You should play the correct melody in you
  musical instrument.

  Valid values are as follows:

  \PYGZhy{}m \PYGZlt{}mode\PYGZgt{} is one of: major, dorian, phrygian, lydian, mixolydian, minor,
  locrian

  \PYGZhy{}t \PYGZlt{}tonic\PYGZgt{} is one of: A, A\PYGZsh{}, Ab, B, Bb, C, C\PYGZsh{}, D, D\PYGZsh{}, Db, E, Eb, F, F\PYGZsh{}, G,
  G\PYGZsh{}, Gb

  \PYGZhy{}p \PYGZlt{}prequestion\PYGZus{}method\PYGZgt{} is one of: none, tonic\PYGZus{}only, progression\PYGZus{}i\PYGZus{}iv\PYGZus{}v\PYGZus{}i

  \PYGZhy{}r \PYGZlt{}resolution\PYGZus{}method\PYGZgt{} is one of: nearest\PYGZus{}tonic, repeat\PYGZus{}only
\end{sphinxVerbatim}


\section{Loading from config/preset files}
\label{\detokenize{index:loading-from-config-preset-files}}

\subsection{Pre-made presets}
\label{\detokenize{index:pre-made-presets}}
\sphinxcode{birdears} cointains some pre-made presets in it’s \sphinxcode{presets/}
subdirectory.

The study for beginners is recommended by following the numeric order of
those files (000, 001, then 002 etc.)


\subsubsection{Pre-made presets description}
\label{\detokenize{index:pre-made-presets-description}}
\sphinxstyleemphasis{write me}


\subsection{Creating new preset files}
\label{\detokenize{index:creating-new-preset-files}}
You can open the files cointained in birdears premade \sphinxcode{presets/}
folder to have an ideia on how config files are made; it is simply the
command line options written in a form \sphinxcode{toml} standard.


\section{Keybindings}
\label{\detokenize{index:keybindings}}

\subsection{On the keybindings}
\label{\detokenize{index:on-the-keybindings}}
The following keyboard diagrams should give you an idea on how the
keybindings work. Please note how the keys on the line from \sphinxcode{z}
(\sphinxstyleemphasis{unison}) to \sphinxcode{,} (comma, \sphinxstyleemphasis{octave}) represent the notes that are
\sphinxstyleemphasis{natural} to the mode, and the line above represent the chromatics.

Also, for exercises with two octaves, the \sphinxstylestrong{uppercased keys represent
the second octave}. For example, \sphinxcode{z} is \sphinxstyleemphasis{unison}, \sphinxcode{,} is the
\sphinxstyleemphasis{octave}, \sphinxcode{Z} (uppercased) is the \sphinxstyleemphasis{double octave}. The same for all the other
intervals.


\subsection{Major (Ionian)}
\label{\detokenize{index:major-ionian}}
\begin{figure}[htbp]
\centering
\capstart

\noindent\sphinxincludegraphics[scale=1.0]{{ionian}.png}
\caption{Keyboard diagram for the \sphinxcode{-{-}mode major} (default).}\label{\detokenize{index:id14}}\end{figure}


\subsection{Dorian}
\label{\detokenize{index:dorian}}
\begin{figure}[htbp]
\centering
\capstart

\noindent\sphinxincludegraphics[scale=1.0]{{dorian}.png}
\caption{Keyboard diagram for the \sphinxcode{-{-}mode dorian}.}\label{\detokenize{index:id15}}\end{figure}


\subsection{Phrygian}
\label{\detokenize{index:phrygian}}
\begin{figure}[htbp]
\centering
\capstart

\noindent\sphinxincludegraphics[scale=1.0]{{phrygian}.png}
\caption{Keyboard diagram for the \sphinxcode{-{-}mode phrygian}.}\label{\detokenize{index:id16}}\end{figure}


\subsection{Lydian}
\label{\detokenize{index:lydian}}
\begin{figure}[htbp]
\centering
\capstart

\noindent\sphinxincludegraphics[scale=1.0]{{lydian}.png}
\caption{Keyboard diagram for the \sphinxcode{-{-}mode lydian}.}\label{\detokenize{index:id17}}\end{figure}


\subsection{Mixolydian}
\label{\detokenize{index:mixolydian}}
\begin{figure}[htbp]
\centering
\capstart

\noindent\sphinxincludegraphics[scale=1.0]{{mixolydian}.png}
\caption{Keyboard diagram for the \sphinxcode{-{-}mode mixolydian}.}\label{\detokenize{index:id18}}\end{figure}


\subsection{Minor (Aeolian)}
\label{\detokenize{index:minor-aeolian}}
\begin{figure}[htbp]
\centering
\capstart

\noindent\sphinxincludegraphics[scale=1.0]{{minor}.png}
\caption{Keyboard diagram for the \sphinxcode{-{-}mode minor}.}\label{\detokenize{index:id19}}\end{figure}


\subsection{Locrian}
\label{\detokenize{index:locrian}}
\begin{figure}[htbp]
\centering
\capstart

\noindent\sphinxincludegraphics[scale=1.0]{{locrian}.png}
\caption{Keyboard diagram for the \sphinxcode{-{-}mode locrian}.}\label{\detokenize{index:id20}}\end{figure}


\chapter{API}
\label{\detokenize{index:api}}

\chapter{birdears package}
\label{\detokenize{index:birdears-package}}\label{\detokenize{index:module-birdears}}\index{birdears (module)}
birdears provides facilities to building musical ear training exercises.
\index{CHROMATIC\_FLAT (in module birdears)}

\begin{fulllineitems}
\phantomsection\label{\detokenize{index:birdears.CHROMATIC_FLAT}}\pysigline{\sphinxcode{birdears.}\sphinxbfcode{CHROMATIC\_FLAT}\sphinxbfcode{ = ('C', 'Db', 'D', 'Eb', 'E', 'F', 'Gb', 'G', 'Ab', 'A', 'Bb', 'B')}}
\sphinxstyleemphasis{tuple} \textendash{} Chromatic notes names using flats.

A mapping of the chromatic note names using flats.

\end{fulllineitems}

\index{CHROMATIC\_SHARP (in module birdears)}

\begin{fulllineitems}
\phantomsection\label{\detokenize{index:birdears.CHROMATIC_SHARP}}\pysigline{\sphinxcode{birdears.}\sphinxbfcode{CHROMATIC\_SHARP}\sphinxbfcode{ = ('C', 'C\#', 'D', 'D\#', 'E', 'F', 'F\#', 'G', 'G\#', 'A', 'A\#', 'B')}}
\sphinxstyleemphasis{tuple} \textendash{} Chromatic notes names using sharps.

A mapping of the chromatic note namesu sing sharps

\end{fulllineitems}

\index{CHROMATIC\_TYPE (in module birdears)}

\begin{fulllineitems}
\phantomsection\label{\detokenize{index:birdears.CHROMATIC_TYPE}}\pysigline{\sphinxcode{birdears.}\sphinxbfcode{CHROMATIC\_TYPE}\sphinxbfcode{ = (0, 1, 2, 3, 4, 5, 6, 7, 8, 9, 10, 11, 12)}}
\sphinxstyleemphasis{tuple} \textendash{} A map of the chromatic chromatic scale.

A map of the the semitones which compound the chromatic scale.

\end{fulllineitems}

\index{CIRCLE\_OF\_FIFTHS (in module birdears)}

\begin{fulllineitems}
\phantomsection\label{\detokenize{index:birdears.CIRCLE_OF_FIFTHS}}\pysigline{\sphinxcode{birdears.}\sphinxbfcode{CIRCLE\_OF\_FIFTHS}\sphinxbfcode{ = {[}('C', 'G', 'D', 'A', 'E', 'B', 'Gb', 'Db', 'Ab', 'Eb', 'Bb', 'F'), ('C', 'F', 'Bb', 'Eb', 'Ab', 'C\#', 'F\#', 'B', 'E', 'A', 'D', 'G'){]}}}
\sphinxstyleemphasis{list of tuples} \textendash{} Circle of fifths.

These are the circle of fifth in both directions.

\end{fulllineitems}

\index{DIATONIC\_MODES (in module birdears)}

\begin{fulllineitems}
\phantomsection\label{\detokenize{index:birdears.DIATONIC_MODES}}\pysigline{\sphinxcode{birdears.}\sphinxbfcode{DIATONIC\_MODES}\sphinxbfcode{ = \{'major': (0, 2, 4, 5, 7, 9, 11, 12), 'dorian': (0, 2, 3, 5, 7, 9, 10, 12), 'phrygian': (0, 1, 3, 5, 7, 8, 10, 12), 'lydian': (0, 2, 4, 6, 7, 9, 11, 12), 'mixolydian': (0, 2, 4, 5, 7, 9, 10, 12), 'minor': (0, 2, 3, 5, 7, 8, 10, 12), 'locrian': (0, 1, 3, 5, 6, 8, 10, 12)\}}}
\sphinxstyleemphasis{dict of tuples} \textendash{} A map of the diatonic scale.

A mapping of the semitones which compound each of the greek modes.

\end{fulllineitems}

\index{INTERVALS (in module birdears)}

\begin{fulllineitems}
\phantomsection\label{\detokenize{index:birdears.INTERVALS}}\pysigline{\sphinxcode{birdears.}\sphinxbfcode{INTERVALS}\sphinxbfcode{ = ((0, 'P1', 'Perfect Unison'), (1, 'm2', 'Minor Second'), (2, 'M2', 'Major Second'), (3, 'm3', 'Minor Third'), (4, 'M3', 'Major Third'), (5, 'P4', 'Perfect Fourth'), (6, 'A4', 'Augmented Fourth'), (7, 'P5', 'Perfect Fifth'), (8, 'm6', 'Minor Sixth'), (9, 'M6', 'Major Sixth'), (10, 'm7', 'Minor Seventh'), (11, 'M7', 'Major Seventh'), (12, 'P8', 'Perfect Octave'), (13, 'A8', 'Minor Ninth'), (14, 'M9', 'Major Ninth'), (15, 'm10', 'Minor Tenth'), (16, 'M10', 'Major Tenth'), (17, 'P11', 'Perfect Eleventh'), (18, 'A11', 'Augmented Eleventh'), (19, 'P12', 'Perfect Twelfth'), (20, 'm13', 'Minor Thirteenth'), (21, 'M13', 'Major Thirteenth'), (22, 'm14', 'Minor Fourteenth'), (23, 'M14', 'Major Fourteenth'), (24, 'P15', 'Perfect Double-octave'), (25, 'A15', 'Minor Sixteenth'), (26, 'M16', 'Major Sixteenth'), (27, 'm17', 'Minor Seventeenth'), (28, 'M17', 'Major Seventeenth'), (29, 'P18', 'Perfect Eighteenth'), (30, 'A18', 'Augmented Eighteenth'), (31, 'P19', 'Perfect Nineteenth'), (32, 'm20', 'Minor Twentieth'), (33, 'M20', 'Major Twentieth'), (34, 'm21', 'Minor Twenty-first'), (35, 'M21', 'Major Twenty-first'), (36, 'P22', 'Perfect Triple-octave'))}}
\sphinxstyleemphasis{tuple of tuples} \textendash{} Data representing intervals.

A tuple of tuples representing data for the intervals with format
(semitones, short name, full name).

\end{fulllineitems}

\index{INTERVAL\_INDEX (in module birdears)}

\begin{fulllineitems}
\phantomsection\label{\detokenize{index:birdears.INTERVAL_INDEX}}\pysigline{\sphinxcode{birdears.}\sphinxbfcode{INTERVAL\_INDEX}\sphinxbfcode{ = \{1: {[}0{]}, 2: {[}1, 2{]}, 3: {[}3, 4{]}, 4: {[}5, 6{]}, 5: {[}6, 7{]}, 6: {[}8, 9{]}, 7: {[}10, 11{]}, 8: {[}12{]}\}}}
\sphinxstyleemphasis{dict of lists} \textendash{} A mapping of semitones of each interval.

A mapping of semitones which index to each interval name, major/minor,
perfect, augmented/diminished

\end{fulllineitems}

\index{KEYS (in module birdears)}

\begin{fulllineitems}
\phantomsection\label{\detokenize{index:birdears.KEYS}}\pysigline{\sphinxcode{birdears.}\sphinxbfcode{KEYS}\sphinxbfcode{ = ('C', 'C\#', 'Db', 'D', 'D\#', 'Eb', 'E', 'F', 'F\#', 'Gb', 'G', 'G\#', 'Ab', 'A', 'A\#', 'Bb', 'B')}}
\sphinxstyleemphasis{tuple} \textendash{} Allowed keys

These are the allowed keys for exercise as comprehended by birdears.

\end{fulllineitems}



\section{Subpackages}
\label{\detokenize{index:subpackages}}

\section{Submodules}
\label{\detokenize{index:submodules}}

\section{birdears.interval module}
\label{\detokenize{index:module-birdears.interval}}\label{\detokenize{index:birdears-interval-module}}\index{birdears.interval (module)}\index{ChromaticInterval (class in birdears.interval)}

\begin{fulllineitems}
\phantomsection\label{\detokenize{index:birdears.interval.ChromaticInterval}}\pysiglinewithargsret{\sphinxbfcode{class }\sphinxcode{birdears.interval.}\sphinxbfcode{ChromaticInterval}}{\emph{mode}, \emph{tonic}, \emph{octave}, \emph{n\_octaves=None}, \emph{descending=None}, \emph{valid\_intervals=None}}{}
Bases: {\hyperref[\detokenize{index:birdears.interval.IntervalBase}]{\sphinxcrossref{\sphinxcode{birdears.interval.IntervalBase}}}}

Chooses a diatonic interval for the question.
\index{tonic\_octave (birdears.interval.ChromaticInterval attribute)}

\begin{fulllineitems}
\phantomsection\label{\detokenize{index:birdears.interval.ChromaticInterval.tonic_octave}}\pysigline{\sphinxbfcode{tonic\_octave}}
\sphinxstyleemphasis{int} \textendash{} Scientific octave for the tonic. For example, if
the tonic is a ‘C4’ then \sphinxtitleref{tonic\_octave} is 4.

\end{fulllineitems}



\begin{fulllineitems}
\pysigline{\sphinxbfcode{interval~octave}}
\sphinxstyleemphasis{int} \textendash{} Scientific octave for the interval. For example,
if the interval is a ‘G5’ then \sphinxtitleref{tonic\_octave} is 5.

\end{fulllineitems}

\index{chromatic\_offset (birdears.interval.ChromaticInterval attribute)}

\begin{fulllineitems}
\phantomsection\label{\detokenize{index:birdears.interval.ChromaticInterval.chromatic_offset}}\pysigline{\sphinxbfcode{chromatic\_offset}}
\sphinxstyleemphasis{int} \textendash{} The offset in semitones inside one octave;
maybe it will be deprecated in favour of \sphinxtitleref{distance{[}‘semitones’{]}}
which is the same.

\end{fulllineitems}

\index{note\_and\_octave (birdears.interval.ChromaticInterval attribute)}

\begin{fulllineitems}
\phantomsection\label{\detokenize{index:birdears.interval.ChromaticInterval.note_and_octave}}\pysigline{\sphinxbfcode{note\_and\_octave}}
\sphinxstyleemphasis{str} \textendash{} Note and octave of the interval, for example, if
the interval is G5 the note name is ‘G5’.

\end{fulllineitems}

\index{note\_name (birdears.interval.ChromaticInterval attribute)}

\begin{fulllineitems}
\phantomsection\label{\detokenize{index:birdears.interval.ChromaticInterval.note_name}}\pysigline{\sphinxbfcode{note\_name}}
\sphinxstyleemphasis{str} \textendash{} The note name of the interval, for example, if the
interval is G5 then the name is ‘G’.

\end{fulllineitems}

\index{semitones (birdears.interval.ChromaticInterval attribute)}

\begin{fulllineitems}
\phantomsection\label{\detokenize{index:birdears.interval.ChromaticInterval.semitones}}\pysigline{\sphinxbfcode{semitones}}
\sphinxstyleemphasis{int} \textendash{} Semitones from tonic to octave. If tonic is C4 and
interval is G5 the number of semitones is 19.

\end{fulllineitems}

\index{is\_chromatic (birdears.interval.ChromaticInterval attribute)}

\begin{fulllineitems}
\phantomsection\label{\detokenize{index:birdears.interval.ChromaticInterval.is_chromatic}}\pysigline{\sphinxbfcode{is\_chromatic}}
\sphinxstyleemphasis{bool} \textendash{} If the current interval is chromatic (True) or if
it exists in the diatonic scale which key is tonic.

\end{fulllineitems}

\index{is\_descending (birdears.interval.ChromaticInterval attribute)}

\begin{fulllineitems}
\phantomsection\label{\detokenize{index:birdears.interval.ChromaticInterval.is_descending}}\pysigline{\sphinxbfcode{is\_descending}}
\sphinxstyleemphasis{bool} \textendash{} If the interval has a descending direction, ie.,
has a lower pitch than the tonic.

\end{fulllineitems}

\index{diatonic\_index (birdears.interval.ChromaticInterval attribute)}

\begin{fulllineitems}
\phantomsection\label{\detokenize{index:birdears.interval.ChromaticInterval.diatonic_index}}\pysigline{\sphinxbfcode{diatonic\_index}}
\sphinxstyleemphasis{int} \textendash{} If the interval is chromatic, this will be the
nearest diatonic interval in the direction of the resolution
(closest tonic.) From II to IV degrees, it is the ditonic interval
before; from V to VII it is the diatonic interval after.

\end{fulllineitems}

\index{distance (birdears.interval.ChromaticInterval attribute)}

\begin{fulllineitems}
\phantomsection\label{\detokenize{index:birdears.interval.ChromaticInterval.distance}}\pysigline{\sphinxbfcode{distance}}
\sphinxstyleemphasis{dict} \textendash{} A dictionary which the distance from tonic to
interval, for example, if tonic is C4 and interval is G5:

\begin{sphinxVerbatim}[commandchars=\\\{\}]
\PYGZob{}
    \PYGZsq{}octaves\PYGZsq{}: 1,
    \PYGZsq{}semitones\PYGZsq{}: 7
\PYGZcb{}
\end{sphinxVerbatim}

\end{fulllineitems}

\index{data (birdears.interval.ChromaticInterval attribute)}

\begin{fulllineitems}
\phantomsection\label{\detokenize{index:birdears.interval.ChromaticInterval.data}}\pysigline{\sphinxbfcode{data}}
\sphinxstyleemphasis{tuple} \textendash{} A tuple representing the interval data in the form of
(semitones, short\_name, long\_name), for example:

\begin{sphinxVerbatim}[commandchars=\\\{\}]
(19, \PYGZsq{}P12\PYGZsq{}, \PYGZsq{}Perfect Twelfth\PYGZsq{})
\end{sphinxVerbatim}

\end{fulllineitems}


\begin{sphinxadmonition}{note}{\label{index:index-0}Todo:}\begin{itemize}
\item {} \begin{description}
\item[{Maybe we should refactor some of the attributes with a tuple}] \leavevmode
(note, octave)

\end{description}

\item {} 
Maybe remove \sphinxtitleref{chromatic\_offset} in favor of \sphinxtitleref{distance{[}‘semitones’{]}{}`}

\end{itemize}
\end{sphinxadmonition}
\index{\_\_init\_\_() (birdears.interval.ChromaticInterval method)}

\begin{fulllineitems}
\phantomsection\label{\detokenize{index:birdears.interval.ChromaticInterval.__init__}}\pysiglinewithargsret{\sphinxbfcode{\_\_init\_\_}}{\emph{mode}, \emph{tonic}, \emph{octave}, \emph{n\_octaves=None}, \emph{descending=None}, \emph{valid\_intervals=None}}{}
Inits the class and choses a random interval with the given args.
\begin{quote}\begin{description}
\item[{Parameters}] \leavevmode\begin{itemize}
\item {} 
\sphinxstyleliteralstrong{mode} (\sphinxstyleliteralemphasis{str}) \textendash{} Diatonic mode for the interval.
(eg.: ‘major’ or ‘minor’)

\item {} 
\sphinxstyleliteralstrong{tonic} (\sphinxstyleliteralemphasis{str}) \textendash{} Tonic of the scale. (eg.: ‘Bb’)

\item {} 
\sphinxstyleliteralstrong{octave} (\sphinxstyleliteralemphasis{str}) \textendash{} Scientific octave of the scale (eg.: 4)

\item {} 
\sphinxstyleliteralstrong{interval} (\sphinxstyleliteralemphasis{str}) \textendash{} Not implemented. The interval.

\item {} 
\sphinxstyleliteralstrong{chromatic} (\sphinxstyleliteralemphasis{bool}) \textendash{} Can have chromatic notes? (eg.: F\# in a key
of C; default: false)

\item {} 
\sphinxstyleliteralstrong{n\_octaves} (\sphinxstyleliteralemphasis{int}) \textendash{} Maximum number os octaves (eg. 2)

\item {} 
\sphinxstyleliteralstrong{descending} (\sphinxstyleliteralemphasis{bool}) \textendash{} Is the interval descending? (default: false)

\item {} 
\sphinxstyleliteralstrong{valid\_intervals} (\sphinxstyleliteralemphasis{int}) \textendash{} A list with inervals valid for random
choice, 1 is 1st, 2 is second etc.

\end{itemize}

\end{description}\end{quote}

\end{fulllineitems}


\end{fulllineitems}

\index{DiatonicInterval (class in birdears.interval)}

\begin{fulllineitems}
\phantomsection\label{\detokenize{index:birdears.interval.DiatonicInterval}}\pysiglinewithargsret{\sphinxbfcode{class }\sphinxcode{birdears.interval.}\sphinxbfcode{DiatonicInterval}}{\emph{mode}, \emph{tonic}, \emph{octave}, \emph{n\_octaves=None}, \emph{descending=None}, \emph{valid\_intervals=None}}{}
Bases: {\hyperref[\detokenize{index:birdears.interval.IntervalBase}]{\sphinxcrossref{\sphinxcode{birdears.interval.IntervalBase}}}}

Chooses a diatonic interval for the question.
\index{tonic\_octave (birdears.interval.DiatonicInterval attribute)}

\begin{fulllineitems}
\phantomsection\label{\detokenize{index:birdears.interval.DiatonicInterval.tonic_octave}}\pysigline{\sphinxbfcode{tonic\_octave}}
\sphinxstyleemphasis{int} \textendash{} Scientific octave for the tonic. For example, if
the tonic is a ‘C4’ then \sphinxtitleref{tonic\_octave} is 4.

\end{fulllineitems}



\begin{fulllineitems}
\pysigline{\sphinxbfcode{interval~octave}}
\sphinxstyleemphasis{int} \textendash{} Scientific octave for the interval. For example,
if the interval is a ‘G5’ then \sphinxtitleref{tonic\_octave} is 5.

\end{fulllineitems}

\index{chromatic\_offset (birdears.interval.DiatonicInterval attribute)}

\begin{fulllineitems}
\phantomsection\label{\detokenize{index:birdears.interval.DiatonicInterval.chromatic_offset}}\pysigline{\sphinxbfcode{chromatic\_offset}}
\sphinxstyleemphasis{int} \textendash{} The offset in semitones inside one octave.
Relative semitones to tonic.

\end{fulllineitems}

\index{note\_and\_octave (birdears.interval.DiatonicInterval attribute)}

\begin{fulllineitems}
\phantomsection\label{\detokenize{index:birdears.interval.DiatonicInterval.note_and_octave}}\pysigline{\sphinxbfcode{note\_and\_octave}}
\sphinxstyleemphasis{str} \textendash{} Note and octave of the interval, for example, if
the interval is G5 the note name is ‘G5’.

\end{fulllineitems}

\index{note\_name (birdears.interval.DiatonicInterval attribute)}

\begin{fulllineitems}
\phantomsection\label{\detokenize{index:birdears.interval.DiatonicInterval.note_name}}\pysigline{\sphinxbfcode{note\_name}}
\sphinxstyleemphasis{str} \textendash{} The note name of the interval, for example, if the
interval is G5 then the name is ‘G’.

\end{fulllineitems}

\index{semitones (birdears.interval.DiatonicInterval attribute)}

\begin{fulllineitems}
\phantomsection\label{\detokenize{index:birdears.interval.DiatonicInterval.semitones}}\pysigline{\sphinxbfcode{semitones}}
\sphinxstyleemphasis{int} \textendash{} Semitones from tonic to octave. If tonic is C4 and
interval is G5 the number of semitones is 19.

\end{fulllineitems}

\index{is\_chromatic (birdears.interval.DiatonicInterval attribute)}

\begin{fulllineitems}
\phantomsection\label{\detokenize{index:birdears.interval.DiatonicInterval.is_chromatic}}\pysigline{\sphinxbfcode{is\_chromatic}}
\sphinxstyleemphasis{bool} \textendash{} If the current interval is chromatic (True) or if
it exists in the diatonic scale which key is tonic.

\end{fulllineitems}

\index{is\_descending (birdears.interval.DiatonicInterval attribute)}

\begin{fulllineitems}
\phantomsection\label{\detokenize{index:birdears.interval.DiatonicInterval.is_descending}}\pysigline{\sphinxbfcode{is\_descending}}
\sphinxstyleemphasis{bool} \textendash{} If the interval has a descending direction, ie.,
has a lower pitch than the tonic.

\end{fulllineitems}

\index{diatonic\_index (birdears.interval.DiatonicInterval attribute)}

\begin{fulllineitems}
\phantomsection\label{\detokenize{index:birdears.interval.DiatonicInterval.diatonic_index}}\pysigline{\sphinxbfcode{diatonic\_index}}
\sphinxstyleemphasis{int} \textendash{} If the interval is chromatic, this will be the
nearest diatonic interval in the direction of the resolution
(closest tonic.) From II to IV degrees, it is the ditonic interval
before; from V to VII it is the diatonic interval after.

\end{fulllineitems}

\index{distance (birdears.interval.DiatonicInterval attribute)}

\begin{fulllineitems}
\phantomsection\label{\detokenize{index:birdears.interval.DiatonicInterval.distance}}\pysigline{\sphinxbfcode{distance}}
\sphinxstyleemphasis{dict} \textendash{} A dictionary which the distance from tonic to
interval, for example, if tonic is C4 and interval is G5:

\begin{sphinxVerbatim}[commandchars=\\\{\}]
\PYGZob{}
    \PYGZsq{}octaves\PYGZsq{}: 1,
    \PYGZsq{}semitones\PYGZsq{}: 7
\PYGZcb{}
\end{sphinxVerbatim}

\end{fulllineitems}

\index{data (birdears.interval.DiatonicInterval attribute)}

\begin{fulllineitems}
\phantomsection\label{\detokenize{index:birdears.interval.DiatonicInterval.data}}\pysigline{\sphinxbfcode{data}}
\sphinxstyleemphasis{tuple} \textendash{} A tuple representing the interval data in the form of
(semitones, short\_name, long\_name), for example:

\begin{sphinxVerbatim}[commandchars=\\\{\}]
(19, \PYGZsq{}P12\PYGZsq{}, \PYGZsq{}Perfect Twelfth\PYGZsq{})
\end{sphinxVerbatim}

\end{fulllineitems}

\index{\_\_init\_\_() (birdears.interval.DiatonicInterval method)}

\begin{fulllineitems}
\phantomsection\label{\detokenize{index:birdears.interval.DiatonicInterval.__init__}}\pysiglinewithargsret{\sphinxbfcode{\_\_init\_\_}}{\emph{mode}, \emph{tonic}, \emph{octave}, \emph{n\_octaves=None}, \emph{descending=None}, \emph{valid\_intervals=None}}{}
Inits the class and choses a random interval with the given args.
\begin{quote}\begin{description}
\item[{Parameters}] \leavevmode\begin{itemize}
\item {} 
\sphinxstyleliteralstrong{mode} (\sphinxstyleliteralemphasis{str}) \textendash{} Diatonic mode for the interval.
(eg.: ‘major’ or ‘minor’)

\item {} 
\sphinxstyleliteralstrong{tonic} (\sphinxstyleliteralemphasis{str}) \textendash{} Tonic of the scale. (eg.: ‘Bb’)

\item {} 
\sphinxstyleliteralstrong{octave} (\sphinxstyleliteralemphasis{str}) \textendash{} Scientific octave of the scale (eg.: 4)

\item {} 
\sphinxstyleliteralstrong{n\_octaves} (\sphinxstyleliteralemphasis{int}) \textendash{} Maximum number os octaves (eg. 2)

\item {} 
\sphinxstyleliteralstrong{descending} (\sphinxstyleliteralemphasis{bool}) \textendash{} Is the interval descending? (default: false)

\item {} 
\sphinxstyleliteralstrong{valid\_intervals} (\sphinxstyleliteralemphasis{int}) \textendash{} A list with intervals (int) valid for random
choice, 1 is 1st, 2 is second etc.

\end{itemize}

\end{description}\end{quote}

\end{fulllineitems}


\end{fulllineitems}

\index{IntervalBase (class in birdears.interval)}

\begin{fulllineitems}
\phantomsection\label{\detokenize{index:birdears.interval.IntervalBase}}\pysigline{\sphinxbfcode{class }\sphinxcode{birdears.interval.}\sphinxbfcode{IntervalBase}}
Bases: \sphinxcode{object}
\index{\_\_init\_\_() (birdears.interval.IntervalBase method)}

\begin{fulllineitems}
\phantomsection\label{\detokenize{index:birdears.interval.IntervalBase.__init__}}\pysiglinewithargsret{\sphinxbfcode{\_\_init\_\_}}{}{}
Base class for interval classes.

\end{fulllineitems}

\index{return\_simple() (birdears.interval.IntervalBase method)}

\begin{fulllineitems}
\phantomsection\label{\detokenize{index:birdears.interval.IntervalBase.return_simple}}\pysiglinewithargsret{\sphinxbfcode{return\_simple}}{\emph{keys}}{}
This method returns a dict with only the values passed to \sphinxtitleref{keys}.

\end{fulllineitems}


\end{fulllineitems}



\section{birdears.logger module}
\label{\detokenize{index:birdears-logger-module}}\label{\detokenize{index:module-birdears.logger}}\index{birdears.logger (module)}
This submodule exports \sphinxtitleref{logger} to log events.

Logging messages which are less severe than \sphinxtitleref{lvl} will be ignored:

\begin{sphinxVerbatim}[commandchars=\\\{\}]
Level       Numeric value
\PYGZhy{}\PYGZhy{}\PYGZhy{}\PYGZhy{}\PYGZhy{}       \PYGZhy{}\PYGZhy{}\PYGZhy{}\PYGZhy{}\PYGZhy{}\PYGZhy{}\PYGZhy{}\PYGZhy{}\PYGZhy{}\PYGZhy{}\PYGZhy{}\PYGZhy{}\PYGZhy{}
CRITICAL    50
ERROR       40
WARNING     30
INFO        20
DEBUG       10
NOTSET      0

Level       When it’s used
\PYGZhy{}\PYGZhy{}\PYGZhy{}\PYGZhy{}\PYGZhy{}       \PYGZhy{}\PYGZhy{}\PYGZhy{}\PYGZhy{}\PYGZhy{}\PYGZhy{}\PYGZhy{}\PYGZhy{}\PYGZhy{}\PYGZhy{}\PYGZhy{}\PYGZhy{}\PYGZhy{}\PYGZhy{}
DEBUG       Detailed information, typically of interest only when
                diagnosing problems.
INFO        Confirmation that things are working as expected.
WARNING     An indication that something unexpected happened, or indicative
                of some problem in the near future (e.g. ‘disk space low’).
                The software is still working as expected.
ERROR       Due to a more serious problem, the software has not been able
                to perform some function.
CRITICAL    A serious error, indicating that the program itself may be
                unable to continue running.
\end{sphinxVerbatim}
\index{log\_event() (in module birdears.logger)}

\begin{fulllineitems}
\phantomsection\label{\detokenize{index:birdears.logger.log_event}}\pysiglinewithargsret{\sphinxcode{birdears.logger.}\sphinxbfcode{log\_event}}{\emph{f}, \emph{*args}, \emph{**kwargs}}{}
Decorator. Functions and method decorated with this decorator will have
their signature logged when birdears is executed with \sphinxtitleref{\textendash{}debug} mode. Both
function signature with their call values and their return will be logged.

\end{fulllineitems}



\section{birdears.prequestion module}
\label{\detokenize{index:module-birdears.prequestion}}\label{\detokenize{index:birdears-prequestion-module}}\index{birdears.prequestion (module)}
This module implements pre-questions’ progressions.

Pre questions are chord progressions or notes played before the question is
played, so to affirmate the sound of the question’s key.

For example a common cadence is chords I-IV-V-I from the diatonic scale, which
in a key of \sphinxtitleref{C} is \sphinxtitleref{CM-FM-GM-CM} and in a key of \sphinxtitleref{A} is \sphinxtitleref{AM-DM-EM-AM}.

Pre-question methods should be decorated with \sphinxtitleref{register\_prequestion\_method}
decorator, so that they will be registered as a valid pre-question method.
\index{PreQuestion (class in birdears.prequestion)}

\begin{fulllineitems}
\phantomsection\label{\detokenize{index:birdears.prequestion.PreQuestion}}\pysiglinewithargsret{\sphinxbfcode{class }\sphinxcode{birdears.prequestion.}\sphinxbfcode{PreQuestion}}{\emph{method}, \emph{question}}{}
Bases: \sphinxcode{object}
\index{\_\_call\_\_() (birdears.prequestion.PreQuestion method)}

\begin{fulllineitems}
\phantomsection\label{\detokenize{index:birdears.prequestion.PreQuestion.__call__}}\pysiglinewithargsret{\sphinxbfcode{\_\_call\_\_}}{\emph{*args}, \emph{**kwargs}}{}
Calls the resolution method and pass arguments to it.

Returns a \sphinxtitleref{birdears.Sequence} object with the pre-question generated by
the method.

\end{fulllineitems}

\index{\_\_init\_\_() (birdears.prequestion.PreQuestion method)}

\begin{fulllineitems}
\phantomsection\label{\detokenize{index:birdears.prequestion.PreQuestion.__init__}}\pysiglinewithargsret{\sphinxbfcode{\_\_init\_\_}}{\emph{method}, \emph{question}}{}
This class implements methods for different types of pre-question
progressions.
\begin{quote}\begin{description}
\item[{Parameters}] \leavevmode\begin{itemize}
\item {} 
\sphinxstyleliteralstrong{method} (\sphinxstyleliteralemphasis{str}) \textendash{} The method used in the pre question.

\item {} 
\sphinxstyleliteralstrong{question} (\sphinxstyleliteralemphasis{obj}) \textendash{} Question object from which to generate the

\item {} 
\sphinxstyleliteralstrong{sequence.} (\sphinxstyleliteralemphasis{pre-question}) \textendash{} 

\end{itemize}

\end{description}\end{quote}

\end{fulllineitems}


\end{fulllineitems}

\index{none() (in module birdears.prequestion)}

\begin{fulllineitems}
\phantomsection\label{\detokenize{index:birdears.prequestion.none}}\pysiglinewithargsret{\sphinxcode{birdears.prequestion.}\sphinxbfcode{none}}{\emph{question}, \emph{*args}, \emph{**kwargs}}{}
Pre-question method that return an empty sequence with no delay.
:param question: Question object from which to generate the
\begin{quote}

pre-question sequence. (this is provided by the \sphinxtitleref{Resolution} class
when it is {\color{red}\bfseries{}{}`}\_\_call\_\_{}`ed)
\end{quote}
\begin{quote}\begin{description}
\end{description}\end{quote}

\end{fulllineitems}

\index{progression\_i\_iv\_v\_i() (in module birdears.prequestion)}

\begin{fulllineitems}
\phantomsection\label{\detokenize{index:birdears.prequestion.progression_i_iv_v_i}}\pysiglinewithargsret{\sphinxcode{birdears.prequestion.}\sphinxbfcode{progression\_i\_iv\_v\_i}}{\emph{question}, \emph{*args}, \emph{**kwargs}}{}
Pre-question method that play’s a chord progression with triad chords built
on the grades I, IV, V the I of the question key.
\begin{quote}\begin{description}
\item[{Parameters}] \leavevmode
\sphinxstyleliteralstrong{question} (\sphinxstyleliteralemphasis{obj}) \textendash{} Question object from which to generate the
pre-question sequence. (this is provided by the \sphinxtitleref{Resolution} class
when it is {\color{red}\bfseries{}{}`}\_\_call\_\_{}`ed)

\end{description}\end{quote}

\end{fulllineitems}

\index{register\_prequestion\_method() (in module birdears.prequestion)}

\begin{fulllineitems}
\phantomsection\label{\detokenize{index:birdears.prequestion.register_prequestion_method}}\pysiglinewithargsret{\sphinxcode{birdears.prequestion.}\sphinxbfcode{register\_prequestion\_method}}{\emph{f}, \emph{*args}, \emph{**kwargs}}{}
Decorator for prequestion method functions.

Functions decorated with this decorator will be registered in the
\sphinxtitleref{PREQUESTION\_METHODS} global dict.

\end{fulllineitems}

\index{tonic\_only() (in module birdears.prequestion)}

\begin{fulllineitems}
\phantomsection\label{\detokenize{index:birdears.prequestion.tonic_only}}\pysiglinewithargsret{\sphinxcode{birdears.prequestion.}\sphinxbfcode{tonic\_only}}{\emph{question}, \emph{*args}, \emph{**kwargs}}{}
Pre-question method that only play’s the question tonic note before the
question.
\begin{quote}\begin{description}
\item[{Parameters}] \leavevmode
\sphinxstyleliteralstrong{question} (\sphinxstyleliteralemphasis{obj}) \textendash{} Question object from which to generate the
pre-question sequence. (this is provided by the \sphinxtitleref{Resolution} class
when it is {\color{red}\bfseries{}{}`}\_\_call\_\_{}`ed)

\end{description}\end{quote}

\end{fulllineitems}



\section{birdears.questionbase module}
\label{\detokenize{index:module-birdears.questionbase}}\label{\detokenize{index:birdears-questionbase-module}}\index{birdears.questionbase (module)}\index{QuestionBase (class in birdears.questionbase)}

\begin{fulllineitems}
\phantomsection\label{\detokenize{index:birdears.questionbase.QuestionBase}}\pysiglinewithargsret{\sphinxbfcode{class }\sphinxcode{birdears.questionbase.}\sphinxbfcode{QuestionBase}}{\emph{mode='major'}, \emph{tonic=None}, \emph{octave=None}, \emph{descending=None}, \emph{chromatic=None}, \emph{n\_octaves=None}, \emph{valid\_intervals=None}, \emph{user\_durations=None}, \emph{prequestion\_method=None}, \emph{resolution\_method=None}, \emph{default\_durations=None}, \emph{*args}, \emph{**kwargs}}{}
Bases: \sphinxcode{object}

Base Class to be subclassed for Question classes.

This class implements attributes and routines to be used in Question
subclasses.
\index{\_\_init\_\_() (birdears.questionbase.QuestionBase method)}

\begin{fulllineitems}
\phantomsection\label{\detokenize{index:birdears.questionbase.QuestionBase.__init__}}\pysiglinewithargsret{\sphinxbfcode{\_\_init\_\_}}{\emph{mode='major'}, \emph{tonic=None}, \emph{octave=None}, \emph{descending=None}, \emph{chromatic=None}, \emph{n\_octaves=None}, \emph{valid\_intervals=None}, \emph{user\_durations=None}, \emph{prequestion\_method=None}, \emph{resolution\_method=None}, \emph{default\_durations=None}, \emph{*args}, \emph{**kwargs}}{}
Inits the class.
\begin{quote}\begin{description}
\item[{Parameters}] \leavevmode\begin{itemize}
\item {} 
\sphinxstyleliteralstrong{mode} (\sphinxstyleliteralemphasis{str}) \textendash{} A string represnting the mode of the question.
Eg., ‘major’ or ‘minor’

\item {} 
\sphinxstyleliteralstrong{tonic} (\sphinxstyleliteralemphasis{str}) \textendash{} A string representing the tonic of the
question, eg.: ‘C’; if omitted, it will be selected
randomly.

\item {} 
\sphinxstyleliteralstrong{octave} (\sphinxstyleliteralemphasis{int}) \textendash{} A scienfic octave notation, for example,
4 for ‘C4’; if not present, it will be randomly chosen.

\item {} 
\sphinxstyleliteralstrong{descending} (\sphinxstyleliteralemphasis{bool}) \textendash{} Is the question direction in descending,
ie., intervals have lower pitch than the tonic.

\item {} 
\sphinxstyleliteralstrong{chromatic} (\sphinxstyleliteralemphasis{bool}) \textendash{} If the question can have (True) or not
(False) chromatic intervals, ie., intervals not in the
diatonic scale of tonic/mode.

\item {} 
\sphinxstyleliteralstrong{n\_octaves} (\sphinxstyleliteralemphasis{int}) \textendash{} Maximum numbr of octaves of the question.

\item {} 
\sphinxstyleliteralstrong{valid\_intervals} (\sphinxstyleliteralemphasis{list}) \textendash{} A list with intervals (int) valid for
random choice, 1 is 1st, 2 is second etc. Eg. {[}1, 4, 5{]} to
allow only tonics, fourths and fifths.

\item {} 
\sphinxstyleliteralstrong{user\_durations} (\sphinxstyleliteralemphasis{dict}) \textendash{} 
A string with 9 comma-separated \sphinxtitleref{int} or
\sphinxtitleref{float{}`s to set the default duration for the notes played. The
values are respectively for: pre-question duration (1st),
pre-question delay (2nd), and pre-question pos-delay (3rd);
question duration (4th), question delay (5th), and question
pos-delay (6th); resolution duration (7th), resolution
delay (8th), and resolution pos-delay (9th).
duration is the duration in of the note in seconds; delay is
the time to wait before playing the next note, and pos\_delay is
the time to wait after all the notes of the respective sequence
have been played. If any of the user durations is {}`n}, the
default duration for the type of question will be used instead.
Example:

\begin{sphinxVerbatim}[commandchars=\\\{\}]
\PYGZdq{}2,0.5,1,2,n,0,2.5,n,1\PYGZdq{}
\end{sphinxVerbatim}


\item {} 
\sphinxstyleliteralstrong{prequestion\_method} (\sphinxstyleliteralemphasis{str}) \textendash{} Method of playing a cadence or the
exercise tonic before the question so to affirm the question
musical tonic key to the ear. Valid ones are registered in the
\sphinxtitleref{birdears.prequestion.PREQUESION\_METHODS} global dict.

\item {} 
\sphinxstyleliteralstrong{resolution\_method} (\sphinxstyleliteralemphasis{str}) \textendash{} Method of playing the resolution of an
exercise Valid ones are registered in the
\sphinxtitleref{birdears.resolution.RESOLUTION\_METHODS} global dict.

\item {} 
\sphinxstyleliteralstrong{user\_durations} \textendash{} Dictionary with the default durations for
each type of sequence. This is provided by the subclasses.

\end{itemize}

\end{description}\end{quote}

\end{fulllineitems}

\index{check\_question() (birdears.questionbase.QuestionBase method)}

\begin{fulllineitems}
\phantomsection\label{\detokenize{index:birdears.questionbase.QuestionBase.check_question}}\pysiglinewithargsret{\sphinxbfcode{check\_question}}{}{}
This method should be overwritten by the question subclasses.

\end{fulllineitems}

\index{get\_valid\_semitones() (birdears.questionbase.QuestionBase method)}

\begin{fulllineitems}
\phantomsection\label{\detokenize{index:birdears.questionbase.QuestionBase.get_valid_semitones}}\pysiglinewithargsret{\sphinxbfcode{get\_valid\_semitones}}{}{}
Returns a list with valid semitones for question.

\end{fulllineitems}

\index{make\_question() (birdears.questionbase.QuestionBase method)}

\begin{fulllineitems}
\phantomsection\label{\detokenize{index:birdears.questionbase.QuestionBase.make_question}}\pysiglinewithargsret{\sphinxbfcode{make\_question}}{}{}
This method should be overwritten by the question subclasses.

\end{fulllineitems}

\index{make\_resolution() (birdears.questionbase.QuestionBase method)}

\begin{fulllineitems}
\phantomsection\label{\detokenize{index:birdears.questionbase.QuestionBase.make_resolution}}\pysiglinewithargsret{\sphinxbfcode{make\_resolution}}{}{}
This method should be overwritten by the question subclasses.

\end{fulllineitems}

\index{play\_question() (birdears.questionbase.QuestionBase method)}

\begin{fulllineitems}
\phantomsection\label{\detokenize{index:birdears.questionbase.QuestionBase.play_question}}\pysiglinewithargsret{\sphinxbfcode{play\_question}}{}{}
This method should be overwritten by the question subclasses.

\end{fulllineitems}


\end{fulllineitems}

\index{register\_question\_class() (in module birdears.questionbase)}

\begin{fulllineitems}
\phantomsection\label{\detokenize{index:birdears.questionbase.register_question_class}}\pysiglinewithargsret{\sphinxcode{birdears.questionbase.}\sphinxbfcode{register\_question\_class}}{\emph{f}, \emph{*args}, \emph{**kwargs}}{}
Decorator for question classes.

Classes decorated with this decorator will be registered in the
\sphinxtitleref{QUESTION\_CLASSES} global.

\end{fulllineitems}



\section{birdears.resolution module}
\label{\detokenize{index:birdears-resolution-module}}\label{\detokenize{index:module-birdears.resolution}}\index{birdears.resolution (module)}\index{Resolution (class in birdears.resolution)}

\begin{fulllineitems}
\phantomsection\label{\detokenize{index:birdears.resolution.Resolution}}\pysiglinewithargsret{\sphinxbfcode{class }\sphinxcode{birdears.resolution.}\sphinxbfcode{Resolution}}{\emph{method}, \emph{question}}{}
Bases: \sphinxcode{object}

This class implements methods for different types of question
resolutions.

A resolution is an answer to a question. It aims to create a mnemonic on
how the inverval resvolver to the tonic.
\index{\_\_call\_\_() (birdears.resolution.Resolution method)}

\begin{fulllineitems}
\phantomsection\label{\detokenize{index:birdears.resolution.Resolution.__call__}}\pysiglinewithargsret{\sphinxbfcode{\_\_call\_\_}}{\emph{*args}, \emph{**kwargs}}{}
Calls the resolution method and pass arguments to it.

Returns a \sphinxtitleref{birdears.Sequence} object with the resolution generated by
the.method.

\end{fulllineitems}

\index{\_\_init\_\_() (birdears.resolution.Resolution method)}

\begin{fulllineitems}
\phantomsection\label{\detokenize{index:birdears.resolution.Resolution.__init__}}\pysiglinewithargsret{\sphinxbfcode{\_\_init\_\_}}{\emph{method}, \emph{question}}{}
Inits the resolution class.
\begin{quote}\begin{description}
\item[{Parameters}] \leavevmode\begin{itemize}
\item {} 
\sphinxstyleliteralstrong{method} (\sphinxstyleliteralemphasis{str}) \textendash{} The method used in the resolution.

\item {} 
\sphinxstyleliteralstrong{question} (\sphinxstyleliteralemphasis{obj}) \textendash{} Question object from which to generate the

\item {} 
\sphinxstyleliteralstrong{sequence.} ({\hyperref[\detokenize{index:module-birdears.resolution}]{\sphinxcrossref{\sphinxstyleliteralemphasis{resolution}}}}) \textendash{} 

\end{itemize}

\end{description}\end{quote}

\end{fulllineitems}


\end{fulllineitems}

\index{nearest\_tonic() (in module birdears.resolution)}

\begin{fulllineitems}
\phantomsection\label{\detokenize{index:birdears.resolution.nearest_tonic}}\pysiglinewithargsret{\sphinxcode{birdears.resolution.}\sphinxbfcode{nearest\_tonic}}{\emph{question}}{}
Resolution method that resolve the intervals to their nearest tonics.
\begin{quote}\begin{description}
\item[{Parameters}] \leavevmode
\sphinxstyleliteralstrong{question} (\sphinxstyleliteralemphasis{obj}) \textendash{} Question object from which to generate the
resolution sequence. (this is provided by the \sphinxtitleref{Prequestion} class
when it is {\color{red}\bfseries{}{}`}\_\_call\_\_{}`ed)

\end{description}\end{quote}

\end{fulllineitems}

\index{register\_resolution\_method() (in module birdears.resolution)}

\begin{fulllineitems}
\phantomsection\label{\detokenize{index:birdears.resolution.register_resolution_method}}\pysiglinewithargsret{\sphinxcode{birdears.resolution.}\sphinxbfcode{register\_resolution\_method}}{\emph{f}, \emph{*args}, \emph{**kwargs}}{}
Decorator for resolution method functions.

Functions decorated with this decorator will be registered in the
\sphinxtitleref{RESOLUTION\_METHODS} global dict.

\end{fulllineitems}

\index{repeat\_only() (in module birdears.resolution)}

\begin{fulllineitems}
\phantomsection\label{\detokenize{index:birdears.resolution.repeat_only}}\pysiglinewithargsret{\sphinxcode{birdears.resolution.}\sphinxbfcode{repeat\_only}}{\emph{question}}{}
Resolution method that only repeats the sequence elements with given
durations.
\begin{quote}\begin{description}
\item[{Parameters}] \leavevmode
\sphinxstyleliteralstrong{question} (\sphinxstyleliteralemphasis{obj}) \textendash{} Question object from which to generate the
resolution sequence. (this is provided by the \sphinxtitleref{Prequestion} class
when it is {\color{red}\bfseries{}{}`}\_\_call\_\_{}`ed)

\end{description}\end{quote}

\end{fulllineitems}



\section{birdears.scale module}
\label{\detokenize{index:module-birdears.scale}}\label{\detokenize{index:birdears-scale-module}}\index{birdears.scale (module)}\index{ChromaticScale (class in birdears.scale)}

\begin{fulllineitems}
\phantomsection\label{\detokenize{index:birdears.scale.ChromaticScale}}\pysiglinewithargsret{\sphinxbfcode{class }\sphinxcode{birdears.scale.}\sphinxbfcode{ChromaticScale}}{\emph{tonic}, \emph{octave=None}, \emph{n\_octaves=None}, \emph{descending=None}, \emph{dont\_repeat\_tonic=None}}{}
Bases: {\hyperref[\detokenize{index:birdears.scale.ScaleBase}]{\sphinxcrossref{\sphinxcode{birdears.scale.ScaleBase}}}}

Builds a musical chromatic scale.
\index{scale (birdears.scale.ChromaticScale attribute)}

\begin{fulllineitems}
\phantomsection\label{\detokenize{index:birdears.scale.ChromaticScale.scale}}\pysigline{\sphinxbfcode{scale}}
\sphinxstyleemphasis{array\_type} \textendash{} The array of notes representing the scale.

\end{fulllineitems}

\index{\_\_init\_\_() (birdears.scale.ChromaticScale method)}

\begin{fulllineitems}
\phantomsection\label{\detokenize{index:birdears.scale.ChromaticScale.__init__}}\pysiglinewithargsret{\sphinxbfcode{\_\_init\_\_}}{\emph{tonic}, \emph{octave=None}, \emph{n\_octaves=None}, \emph{descending=None}, \emph{dont\_repeat\_tonic=None}}{}
Returns a chromatic scale from tonic.
\begin{quote}\begin{description}
\item[{Parameters}] \leavevmode\begin{itemize}
\item {} 
\sphinxstyleliteralstrong{tonic} (\sphinxstyleliteralemphasis{str}) \textendash{} The note which the scale will be built upon.

\item {} 
\sphinxstyleliteralstrong{octave} (\sphinxstyleliteralemphasis{int}) \textendash{} The scientific octave the scale will be built upon.

\item {} 
\sphinxstyleliteralstrong{n\_octaves} (\sphinxstyleliteralemphasis{int}) \textendash{} The number of octaves the scale will contain.

\item {} 
\sphinxstyleliteralstrong{descending} (\sphinxstyleliteralemphasis{bool}) \textendash{} Whether the scale is descending.

\item {} 
\sphinxstyleliteralstrong{dont\_repeat\_tonic} (\sphinxstyleliteralemphasis{bool}) \textendash{} Whether to skip appending the last
note (octave) to the scale.

\end{itemize}

\end{description}\end{quote}

\end{fulllineitems}

\index{get\_triad() (birdears.scale.ChromaticScale method)}

\begin{fulllineitems}
\phantomsection\label{\detokenize{index:birdears.scale.ChromaticScale.get_triad}}\pysiglinewithargsret{\sphinxbfcode{get\_triad}}{\emph{mode}, \emph{index=0}, \emph{degree=None}}{}
Returns an array with notes from a scale’s triad.
\begin{quote}\begin{description}
\item[{Parameters}] \leavevmode\begin{itemize}
\item {} 
\sphinxstyleliteralstrong{mode} (\sphinxstyleliteralemphasis{str}) \textendash{} Mode of the scale (eg. ‘major’ or ‘minor’)

\item {} 
\sphinxstyleliteralstrong{index} (\sphinxstyleliteralemphasis{int}) \textendash{} Triad index (eg.: 0 for 1st degree triad.)

\item {} 
\sphinxstyleliteralstrong{degree} (\sphinxstyleliteralemphasis{int}) \textendash{} Degree of the scale. If provided, overrides the
\sphinxtitleref{index} argument. (eg.: \sphinxtitleref{1} for the 1st degree triad.)

\end{itemize}

\item[{Returns}] \leavevmode
A list with three pitches (str), one for each note of the triad.

\end{description}\end{quote}

\end{fulllineitems}


\end{fulllineitems}

\index{DiatonicScale (class in birdears.scale)}

\begin{fulllineitems}
\phantomsection\label{\detokenize{index:birdears.scale.DiatonicScale}}\pysiglinewithargsret{\sphinxbfcode{class }\sphinxcode{birdears.scale.}\sphinxbfcode{DiatonicScale}}{\emph{tonic}, \emph{mode=None}, \emph{octave=None}, \emph{n\_octaves=None}, \emph{descending=None}, \emph{dont\_repeat\_tonic=None}}{}
Bases: {\hyperref[\detokenize{index:birdears.scale.ScaleBase}]{\sphinxcrossref{\sphinxcode{birdears.scale.ScaleBase}}}}

Builds a musical diatonic scale.
\index{scale (birdears.scale.DiatonicScale attribute)}

\begin{fulllineitems}
\phantomsection\label{\detokenize{index:birdears.scale.DiatonicScale.scale}}\pysigline{\sphinxbfcode{scale}}
\sphinxstyleemphasis{array\_type} \textendash{} The array of notes representing the scale.

\end{fulllineitems}

\index{\_\_init\_\_() (birdears.scale.DiatonicScale method)}

\begin{fulllineitems}
\phantomsection\label{\detokenize{index:birdears.scale.DiatonicScale.__init__}}\pysiglinewithargsret{\sphinxbfcode{\_\_init\_\_}}{\emph{tonic}, \emph{mode=None}, \emph{octave=None}, \emph{n\_octaves=None}, \emph{descending=None}, \emph{dont\_repeat\_tonic=None}}{}
Returns a diatonic scale from tonic and mode.
\begin{quote}\begin{description}
\item[{Parameters}] \leavevmode\begin{itemize}
\item {} 
\sphinxstyleliteralstrong{tonic} (\sphinxstyleliteralemphasis{str}) \textendash{} The note which the scale will be built upon.

\item {} 
\sphinxstyleliteralstrong{mode} (\sphinxstyleliteralemphasis{str}) \textendash{} The mode the scale will be built upon.
(‘major’ or ‘minor’)

\item {} 
\sphinxstyleliteralstrong{octave} (\sphinxstyleliteralemphasis{int}) \textendash{} The scientific octave the scale will be built upon.

\item {} 
\sphinxstyleliteralstrong{n\_octaves} (\sphinxstyleliteralemphasis{int}) \textendash{} The number of octaves the scale will contain.

\item {} 
\sphinxstyleliteralstrong{descending} (\sphinxstyleliteralemphasis{bool}) \textendash{} Whether the scale is descending.

\item {} 
\sphinxstyleliteralstrong{dont\_repeat\_tonic} (\sphinxstyleliteralemphasis{bool}) \textendash{} Whether to skip appending the last
note (octave) to the scale.

\end{itemize}

\end{description}\end{quote}

\end{fulllineitems}

\index{get\_triad() (birdears.scale.DiatonicScale method)}

\begin{fulllineitems}
\phantomsection\label{\detokenize{index:birdears.scale.DiatonicScale.get_triad}}\pysiglinewithargsret{\sphinxbfcode{get\_triad}}{\emph{index=0}, \emph{degree=None}}{}
Returns an array with notes from a scale’s triad.
\begin{quote}\begin{description}
\item[{Parameters}] \leavevmode\begin{itemize}
\item {} 
\sphinxstyleliteralstrong{index} (\sphinxstyleliteralemphasis{int}) \textendash{} triad index (eg.: 0 for 1st degree triad.)

\item {} 
\sphinxstyleliteralstrong{degree} (\sphinxstyleliteralemphasis{int}) \textendash{} Degree of the scale. If provided, overrides the
\sphinxtitleref{index} argument. (eg.: \sphinxtitleref{1} for the 1st degree triad.)

\end{itemize}

\item[{Returns}] \leavevmode
An array with three pitches, one for each note of the triad.

\end{description}\end{quote}

\end{fulllineitems}


\end{fulllineitems}

\index{ScaleBase (class in birdears.scale)}

\begin{fulllineitems}
\phantomsection\label{\detokenize{index:birdears.scale.ScaleBase}}\pysigline{\sphinxbfcode{class }\sphinxcode{birdears.scale.}\sphinxbfcode{ScaleBase}}
Bases: \sphinxcode{object}

\end{fulllineitems}



\section{birdears.sequence module}
\label{\detokenize{index:birdears-sequence-module}}\label{\detokenize{index:module-birdears.sequence}}\index{birdears.sequence (module)}\index{Sequence (class in birdears.sequence)}

\begin{fulllineitems}
\phantomsection\label{\detokenize{index:birdears.sequence.Sequence}}\pysiglinewithargsret{\sphinxbfcode{class }\sphinxcode{birdears.sequence.}\sphinxbfcode{Sequence}}{\emph{elements={[}{]}}, \emph{duration=2}, \emph{delay=1.5}, \emph{pos\_delay=1}}{}
Bases: \sphinxcode{object}

Register a Sequence of notes and/or chords.
\index{elements (birdears.sequence.Sequence attribute)}

\begin{fulllineitems}
\phantomsection\label{\detokenize{index:birdears.sequence.Sequence.elements}}\pysigline{\sphinxbfcode{elements}}
\sphinxstyleemphasis{array\_type} \textendash{} List of notes (strings) ou chords (list of
strings) in this Sequence.

\end{fulllineitems}

\index{append() (birdears.sequence.Sequence method)}

\begin{fulllineitems}
\phantomsection\label{\detokenize{index:birdears.sequence.Sequence.append}}\pysiglinewithargsret{\sphinxbfcode{append}}{\emph{elements}}{}
Appends \sphinxtitleref{elements} to Sequence.elements
\begin{quote}\begin{description}
\item[{Parameters}] \leavevmode
\sphinxstyleliteralstrong{elements} (\sphinxstyleliteralemphasis{array\_type}) \textendash{} Elements to be appended to the class.

\end{description}\end{quote}

\end{fulllineitems}

\index{async\_play() (birdears.sequence.Sequence method)}

\begin{fulllineitems}
\phantomsection\label{\detokenize{index:birdears.sequence.Sequence.async_play}}\pysiglinewithargsret{\sphinxbfcode{async\_play}}{\emph{callback}, \emph{end\_callback}}{}
Plays the Sequence elements of notes and/or chords and wait for
\sphinxtitleref{Sequence.pos\_delay} seconds.

\end{fulllineitems}

\index{extend() (birdears.sequence.Sequence method)}

\begin{fulllineitems}
\phantomsection\label{\detokenize{index:birdears.sequence.Sequence.extend}}\pysiglinewithargsret{\sphinxbfcode{extend}}{\emph{elements}}{}
Extends Sequence.elements with \sphinxtitleref{elements}.
\begin{quote}\begin{description}
\item[{Parameters}] \leavevmode
\sphinxstyleliteralstrong{elements} (\sphinxstyleliteralemphasis{array\_type}) \textendash{} elements extend the class with.

\end{description}\end{quote}

\end{fulllineitems}

\index{make\_chord\_progression() (birdears.sequence.Sequence method)}

\begin{fulllineitems}
\phantomsection\label{\detokenize{index:birdears.sequence.Sequence.make_chord_progression}}\pysiglinewithargsret{\sphinxbfcode{make\_chord\_progression}}{\emph{tonic}, \emph{mode}, \emph{degrees}}{}
Appends triad chord(s) to the Sequence.
\begin{quote}\begin{description}
\item[{Parameters}] \leavevmode\begin{itemize}
\item {} 
\sphinxstyleliteralstrong{tonic} (\sphinxstyleliteralemphasis{str}) \textendash{} Tonic note of the scale.

\item {} 
\sphinxstyleliteralstrong{mode} (\sphinxstyleliteralemphasis{str}) \textendash{} Mode of the scale from which build the triads upon.

\item {} 
\sphinxstyleliteralstrong{degrees} (\sphinxstyleliteralemphasis{array\_type}) \textendash{} List with integers represending the degrees
of each triad.

\end{itemize}

\end{description}\end{quote}

\end{fulllineitems}

\index{play() (birdears.sequence.Sequence method)}

\begin{fulllineitems}
\phantomsection\label{\detokenize{index:birdears.sequence.Sequence.play}}\pysiglinewithargsret{\sphinxbfcode{play}}{\emph{callback=None}, \emph{end\_callback=None}}{}
\end{fulllineitems}

\index{play\_element() (birdears.sequence.Sequence method)}

\begin{fulllineitems}
\phantomsection\label{\detokenize{index:birdears.sequence.Sequence.play_element}}\pysiglinewithargsret{\sphinxbfcode{play\_element}}{\emph{index}}{}
Plays element \sphinxtitleref{sequence.elements{[}index{]}.}

\end{fulllineitems}


\end{fulllineitems}



\chapter{birdears.questions package}
\label{\detokenize{index:birdears-questions-package}}\label{\detokenize{index:module-birdears.questions}}\index{birdears.questions (module)}

\section{Submodules}
\label{\detokenize{index:id12}}

\section{birdears.questions.harmonicinterval module}
\label{\detokenize{index:birdears-questions-harmonicinterval-module}}\label{\detokenize{index:module-birdears.questions.harmonicinterval}}\index{birdears.questions.harmonicinterval (module)}\index{HarmonicIntervalQuestion (class in birdears.questions.harmonicinterval)}

\begin{fulllineitems}
\phantomsection\label{\detokenize{index:birdears.questions.harmonicinterval.HarmonicIntervalQuestion}}\pysiglinewithargsret{\sphinxbfcode{class }\sphinxcode{birdears.questions.harmonicinterval.}\sphinxbfcode{HarmonicIntervalQuestion}}{\emph{mode='major'}, \emph{tonic=None}, \emph{octave=None}, \emph{descending=None}, \emph{chromatic=None}, \emph{n\_octaves=None}, \emph{valid\_intervals=None}, \emph{user\_durations=None}, \emph{prequestion\_method='none'}, \emph{resolution\_method='nearest\_tonic'}, \emph{*args}, \emph{**kwargs}}{}
Bases: {\hyperref[\detokenize{index:birdears.questionbase.QuestionBase}]{\sphinxcrossref{\sphinxcode{birdears.questionbase.QuestionBase}}}}

Implements a Harmonic Interval test.
\index{\_\_init\_\_() (birdears.questions.harmonicinterval.HarmonicIntervalQuestion method)}

\begin{fulllineitems}
\phantomsection\label{\detokenize{index:birdears.questions.harmonicinterval.HarmonicIntervalQuestion.__init__}}\pysiglinewithargsret{\sphinxbfcode{\_\_init\_\_}}{\emph{mode='major'}, \emph{tonic=None}, \emph{octave=None}, \emph{descending=None}, \emph{chromatic=None}, \emph{n\_octaves=None}, \emph{valid\_intervals=None}, \emph{user\_durations=None}, \emph{prequestion\_method='none'}, \emph{resolution\_method='nearest\_tonic'}, \emph{*args}, \emph{**kwargs}}{}
Inits the class.
\begin{quote}\begin{description}
\item[{Parameters}] \leavevmode\begin{itemize}
\item {} 
\sphinxstyleliteralstrong{mode} (\sphinxstyleliteralemphasis{str}) \textendash{} A string representing the mode of the question.
Eg., ‘major’ or ‘minor’

\item {} 
\sphinxstyleliteralstrong{tonic} (\sphinxstyleliteralemphasis{str}) \textendash{} A string representing the tonic of the question,
eg.: ‘C’; if omitted, it will be selected randomly.

\item {} 
\sphinxstyleliteralstrong{octave} (\sphinxstyleliteralemphasis{int}) \textendash{} A scienfic octave notation, for example, 4 for ‘C4’;
if not present, it will be randomly chosen.

\item {} 
\sphinxstyleliteralstrong{descending} (\sphinxstyleliteralemphasis{bool}) \textendash{} Is the question direction in descending, ie.,
intervals have lower pitch than the tonic.

\item {} 
\sphinxstyleliteralstrong{chromatic} (\sphinxstyleliteralemphasis{bool}) \textendash{} If the question can have (True) or not (False)
chromatic intervals, ie., intervals not in the diatonic scale
of tonic/mode.

\item {} 
\sphinxstyleliteralstrong{n\_octaves} (\sphinxstyleliteralemphasis{int}) \textendash{} Maximum number of octaves of the question.

\item {} 
\sphinxstyleliteralstrong{valid\_intervals} (\sphinxstyleliteralemphasis{list}) \textendash{} A list with intervals (int) valid for
random choice, 1 is 1st, 2 is second etc. Eg. {[}1, 4, 5{]} to
allow only tonics, fourths and fifths.

\item {} 
\sphinxstyleliteralstrong{user\_durations} (\sphinxstyleliteralemphasis{str}) \textendash{} 
A string with 9 comma-separated \sphinxtitleref{int} or
\sphinxtitleref{float{}`s to set the default duration for the notes played. The
values are respectively for: pre-question duration (1st),
pre-question delay (2nd), and pre-question pos-delay (3rd);
question duration (4th), question delay (5th), and question
pos-delay (6th); resolution duration (7th), resolution
delay (8th), and resolution pos-delay (9th).
duration is the duration in of the note in seconds; delay is
the time to wait before playing the next note, and pos\_delay is
the time to wait after all the notes of the respective sequence
have been played. If any of the user durations is {}`n}, the
default duration for the type of question will be used instead.
Example:

\begin{sphinxVerbatim}[commandchars=\\\{\}]
\PYGZdq{}2,0.5,1,2,n,0,2.5,n,1\PYGZdq{}
\end{sphinxVerbatim}


\item {} 
\sphinxstyleliteralstrong{prequestion\_method} (\sphinxstyleliteralemphasis{str}) \textendash{} Method of playing a cadence or the
exercise tonic before the question so to affirm the question
musical tonic key to the ear. Valid ones are registered in the
\sphinxtitleref{birdears.prequestion.PREQUESION\_METHODS} global dict.

\item {} 
\sphinxstyleliteralstrong{resolution\_method} (\sphinxstyleliteralemphasis{str}) \textendash{} Method of playing the resolution of an
exercise. Valid ones are registered in the
\sphinxtitleref{birdears.resolution.RESOLUTION\_METHODS} global dict.

\end{itemize}

\end{description}\end{quote}

\end{fulllineitems}

\index{check\_question() (birdears.questions.harmonicinterval.HarmonicIntervalQuestion method)}

\begin{fulllineitems}
\phantomsection\label{\detokenize{index:birdears.questions.harmonicinterval.HarmonicIntervalQuestion.check_question}}\pysiglinewithargsret{\sphinxbfcode{check\_question}}{\emph{user\_input\_char}}{}
Checks whether the given answer is correct.

\end{fulllineitems}

\index{make\_pre\_question() (birdears.questions.harmonicinterval.HarmonicIntervalQuestion method)}

\begin{fulllineitems}
\phantomsection\label{\detokenize{index:birdears.questions.harmonicinterval.HarmonicIntervalQuestion.make_pre_question}}\pysiglinewithargsret{\sphinxbfcode{make\_pre\_question}}{\emph{method}}{}
\end{fulllineitems}

\index{make\_question() (birdears.questions.harmonicinterval.HarmonicIntervalQuestion method)}

\begin{fulllineitems}
\phantomsection\label{\detokenize{index:birdears.questions.harmonicinterval.HarmonicIntervalQuestion.make_question}}\pysiglinewithargsret{\sphinxbfcode{make\_question}}{}{}
\end{fulllineitems}

\index{make\_resolution() (birdears.questions.harmonicinterval.HarmonicIntervalQuestion method)}

\begin{fulllineitems}
\phantomsection\label{\detokenize{index:birdears.questions.harmonicinterval.HarmonicIntervalQuestion.make_resolution}}\pysiglinewithargsret{\sphinxbfcode{make\_resolution}}{\emph{method}}{}
\end{fulllineitems}

\index{play\_question() (birdears.questions.harmonicinterval.HarmonicIntervalQuestion method)}

\begin{fulllineitems}
\phantomsection\label{\detokenize{index:birdears.questions.harmonicinterval.HarmonicIntervalQuestion.play_question}}\pysiglinewithargsret{\sphinxbfcode{play\_question}}{}{}
\end{fulllineitems}

\index{play\_resolution() (birdears.questions.harmonicinterval.HarmonicIntervalQuestion method)}

\begin{fulllineitems}
\phantomsection\label{\detokenize{index:birdears.questions.harmonicinterval.HarmonicIntervalQuestion.play_resolution}}\pysiglinewithargsret{\sphinxbfcode{play\_resolution}}{}{}
\end{fulllineitems}


\end{fulllineitems}



\section{birdears.questions.instrumentaldictation module}
\label{\detokenize{index:module-birdears.questions.instrumentaldictation}}\label{\detokenize{index:birdears-questions-instrumentaldictation-module}}\index{birdears.questions.instrumentaldictation (module)}\index{InstrumentalDictationQuestion (class in birdears.questions.instrumentaldictation)}

\begin{fulllineitems}
\phantomsection\label{\detokenize{index:birdears.questions.instrumentaldictation.InstrumentalDictationQuestion}}\pysiglinewithargsret{\sphinxbfcode{class }\sphinxcode{birdears.questions.instrumentaldictation.}\sphinxbfcode{InstrumentalDictationQuestion}}{\emph{mode='major'}, \emph{wait\_time=11}, \emph{n\_repeats=1}, \emph{max\_intervals=3}, \emph{n\_notes=4}, \emph{tonic=None}, \emph{octave=None}, \emph{descending=None}, \emph{chromatic=None}, \emph{n\_octaves=None}, \emph{valid\_intervals=None}, \emph{user\_durations=None}, \emph{prequestion\_method='progression\_i\_iv\_v\_i'}, \emph{resolution\_method='repeat\_only'}, \emph{*args}, \emph{**kwargs}}{}
Bases: {\hyperref[\detokenize{index:birdears.questionbase.QuestionBase}]{\sphinxcrossref{\sphinxcode{birdears.questionbase.QuestionBase}}}}

Implements an instrumental dictation test.
\index{\_\_init\_\_() (birdears.questions.instrumentaldictation.InstrumentalDictationQuestion method)}

\begin{fulllineitems}
\phantomsection\label{\detokenize{index:birdears.questions.instrumentaldictation.InstrumentalDictationQuestion.__init__}}\pysiglinewithargsret{\sphinxbfcode{\_\_init\_\_}}{\emph{mode='major'}, \emph{wait\_time=11}, \emph{n\_repeats=1}, \emph{max\_intervals=3}, \emph{n\_notes=4}, \emph{tonic=None}, \emph{octave=None}, \emph{descending=None}, \emph{chromatic=None}, \emph{n\_octaves=None}, \emph{valid\_intervals=None}, \emph{user\_durations=None}, \emph{prequestion\_method='progression\_i\_iv\_v\_i'}, \emph{resolution\_method='repeat\_only'}, \emph{*args}, \emph{**kwargs}}{}
Inits the class.
\begin{quote}\begin{description}
\item[{Parameters}] \leavevmode\begin{itemize}
\item {} 
\sphinxstyleliteralstrong{mode} (\sphinxstyleliteralemphasis{str}) \textendash{} A string representing the mode of the question.
Eg., ‘major’ or ‘minor’.

\item {} 
\sphinxstyleliteralstrong{wait\_time} (\sphinxstyleliteralemphasis{float}) \textendash{} Wait time in seconds for the next question or
repeat.

\item {} 
\sphinxstyleliteralstrong{n\_repeats} (\sphinxstyleliteralemphasis{int}) \textendash{} Number of times the same dictation will be
repeated before the end of the exercise.

\item {} 
\sphinxstyleliteralstrong{max\_intervals} (\sphinxstyleliteralemphasis{int}) \textendash{} The maximum number of random intervals the
question will have.

\item {} 
\sphinxstyleliteralstrong{n\_notes} (\sphinxstyleliteralemphasis{int}) \textendash{} The number of notes the melodic dictation will have.

\item {} 
\sphinxstyleliteralstrong{tonic} (\sphinxstyleliteralemphasis{str}) \textendash{} A string representing the tonic of the question,
eg.: ‘C’; if omitted, it will be selected randomly.

\item {} 
\sphinxstyleliteralstrong{octave} (\sphinxstyleliteralemphasis{int}) \textendash{} A scienfic octave notation, for example, 4 for ‘C4’;
if not present, it will be randomly chosen.

\item {} 
\sphinxstyleliteralstrong{descending} (\sphinxstyleliteralemphasis{bool}) \textendash{} Is the question direction in descending, ie.,
intervals have lower pitch than the tonic.

\item {} 
\sphinxstyleliteralstrong{chromatic} (\sphinxstyleliteralemphasis{bool}) \textendash{} If the question can have (True) or not (False)
chromatic intervals, ie., intervals not in the diatonic scale
of tonic/mode.

\item {} 
\sphinxstyleliteralstrong{n\_octaves} (\sphinxstyleliteralemphasis{int}) \textendash{} Maximum number of octaves of the question.

\item {} 
\sphinxstyleliteralstrong{valid\_intervals} (\sphinxstyleliteralemphasis{list}) \textendash{} A list with intervals (int) valid for
random choice, 1 is 1st, 2 is second etc. Eg. {[}1, 4, 5{]} to
allow only tonics, fourths and fifths.

\item {} 
\sphinxstyleliteralstrong{user\_durations} (\sphinxstyleliteralemphasis{str}) \textendash{} 
A string with 9 comma-separated \sphinxtitleref{int} or
\sphinxtitleref{float{}`s to set the default duration for the notes played. The
values are respectively for: pre-question duration (1st),
pre-question delay (2nd), and pre-question pos-delay (3rd);
question duration (4th), question delay (5th), and question
pos-delay (6th); resolution duration (7th), resolution
delay (8th), and resolution pos-delay (9th).
duration is the duration in of the note in seconds; delay is
the time to wait before playing the next note, and pos\_delay is
the time to wait after all the notes of the respective sequence
have been played. If any of the user durations is {}`n}, the
default duration for the type of question will be used instead.
Example:

\begin{sphinxVerbatim}[commandchars=\\\{\}]
\PYGZdq{}2,0.5,1,2,n,0,2.5,n,1\PYGZdq{}
\end{sphinxVerbatim}


\item {} 
\sphinxstyleliteralstrong{prequestion\_method} (\sphinxstyleliteralemphasis{str}) \textendash{} Method of playing a cadence or the
exercise tonic before the question so to affirm the question
musical tonic key to the ear. Valid ones are registered in the
\sphinxtitleref{birdears.prequestion.PREQUESION\_METHODS} global dict.

\item {} 
\sphinxstyleliteralstrong{resolution\_method} (\sphinxstyleliteralemphasis{str}) \textendash{} Method of playing the resolution of an
exercise. Valid ones are registered in the
\sphinxtitleref{birdears.resolution.RESOLUTION\_METHODS} global dict.

\end{itemize}

\end{description}\end{quote}

\end{fulllineitems}

\index{check\_question() (birdears.questions.instrumentaldictation.InstrumentalDictationQuestion method)}

\begin{fulllineitems}
\phantomsection\label{\detokenize{index:birdears.questions.instrumentaldictation.InstrumentalDictationQuestion.check_question}}\pysiglinewithargsret{\sphinxbfcode{check\_question}}{}{}
Checks whether the given answer is correct.

This currently doesn’t applies to instrumental dictation questions.

\end{fulllineitems}

\index{make\_pre\_question() (birdears.questions.instrumentaldictation.InstrumentalDictationQuestion method)}

\begin{fulllineitems}
\phantomsection\label{\detokenize{index:birdears.questions.instrumentaldictation.InstrumentalDictationQuestion.make_pre_question}}\pysiglinewithargsret{\sphinxbfcode{make\_pre\_question}}{\emph{method}}{}
\end{fulllineitems}

\index{make\_question() (birdears.questions.instrumentaldictation.InstrumentalDictationQuestion method)}

\begin{fulllineitems}
\phantomsection\label{\detokenize{index:birdears.questions.instrumentaldictation.InstrumentalDictationQuestion.make_question}}\pysiglinewithargsret{\sphinxbfcode{make\_question}}{\emph{phrase\_semitones}}{}
\end{fulllineitems}

\index{make\_resolution() (birdears.questions.instrumentaldictation.InstrumentalDictationQuestion method)}

\begin{fulllineitems}
\phantomsection\label{\detokenize{index:birdears.questions.instrumentaldictation.InstrumentalDictationQuestion.make_resolution}}\pysiglinewithargsret{\sphinxbfcode{make\_resolution}}{\emph{method}}{}
\end{fulllineitems}

\index{play\_question() (birdears.questions.instrumentaldictation.InstrumentalDictationQuestion method)}

\begin{fulllineitems}
\phantomsection\label{\detokenize{index:birdears.questions.instrumentaldictation.InstrumentalDictationQuestion.play_question}}\pysiglinewithargsret{\sphinxbfcode{play\_question}}{}{}
\end{fulllineitems}


\end{fulllineitems}



\section{birdears.questions.melodicdictation module}
\label{\detokenize{index:birdears-questions-melodicdictation-module}}\label{\detokenize{index:module-birdears.questions.melodicdictation}}\index{birdears.questions.melodicdictation (module)}\index{MelodicDictationQuestion (class in birdears.questions.melodicdictation)}

\begin{fulllineitems}
\phantomsection\label{\detokenize{index:birdears.questions.melodicdictation.MelodicDictationQuestion}}\pysiglinewithargsret{\sphinxbfcode{class }\sphinxcode{birdears.questions.melodicdictation.}\sphinxbfcode{MelodicDictationQuestion}}{\emph{mode='major'}, \emph{max\_intervals=3}, \emph{n\_notes=4}, \emph{tonic=None}, \emph{octave=None}, \emph{descending=None}, \emph{chromatic=None}, \emph{n\_octaves=None}, \emph{valid\_intervals=None}, \emph{user\_durations=None}, \emph{prequestion\_method='progression\_i\_iv\_v\_i'}, \emph{resolution\_method='repeat\_only'}, \emph{*args}, \emph{**kwargs}}{}
Bases: {\hyperref[\detokenize{index:birdears.questionbase.QuestionBase}]{\sphinxcrossref{\sphinxcode{birdears.questionbase.QuestionBase}}}}

Implements a melodic dictation test.
\index{\_\_init\_\_() (birdears.questions.melodicdictation.MelodicDictationQuestion method)}

\begin{fulllineitems}
\phantomsection\label{\detokenize{index:birdears.questions.melodicdictation.MelodicDictationQuestion.__init__}}\pysiglinewithargsret{\sphinxbfcode{\_\_init\_\_}}{\emph{mode='major'}, \emph{max\_intervals=3}, \emph{n\_notes=4}, \emph{tonic=None}, \emph{octave=None}, \emph{descending=None}, \emph{chromatic=None}, \emph{n\_octaves=None}, \emph{valid\_intervals=None}, \emph{user\_durations=None}, \emph{prequestion\_method='progression\_i\_iv\_v\_i'}, \emph{resolution\_method='repeat\_only'}, \emph{*args}, \emph{**kwargs}}{}
Inits the class.
\begin{quote}\begin{description}
\item[{Parameters}] \leavevmode\begin{itemize}
\item {} 
\sphinxstyleliteralstrong{mode} (\sphinxstyleliteralemphasis{str}) \textendash{} A string representing the mode of the question.
Eg., ‘major’ or ‘minor’.

\item {} 
\sphinxstyleliteralstrong{max\_intervals} (\sphinxstyleliteralemphasis{int}) \textendash{} The maximum number of random intervals
the question will have.

\item {} 
\sphinxstyleliteralstrong{n\_notes} (\sphinxstyleliteralemphasis{int}) \textendash{} The number of notes the melodic dictation will have.

\item {} 
\sphinxstyleliteralstrong{tonic} (\sphinxstyleliteralemphasis{str}) \textendash{} A string representing the tonic of the question,
eg.: ‘C’; if omitted, it will be selected randomly.

\item {} 
\sphinxstyleliteralstrong{octave} (\sphinxstyleliteralemphasis{int}) \textendash{} A scienfic octave notation, for example, 4 for ‘C4’;
if not present, it will be randomly chosen.

\item {} 
\sphinxstyleliteralstrong{descending} (\sphinxstyleliteralemphasis{bool}) \textendash{} Is the question direction in descending, ie.,
intervals have lower pitch than the tonic.

\item {} 
\sphinxstyleliteralstrong{chromatic} (\sphinxstyleliteralemphasis{bool}) \textendash{} If the question can have (True) or not (False)
chromatic intervals, ie., intervals not in the diatonic scale
of tonic/mode.

\item {} 
\sphinxstyleliteralstrong{n\_octaves} (\sphinxstyleliteralemphasis{int}) \textendash{} Maximum number of octaves of the question.

\item {} 
\sphinxstyleliteralstrong{valid\_intervals} (\sphinxstyleliteralemphasis{list}) \textendash{} A list with intervals (int) valid for
random choice, 1 is 1st, 2 is second etc. Eg. {[}1, 4, 5{]} to
allow only tonics, fourths and fifths.

\item {} 
\sphinxstyleliteralstrong{user\_durations} (\sphinxstyleliteralemphasis{str}) \textendash{} 
A string with 9 comma-separated \sphinxtitleref{int} or
\sphinxtitleref{float{}`s to set the default duration for the notes played. The
values are respectively for: pre-question duration (1st),
pre-question delay (2nd), and pre-question pos-delay (3rd);
question duration (4th), question delay (5th), and question
pos-delay (6th); resolution duration (7th), resolution
delay (8th), and resolution pos-delay (9th).
duration is the duration in of the note in seconds; delay is
the time to wait before playing the next note, and pos\_delay is
the time to wait after all the notes of the respective sequence
have been played. If any of the user durations is {}`n}, the
default duration for the type of question will be used instead.
Example:

\begin{sphinxVerbatim}[commandchars=\\\{\}]
\PYGZdq{}2,0.5,1,2,n,0,2.5,n,1\PYGZdq{}
\end{sphinxVerbatim}


\item {} 
\sphinxstyleliteralstrong{prequestion\_method} (\sphinxstyleliteralemphasis{str}) \textendash{} Method of playing a cadence or the
exercise tonic before the question so to affirm the question
musical tonic key to the ear. Valid ones are registered in the
\sphinxtitleref{birdears.prequestion.PREQUESION\_METHODS} global dict.

\item {} 
\sphinxstyleliteralstrong{resolution\_method} (\sphinxstyleliteralemphasis{str}) \textendash{} Method of playing the resolution of an
exercise. Valid ones are registered in the
\sphinxtitleref{birdears.resolution.RESOLUTION\_METHODS} global dict.

\end{itemize}

\end{description}\end{quote}

\end{fulllineitems}

\index{check\_question() (birdears.questions.melodicdictation.MelodicDictationQuestion method)}

\begin{fulllineitems}
\phantomsection\label{\detokenize{index:birdears.questions.melodicdictation.MelodicDictationQuestion.check_question}}\pysiglinewithargsret{\sphinxbfcode{check\_question}}{\emph{user\_input\_keys}}{}
Checks whether the given answer is correct.

\end{fulllineitems}

\index{make\_pre\_question() (birdears.questions.melodicdictation.MelodicDictationQuestion method)}

\begin{fulllineitems}
\phantomsection\label{\detokenize{index:birdears.questions.melodicdictation.MelodicDictationQuestion.make_pre_question}}\pysiglinewithargsret{\sphinxbfcode{make\_pre\_question}}{\emph{method}}{}
\end{fulllineitems}

\index{make\_question() (birdears.questions.melodicdictation.MelodicDictationQuestion method)}

\begin{fulllineitems}
\phantomsection\label{\detokenize{index:birdears.questions.melodicdictation.MelodicDictationQuestion.make_question}}\pysiglinewithargsret{\sphinxbfcode{make\_question}}{\emph{phrase\_semitones}}{}
\end{fulllineitems}

\index{make\_resolution() (birdears.questions.melodicdictation.MelodicDictationQuestion method)}

\begin{fulllineitems}
\phantomsection\label{\detokenize{index:birdears.questions.melodicdictation.MelodicDictationQuestion.make_resolution}}\pysiglinewithargsret{\sphinxbfcode{make\_resolution}}{\emph{method}}{}
\end{fulllineitems}

\index{play\_question() (birdears.questions.melodicdictation.MelodicDictationQuestion method)}

\begin{fulllineitems}
\phantomsection\label{\detokenize{index:birdears.questions.melodicdictation.MelodicDictationQuestion.play_question}}\pysiglinewithargsret{\sphinxbfcode{play\_question}}{}{}
\end{fulllineitems}

\index{play\_resolution() (birdears.questions.melodicdictation.MelodicDictationQuestion method)}

\begin{fulllineitems}
\phantomsection\label{\detokenize{index:birdears.questions.melodicdictation.MelodicDictationQuestion.play_resolution}}\pysiglinewithargsret{\sphinxbfcode{play\_resolution}}{}{}
\end{fulllineitems}


\end{fulllineitems}



\section{birdears.questions.melodicinterval module}
\label{\detokenize{index:birdears-questions-melodicinterval-module}}\label{\detokenize{index:module-birdears.questions.melodicinterval}}\index{birdears.questions.melodicinterval (module)}\index{MelodicIntervalQuestion (class in birdears.questions.melodicinterval)}

\begin{fulllineitems}
\phantomsection\label{\detokenize{index:birdears.questions.melodicinterval.MelodicIntervalQuestion}}\pysiglinewithargsret{\sphinxbfcode{class }\sphinxcode{birdears.questions.melodicinterval.}\sphinxbfcode{MelodicIntervalQuestion}}{\emph{mode='major'}, \emph{tonic=None}, \emph{octave=None}, \emph{descending=None}, \emph{chromatic=None}, \emph{n\_octaves=None}, \emph{valid\_intervals=None}, \emph{user\_durations=None}, \emph{prequestion\_method='tonic\_only'}, \emph{resolution\_method='nearest\_tonic'}, \emph{*args}, \emph{**kwargs}}{}
Bases: {\hyperref[\detokenize{index:birdears.questionbase.QuestionBase}]{\sphinxcrossref{\sphinxcode{birdears.questionbase.QuestionBase}}}}

Implements a Melodic Interval test.
\index{check\_question() (birdears.questions.melodicinterval.MelodicIntervalQuestion method)}

\begin{fulllineitems}
\phantomsection\label{\detokenize{index:birdears.questions.melodicinterval.MelodicIntervalQuestion.check_question}}\pysiglinewithargsret{\sphinxbfcode{check\_question}}{\emph{user\_input\_char}}{}
Checks whether the given answer is correct.

\end{fulllineitems}

\index{make\_pre\_question() (birdears.questions.melodicinterval.MelodicIntervalQuestion method)}

\begin{fulllineitems}
\phantomsection\label{\detokenize{index:birdears.questions.melodicinterval.MelodicIntervalQuestion.make_pre_question}}\pysiglinewithargsret{\sphinxbfcode{make\_pre\_question}}{\emph{method}}{}
\end{fulllineitems}

\index{make\_question() (birdears.questions.melodicinterval.MelodicIntervalQuestion method)}

\begin{fulllineitems}
\phantomsection\label{\detokenize{index:birdears.questions.melodicinterval.MelodicIntervalQuestion.make_question}}\pysiglinewithargsret{\sphinxbfcode{make\_question}}{}{}
\end{fulllineitems}

\index{make\_resolution() (birdears.questions.melodicinterval.MelodicIntervalQuestion method)}

\begin{fulllineitems}
\phantomsection\label{\detokenize{index:birdears.questions.melodicinterval.MelodicIntervalQuestion.make_resolution}}\pysiglinewithargsret{\sphinxbfcode{make\_resolution}}{\emph{method}}{}
\end{fulllineitems}

\index{play\_question() (birdears.questions.melodicinterval.MelodicIntervalQuestion method)}

\begin{fulllineitems}
\phantomsection\label{\detokenize{index:birdears.questions.melodicinterval.MelodicIntervalQuestion.play_question}}\pysiglinewithargsret{\sphinxbfcode{play\_question}}{}{}
\end{fulllineitems}

\index{play\_resolution() (birdears.questions.melodicinterval.MelodicIntervalQuestion method)}

\begin{fulllineitems}
\phantomsection\label{\detokenize{index:birdears.questions.melodicinterval.MelodicIntervalQuestion.play_resolution}}\pysiglinewithargsret{\sphinxbfcode{play\_resolution}}{}{}
\end{fulllineitems}


\end{fulllineitems}



\chapter{birdears.interfaces package}
\label{\detokenize{index:birdears-interfaces-package}}\label{\detokenize{index:module-birdears.interfaces}}\index{birdears.interfaces (module)}

\section{Submodules}
\label{\detokenize{index:id13}}

\section{birdears.interfaces.commandline module}
\label{\detokenize{index:birdears-interfaces-commandline-module}}\label{\detokenize{index:module-birdears.interfaces.commandline}}\index{birdears.interfaces.commandline (module)}\index{CommandLine() (in module birdears.interfaces.commandline)}

\begin{fulllineitems}
\phantomsection\label{\detokenize{index:birdears.interfaces.commandline.CommandLine}}\pysiglinewithargsret{\sphinxcode{birdears.interfaces.commandline.}\sphinxbfcode{CommandLine}}{\emph{exercise}, \emph{**kwargs}}{}
This function implements the birdears loop for command line.
\begin{quote}\begin{description}
\item[{Parameters}] \leavevmode\begin{itemize}
\item {} 
\sphinxstyleliteralstrong{exercise} (\sphinxstyleliteralemphasis{str}) \textendash{} The question name.

\item {} 
\sphinxstyleliteralstrong{**kwargs} (\sphinxstyleliteralemphasis{kwargs}) \textendash{} FIXME: The kwargs can contain options for specific
questions.

\end{itemize}

\end{description}\end{quote}

\end{fulllineitems}

\index{center\_text() (in module birdears.interfaces.commandline)}

\begin{fulllineitems}
\phantomsection\label{\detokenize{index:birdears.interfaces.commandline.center_text}}\pysiglinewithargsret{\sphinxcode{birdears.interfaces.commandline.}\sphinxbfcode{center\_text}}{\emph{text}, \emph{sep=True}, \emph{nl=0}}{}
This function returns input text centered according to terminal columns.
\begin{quote}\begin{description}
\item[{Parameters}] \leavevmode\begin{itemize}
\item {} 
\sphinxstyleliteralstrong{text} (\sphinxstyleliteralemphasis{str}) \textendash{} The string to be centered, it can have multiple lines.

\item {} 
\sphinxstyleliteralstrong{sep} (\sphinxstyleliteralemphasis{bool}) \textendash{} Add line separator after centered text (True) or
not (False).

\item {} 
\sphinxstyleliteralstrong{nl} (\sphinxstyleliteralemphasis{int}) \textendash{} How many new lines to add after text.

\end{itemize}

\end{description}\end{quote}

\end{fulllineitems}

\index{make\_input\_str() (in module birdears.interfaces.commandline)}

\begin{fulllineitems}
\phantomsection\label{\detokenize{index:birdears.interfaces.commandline.make_input_str}}\pysiglinewithargsret{\sphinxcode{birdears.interfaces.commandline.}\sphinxbfcode{make\_input\_str}}{\emph{user\_input}, \emph{keyboard\_index}}{}
Makes a string representing intervals entered by the user.

This function is to be used by questions which takes more than one interval
input as MelodicDictation, and formats the intervals already entered.
\begin{quote}\begin{description}
\item[{Parameters}] \leavevmode\begin{itemize}
\item {} 
\sphinxstyleliteralstrong{user\_input} (\sphinxstyleliteralemphasis{array\_type}) \textendash{} The list of keyboard keys entered by user.

\item {} 
\sphinxstyleliteralstrong{keyboard\_index} (\sphinxstyleliteralemphasis{array\_type}) \textendash{} The keyboard mapping used by question.

\end{itemize}

\end{description}\end{quote}

\end{fulllineitems}

\index{print\_instrumental() (in module birdears.interfaces.commandline)}

\begin{fulllineitems}
\phantomsection\label{\detokenize{index:birdears.interfaces.commandline.print_instrumental}}\pysiglinewithargsret{\sphinxcode{birdears.interfaces.commandline.}\sphinxbfcode{print\_instrumental}}{\emph{response}}{}
Prints the formatted response for ‘instrumental’ exercise.
\begin{quote}\begin{description}
\item[{Parameters}] \leavevmode
\sphinxstyleliteralstrong{response} (\sphinxstyleliteralemphasis{dict}) \textendash{} A response returned by question’s check\_question()

\end{description}\end{quote}

\end{fulllineitems}

\index{print\_question() (in module birdears.interfaces.commandline)}

\begin{fulllineitems}
\phantomsection\label{\detokenize{index:birdears.interfaces.commandline.print_question}}\pysiglinewithargsret{\sphinxcode{birdears.interfaces.commandline.}\sphinxbfcode{print\_question}}{\emph{question}}{}
Prints the question to the user.
\begin{quote}\begin{description}
\item[{Parameters}] \leavevmode
\sphinxstyleliteralstrong{question} (\sphinxstyleliteralemphasis{obj}) \textendash{} A Question class with the question to be printed.

\end{description}\end{quote}

\end{fulllineitems}

\index{print\_response() (in module birdears.interfaces.commandline)}

\begin{fulllineitems}
\phantomsection\label{\detokenize{index:birdears.interfaces.commandline.print_response}}\pysiglinewithargsret{\sphinxcode{birdears.interfaces.commandline.}\sphinxbfcode{print\_response}}{\emph{response}}{}
Prints the formatted response.
\begin{quote}\begin{description}
\item[{Parameters}] \leavevmode
\sphinxstyleliteralstrong{response} (\sphinxstyleliteralemphasis{dict}) \textendash{} A response returned by question’s check\_question()

\end{description}\end{quote}

\end{fulllineitems}



\renewcommand{\indexname}{Python Module Index}
\begin{sphinxtheindex}
\def\bigletter#1{{\Large\sffamily#1}\nopagebreak\vspace{1mm}}
\bigletter{b}
\item {\sphinxstyleindexentry{birdears}}\sphinxstyleindexpageref{index:\detokenize{module-birdears}}
\item {\sphinxstyleindexentry{birdears.interfaces}}\sphinxstyleindexpageref{index:\detokenize{module-birdears.interfaces}}
\item {\sphinxstyleindexentry{birdears.interfaces.commandline}}\sphinxstyleindexpageref{index:\detokenize{module-birdears.interfaces.commandline}}
\item {\sphinxstyleindexentry{birdears.interval}}\sphinxstyleindexpageref{index:\detokenize{module-birdears.interval}}
\item {\sphinxstyleindexentry{birdears.logger}}\sphinxstyleindexpageref{index:\detokenize{module-birdears.logger}}
\item {\sphinxstyleindexentry{birdears.prequestion}}\sphinxstyleindexpageref{index:\detokenize{module-birdears.prequestion}}
\item {\sphinxstyleindexentry{birdears.questionbase}}\sphinxstyleindexpageref{index:\detokenize{module-birdears.questionbase}}
\item {\sphinxstyleindexentry{birdears.questions}}\sphinxstyleindexpageref{index:\detokenize{module-birdears.questions}}
\item {\sphinxstyleindexentry{birdears.questions.harmonicinterval}}\sphinxstyleindexpageref{index:\detokenize{module-birdears.questions.harmonicinterval}}
\item {\sphinxstyleindexentry{birdears.questions.instrumentaldictation}}\sphinxstyleindexpageref{index:\detokenize{module-birdears.questions.instrumentaldictation}}
\item {\sphinxstyleindexentry{birdears.questions.melodicdictation}}\sphinxstyleindexpageref{index:\detokenize{module-birdears.questions.melodicdictation}}
\item {\sphinxstyleindexentry{birdears.questions.melodicinterval}}\sphinxstyleindexpageref{index:\detokenize{module-birdears.questions.melodicinterval}}
\item {\sphinxstyleindexentry{birdears.resolution}}\sphinxstyleindexpageref{index:\detokenize{module-birdears.resolution}}
\item {\sphinxstyleindexentry{birdears.scale}}\sphinxstyleindexpageref{index:\detokenize{module-birdears.scale}}
\item {\sphinxstyleindexentry{birdears.sequence}}\sphinxstyleindexpageref{index:\detokenize{module-birdears.sequence}}
\end{sphinxtheindex}

\renewcommand{\indexname}{Index}
\printindex
\end{document}