%% Generated by Sphinx.
\def\sphinxdocclass{report}
\documentclass[letterpaper,10pt,english]{sphinxmanual}
\ifdefined\pdfpxdimen
   \let\sphinxpxdimen\pdfpxdimen\else\newdimen\sphinxpxdimen
\fi \sphinxpxdimen=.75bp\relax
\ifdefined\pdfimageresolution
    \pdfimageresolution= \numexpr \dimexpr1in\relax/\sphinxpxdimen\relax
\fi
%% let collapsible pdf bookmarks panel have high depth per default
\PassOptionsToPackage{bookmarksdepth=5}{hyperref}

\PassOptionsToPackage{warn}{textcomp}
\usepackage[utf8]{inputenc}
\ifdefined\DeclareUnicodeCharacter
% support both utf8 and utf8x syntaxes
  \ifdefined\DeclareUnicodeCharacterAsOptional
    \def\sphinxDUC#1{\DeclareUnicodeCharacter{"#1}}
  \else
    \let\sphinxDUC\DeclareUnicodeCharacter
  \fi
  \sphinxDUC{00A0}{\nobreakspace}
  \sphinxDUC{2500}{\sphinxunichar{2500}}
  \sphinxDUC{2502}{\sphinxunichar{2502}}
  \sphinxDUC{2514}{\sphinxunichar{2514}}
  \sphinxDUC{251C}{\sphinxunichar{251C}}
  \sphinxDUC{2572}{\textbackslash}
\fi
\usepackage{cmap}
\usepackage[T1]{fontenc}
\usepackage{amsmath,amssymb,amstext}
\usepackage{babel}



\usepackage{tgtermes}
\usepackage{tgheros}
\renewcommand{\ttdefault}{txtt}



\usepackage[Bjarne]{fncychap}
\usepackage{sphinx}

\fvset{fontsize=auto}
\usepackage{geometry}


% Include hyperref last.
\usepackage{hyperref}
% Fix anchor placement for figures with captions.
\usepackage{hypcap}% it must be loaded after hyperref.
% Set up styles of URL: it should be placed after hyperref.
\urlstyle{same}


\usepackage{sphinxmessages}
\setcounter{tocdepth}{2}



\title{birdears Documentation}
\date{Nov 02, 2021}
\release{0.2.1}
\author{Iacchus Mercurius}
\newcommand{\sphinxlogo}{\vbox{}}
\renewcommand{\releasename}{Release}
\makeindex
\begin{document}

\pagestyle{empty}
\sphinxmaketitle
\pagestyle{plain}
\sphinxtableofcontents
\pagestyle{normal}
\phantomsection\label{\detokenize{index::doc}}


\sphinxAtStartPar
Welcome to birdears documentation.

\sphinxAtStartPar
\sphinxcode{\sphinxupquote{birdears}} is a software written in Python 3 for ear training for
musicians (musical intelligence, transcribing music, composing). It is a
clone of the method used by \sphinxhref{https://play.google.com/store/apps/details?id=com.kaizen9.fet.android}{Funcitional Ear
Trainer}
app for Android.

\sphinxAtStartPar
It comes with four modes, or four kind of exercises, which are:
\sphinxcode{\sphinxupquote{melodic}}, \sphinxcode{\sphinxupquote{harmonic}}, \sphinxcode{\sphinxupquote{dictation}} and \sphinxcode{\sphinxupquote{instrumental}}.

\sphinxAtStartPar
In resume, with the \sphinxstyleemphasis{melodic} mode two notes are played one after the
other and you have to guess the interval; with the \sphinxstyleemphasis{harmonic} mode,
two notes are played simoutaneously (harmonically) and you should guess
the interval.

\sphinxAtStartPar
With the \sphinxstyleemphasis{dictation} mode, more than 2 notes are played (\sphinxstyleemphasis{ie}., a
melodic dictation) and you should tell what are the intervals between
them.

\sphinxAtStartPar
With the \sphinxstyleemphasis{instrumental} mode, it is a like the \sphinxstyleemphasis{dictation}, but you will
be expected to play the notes on your instrument, \sphinxstyleemphasis{ie}., birdears will
not wait for a typed reply and you should prectice with your own
judgement. The melody can be repeat any times and you can have as much
time as you want to try it out.

\sphinxAtStartPar
Project at \sphinxhref{https://github.com/iacchus/birdears}{GitHub}.

\sphinxAtStartPar
Download the PDF version of this book. Clicking \sphinxhref{https://github.com/iacchus/birdears/raw/master/docs/sphinx/\_build/latex/birdears.pdf}{here}.




\chapter{Support}
\label{\detokenize{community:support}}\label{\detokenize{community::doc}}
\sphinxAtStartPar
If you need help you can get in touch via IRC or file an issue on any matter regarding birdears at Github.


\begin{savenotes}\sphinxattablestart
\centering
\begin{tabulary}{\linewidth}[t]{|T|T|}
\hline
\sphinxstyletheadfamily 
\sphinxAtStartPar
Media
&\sphinxstyletheadfamily 
\sphinxAtStartPar
Channel
\\
\hline
\sphinxAtStartPar
IRC
&
\sphinxAtStartPar
\sphinxhref{https://webchat.freenode.net/?randomnick=1\&channels=\%23birdears\&uio=MTY9dHJ1ZSYxMT0yNDY57}{\#birdears} at irc.freenode.org/6697 \sphinxhyphen{}ssl
\\
\hline
\sphinxAtStartPar
GitHub
&
\sphinxAtStartPar
\sphinxurl{https://github.com/iacchus/birdears}
\\
\hline
\sphinxAtStartPar
GH issues
&
\sphinxAtStartPar
\sphinxurl{https://github.com/iacchus/birdears/issues}
\\
\hline
\sphinxAtStartPar
ReadTheDocs
&
\sphinxAtStartPar
\sphinxurl{https://birdears.readthedocs.io}
\\
\hline
\sphinxAtStartPar
PyPI
&
\sphinxAtStartPar
\sphinxurl{https://pypi.python.org/pypi/birdears}
\\
\hline
\sphinxAtStartPar
TravisCI
&
\sphinxAtStartPar
\sphinxurl{https://travis-ci.org/iacchus/birdears}
\\
\hline
\sphinxAtStartPar
Coveralls
&
\sphinxAtStartPar
\sphinxurl{https://coveralls.io/github/iacchus/birdears}
\\
\hline
\end{tabulary}
\par
\sphinxattableend\end{savenotes}


\chapter{Features}
\label{\detokenize{features:features}}\label{\detokenize{features::doc}}\begin{itemize}
\item {} 
\sphinxAtStartPar
questions

\item {} 
\sphinxAtStartPar
pretty much configurable

\item {} 
\sphinxAtStartPar
load from config file

\item {} 
\sphinxAtStartPar
you can make your own presets

\item {} 
\sphinxAtStartPar
can be used interactively \sphinxstyleemphasis{(docs needed)}

\item {} 
\sphinxAtStartPar
can be used as a library \sphinxstyleemphasis{(docs needed)}

\end{itemize}


\chapter{Installing birdears}
\label{\detokenize{installing:installing-birdears}}\label{\detokenize{installing::doc}}

\section{Installing the dependencies}
\label{\detokenize{installing:installing-the-dependencies}}

\subsection{Arch Linux}
\label{\detokenize{installing:arch-linux}}
\begin{sphinxVerbatim}[commandchars=\\\{\}]
sudo pacman \PYGZhy{}Syu sox python python\PYGZhy{}pip
\end{sphinxVerbatim}


\section{Installing birdears}
\label{\detokenize{installing:id1}}
\sphinxAtStartPar
To install,simple do this command with pip3

\begin{sphinxVerbatim}[commandchars=\\\{\}]
pip3 install \PYGZhy{}\PYGZhy{}user \PYGZhy{}\PYGZhy{}upgrade \PYGZhy{}\PYGZhy{}no\PYGZhy{}cache\PYGZhy{}dir birdears
\end{sphinxVerbatim}


\subsection{In\sphinxhyphen{}depth installation}
\label{\detokenize{installing:in-depth-installation}}
\sphinxAtStartPar
You can choose to use a virtualenv to use birdears; this should give you
an idea on how to setup one virtualenv.

\sphinxAtStartPar
You should first install virtualenv (for python3) using your
distribution’s package (supposing you’re on linux), then issue on terminal:

\begin{sphinxVerbatim}[commandchars=\\\{\}]
virtualenv \PYGZhy{}p python3 \PYGZti{}/.venv \PYGZsh{} use the directory \PYGZti{}/.venv/ for the virtualenv

source \PYGZti{}/.venv/bin/activate   \PYGZsh{} activate the virtualenv; this should be done
                              \PYGZsh{} every time you may want to run the software
                              \PYGZsh{} installed here.

pip3 install birdears         \PYGZsh{} this will install the software

birdears \PYGZhy{}\PYGZhy{}help               \PYGZsh{} and this will run it
\end{sphinxVerbatim}


\chapter{Using birdears}
\label{\detokenize{using:using-birdears}}\label{\detokenize{using::doc}}

\section{What is Functional Ear Training}
\label{\detokenize{using:what-is-functional-ear-training}}
\sphinxAtStartPar
\sphinxstyleemphasis{write me!}


\section{The method}
\label{\detokenize{using:the-method}}
\sphinxAtStartPar
We can use abc language to notate music within the documentation, ok

\begin{sphinxVerbatim}[commandchars=\\\{\}]
X: 1
T: Banish Misfortune
R: jig
M: 6/8
L: 1/8
K: Dmix
fed cAG| A2d cAG| F2D DED| FEF GFG|
AGA cAG| AGA cde|fed cAG| Ad\PYGZca{}c d3:|
f2d d\PYGZca{}cd| f2g agf| e2c cBc|e2f gfe|
f2g agf| e2f gfe|fed cAG|Ad\PYGZca{}c d3:|
f2g e2f| d2e c2d|ABA GAG| F2F GED|
c3 cAG| AGA cde| fed cAG| Ad\PYGZca{}c d3:|
\end{sphinxVerbatim}


\section{birdears modes and basic usage}
\label{\detokenize{using:birdears-modes-and-basic-usage}}
\sphinxAtStartPar
birdears actually has four modes:
\begin{itemize}
\item {} 
\sphinxAtStartPar
melodic interval question

\item {} 
\sphinxAtStartPar
harmonic interval question

\item {} 
\sphinxAtStartPar
melodic dictation question

\item {} 
\sphinxAtStartPar
instrumental dictation question

\end{itemize}

\sphinxAtStartPar
To see the commands avaliable just invoke the command without any arguments:

\begin{sphinxVerbatim}[commandchars=\\\{\}]
birdears
\end{sphinxVerbatim}

\begin{sphinxVerbatim}[commandchars=\\\{\}]
Usage: birdears  \PYGZlt{}command\PYGZgt{} [options]

  birdears ─ Functional Ear Training for Musicians!

Options:
  \PYGZhy{}\PYGZhy{}debug / \PYGZhy{}\PYGZhy{}no\PYGZhy{}debug  Turns on debugging; instead you can set DEBUG=1.
  \PYGZhy{}h, \PYGZhy{}\PYGZhy{}help            Show this message and exit.

Commands:
  dictation     Melodic dictation
  harmonic      Harmonic interval recognition
  instrumental  Instrumental melodic time\PYGZhy{}based dictation
  load          Loads exercise from .toml config file...
  melodic       Melodic interval recognition

  You can use \PYGZsq{}birdears \PYGZlt{}command\PYGZgt{} \PYGZhy{}\PYGZhy{}help\PYGZsq{} to show options for a specific
  command.

  More info at https://github.com/iacchus/birdears
\end{sphinxVerbatim}

\begin{sphinxVerbatim}[commandchars=\\\{\}]
birdears \PYGZlt{}command\PYGZgt{} \PYGZhy{}\PYGZhy{}help
\end{sphinxVerbatim}


\subsection{melodic}
\label{\detokenize{using:melodic}}
\sphinxAtStartPar
In this exercise birdears will play two notes, the tonic and the interval
melodically, ie., one after the other and you should reply which is the
correct distance between the two.

\begin{sphinxVerbatim}[commandchars=\\\{\}]
birdears melodic \PYGZhy{}\PYGZhy{}help
\end{sphinxVerbatim}

\begin{sphinxVerbatim}[commandchars=\\\{\}]
Usage: birdears melodic [options]

  Melodic interval recognition

Options:
  \PYGZhy{}m, \PYGZhy{}\PYGZhy{}mode \PYGZlt{}mode\PYGZgt{}               Mode of the question.
  \PYGZhy{}t, \PYGZhy{}\PYGZhy{}tonic \PYGZlt{}tonic\PYGZgt{}             Tonic of the question.
  \PYGZhy{}o, \PYGZhy{}\PYGZhy{}octave \PYGZlt{}octave\PYGZgt{}           Octave of the question.
  \PYGZhy{}d, \PYGZhy{}\PYGZhy{}descending                Whether the question interval is descending.
  \PYGZhy{}c, \PYGZhy{}\PYGZhy{}chromatic                 If chosen, question has chromatic notes.
  \PYGZhy{}n, \PYGZhy{}\PYGZhy{}n\PYGZus{}octaves \PYGZlt{}n max\PYGZgt{}         Maximum number of octaves.
  \PYGZhy{}v, \PYGZhy{}\PYGZhy{}valid\PYGZus{}intervals \PYGZlt{}1,2,..\PYGZgt{}  A comma\PYGZhy{}separated list without spaces
                                  of valid scale degrees to be chosen for the
                                  question.
  \PYGZhy{}q, \PYGZhy{}\PYGZhy{}user\PYGZus{}durations \PYGZlt{}1,0.5,n..\PYGZgt{}
                                  A comma\PYGZhy{}separated list without
                                  spaces with PRECISLY 9 floating values. Or
                                  \PYGZsq{}n\PYGZsq{} for default              duration.
  \PYGZhy{}p, \PYGZhy{}\PYGZhy{}prequestion\PYGZus{}method \PYGZlt{}prequestion\PYGZus{}method\PYGZgt{}
                                  The name of a pre\PYGZhy{}question method.
  \PYGZhy{}r, \PYGZhy{}\PYGZhy{}resolution\PYGZus{}method \PYGZlt{}resolution\PYGZus{}method\PYGZgt{}
                                  The name of a resolution method.
  \PYGZhy{}h, \PYGZhy{}\PYGZhy{}help                      Show this message and exit.

  In this exercise birdears will play two notes, the tonic and the interval
  melodically, ie., one after the other and you should reply which is the
  correct distance between the two.

  Valid values are as follows:

  \PYGZhy{}m \PYGZlt{}mode\PYGZgt{} is one of: major, dorian, phrygian, lydian, mixolydian, minor,
  locrian

  \PYGZhy{}t \PYGZlt{}tonic\PYGZgt{} is one of: A, A\PYGZsh{}, Ab, B, Bb, C, C\PYGZsh{}, D, D\PYGZsh{}, Db, E, Eb, F, F\PYGZsh{}, G,
  G\PYGZsh{}, Gb

  \PYGZhy{}p \PYGZlt{}prequestion\PYGZus{}method\PYGZgt{} is one of: none, tonic\PYGZus{}only, progression\PYGZus{}i\PYGZus{}iv\PYGZus{}v\PYGZus{}i

  \PYGZhy{}r \PYGZlt{}resolution\PYGZus{}method\PYGZgt{} is one of: nearest\PYGZus{}tonic, repeat\PYGZus{}only
\end{sphinxVerbatim}


\subsection{harmonic}
\label{\detokenize{using:harmonic}}
\sphinxAtStartPar
In this exercise birdears will play two notes, the tonic and the interval
harmonically, ie., both on the same time and you should reply which is the
correct distance between the two.

\begin{sphinxVerbatim}[commandchars=\\\{\}]
birdears harmonic \PYGZhy{}\PYGZhy{}help
\end{sphinxVerbatim}

\begin{sphinxVerbatim}[commandchars=\\\{\}]
Usage: birdears harmonic [options]

  Harmonic interval recognition

Options:
  \PYGZhy{}m, \PYGZhy{}\PYGZhy{}mode \PYGZlt{}mode\PYGZgt{}               Mode of the question.
  \PYGZhy{}t, \PYGZhy{}\PYGZhy{}tonic \PYGZlt{}note\PYGZgt{}              Tonic of the question.
  \PYGZhy{}o, \PYGZhy{}\PYGZhy{}octave \PYGZlt{}octave\PYGZgt{}           Octave of the question.
  \PYGZhy{}d, \PYGZhy{}\PYGZhy{}descending                Whether the question interval is descending.
  \PYGZhy{}c, \PYGZhy{}\PYGZhy{}chromatic                 If chosen, question has chromatic notes.
  \PYGZhy{}n, \PYGZhy{}\PYGZhy{}n\PYGZus{}octaves \PYGZlt{}n max\PYGZgt{}         Maximum number of octaves.
  \PYGZhy{}v, \PYGZhy{}\PYGZhy{}valid\PYGZus{}intervals \PYGZlt{}1,2,..\PYGZgt{}  A comma\PYGZhy{}separated list without spaces
                                  of valid scale degrees to be chosen for the
                                  question.
  \PYGZhy{}q, \PYGZhy{}\PYGZhy{}user\PYGZus{}durations \PYGZlt{}1,0.5,n..\PYGZgt{}
                                  A comma\PYGZhy{}separated list without
                                  spaces with PRECISLY 9 floating values. Or
                                  \PYGZsq{}n\PYGZsq{} for default              duration.
  \PYGZhy{}p, \PYGZhy{}\PYGZhy{}prequestion\PYGZus{}method \PYGZlt{}prequestion\PYGZus{}method\PYGZgt{}
                                  The name of a pre\PYGZhy{}question method.
  \PYGZhy{}r, \PYGZhy{}\PYGZhy{}resolution\PYGZus{}method \PYGZlt{}resolution\PYGZus{}method\PYGZgt{}
                                  The name of a resolution method.
  \PYGZhy{}h, \PYGZhy{}\PYGZhy{}help                      Show this message and exit.

  In this exercise birdears will play two notes, the tonic and the interval
  harmonically, ie., both on the same time and you should reply which is the
  correct distance between the two.

  Valid values are as follows:

  \PYGZhy{}m \PYGZlt{}mode\PYGZgt{} is one of: major, dorian, phrygian, lydian, mixolydian, minor,
  locrian

  \PYGZhy{}t \PYGZlt{}tonic\PYGZgt{} is one of: A, A\PYGZsh{}, Ab, B, Bb, C, C\PYGZsh{}, D, D\PYGZsh{}, Db, E, Eb, F, F\PYGZsh{}, G,
  G\PYGZsh{}, Gb

  \PYGZhy{}p \PYGZlt{}prequestion\PYGZus{}method\PYGZgt{} is one of: none, tonic\PYGZus{}only, progression\PYGZus{}i\PYGZus{}iv\PYGZus{}v\PYGZus{}i

  \PYGZhy{}r \PYGZlt{}resolution\PYGZus{}method\PYGZgt{} is one of: nearest\PYGZus{}tonic, repeat\PYGZus{}only
\end{sphinxVerbatim}


\subsection{dictation}
\label{\detokenize{using:dictation}}
\sphinxAtStartPar
In this exercise birdears will choose some random intervals and create a
melodic dictation with them. You should reply the correct intervals of the
melodic dictation.

\begin{sphinxVerbatim}[commandchars=\\\{\}]
birdears dictation \PYGZhy{}\PYGZhy{}help
\end{sphinxVerbatim}

\begin{sphinxVerbatim}[commandchars=\\\{\}]
Usage: birdears dictation [options]

  Melodic dictation

Options:
  \PYGZhy{}m, \PYGZhy{}\PYGZhy{}mode \PYGZlt{}mode\PYGZgt{}               Mode of the question.
  \PYGZhy{}i, \PYGZhy{}\PYGZhy{}max\PYGZus{}intervals \PYGZlt{}n max\PYGZgt{}     Max random intervals for the dictation.
  \PYGZhy{}x, \PYGZhy{}\PYGZhy{}n\PYGZus{}notes \PYGZlt{}n notes\PYGZgt{}         Number of notes for the dictation.
  \PYGZhy{}t, \PYGZhy{}\PYGZhy{}tonic \PYGZlt{}note\PYGZgt{}              Tonic of the question.
  \PYGZhy{}o, \PYGZhy{}\PYGZhy{}octave \PYGZlt{}octave\PYGZgt{}           Octave of the question.
  \PYGZhy{}d, \PYGZhy{}\PYGZhy{}descending                Wether the question interval is descending.
  \PYGZhy{}c, \PYGZhy{}\PYGZhy{}chromatic                 If chosen, question has chromatic notes.
  \PYGZhy{}n, \PYGZhy{}\PYGZhy{}n\PYGZus{}octaves \PYGZlt{}n max\PYGZgt{}         Maximum number of octaves.
  \PYGZhy{}v, \PYGZhy{}\PYGZhy{}valid\PYGZus{}intervals \PYGZlt{}1,2,..\PYGZgt{}  A comma\PYGZhy{}separated list without spaces
                                  of valid scale degrees to be chosen for the
                                  question.
  \PYGZhy{}q, \PYGZhy{}\PYGZhy{}user\PYGZus{}durations \PYGZlt{}1,0.5,n..\PYGZgt{}
                                  A comma\PYGZhy{}separated list without
                                  spaces with PRECISLY 9 floating values. Or
                                  \PYGZsq{}n\PYGZsq{} for default              duration.
  \PYGZhy{}p, \PYGZhy{}\PYGZhy{}prequestion\PYGZus{}method \PYGZlt{}prequestion\PYGZus{}method\PYGZgt{}
                                  The name of a pre\PYGZhy{}question method.
  \PYGZhy{}r, \PYGZhy{}\PYGZhy{}resolution\PYGZus{}method \PYGZlt{}resolution\PYGZus{}method\PYGZgt{}
                                  The name of a resolution method.
  \PYGZhy{}h, \PYGZhy{}\PYGZhy{}help                      Show this message and exit.

  In this exercise birdears will choose some random intervals and create a
  melodic dictation with them. You should reply the correct intervals of the
  melodic dictation.

  Valid values are as follows:

  \PYGZhy{}m \PYGZlt{}mode\PYGZgt{} is one of: major, dorian, phrygian, lydian, mixolydian, minor,
  locrian

  \PYGZhy{}t \PYGZlt{}tonic\PYGZgt{} is one of: A, A\PYGZsh{}, Ab, B, Bb, C, C\PYGZsh{}, D, D\PYGZsh{}, Db, E, Eb, F, F\PYGZsh{}, G,
  G\PYGZsh{}, Gb

  \PYGZhy{}p \PYGZlt{}prequestion\PYGZus{}method\PYGZgt{} is one of: none, tonic\PYGZus{}only, progression\PYGZus{}i\PYGZus{}iv\PYGZus{}v\PYGZus{}i

  \PYGZhy{}r \PYGZlt{}resolution\PYGZus{}method\PYGZgt{} is one of: nearest\PYGZus{}tonic, repeat\PYGZus{}only
\end{sphinxVerbatim}


\subsection{instrumental}
\label{\detokenize{using:instrumental}}
\sphinxAtStartPar
In this exercise birdears will choose some random intervals and create a
melodic dictation with them. You should play the correct melody in you
musical instrument.

\begin{sphinxVerbatim}[commandchars=\\\{\}]
birdears instrumental \PYGZhy{}\PYGZhy{}help
\end{sphinxVerbatim}

\begin{sphinxVerbatim}[commandchars=\\\{\}]
Usage: birdears instrumental [options]

  Instrumental melodic time\PYGZhy{}based dictation

Options:
  \PYGZhy{}m, \PYGZhy{}\PYGZhy{}mode \PYGZlt{}mode\PYGZgt{}               Mode of the question.
  \PYGZhy{}w, \PYGZhy{}\PYGZhy{}wait\PYGZus{}time \PYGZlt{}seconds\PYGZgt{}       Time in seconds for next question/repeat.
  \PYGZhy{}u, \PYGZhy{}\PYGZhy{}n\PYGZus{}repeats \PYGZlt{}times\PYGZgt{}         Times to repeat question.
  \PYGZhy{}i, \PYGZhy{}\PYGZhy{}max\PYGZus{}intervals \PYGZlt{}n max\PYGZgt{}     Max random intervals for the dictation.
  \PYGZhy{}x, \PYGZhy{}\PYGZhy{}n\PYGZus{}notes \PYGZlt{}n notes\PYGZgt{}         Number of notes for the dictation.
  \PYGZhy{}t, \PYGZhy{}\PYGZhy{}tonic \PYGZlt{}note\PYGZgt{}              Tonic of the question.
  \PYGZhy{}o, \PYGZhy{}\PYGZhy{}octave \PYGZlt{}octave\PYGZgt{}           Octave of the question.
  \PYGZhy{}d, \PYGZhy{}\PYGZhy{}descending                Wether the question interval is descending.
  \PYGZhy{}c, \PYGZhy{}\PYGZhy{}chromatic                 If chosen, question has chromatic notes.
  \PYGZhy{}n, \PYGZhy{}\PYGZhy{}n\PYGZus{}octaves \PYGZlt{}n max\PYGZgt{}         Maximum number of octaves.
  \PYGZhy{}v, \PYGZhy{}\PYGZhy{}valid\PYGZus{}intervals \PYGZlt{}1,2,..\PYGZgt{}  A comma\PYGZhy{}separated list without spaces
                                  of valid scale degrees to be chosen for the
                                  question.
  \PYGZhy{}q, \PYGZhy{}\PYGZhy{}user\PYGZus{}durations \PYGZlt{}1,0.5,n..\PYGZgt{}
                                  A comma\PYGZhy{}separated list without
                                  spaces with PRECISLY 9 floating values. Or
                                  \PYGZsq{}n\PYGZsq{} for default              duration.
  \PYGZhy{}p, \PYGZhy{}\PYGZhy{}prequestion\PYGZus{}method \PYGZlt{}prequestion\PYGZus{}method\PYGZgt{}
                                  The name of a pre\PYGZhy{}question method.
  \PYGZhy{}r, \PYGZhy{}\PYGZhy{}resolution\PYGZus{}method \PYGZlt{}resolution\PYGZus{}method\PYGZgt{}
                                  The name of a resolution method.
  \PYGZhy{}h, \PYGZhy{}\PYGZhy{}help                      Show this message and exit.

  In this exercise birdears will choose some random intervals and create a
  melodic dictation with them. You should play the correct melody in you
  musical instrument.

  Valid values are as follows:

  \PYGZhy{}m \PYGZlt{}mode\PYGZgt{} is one of: major, dorian, phrygian, lydian, mixolydian, minor,
  locrian

  \PYGZhy{}t \PYGZlt{}tonic\PYGZgt{} is one of: A, A\PYGZsh{}, Ab, B, Bb, C, C\PYGZsh{}, D, D\PYGZsh{}, Db, E, Eb, F, F\PYGZsh{}, G,
  G\PYGZsh{}, Gb

  \PYGZhy{}p \PYGZlt{}prequestion\PYGZus{}method\PYGZgt{} is one of: none, tonic\PYGZus{}only, progression\PYGZus{}i\PYGZus{}iv\PYGZus{}v\PYGZus{}i

  \PYGZhy{}r \PYGZlt{}resolution\PYGZus{}method\PYGZgt{} is one of: nearest\PYGZus{}tonic, repeat\PYGZus{}only
\end{sphinxVerbatim}


\section{Loading from config/preset files}
\label{\detokenize{using:loading-from-config-preset-files}}

\subsection{Pre\sphinxhyphen{}made presets}
\label{\detokenize{using:pre-made-presets}}
\sphinxAtStartPar
\sphinxcode{\sphinxupquote{birdears}} cointains some pre\sphinxhyphen{}made presets in it’s \sphinxcode{\sphinxupquote{presets/}}
subdirectory.

\sphinxAtStartPar
The study for beginners is recommended by following the numeric order of
those files (000, 001, then 002 etc.)


\subsubsection{Pre\sphinxhyphen{}made presets description}
\label{\detokenize{using:pre-made-presets-description}}
\sphinxAtStartPar
\sphinxstyleemphasis{write me}


\subsection{Creating new preset files}
\label{\detokenize{using:creating-new-preset-files}}
\sphinxAtStartPar
You can open the files cointained in birdears premade \sphinxcode{\sphinxupquote{presets/}}
folder to have an ideia on how config files are made; it is simply the
command line options written in a form \sphinxcode{\sphinxupquote{toml}} standard.


\section{Keybindings}
\label{\detokenize{using:keybindings}}

\subsection{On the keybindings}
\label{\detokenize{using:on-the-keybindings}}
\sphinxAtStartPar
The following keyboard diagrams should give you an idea on how the
keybindings work. Please note how the keys on the line from \sphinxcode{\sphinxupquote{z}}
(\sphinxstyleemphasis{unison}) to \sphinxcode{\sphinxupquote{,}} (comma, \sphinxstyleemphasis{octave}) represent the notes that are
\sphinxstyleemphasis{natural} to the mode, and the line above represent the chromatics.

\sphinxAtStartPar
Also, for exercises with two octaves, the \sphinxstylestrong{uppercased keys represent
the second octave}. For example, \sphinxcode{\sphinxupquote{z}} is \sphinxstyleemphasis{unison}, \sphinxcode{\sphinxupquote{,}} is the
\sphinxstyleemphasis{octave}, \sphinxcode{\sphinxupquote{Z}} (uppercased) is the \sphinxstyleemphasis{double octave}. The same for all the other
intervals.


\subsection{Major (Ionian)}
\label{\detokenize{using:major-ionian}}
\begin{figure}[htbp]
\centering
\capstart

\noindent\sphinxincludegraphics[scale=1.0]{{ionian}.png}
\caption{Keyboard diagram for the \sphinxcode{\sphinxupquote{\sphinxhyphen{}\sphinxhyphen{}mode major}} (default).}\label{\detokenize{using:id1}}\end{figure}


\subsection{Dorian}
\label{\detokenize{using:dorian}}
\begin{figure}[htbp]
\centering
\capstart

\noindent\sphinxincludegraphics[scale=1.0]{{dorian}.png}
\caption{Keyboard diagram for the \sphinxcode{\sphinxupquote{\sphinxhyphen{}\sphinxhyphen{}mode dorian}}.}\label{\detokenize{using:id2}}\end{figure}


\subsection{Phrygian}
\label{\detokenize{using:phrygian}}
\begin{figure}[htbp]
\centering
\capstart

\noindent\sphinxincludegraphics[scale=1.0]{{phrygian}.png}
\caption{Keyboard diagram for the \sphinxcode{\sphinxupquote{\sphinxhyphen{}\sphinxhyphen{}mode phrygian}}.}\label{\detokenize{using:id3}}\end{figure}


\subsection{Lydian}
\label{\detokenize{using:lydian}}
\begin{figure}[htbp]
\centering
\capstart

\noindent\sphinxincludegraphics[scale=1.0]{{lydian}.png}
\caption{Keyboard diagram for the \sphinxcode{\sphinxupquote{\sphinxhyphen{}\sphinxhyphen{}mode lydian}}.}\label{\detokenize{using:id4}}\end{figure}


\subsection{Mixolydian}
\label{\detokenize{using:mixolydian}}
\begin{figure}[htbp]
\centering
\capstart

\noindent\sphinxincludegraphics[scale=1.0]{{mixolydian}.png}
\caption{Keyboard diagram for the \sphinxcode{\sphinxupquote{\sphinxhyphen{}\sphinxhyphen{}mode mixolydian}}.}\label{\detokenize{using:id5}}\end{figure}


\subsection{Minor (Aeolian)}
\label{\detokenize{using:minor-aeolian}}
\begin{figure}[htbp]
\centering
\capstart

\noindent\sphinxincludegraphics[scale=1.0]{{minor}.png}
\caption{Keyboard diagram for the \sphinxcode{\sphinxupquote{\sphinxhyphen{}\sphinxhyphen{}mode minor}}.}\label{\detokenize{using:id6}}\end{figure}


\subsection{Locrian}
\label{\detokenize{using:locrian}}
\begin{figure}[htbp]
\centering
\capstart

\noindent\sphinxincludegraphics[scale=1.0]{{locrian}.png}
\caption{Keyboard diagram for the \sphinxcode{\sphinxupquote{\sphinxhyphen{}\sphinxhyphen{}mode locrian}}.}\label{\detokenize{using:id7}}\end{figure}


\chapter{birdears package}
\label{\detokenize{birdears:module-birdears}}\label{\detokenize{birdears:birdears-package}}\label{\detokenize{birdears::doc}}\index{module@\spxentry{module}!birdears@\spxentry{birdears}}\index{birdears@\spxentry{birdears}!module@\spxentry{module}}
\sphinxAtStartPar
birdears provides facilities to building musical ear training exercises.
\index{CHROMATIC\_FLAT (in module birdears)@\spxentry{CHROMATIC\_FLAT}\spxextra{in module birdears}}

\begin{fulllineitems}
\phantomsection\label{\detokenize{birdears:birdears.CHROMATIC_FLAT}}\pysigline{\sphinxcode{\sphinxupquote{birdears.}}\sphinxbfcode{\sphinxupquote{CHROMATIC\_FLAT}}\sphinxbfcode{\sphinxupquote{ = (\textquotesingle{}C\textquotesingle{}, \textquotesingle{}Db\textquotesingle{}, \textquotesingle{}D\textquotesingle{}, \textquotesingle{}Eb\textquotesingle{}, \textquotesingle{}E\textquotesingle{}, \textquotesingle{}F\textquotesingle{}, \textquotesingle{}Gb\textquotesingle{}, \textquotesingle{}G\textquotesingle{}, \textquotesingle{}Ab\textquotesingle{}, \textquotesingle{}A\textquotesingle{}, \textquotesingle{}Bb\textquotesingle{}, \textquotesingle{}B\textquotesingle{})}}}
\sphinxAtStartPar
Chromatic notes names using flats.

\sphinxAtStartPar
A mapping of the chromatic note names using flats.
\begin{quote}\begin{description}
\item[{Type}] \leavevmode
\sphinxAtStartPar
tuple

\end{description}\end{quote}

\end{fulllineitems}

\index{CHROMATIC\_SHARP (in module birdears)@\spxentry{CHROMATIC\_SHARP}\spxextra{in module birdears}}

\begin{fulllineitems}
\phantomsection\label{\detokenize{birdears:birdears.CHROMATIC_SHARP}}\pysigline{\sphinxcode{\sphinxupquote{birdears.}}\sphinxbfcode{\sphinxupquote{CHROMATIC\_SHARP}}\sphinxbfcode{\sphinxupquote{ = (\textquotesingle{}C\textquotesingle{}, \textquotesingle{}C\#\textquotesingle{}, \textquotesingle{}D\textquotesingle{}, \textquotesingle{}D\#\textquotesingle{}, \textquotesingle{}E\textquotesingle{}, \textquotesingle{}F\textquotesingle{}, \textquotesingle{}F\#\textquotesingle{}, \textquotesingle{}G\textquotesingle{}, \textquotesingle{}G\#\textquotesingle{}, \textquotesingle{}A\textquotesingle{}, \textquotesingle{}A\#\textquotesingle{}, \textquotesingle{}B\textquotesingle{})}}}
\sphinxAtStartPar
Chromatic notes names using sharps.

\sphinxAtStartPar
A mapping of the chromatic note namesu sing sharps
\begin{quote}\begin{description}
\item[{Type}] \leavevmode
\sphinxAtStartPar
tuple

\end{description}\end{quote}

\end{fulllineitems}

\index{CHROMATIC\_TYPE (in module birdears)@\spxentry{CHROMATIC\_TYPE}\spxextra{in module birdears}}

\begin{fulllineitems}
\phantomsection\label{\detokenize{birdears:birdears.CHROMATIC_TYPE}}\pysigline{\sphinxcode{\sphinxupquote{birdears.}}\sphinxbfcode{\sphinxupquote{CHROMATIC\_TYPE}}\sphinxbfcode{\sphinxupquote{ = (0, 1, 2, 3, 4, 5, 6, 7, 8, 9, 10, 11)}}}
\sphinxAtStartPar
A map of the chromatic scale.

\sphinxAtStartPar
A map of the the semitones which compound the chromatic scale.
\begin{quote}\begin{description}
\item[{Type}] \leavevmode
\sphinxAtStartPar
tuple

\end{description}\end{quote}

\end{fulllineitems}

\index{CIRCLE\_OF\_FIFTHS (in module birdears)@\spxentry{CIRCLE\_OF\_FIFTHS}\spxextra{in module birdears}}

\begin{fulllineitems}
\phantomsection\label{\detokenize{birdears:birdears.CIRCLE_OF_FIFTHS}}\pysigline{\sphinxcode{\sphinxupquote{birdears.}}\sphinxbfcode{\sphinxupquote{CIRCLE\_OF\_FIFTHS}}\sphinxbfcode{\sphinxupquote{ = {[}(\textquotesingle{}C\textquotesingle{}, \textquotesingle{}G\textquotesingle{}, \textquotesingle{}D\textquotesingle{}, \textquotesingle{}A\textquotesingle{}, \textquotesingle{}E\textquotesingle{}, \textquotesingle{}B\textquotesingle{}, \textquotesingle{}Gb\textquotesingle{}, \textquotesingle{}Db\textquotesingle{}, \textquotesingle{}Ab\textquotesingle{}, \textquotesingle{}Eb\textquotesingle{}, \textquotesingle{}Bb\textquotesingle{}, \textquotesingle{}F\textquotesingle{}), (\textquotesingle{}C\textquotesingle{}, \textquotesingle{}F\textquotesingle{}, \textquotesingle{}Bb\textquotesingle{}, \textquotesingle{}Eb\textquotesingle{}, \textquotesingle{}Ab\textquotesingle{}, \textquotesingle{}C\#\textquotesingle{}, \textquotesingle{}F\#\textquotesingle{}, \textquotesingle{}B\textquotesingle{}, \textquotesingle{}E\textquotesingle{}, \textquotesingle{}A\textquotesingle{}, \textquotesingle{}D\textquotesingle{}, \textquotesingle{}G\textquotesingle{}){]}}}}
\sphinxAtStartPar
Circle of fifths.

\sphinxAtStartPar
These are the circle of fifth in both directions.
\begin{quote}\begin{description}
\item[{Type}] \leavevmode
\sphinxAtStartPar
list of tuples

\end{description}\end{quote}

\end{fulllineitems}

\index{D() (in module birdears)@\spxentry{D()}\spxextra{in module birdears}}

\begin{fulllineitems}
\phantomsection\label{\detokenize{birdears:birdears.D}}\pysiglinewithargsret{\sphinxcode{\sphinxupquote{birdears.}}\sphinxbfcode{\sphinxupquote{D}}}{\emph{\DUrole{n}{data}}, \emph{\DUrole{n}{nlines}\DUrole{o}{=}\DUrole{default_value}{0}}}{}
\end{fulllineitems}

\index{DEGREE\_INDEX (in module birdears)@\spxentry{DEGREE\_INDEX}\spxextra{in module birdears}}

\begin{fulllineitems}
\phantomsection\label{\detokenize{birdears:birdears.DEGREE_INDEX}}\pysigline{\sphinxcode{\sphinxupquote{birdears.}}\sphinxbfcode{\sphinxupquote{DEGREE\_INDEX}}\sphinxbfcode{\sphinxupquote{ = \{\textquotesingle{}i\textquotesingle{}: {[}0{]}, \textquotesingle{}ii\textquotesingle{}: {[}1, 2{]}, \textquotesingle{}iii\textquotesingle{}: {[}3, 4{]}, \textquotesingle{}iv\textquotesingle{}: {[}5, 6{]}, \textquotesingle{}v\textquotesingle{}: {[}6, 7{]}, \textquotesingle{}vi\textquotesingle{}: {[}8, 9{]}, \textquotesingle{}vii\textquotesingle{}: {[}10, 11{]}, \textquotesingle{}viii\textquotesingle{}: {[}12{]}\}}}}
\sphinxAtStartPar
A mapping of semitones of each degree.

\sphinxAtStartPar
A mapping of semitones which index to each degree roman numeral, major/minor,
perfect, augmented/diminished
\begin{quote}\begin{description}
\item[{Type}] \leavevmode
\sphinxAtStartPar
dict of lists

\end{description}\end{quote}

\end{fulllineitems}

\index{DIATONIC\_MASK (in module birdears)@\spxentry{DIATONIC\_MASK}\spxextra{in module birdears}}

\begin{fulllineitems}
\phantomsection\label{\detokenize{birdears:birdears.DIATONIC_MASK}}\pysigline{\sphinxcode{\sphinxupquote{birdears.}}\sphinxbfcode{\sphinxupquote{DIATONIC\_MASK}}\sphinxbfcode{\sphinxupquote{ = \{\textquotesingle{}dorian\textquotesingle{}: (1, 0, 1, 1, 0, 1, 0, 1, 0, 1, 1, 0), \textquotesingle{}locrian\textquotesingle{}: (1, 1, 0, 1, 0, 1, 1, 0, 1, 0, 1, 0), \textquotesingle{}lydian\textquotesingle{}: (1, 0, 1, 0, 1, 0, 1, 1, 0, 1, 0, 1), \textquotesingle{}major\textquotesingle{}: (1, 0, 1, 0, 1, 1, 0, 1, 0, 1, 0, 1), \textquotesingle{}minor\textquotesingle{}: (1, 0, 1, 1, 0, 1, 0, 1, 1, 0, 1, 0), \textquotesingle{}mixolydian\textquotesingle{}: (1, 0, 1, 0, 1, 1, 0, 1, 0, 1, 1, 0), \textquotesingle{}phrygian\textquotesingle{}: (1, 1, 0, 1, 0, 1, 0, 1, 1, 0, 1, 0)\}}}}
\sphinxAtStartPar
A map of the diatonic scale.

\sphinxAtStartPar
A mapping of the semitones which compound each of the greek modes.
\begin{quote}\begin{description}
\item[{Type}] \leavevmode
\sphinxAtStartPar
dict of tuples

\end{description}\end{quote}

\end{fulllineitems}

\index{INTERVALS (in module birdears)@\spxentry{INTERVALS}\spxextra{in module birdears}}

\begin{fulllineitems}
\phantomsection\label{\detokenize{birdears:birdears.INTERVALS}}\pysigline{\sphinxcode{\sphinxupquote{birdears.}}\sphinxbfcode{\sphinxupquote{INTERVALS}}\sphinxbfcode{\sphinxupquote{ = ((0, \textquotesingle{}P1\textquotesingle{}, \textquotesingle{}Perfect Unison\textquotesingle{}), (1, \textquotesingle{}m2\textquotesingle{}, \textquotesingle{}Minor Second\textquotesingle{}), (2, \textquotesingle{}M2\textquotesingle{}, \textquotesingle{}Major Second\textquotesingle{}), (3, \textquotesingle{}m3\textquotesingle{}, \textquotesingle{}Minor Third\textquotesingle{}), (4, \textquotesingle{}M3\textquotesingle{}, \textquotesingle{}Major Third\textquotesingle{}), (5, \textquotesingle{}P4\textquotesingle{}, \textquotesingle{}Perfect Fourth\textquotesingle{}), (6, \textquotesingle{}A4\textquotesingle{}, \textquotesingle{}Augmented Fourth\textquotesingle{}), (7, \textquotesingle{}P5\textquotesingle{}, \textquotesingle{}Perfect Fifth\textquotesingle{}), (8, \textquotesingle{}m6\textquotesingle{}, \textquotesingle{}Minor Sixth\textquotesingle{}), (9, \textquotesingle{}M6\textquotesingle{}, \textquotesingle{}Major Sixth\textquotesingle{}), (10, \textquotesingle{}m7\textquotesingle{}, \textquotesingle{}Minor Seventh\textquotesingle{}), (11, \textquotesingle{}M7\textquotesingle{}, \textquotesingle{}Major Seventh\textquotesingle{}), (12, \textquotesingle{}P8\textquotesingle{}, \textquotesingle{}Perfect Octave\textquotesingle{}), (13, \textquotesingle{}A8\textquotesingle{}, \textquotesingle{}Minor Ninth\textquotesingle{}), (14, \textquotesingle{}M9\textquotesingle{}, \textquotesingle{}Major Ninth\textquotesingle{}), (15, \textquotesingle{}m10\textquotesingle{}, \textquotesingle{}Minor Tenth\textquotesingle{}), (16, \textquotesingle{}M10\textquotesingle{}, \textquotesingle{}Major Tenth\textquotesingle{}), (17, \textquotesingle{}P11\textquotesingle{}, \textquotesingle{}Perfect Eleventh\textquotesingle{}), (18, \textquotesingle{}A11\textquotesingle{}, \textquotesingle{}Augmented Eleventh\textquotesingle{}), (19, \textquotesingle{}P12\textquotesingle{}, \textquotesingle{}Perfect Twelfth\textquotesingle{}), (20, \textquotesingle{}m13\textquotesingle{}, \textquotesingle{}Minor Thirteenth\textquotesingle{}), (21, \textquotesingle{}M13\textquotesingle{}, \textquotesingle{}Major Thirteenth\textquotesingle{}), (22, \textquotesingle{}m14\textquotesingle{}, \textquotesingle{}Minor Fourteenth\textquotesingle{}), (23, \textquotesingle{}M14\textquotesingle{}, \textquotesingle{}Major Fourteenth\textquotesingle{}), (24, \textquotesingle{}P15\textquotesingle{}, \textquotesingle{}Perfect Double\sphinxhyphen{}octave\textquotesingle{}), (25, \textquotesingle{}A15\textquotesingle{}, \textquotesingle{}Minor Sixteenth\textquotesingle{}), (26, \textquotesingle{}M16\textquotesingle{}, \textquotesingle{}Major Sixteenth\textquotesingle{}), (27, \textquotesingle{}m17\textquotesingle{}, \textquotesingle{}Minor Seventeenth\textquotesingle{}), (28, \textquotesingle{}M17\textquotesingle{}, \textquotesingle{}Major Seventeenth\textquotesingle{}), (29, \textquotesingle{}P18\textquotesingle{}, \textquotesingle{}Perfect Eighteenth\textquotesingle{}), (30, \textquotesingle{}A18\textquotesingle{}, \textquotesingle{}Augmented Eighteenth\textquotesingle{}), (31, \textquotesingle{}P19\textquotesingle{}, \textquotesingle{}Perfect Nineteenth\textquotesingle{}), (32, \textquotesingle{}m20\textquotesingle{}, \textquotesingle{}Minor Twentieth\textquotesingle{}), (33, \textquotesingle{}M20\textquotesingle{}, \textquotesingle{}Major Twentieth\textquotesingle{}), (34, \textquotesingle{}m21\textquotesingle{}, \textquotesingle{}Minor Twenty\sphinxhyphen{}first\textquotesingle{}), (35, \textquotesingle{}M21\textquotesingle{}, \textquotesingle{}Major Twenty\sphinxhyphen{}first\textquotesingle{}), (36, \textquotesingle{}P22\textquotesingle{}, \textquotesingle{}Perfect Triple\sphinxhyphen{}octave\textquotesingle{}))}}}
\sphinxAtStartPar
Data representing intervals.

\sphinxAtStartPar
A tuple of tuples representing data for the intervals with format
(semitones, short name, full name).
\begin{quote}\begin{description}
\item[{Type}] \leavevmode
\sphinxAtStartPar
tuple of tuples

\end{description}\end{quote}

\end{fulllineitems}

\index{INTERVAL\_INDEX (in module birdears)@\spxentry{INTERVAL\_INDEX}\spxextra{in module birdears}}

\begin{fulllineitems}
\phantomsection\label{\detokenize{birdears:birdears.INTERVAL_INDEX}}\pysigline{\sphinxcode{\sphinxupquote{birdears.}}\sphinxbfcode{\sphinxupquote{INTERVAL\_INDEX}}\sphinxbfcode{\sphinxupquote{ = \{1: {[}0{]}, 2: {[}1, 2{]}, 3: {[}3, 4{]}, 4: {[}5, 6{]}, 5: {[}6, 7{]}, 6: {[}8, 9{]}, 7: {[}10, 11{]}, 8: {[}12{]}\}}}}
\sphinxAtStartPar
A mapping of semitones of each interval.

\sphinxAtStartPar
A mapping of semitones which index to each interval name, major/minor,
perfect, augmented/diminished
\begin{quote}\begin{description}
\item[{Type}] \leavevmode
\sphinxAtStartPar
dict of lists

\end{description}\end{quote}

\end{fulllineitems}

\index{KEYS (in module birdears)@\spxentry{KEYS}\spxextra{in module birdears}}

\begin{fulllineitems}
\phantomsection\label{\detokenize{birdears:birdears.KEYS}}\pysigline{\sphinxcode{\sphinxupquote{birdears.}}\sphinxbfcode{\sphinxupquote{KEYS}}\sphinxbfcode{\sphinxupquote{ = (\textquotesingle{}C\textquotesingle{}, \textquotesingle{}C\#\textquotesingle{}, \textquotesingle{}Db\textquotesingle{}, \textquotesingle{}D\textquotesingle{}, \textquotesingle{}D\#\textquotesingle{}, \textquotesingle{}Eb\textquotesingle{}, \textquotesingle{}E\textquotesingle{}, \textquotesingle{}F\textquotesingle{}, \textquotesingle{}F\#\textquotesingle{}, \textquotesingle{}Gb\textquotesingle{}, \textquotesingle{}G\textquotesingle{}, \textquotesingle{}G\#\textquotesingle{}, \textquotesingle{}Ab\textquotesingle{}, \textquotesingle{}A\textquotesingle{}, \textquotesingle{}A\#\textquotesingle{}, \textquotesingle{}Bb\textquotesingle{}, \textquotesingle{}B\textquotesingle{})}}}
\sphinxAtStartPar
Allowed keys

\sphinxAtStartPar
These are the allowed keys for exercise as comprehended by birdears.
\begin{quote}\begin{description}
\item[{Type}] \leavevmode
\sphinxAtStartPar
tuple

\end{description}\end{quote}

\end{fulllineitems}



\section{Subpackages}
\label{\detokenize{birdears:subpackages}}

\subsection{birdears.interfaces package}
\label{\detokenize{birdears.interfaces:module-birdears.interfaces}}\label{\detokenize{birdears.interfaces:birdears-interfaces-package}}\label{\detokenize{birdears.interfaces::doc}}\index{module@\spxentry{module}!birdears.interfaces@\spxentry{birdears.interfaces}}\index{birdears.interfaces@\spxentry{birdears.interfaces}!module@\spxentry{module}}

\subsubsection{Submodules}
\label{\detokenize{birdears.interfaces:submodules}}

\subsubsection{birdears.interfaces.commandline module}
\label{\detokenize{birdears.interfaces:module-birdears.interfaces.commandline}}\label{\detokenize{birdears.interfaces:birdears-interfaces-commandline-module}}\index{module@\spxentry{module}!birdears.interfaces.commandline@\spxentry{birdears.interfaces.commandline}}\index{birdears.interfaces.commandline@\spxentry{birdears.interfaces.commandline}!module@\spxentry{module}}\index{CommandLine (class in birdears.interfaces.commandline)@\spxentry{CommandLine}\spxextra{class in birdears.interfaces.commandline}}

\begin{fulllineitems}
\phantomsection\label{\detokenize{birdears.interfaces:birdears.interfaces.commandline.CommandLine}}\pysiglinewithargsret{\sphinxbfcode{\sphinxupquote{class }}\sphinxcode{\sphinxupquote{birdears.interfaces.commandline.}}\sphinxbfcode{\sphinxupquote{CommandLine}}}{\emph{\DUrole{n}{exercise}\DUrole{o}{=}\DUrole{default_value}{None}}, \emph{\DUrole{o}{*}\DUrole{n}{args}}, \emph{\DUrole{o}{**}\DUrole{n}{kwargs}}}{}
\sphinxAtStartPar
Bases: \sphinxcode{\sphinxupquote{object}}
\index{\_\_init\_\_() (birdears.interfaces.commandline.CommandLine method)@\spxentry{\_\_init\_\_()}\spxextra{birdears.interfaces.commandline.CommandLine method}}

\begin{fulllineitems}
\phantomsection\label{\detokenize{birdears.interfaces:birdears.interfaces.commandline.CommandLine.__init__}}\pysiglinewithargsret{\sphinxbfcode{\sphinxupquote{\_\_init\_\_}}}{\emph{\DUrole{n}{exercise}\DUrole{o}{=}\DUrole{default_value}{None}}, \emph{\DUrole{o}{*}\DUrole{n}{args}}, \emph{\DUrole{o}{**}\DUrole{n}{kwargs}}}{}
\sphinxAtStartPar
This function implements the birdears loop for command line.
\begin{quote}\begin{description}
\item[{Parameters}] \leavevmode\begin{itemize}
\item {} 
\sphinxAtStartPar
\sphinxstyleliteralstrong{\sphinxupquote{exercise}} (\sphinxstyleliteralemphasis{\sphinxupquote{str}}) \textendash{} The question name.

\item {} 
\sphinxAtStartPar
\sphinxstyleliteralstrong{\sphinxupquote{**kwargs}} (\sphinxstyleliteralemphasis{\sphinxupquote{kwargs}}) \textendash{} FIXME: The kwargs can contain options for specific
questions.

\end{itemize}

\end{description}\end{quote}

\end{fulllineitems}

\index{process\_key() (birdears.interfaces.commandline.CommandLine method)@\spxentry{process\_key()}\spxextra{birdears.interfaces.commandline.CommandLine method}}

\begin{fulllineitems}
\phantomsection\label{\detokenize{birdears.interfaces:birdears.interfaces.commandline.CommandLine.process_key}}\pysiglinewithargsret{\sphinxbfcode{\sphinxupquote{process\_key}}}{\emph{\DUrole{n}{user\_input}}}{}
\end{fulllineitems}


\end{fulllineitems}

\index{center\_text() (in module birdears.interfaces.commandline)@\spxentry{center\_text()}\spxextra{in module birdears.interfaces.commandline}}

\begin{fulllineitems}
\phantomsection\label{\detokenize{birdears.interfaces:birdears.interfaces.commandline.center_text}}\pysiglinewithargsret{\sphinxcode{\sphinxupquote{birdears.interfaces.commandline.}}\sphinxbfcode{\sphinxupquote{center\_text}}}{\emph{\DUrole{n}{text}}, \emph{\DUrole{n}{sep}\DUrole{o}{=}\DUrole{default_value}{True}}, \emph{\DUrole{n}{nl}\DUrole{o}{=}\DUrole{default_value}{0}}}{}
\sphinxAtStartPar
This function returns input text centered according to terminal columns.
\begin{quote}\begin{description}
\item[{Parameters}] \leavevmode\begin{itemize}
\item {} 
\sphinxAtStartPar
\sphinxstyleliteralstrong{\sphinxupquote{text}} (\sphinxstyleliteralemphasis{\sphinxupquote{str}}) \textendash{} The string to be centered, it can have multiple lines.

\item {} 
\sphinxAtStartPar
\sphinxstyleliteralstrong{\sphinxupquote{sep}} (\sphinxstyleliteralemphasis{\sphinxupquote{bool}}) \textendash{} Add line separator after centered text (True) or
not (False).

\item {} 
\sphinxAtStartPar
\sphinxstyleliteralstrong{\sphinxupquote{nl}} (\sphinxstyleliteralemphasis{\sphinxupquote{int}}) \textendash{} How many new lines to add after text.

\end{itemize}

\end{description}\end{quote}

\end{fulllineitems}

\index{make\_input\_str() (in module birdears.interfaces.commandline)@\spxentry{make\_input\_str()}\spxextra{in module birdears.interfaces.commandline}}

\begin{fulllineitems}
\phantomsection\label{\detokenize{birdears.interfaces:birdears.interfaces.commandline.make_input_str}}\pysiglinewithargsret{\sphinxcode{\sphinxupquote{birdears.interfaces.commandline.}}\sphinxbfcode{\sphinxupquote{make\_input\_str}}}{\emph{\DUrole{n}{user\_input}}, \emph{\DUrole{n}{keyboard\_index}}}{}
\sphinxAtStartPar
Makes a string representing intervals entered by the user.

\sphinxAtStartPar
This function is to be used by questions which takes more than one interval
input as MelodicDictation, and formats the intervals already entered.
\begin{quote}\begin{description}
\item[{Parameters}] \leavevmode\begin{itemize}
\item {} 
\sphinxAtStartPar
\sphinxstyleliteralstrong{\sphinxupquote{user\_input}} (\sphinxstyleliteralemphasis{\sphinxupquote{array\_type}}) \textendash{} The list of keyboard keys entered by user.

\item {} 
\sphinxAtStartPar
\sphinxstyleliteralstrong{\sphinxupquote{keyboard\_index}} (\sphinxstyleliteralemphasis{\sphinxupquote{array\_type}}) \textendash{} The keyboard mapping used by question.

\end{itemize}

\end{description}\end{quote}

\end{fulllineitems}

\index{print\_instrumental() (in module birdears.interfaces.commandline)@\spxentry{print\_instrumental()}\spxextra{in module birdears.interfaces.commandline}}

\begin{fulllineitems}
\phantomsection\label{\detokenize{birdears.interfaces:birdears.interfaces.commandline.print_instrumental}}\pysiglinewithargsret{\sphinxcode{\sphinxupquote{birdears.interfaces.commandline.}}\sphinxbfcode{\sphinxupquote{print\_instrumental}}}{\emph{\DUrole{n}{response}}}{}
\sphinxAtStartPar
Prints the formatted response for ‘instrumental’ exercise.
\begin{quote}\begin{description}
\item[{Parameters}] \leavevmode
\sphinxAtStartPar
\sphinxstyleliteralstrong{\sphinxupquote{response}} (\sphinxstyleliteralemphasis{\sphinxupquote{dict}}) \textendash{} A response returned by question’s check\_question()

\end{description}\end{quote}

\end{fulllineitems}

\index{print\_question() (in module birdears.interfaces.commandline)@\spxentry{print\_question()}\spxextra{in module birdears.interfaces.commandline}}

\begin{fulllineitems}
\phantomsection\label{\detokenize{birdears.interfaces:birdears.interfaces.commandline.print_question}}\pysiglinewithargsret{\sphinxcode{\sphinxupquote{birdears.interfaces.commandline.}}\sphinxbfcode{\sphinxupquote{print\_question}}}{\emph{\DUrole{n}{question}}}{}
\sphinxAtStartPar
Prints the question to the user.
\begin{quote}\begin{description}
\item[{Parameters}] \leavevmode
\sphinxAtStartPar
\sphinxstyleliteralstrong{\sphinxupquote{question}} (\sphinxstyleliteralemphasis{\sphinxupquote{obj}}) \textendash{} A Question class with the question to be printed.

\end{description}\end{quote}

\end{fulllineitems}

\index{print\_response() (in module birdears.interfaces.commandline)@\spxentry{print\_response()}\spxextra{in module birdears.interfaces.commandline}}

\begin{fulllineitems}
\phantomsection\label{\detokenize{birdears.interfaces:birdears.interfaces.commandline.print_response}}\pysiglinewithargsret{\sphinxcode{\sphinxupquote{birdears.interfaces.commandline.}}\sphinxbfcode{\sphinxupquote{print\_response}}}{\emph{\DUrole{n}{response}}}{}
\sphinxAtStartPar
Prints the formatted response.
\begin{quote}\begin{description}
\item[{Parameters}] \leavevmode
\sphinxAtStartPar
\sphinxstyleliteralstrong{\sphinxupquote{response}} (\sphinxstyleliteralemphasis{\sphinxupquote{dict}}) \textendash{} A response returned by question’s check\_question()

\end{description}\end{quote}

\end{fulllineitems}



\subsection{birdears.questions package}
\label{\detokenize{birdears.questions:module-birdears.questions}}\label{\detokenize{birdears.questions:birdears-questions-package}}\label{\detokenize{birdears.questions::doc}}\index{module@\spxentry{module}!birdears.questions@\spxentry{birdears.questions}}\index{birdears.questions@\spxentry{birdears.questions}!module@\spxentry{module}}

\subsubsection{Submodules}
\label{\detokenize{birdears.questions:submodules}}

\subsubsection{birdears.questions.harmonicinterval module}
\label{\detokenize{birdears.questions:module-birdears.questions.harmonicinterval}}\label{\detokenize{birdears.questions:birdears-questions-harmonicinterval-module}}\index{module@\spxentry{module}!birdears.questions.harmonicinterval@\spxentry{birdears.questions.harmonicinterval}}\index{birdears.questions.harmonicinterval@\spxentry{birdears.questions.harmonicinterval}!module@\spxentry{module}}\index{HarmonicIntervalQuestion (class in birdears.questions.harmonicinterval)@\spxentry{HarmonicIntervalQuestion}\spxextra{class in birdears.questions.harmonicinterval}}

\begin{fulllineitems}
\phantomsection\label{\detokenize{birdears.questions:birdears.questions.harmonicinterval.HarmonicIntervalQuestion}}\pysiglinewithargsret{\sphinxbfcode{\sphinxupquote{class }}\sphinxcode{\sphinxupquote{birdears.questions.harmonicinterval.}}\sphinxbfcode{\sphinxupquote{HarmonicIntervalQuestion}}}{\emph{\DUrole{n}{mode}\DUrole{o}{=}\DUrole{default_value}{\textquotesingle{}major\textquotesingle{}}}, \emph{\DUrole{n}{tonic}\DUrole{o}{=}\DUrole{default_value}{\textquotesingle{}C\textquotesingle{}}}, \emph{\DUrole{n}{octave}\DUrole{o}{=}\DUrole{default_value}{4}}, \emph{\DUrole{n}{descending}\DUrole{o}{=}\DUrole{default_value}{False}}, \emph{\DUrole{n}{chromatic}\DUrole{o}{=}\DUrole{default_value}{False}}, \emph{\DUrole{n}{n\_octaves}\DUrole{o}{=}\DUrole{default_value}{1}}, \emph{\DUrole{n}{valid\_intervals}\DUrole{o}{=}\DUrole{default_value}{(0, 1, 2, 3, 4, 5, 6, 7, 8, 9, 10, 11)}}, \emph{\DUrole{n}{user\_durations}\DUrole{o}{=}\DUrole{default_value}{None}}, \emph{\DUrole{n}{prequestion\_method}\DUrole{o}{=}\DUrole{default_value}{\textquotesingle{}none\textquotesingle{}}}, \emph{\DUrole{n}{resolution\_method}\DUrole{o}{=}\DUrole{default_value}{\textquotesingle{}nearest\_tonic\textquotesingle{}}}, \emph{\DUrole{o}{*}\DUrole{n}{args}}, \emph{\DUrole{o}{**}\DUrole{n}{kwargs}}}{}
\sphinxAtStartPar
Bases: {\hyperref[\detokenize{index:birdears.questionbase.QuestionBase}]{\sphinxcrossref{\sphinxcode{\sphinxupquote{birdears.questionbase.QuestionBase}}}}}

\sphinxAtStartPar
Implements a Harmonic Interval test.
\index{\_\_init\_\_() (birdears.questions.harmonicinterval.HarmonicIntervalQuestion method)@\spxentry{\_\_init\_\_()}\spxextra{birdears.questions.harmonicinterval.HarmonicIntervalQuestion method}}

\begin{fulllineitems}
\phantomsection\label{\detokenize{birdears.questions:birdears.questions.harmonicinterval.HarmonicIntervalQuestion.__init__}}\pysiglinewithargsret{\sphinxbfcode{\sphinxupquote{\_\_init\_\_}}}{\emph{\DUrole{n}{mode}\DUrole{o}{=}\DUrole{default_value}{\textquotesingle{}major\textquotesingle{}}}, \emph{\DUrole{n}{tonic}\DUrole{o}{=}\DUrole{default_value}{\textquotesingle{}C\textquotesingle{}}}, \emph{\DUrole{n}{octave}\DUrole{o}{=}\DUrole{default_value}{4}}, \emph{\DUrole{n}{descending}\DUrole{o}{=}\DUrole{default_value}{False}}, \emph{\DUrole{n}{chromatic}\DUrole{o}{=}\DUrole{default_value}{False}}, \emph{\DUrole{n}{n\_octaves}\DUrole{o}{=}\DUrole{default_value}{1}}, \emph{\DUrole{n}{valid\_intervals}\DUrole{o}{=}\DUrole{default_value}{(0, 1, 2, 3, 4, 5, 6, 7, 8, 9, 10, 11)}}, \emph{\DUrole{n}{user\_durations}\DUrole{o}{=}\DUrole{default_value}{None}}, \emph{\DUrole{n}{prequestion\_method}\DUrole{o}{=}\DUrole{default_value}{\textquotesingle{}none\textquotesingle{}}}, \emph{\DUrole{n}{resolution\_method}\DUrole{o}{=}\DUrole{default_value}{\textquotesingle{}nearest\_tonic\textquotesingle{}}}, \emph{\DUrole{o}{*}\DUrole{n}{args}}, \emph{\DUrole{o}{**}\DUrole{n}{kwargs}}}{}
\sphinxAtStartPar
Inits the class.
\begin{quote}\begin{description}
\item[{Parameters}] \leavevmode\begin{itemize}
\item {} 
\sphinxAtStartPar
\sphinxstyleliteralstrong{\sphinxupquote{mode}} (\sphinxstyleliteralemphasis{\sphinxupquote{str}}) \textendash{} A string representing the mode of the question.
Eg., ‘major’ or ‘minor’

\item {} 
\sphinxAtStartPar
\sphinxstyleliteralstrong{\sphinxupquote{tonic}} (\sphinxstyleliteralemphasis{\sphinxupquote{str}}) \textendash{} A string representing the tonic of the question,
eg.: ‘C’; if omitted, it will be selected randomly.

\item {} 
\sphinxAtStartPar
\sphinxstyleliteralstrong{\sphinxupquote{octave}} (\sphinxstyleliteralemphasis{\sphinxupquote{int}}) \textendash{} A scienfic octave notation, for example, 4 for ‘C4’;
if not present, it will be randomly chosen.

\item {} 
\sphinxAtStartPar
\sphinxstyleliteralstrong{\sphinxupquote{descending}} (\sphinxstyleliteralemphasis{\sphinxupquote{bool}}) \textendash{} Is the question direction in descending, ie.,
intervals have lower pitch than the tonic.

\item {} 
\sphinxAtStartPar
\sphinxstyleliteralstrong{\sphinxupquote{chromatic}} (\sphinxstyleliteralemphasis{\sphinxupquote{bool}}) \textendash{} If the question can have (True) or not (False)
chromatic intervals, ie., intervals not in the diatonic scale
of tonic/mode.

\item {} 
\sphinxAtStartPar
\sphinxstyleliteralstrong{\sphinxupquote{n\_octaves}} (\sphinxstyleliteralemphasis{\sphinxupquote{int}}) \textendash{} Maximum number of octaves of the question.

\item {} 
\sphinxAtStartPar
\sphinxstyleliteralstrong{\sphinxupquote{valid\_intervals}} (\sphinxstyleliteralemphasis{\sphinxupquote{list}}) \textendash{} A list with intervals (int) valid for
random choice, 1 is 1st, 2 is second etc. Eg. {[}1, 4, 5{]} to
allow only tonics, fourths and fifths.

\item {} 
\sphinxAtStartPar
\sphinxstyleliteralstrong{\sphinxupquote{user\_durations}} (\sphinxstyleliteralemphasis{\sphinxupquote{str}}) \textendash{} 
\sphinxAtStartPar
A string with 9 comma\sphinxhyphen{}separated \sphinxtitleref{int} or
\sphinxtitleref{float\textasciigrave{}s to set the default duration for the notes played. The
values are respectively for: pre\sphinxhyphen{}question duration (1st),
pre\sphinxhyphen{}question delay (2nd), and pre\sphinxhyphen{}question pos\sphinxhyphen{}delay (3rd);
question duration (4th), question delay (5th), and question
pos\sphinxhyphen{}delay (6th); resolution duration (7th), resolution
delay (8th), and resolution pos\sphinxhyphen{}delay (9th).
duration is the duration in of the note in seconds; delay is
the time to wait before playing the next note, and pos\_delay is
the time to wait after all the notes of the respective sequence
have been played. If any of the user durations is \textasciigrave{}n}, the
default duration for the type of question will be used instead.
Example:

\begin{sphinxVerbatim}[commandchars=\\\{\}]
\PYGZdq{}2,0.5,1,2,n,0,2.5,n,1\PYGZdq{}
\end{sphinxVerbatim}


\item {} 
\sphinxAtStartPar
\sphinxstyleliteralstrong{\sphinxupquote{prequestion\_method}} (\sphinxstyleliteralemphasis{\sphinxupquote{str}}) \textendash{} Method of playing a cadence or the
exercise tonic before the question so to affirm the question
musical tonic key to the ear. Valid ones are registered in the
\sphinxtitleref{birdears.prequestion.PREQUESION\_METHODS} global dict.

\item {} 
\sphinxAtStartPar
\sphinxstyleliteralstrong{\sphinxupquote{resolution\_method}} (\sphinxstyleliteralemphasis{\sphinxupquote{str}}) \textendash{} Method of playing the resolution of an
exercise. Valid ones are registered in the
\sphinxtitleref{birdears.resolution.RESOLUTION\_METHODS} global dict.

\end{itemize}

\end{description}\end{quote}

\end{fulllineitems}

\index{check\_question() (birdears.questions.harmonicinterval.HarmonicIntervalQuestion method)@\spxentry{check\_question()}\spxextra{birdears.questions.harmonicinterval.HarmonicIntervalQuestion method}}

\begin{fulllineitems}
\phantomsection\label{\detokenize{birdears.questions:birdears.questions.harmonicinterval.HarmonicIntervalQuestion.check_question}}\pysiglinewithargsret{\sphinxbfcode{\sphinxupquote{check\_question}}}{\emph{\DUrole{n}{user\_input\_char}}}{}
\sphinxAtStartPar
Checks whether the given answer is correct.

\end{fulllineitems}

\index{make\_pre\_question() (birdears.questions.harmonicinterval.HarmonicIntervalQuestion method)@\spxentry{make\_pre\_question()}\spxextra{birdears.questions.harmonicinterval.HarmonicIntervalQuestion method}}

\begin{fulllineitems}
\phantomsection\label{\detokenize{birdears.questions:birdears.questions.harmonicinterval.HarmonicIntervalQuestion.make_pre_question}}\pysiglinewithargsret{\sphinxbfcode{\sphinxupquote{make\_pre\_question}}}{\emph{\DUrole{n}{method}}}{}
\end{fulllineitems}

\index{make\_question() (birdears.questions.harmonicinterval.HarmonicIntervalQuestion method)@\spxentry{make\_question()}\spxextra{birdears.questions.harmonicinterval.HarmonicIntervalQuestion method}}

\begin{fulllineitems}
\phantomsection\label{\detokenize{birdears.questions:birdears.questions.harmonicinterval.HarmonicIntervalQuestion.make_question}}\pysiglinewithargsret{\sphinxbfcode{\sphinxupquote{make\_question}}}{}{}
\sphinxAtStartPar
This method should be overwritten by the question subclasses.

\end{fulllineitems}

\index{make\_resolution() (birdears.questions.harmonicinterval.HarmonicIntervalQuestion method)@\spxentry{make\_resolution()}\spxextra{birdears.questions.harmonicinterval.HarmonicIntervalQuestion method}}

\begin{fulllineitems}
\phantomsection\label{\detokenize{birdears.questions:birdears.questions.harmonicinterval.HarmonicIntervalQuestion.make_resolution}}\pysiglinewithargsret{\sphinxbfcode{\sphinxupquote{make\_resolution}}}{\emph{\DUrole{n}{method}}}{}
\sphinxAtStartPar
This method should be overwritten by the question subclasses.

\end{fulllineitems}

\index{name (birdears.questions.harmonicinterval.HarmonicIntervalQuestion attribute)@\spxentry{name}\spxextra{birdears.questions.harmonicinterval.HarmonicIntervalQuestion attribute}}

\begin{fulllineitems}
\phantomsection\label{\detokenize{birdears.questions:birdears.questions.harmonicinterval.HarmonicIntervalQuestion.name}}\pysigline{\sphinxbfcode{\sphinxupquote{name}}\sphinxbfcode{\sphinxupquote{ = \textquotesingle{}harmonic\textquotesingle{}}}}
\end{fulllineitems}

\index{play\_question() (birdears.questions.harmonicinterval.HarmonicIntervalQuestion method)@\spxentry{play\_question()}\spxextra{birdears.questions.harmonicinterval.HarmonicIntervalQuestion method}}

\begin{fulllineitems}
\phantomsection\label{\detokenize{birdears.questions:birdears.questions.harmonicinterval.HarmonicIntervalQuestion.play_question}}\pysiglinewithargsret{\sphinxbfcode{\sphinxupquote{play\_question}}}{\emph{\DUrole{n}{callback}\DUrole{o}{=}\DUrole{default_value}{None}}, \emph{\DUrole{n}{end\_callback}\DUrole{o}{=}\DUrole{default_value}{None}}, \emph{\DUrole{o}{*}\DUrole{n}{args}}, \emph{\DUrole{o}{**}\DUrole{n}{kwargs}}}{}
\sphinxAtStartPar
This method should be overwritten by the question subclasses.

\end{fulllineitems}

\index{play\_resolution() (birdears.questions.harmonicinterval.HarmonicIntervalQuestion method)@\spxentry{play\_resolution()}\spxextra{birdears.questions.harmonicinterval.HarmonicIntervalQuestion method}}

\begin{fulllineitems}
\phantomsection\label{\detokenize{birdears.questions:birdears.questions.harmonicinterval.HarmonicIntervalQuestion.play_resolution}}\pysiglinewithargsret{\sphinxbfcode{\sphinxupquote{play\_resolution}}}{\emph{\DUrole{n}{callback}\DUrole{o}{=}\DUrole{default_value}{None}}, \emph{\DUrole{n}{end\_callback}\DUrole{o}{=}\DUrole{default_value}{None}}, \emph{\DUrole{o}{*}\DUrole{n}{args}}, \emph{\DUrole{o}{**}\DUrole{n}{kwargs}}}{}
\end{fulllineitems}


\end{fulllineitems}



\subsubsection{birdears.questions.instrumentaldictation module}
\label{\detokenize{birdears.questions:module-birdears.questions.instrumentaldictation}}\label{\detokenize{birdears.questions:birdears-questions-instrumentaldictation-module}}\index{module@\spxentry{module}!birdears.questions.instrumentaldictation@\spxentry{birdears.questions.instrumentaldictation}}\index{birdears.questions.instrumentaldictation@\spxentry{birdears.questions.instrumentaldictation}!module@\spxentry{module}}\index{InstrumentalDictationQuestion (class in birdears.questions.instrumentaldictation)@\spxentry{InstrumentalDictationQuestion}\spxextra{class in birdears.questions.instrumentaldictation}}

\begin{fulllineitems}
\phantomsection\label{\detokenize{birdears.questions:birdears.questions.instrumentaldictation.InstrumentalDictationQuestion}}\pysiglinewithargsret{\sphinxbfcode{\sphinxupquote{class }}\sphinxcode{\sphinxupquote{birdears.questions.instrumentaldictation.}}\sphinxbfcode{\sphinxupquote{InstrumentalDictationQuestion}}}{\emph{\DUrole{n}{mode}\DUrole{o}{=}\DUrole{default_value}{\textquotesingle{}major\textquotesingle{}}}, \emph{\DUrole{n}{wait\_time}\DUrole{o}{=}\DUrole{default_value}{11}}, \emph{\DUrole{n}{n\_repeats}\DUrole{o}{=}\DUrole{default_value}{1}}, \emph{\DUrole{n}{max\_intervals}\DUrole{o}{=}\DUrole{default_value}{3}}, \emph{\DUrole{n}{n\_notes}\DUrole{o}{=}\DUrole{default_value}{4}}, \emph{\DUrole{n}{tonic}\DUrole{o}{=}\DUrole{default_value}{\textquotesingle{}C\textquotesingle{}}}, \emph{\DUrole{n}{octave}\DUrole{o}{=}\DUrole{default_value}{4}}, \emph{\DUrole{n}{descending}\DUrole{o}{=}\DUrole{default_value}{False}}, \emph{\DUrole{n}{chromatic}\DUrole{o}{=}\DUrole{default_value}{False}}, \emph{\DUrole{n}{n\_octaves}\DUrole{o}{=}\DUrole{default_value}{1}}, \emph{\DUrole{n}{valid\_intervals}\DUrole{o}{=}\DUrole{default_value}{(0, 1, 2, 3, 4, 5, 6, 7, 8, 9, 10, 11)}}, \emph{\DUrole{n}{user\_durations}\DUrole{o}{=}\DUrole{default_value}{None}}, \emph{\DUrole{n}{prequestion\_method}\DUrole{o}{=}\DUrole{default_value}{\textquotesingle{}progression\_i\_iv\_v\_i\textquotesingle{}}}, \emph{\DUrole{n}{resolution\_method}\DUrole{o}{=}\DUrole{default_value}{\textquotesingle{}repeat\_only\textquotesingle{}}}, \emph{\DUrole{o}{*}\DUrole{n}{args}}, \emph{\DUrole{o}{**}\DUrole{n}{kwargs}}}{}
\sphinxAtStartPar
Bases: {\hyperref[\detokenize{index:birdears.questionbase.QuestionBase}]{\sphinxcrossref{\sphinxcode{\sphinxupquote{birdears.questionbase.QuestionBase}}}}}

\sphinxAtStartPar
Implements an instrumental dictation test.
\index{\_\_init\_\_() (birdears.questions.instrumentaldictation.InstrumentalDictationQuestion method)@\spxentry{\_\_init\_\_()}\spxextra{birdears.questions.instrumentaldictation.InstrumentalDictationQuestion method}}

\begin{fulllineitems}
\phantomsection\label{\detokenize{birdears.questions:birdears.questions.instrumentaldictation.InstrumentalDictationQuestion.__init__}}\pysiglinewithargsret{\sphinxbfcode{\sphinxupquote{\_\_init\_\_}}}{\emph{\DUrole{n}{mode}\DUrole{o}{=}\DUrole{default_value}{\textquotesingle{}major\textquotesingle{}}}, \emph{\DUrole{n}{wait\_time}\DUrole{o}{=}\DUrole{default_value}{11}}, \emph{\DUrole{n}{n\_repeats}\DUrole{o}{=}\DUrole{default_value}{1}}, \emph{\DUrole{n}{max\_intervals}\DUrole{o}{=}\DUrole{default_value}{3}}, \emph{\DUrole{n}{n\_notes}\DUrole{o}{=}\DUrole{default_value}{4}}, \emph{\DUrole{n}{tonic}\DUrole{o}{=}\DUrole{default_value}{\textquotesingle{}C\textquotesingle{}}}, \emph{\DUrole{n}{octave}\DUrole{o}{=}\DUrole{default_value}{4}}, \emph{\DUrole{n}{descending}\DUrole{o}{=}\DUrole{default_value}{False}}, \emph{\DUrole{n}{chromatic}\DUrole{o}{=}\DUrole{default_value}{False}}, \emph{\DUrole{n}{n\_octaves}\DUrole{o}{=}\DUrole{default_value}{1}}, \emph{\DUrole{n}{valid\_intervals}\DUrole{o}{=}\DUrole{default_value}{(0, 1, 2, 3, 4, 5, 6, 7, 8, 9, 10, 11)}}, \emph{\DUrole{n}{user\_durations}\DUrole{o}{=}\DUrole{default_value}{None}}, \emph{\DUrole{n}{prequestion\_method}\DUrole{o}{=}\DUrole{default_value}{\textquotesingle{}progression\_i\_iv\_v\_i\textquotesingle{}}}, \emph{\DUrole{n}{resolution\_method}\DUrole{o}{=}\DUrole{default_value}{\textquotesingle{}repeat\_only\textquotesingle{}}}, \emph{\DUrole{o}{*}\DUrole{n}{args}}, \emph{\DUrole{o}{**}\DUrole{n}{kwargs}}}{}
\sphinxAtStartPar
Inits the class.
\begin{quote}\begin{description}
\item[{Parameters}] \leavevmode\begin{itemize}
\item {} 
\sphinxAtStartPar
\sphinxstyleliteralstrong{\sphinxupquote{mode}} (\sphinxstyleliteralemphasis{\sphinxupquote{str}}) \textendash{} A string representing the mode of the question.
Eg., ‘major’ or ‘minor’.

\item {} 
\sphinxAtStartPar
\sphinxstyleliteralstrong{\sphinxupquote{wait\_time}} (\sphinxstyleliteralemphasis{\sphinxupquote{float}}) \textendash{} Wait time in seconds for the next question or
repeat.

\item {} 
\sphinxAtStartPar
\sphinxstyleliteralstrong{\sphinxupquote{n\_repeats}} (\sphinxstyleliteralemphasis{\sphinxupquote{int}}) \textendash{} Number of times the same dictation will be
repeated before the end of the exercise.

\item {} 
\sphinxAtStartPar
\sphinxstyleliteralstrong{\sphinxupquote{max\_intervals}} (\sphinxstyleliteralemphasis{\sphinxupquote{int}}) \textendash{} The maximum number of random intervals the
question will have.

\item {} 
\sphinxAtStartPar
\sphinxstyleliteralstrong{\sphinxupquote{n\_notes}} (\sphinxstyleliteralemphasis{\sphinxupquote{int}}) \textendash{} The number of notes the melodic dictation will have.

\item {} 
\sphinxAtStartPar
\sphinxstyleliteralstrong{\sphinxupquote{tonic}} (\sphinxstyleliteralemphasis{\sphinxupquote{str}}) \textendash{} A string representing the tonic of the question,
eg.: ‘C’; if omitted, it will be selected randomly.

\item {} 
\sphinxAtStartPar
\sphinxstyleliteralstrong{\sphinxupquote{octave}} (\sphinxstyleliteralemphasis{\sphinxupquote{int}}) \textendash{} A scienfic octave notation, for example, 4 for ‘C4’;
if not present, it will be randomly chosen.

\item {} 
\sphinxAtStartPar
\sphinxstyleliteralstrong{\sphinxupquote{descending}} (\sphinxstyleliteralemphasis{\sphinxupquote{bool}}) \textendash{} Is the question direction in descending, ie.,
intervals have lower pitch than the tonic.

\item {} 
\sphinxAtStartPar
\sphinxstyleliteralstrong{\sphinxupquote{chromatic}} (\sphinxstyleliteralemphasis{\sphinxupquote{bool}}) \textendash{} If the question can have (True) or not (False)
chromatic intervals, ie., intervals not in the diatonic scale
of tonic/mode.

\item {} 
\sphinxAtStartPar
\sphinxstyleliteralstrong{\sphinxupquote{n\_octaves}} (\sphinxstyleliteralemphasis{\sphinxupquote{int}}) \textendash{} Maximum number of octaves of the question.

\item {} 
\sphinxAtStartPar
\sphinxstyleliteralstrong{\sphinxupquote{valid\_intervals}} (\sphinxstyleliteralemphasis{\sphinxupquote{list}}) \textendash{} A list with intervals (int) valid for
random choice, 1 is 1st, 2 is second etc. Eg. {[}1, 4, 5{]} to
allow only tonics, fourths and fifths.

\item {} 
\sphinxAtStartPar
\sphinxstyleliteralstrong{\sphinxupquote{user\_durations}} (\sphinxstyleliteralemphasis{\sphinxupquote{str}}) \textendash{} 
\sphinxAtStartPar
A string with 9 comma\sphinxhyphen{}separated \sphinxtitleref{int} or
\sphinxtitleref{float\textasciigrave{}s to set the default duration for the notes played. The
values are respectively for: pre\sphinxhyphen{}question duration (1st),
pre\sphinxhyphen{}question delay (2nd), and pre\sphinxhyphen{}question pos\sphinxhyphen{}delay (3rd);
question duration (4th), question delay (5th), and question
pos\sphinxhyphen{}delay (6th); resolution duration (7th), resolution
delay (8th), and resolution pos\sphinxhyphen{}delay (9th).
duration is the duration in of the note in seconds; delay is
the time to wait before playing the next note, and pos\_delay is
the time to wait after all the notes of the respective sequence
have been played. If any of the user durations is \textasciigrave{}n}, the
default duration for the type of question will be used instead.
Example:

\begin{sphinxVerbatim}[commandchars=\\\{\}]
\PYGZdq{}2,0.5,1,2,n,0,2.5,n,1\PYGZdq{}
\end{sphinxVerbatim}


\item {} 
\sphinxAtStartPar
\sphinxstyleliteralstrong{\sphinxupquote{prequestion\_method}} (\sphinxstyleliteralemphasis{\sphinxupquote{str}}) \textendash{} Method of playing a cadence or the
exercise tonic before the question so to affirm the question
musical tonic key to the ear. Valid ones are registered in the
\sphinxtitleref{birdears.prequestion.PREQUESION\_METHODS} global dict.

\item {} 
\sphinxAtStartPar
\sphinxstyleliteralstrong{\sphinxupquote{resolution\_method}} (\sphinxstyleliteralemphasis{\sphinxupquote{str}}) \textendash{} Method of playing the resolution of an
exercise. Valid ones are registered in the
\sphinxtitleref{birdears.resolution.RESOLUTION\_METHODS} global dict.

\end{itemize}

\end{description}\end{quote}

\end{fulllineitems}

\index{check\_question() (birdears.questions.instrumentaldictation.InstrumentalDictationQuestion method)@\spxentry{check\_question()}\spxextra{birdears.questions.instrumentaldictation.InstrumentalDictationQuestion method}}

\begin{fulllineitems}
\phantomsection\label{\detokenize{birdears.questions:birdears.questions.instrumentaldictation.InstrumentalDictationQuestion.check_question}}\pysiglinewithargsret{\sphinxbfcode{\sphinxupquote{check\_question}}}{}{}
\sphinxAtStartPar
Checks whether the given answer is correct.

\sphinxAtStartPar
This currently doesn’t applies to instrumental dictation questions.

\end{fulllineitems}

\index{make\_pre\_question() (birdears.questions.instrumentaldictation.InstrumentalDictationQuestion method)@\spxentry{make\_pre\_question()}\spxextra{birdears.questions.instrumentaldictation.InstrumentalDictationQuestion method}}

\begin{fulllineitems}
\phantomsection\label{\detokenize{birdears.questions:birdears.questions.instrumentaldictation.InstrumentalDictationQuestion.make_pre_question}}\pysiglinewithargsret{\sphinxbfcode{\sphinxupquote{make\_pre\_question}}}{\emph{\DUrole{n}{method}}}{}
\end{fulllineitems}

\index{make\_question() (birdears.questions.instrumentaldictation.InstrumentalDictationQuestion method)@\spxentry{make\_question()}\spxextra{birdears.questions.instrumentaldictation.InstrumentalDictationQuestion method}}

\begin{fulllineitems}
\phantomsection\label{\detokenize{birdears.questions:birdears.questions.instrumentaldictation.InstrumentalDictationQuestion.make_question}}\pysiglinewithargsret{\sphinxbfcode{\sphinxupquote{make\_question}}}{}{}
\sphinxAtStartPar
This method should be overwritten by the question subclasses.

\end{fulllineitems}

\index{make\_resolution() (birdears.questions.instrumentaldictation.InstrumentalDictationQuestion method)@\spxentry{make\_resolution()}\spxextra{birdears.questions.instrumentaldictation.InstrumentalDictationQuestion method}}

\begin{fulllineitems}
\phantomsection\label{\detokenize{birdears.questions:birdears.questions.instrumentaldictation.InstrumentalDictationQuestion.make_resolution}}\pysiglinewithargsret{\sphinxbfcode{\sphinxupquote{make\_resolution}}}{\emph{\DUrole{n}{method}}}{}
\sphinxAtStartPar
This method should be overwritten by the question subclasses.

\end{fulllineitems}

\index{name (birdears.questions.instrumentaldictation.InstrumentalDictationQuestion attribute)@\spxentry{name}\spxextra{birdears.questions.instrumentaldictation.InstrumentalDictationQuestion attribute}}

\begin{fulllineitems}
\phantomsection\label{\detokenize{birdears.questions:birdears.questions.instrumentaldictation.InstrumentalDictationQuestion.name}}\pysigline{\sphinxbfcode{\sphinxupquote{name}}\sphinxbfcode{\sphinxupquote{ = \textquotesingle{}instrumental\textquotesingle{}}}}
\end{fulllineitems}

\index{play\_question() (birdears.questions.instrumentaldictation.InstrumentalDictationQuestion method)@\spxentry{play\_question()}\spxextra{birdears.questions.instrumentaldictation.InstrumentalDictationQuestion method}}

\begin{fulllineitems}
\phantomsection\label{\detokenize{birdears.questions:birdears.questions.instrumentaldictation.InstrumentalDictationQuestion.play_question}}\pysiglinewithargsret{\sphinxbfcode{\sphinxupquote{play\_question}}}{\emph{\DUrole{n}{callback}\DUrole{o}{=}\DUrole{default_value}{None}}, \emph{\DUrole{n}{end\_callback}\DUrole{o}{=}\DUrole{default_value}{None}}, \emph{\DUrole{o}{*}\DUrole{n}{args}}, \emph{\DUrole{o}{**}\DUrole{n}{kwargs}}}{}
\sphinxAtStartPar
This method should be overwritten by the question subclasses.

\end{fulllineitems}


\end{fulllineitems}



\subsubsection{birdears.questions.melodicdictation module}
\label{\detokenize{birdears.questions:module-birdears.questions.melodicdictation}}\label{\detokenize{birdears.questions:birdears-questions-melodicdictation-module}}\index{module@\spxentry{module}!birdears.questions.melodicdictation@\spxentry{birdears.questions.melodicdictation}}\index{birdears.questions.melodicdictation@\spxentry{birdears.questions.melodicdictation}!module@\spxentry{module}}\index{MelodicDictationQuestion (class in birdears.questions.melodicdictation)@\spxentry{MelodicDictationQuestion}\spxextra{class in birdears.questions.melodicdictation}}

\begin{fulllineitems}
\phantomsection\label{\detokenize{birdears.questions:birdears.questions.melodicdictation.MelodicDictationQuestion}}\pysiglinewithargsret{\sphinxbfcode{\sphinxupquote{class }}\sphinxcode{\sphinxupquote{birdears.questions.melodicdictation.}}\sphinxbfcode{\sphinxupquote{MelodicDictationQuestion}}}{\emph{\DUrole{n}{mode}\DUrole{o}{=}\DUrole{default_value}{\textquotesingle{}major\textquotesingle{}}}, \emph{\DUrole{n}{max\_intervals}\DUrole{o}{=}\DUrole{default_value}{3}}, \emph{\DUrole{n}{n\_notes}\DUrole{o}{=}\DUrole{default_value}{4}}, \emph{\DUrole{n}{tonic}\DUrole{o}{=}\DUrole{default_value}{\textquotesingle{}C\textquotesingle{}}}, \emph{\DUrole{n}{octave}\DUrole{o}{=}\DUrole{default_value}{4}}, \emph{\DUrole{n}{descending}\DUrole{o}{=}\DUrole{default_value}{False}}, \emph{\DUrole{n}{chromatic}\DUrole{o}{=}\DUrole{default_value}{False}}, \emph{\DUrole{n}{n\_octaves}\DUrole{o}{=}\DUrole{default_value}{1}}, \emph{\DUrole{n}{valid\_intervals}\DUrole{o}{=}\DUrole{default_value}{(0, 1, 2, 3, 4, 5, 6, 7, 8, 9, 10, 11)}}, \emph{\DUrole{n}{user\_durations}\DUrole{o}{=}\DUrole{default_value}{None}}, \emph{\DUrole{n}{prequestion\_method}\DUrole{o}{=}\DUrole{default_value}{\textquotesingle{}progression\_i\_iv\_v\_i\textquotesingle{}}}, \emph{\DUrole{n}{resolution\_method}\DUrole{o}{=}\DUrole{default_value}{\textquotesingle{}repeat\_only\textquotesingle{}}}, \emph{\DUrole{o}{*}\DUrole{n}{args}}, \emph{\DUrole{o}{**}\DUrole{n}{kwargs}}}{}
\sphinxAtStartPar
Bases: {\hyperref[\detokenize{index:birdears.questionbase.QuestionBase}]{\sphinxcrossref{\sphinxcode{\sphinxupquote{birdears.questionbase.QuestionBase}}}}}

\sphinxAtStartPar
Implements a melodic dictation test.
\index{\_\_init\_\_() (birdears.questions.melodicdictation.MelodicDictationQuestion method)@\spxentry{\_\_init\_\_()}\spxextra{birdears.questions.melodicdictation.MelodicDictationQuestion method}}

\begin{fulllineitems}
\phantomsection\label{\detokenize{birdears.questions:birdears.questions.melodicdictation.MelodicDictationQuestion.__init__}}\pysiglinewithargsret{\sphinxbfcode{\sphinxupquote{\_\_init\_\_}}}{\emph{\DUrole{n}{mode}\DUrole{o}{=}\DUrole{default_value}{\textquotesingle{}major\textquotesingle{}}}, \emph{\DUrole{n}{max\_intervals}\DUrole{o}{=}\DUrole{default_value}{3}}, \emph{\DUrole{n}{n\_notes}\DUrole{o}{=}\DUrole{default_value}{4}}, \emph{\DUrole{n}{tonic}\DUrole{o}{=}\DUrole{default_value}{\textquotesingle{}C\textquotesingle{}}}, \emph{\DUrole{n}{octave}\DUrole{o}{=}\DUrole{default_value}{4}}, \emph{\DUrole{n}{descending}\DUrole{o}{=}\DUrole{default_value}{False}}, \emph{\DUrole{n}{chromatic}\DUrole{o}{=}\DUrole{default_value}{False}}, \emph{\DUrole{n}{n\_octaves}\DUrole{o}{=}\DUrole{default_value}{1}}, \emph{\DUrole{n}{valid\_intervals}\DUrole{o}{=}\DUrole{default_value}{(0, 1, 2, 3, 4, 5, 6, 7, 8, 9, 10, 11)}}, \emph{\DUrole{n}{user\_durations}\DUrole{o}{=}\DUrole{default_value}{None}}, \emph{\DUrole{n}{prequestion\_method}\DUrole{o}{=}\DUrole{default_value}{\textquotesingle{}progression\_i\_iv\_v\_i\textquotesingle{}}}, \emph{\DUrole{n}{resolution\_method}\DUrole{o}{=}\DUrole{default_value}{\textquotesingle{}repeat\_only\textquotesingle{}}}, \emph{\DUrole{o}{*}\DUrole{n}{args}}, \emph{\DUrole{o}{**}\DUrole{n}{kwargs}}}{}
\sphinxAtStartPar
Inits the class.
\begin{quote}\begin{description}
\item[{Parameters}] \leavevmode\begin{itemize}
\item {} 
\sphinxAtStartPar
\sphinxstyleliteralstrong{\sphinxupquote{mode}} (\sphinxstyleliteralemphasis{\sphinxupquote{str}}) \textendash{} A string representing the mode of the question.
Eg., ‘major’ or ‘minor’.

\item {} 
\sphinxAtStartPar
\sphinxstyleliteralstrong{\sphinxupquote{max\_intervals}} (\sphinxstyleliteralemphasis{\sphinxupquote{int}}) \textendash{} The maximum number of random intervals
the question will have.

\item {} 
\sphinxAtStartPar
\sphinxstyleliteralstrong{\sphinxupquote{n\_notes}} (\sphinxstyleliteralemphasis{\sphinxupquote{int}}) \textendash{} The number of notes the melodic dictation will have.

\item {} 
\sphinxAtStartPar
\sphinxstyleliteralstrong{\sphinxupquote{tonic}} (\sphinxstyleliteralemphasis{\sphinxupquote{str}}) \textendash{} A string representing the tonic of the question,
eg.: ‘C’; if omitted, it will be selected randomly.

\item {} 
\sphinxAtStartPar
\sphinxstyleliteralstrong{\sphinxupquote{octave}} (\sphinxstyleliteralemphasis{\sphinxupquote{int}}) \textendash{} A scienfic octave notation, for example, 4 for ‘C4’;
if not present, it will be randomly chosen.

\item {} 
\sphinxAtStartPar
\sphinxstyleliteralstrong{\sphinxupquote{descending}} (\sphinxstyleliteralemphasis{\sphinxupquote{bool}}) \textendash{} Is the question direction in descending, ie.,
intervals have lower pitch than the tonic.

\item {} 
\sphinxAtStartPar
\sphinxstyleliteralstrong{\sphinxupquote{chromatic}} (\sphinxstyleliteralemphasis{\sphinxupquote{bool}}) \textendash{} If the question can have (True) or not (False)
chromatic intervals, ie., intervals not in the diatonic scale
of tonic/mode.

\item {} 
\sphinxAtStartPar
\sphinxstyleliteralstrong{\sphinxupquote{n\_octaves}} (\sphinxstyleliteralemphasis{\sphinxupquote{int}}) \textendash{} Maximum number of octaves of the question.

\item {} 
\sphinxAtStartPar
\sphinxstyleliteralstrong{\sphinxupquote{valid\_intervals}} (\sphinxstyleliteralemphasis{\sphinxupquote{list}}) \textendash{} A list with intervals (int) valid for
random choice, 1 is 1st, 2 is second etc. Eg. {[}1, 4, 5{]} to
allow only tonics, fourths and fifths.

\item {} 
\sphinxAtStartPar
\sphinxstyleliteralstrong{\sphinxupquote{user\_durations}} (\sphinxstyleliteralemphasis{\sphinxupquote{str}}) \textendash{} 
\sphinxAtStartPar
A string with 9 comma\sphinxhyphen{}separated \sphinxtitleref{int} or
\sphinxtitleref{float\textasciigrave{}s to set the default duration for the notes played. The
values are respectively for: pre\sphinxhyphen{}question duration (1st),
pre\sphinxhyphen{}question delay (2nd), and pre\sphinxhyphen{}question pos\sphinxhyphen{}delay (3rd);
question duration (4th), question delay (5th), and question
pos\sphinxhyphen{}delay (6th); resolution duration (7th), resolution
delay (8th), and resolution pos\sphinxhyphen{}delay (9th).
duration is the duration in of the note in seconds; delay is
the time to wait before playing the next note, and pos\_delay is
the time to wait after all the notes of the respective sequence
have been played. If any of the user durations is \textasciigrave{}n}, the
default duration for the type of question will be used instead.
Example:

\begin{sphinxVerbatim}[commandchars=\\\{\}]
\PYGZdq{}2,0.5,1,2,n,0,2.5,n,1\PYGZdq{}
\end{sphinxVerbatim}


\item {} 
\sphinxAtStartPar
\sphinxstyleliteralstrong{\sphinxupquote{prequestion\_method}} (\sphinxstyleliteralemphasis{\sphinxupquote{str}}) \textendash{} Method of playing a cadence or the
exercise tonic before the question so to affirm the question
musical tonic key to the ear. Valid ones are registered in the
\sphinxtitleref{birdears.prequestion.PREQUESION\_METHODS} global dict.

\item {} 
\sphinxAtStartPar
\sphinxstyleliteralstrong{\sphinxupquote{resolution\_method}} (\sphinxstyleliteralemphasis{\sphinxupquote{str}}) \textendash{} Method of playing the resolution of an
exercise. Valid ones are registered in the
\sphinxtitleref{birdears.resolution.RESOLUTION\_METHODS} global dict.

\end{itemize}

\end{description}\end{quote}

\end{fulllineitems}

\index{check\_question() (birdears.questions.melodicdictation.MelodicDictationQuestion method)@\spxentry{check\_question()}\spxextra{birdears.questions.melodicdictation.MelodicDictationQuestion method}}

\begin{fulllineitems}
\phantomsection\label{\detokenize{birdears.questions:birdears.questions.melodicdictation.MelodicDictationQuestion.check_question}}\pysiglinewithargsret{\sphinxbfcode{\sphinxupquote{check\_question}}}{\emph{\DUrole{n}{user\_input\_keys}}}{}
\sphinxAtStartPar
Checks whether the given answer is correct.

\end{fulllineitems}

\index{make\_pre\_question() (birdears.questions.melodicdictation.MelodicDictationQuestion method)@\spxentry{make\_pre\_question()}\spxextra{birdears.questions.melodicdictation.MelodicDictationQuestion method}}

\begin{fulllineitems}
\phantomsection\label{\detokenize{birdears.questions:birdears.questions.melodicdictation.MelodicDictationQuestion.make_pre_question}}\pysiglinewithargsret{\sphinxbfcode{\sphinxupquote{make\_pre\_question}}}{\emph{\DUrole{n}{method}}}{}
\end{fulllineitems}

\index{make\_question() (birdears.questions.melodicdictation.MelodicDictationQuestion method)@\spxentry{make\_question()}\spxextra{birdears.questions.melodicdictation.MelodicDictationQuestion method}}

\begin{fulllineitems}
\phantomsection\label{\detokenize{birdears.questions:birdears.questions.melodicdictation.MelodicDictationQuestion.make_question}}\pysiglinewithargsret{\sphinxbfcode{\sphinxupquote{make\_question}}}{}{}
\sphinxAtStartPar
This method should be overwritten by the question subclasses.

\end{fulllineitems}

\index{make\_resolution() (birdears.questions.melodicdictation.MelodicDictationQuestion method)@\spxentry{make\_resolution()}\spxextra{birdears.questions.melodicdictation.MelodicDictationQuestion method}}

\begin{fulllineitems}
\phantomsection\label{\detokenize{birdears.questions:birdears.questions.melodicdictation.MelodicDictationQuestion.make_resolution}}\pysiglinewithargsret{\sphinxbfcode{\sphinxupquote{make\_resolution}}}{\emph{\DUrole{n}{method}}}{}
\sphinxAtStartPar
This method should be overwritten by the question subclasses.

\end{fulllineitems}

\index{name (birdears.questions.melodicdictation.MelodicDictationQuestion attribute)@\spxentry{name}\spxextra{birdears.questions.melodicdictation.MelodicDictationQuestion attribute}}

\begin{fulllineitems}
\phantomsection\label{\detokenize{birdears.questions:birdears.questions.melodicdictation.MelodicDictationQuestion.name}}\pysigline{\sphinxbfcode{\sphinxupquote{name}}\sphinxbfcode{\sphinxupquote{ = \textquotesingle{}dictation\textquotesingle{}}}}
\end{fulllineitems}

\index{play\_question() (birdears.questions.melodicdictation.MelodicDictationQuestion method)@\spxentry{play\_question()}\spxextra{birdears.questions.melodicdictation.MelodicDictationQuestion method}}

\begin{fulllineitems}
\phantomsection\label{\detokenize{birdears.questions:birdears.questions.melodicdictation.MelodicDictationQuestion.play_question}}\pysiglinewithargsret{\sphinxbfcode{\sphinxupquote{play\_question}}}{\emph{\DUrole{n}{callback}\DUrole{o}{=}\DUrole{default_value}{None}}, \emph{\DUrole{n}{end\_callback}\DUrole{o}{=}\DUrole{default_value}{None}}, \emph{\DUrole{o}{*}\DUrole{n}{args}}, \emph{\DUrole{o}{**}\DUrole{n}{kwargs}}}{}
\sphinxAtStartPar
This method should be overwritten by the question subclasses.

\end{fulllineitems}

\index{play\_resolution() (birdears.questions.melodicdictation.MelodicDictationQuestion method)@\spxentry{play\_resolution()}\spxextra{birdears.questions.melodicdictation.MelodicDictationQuestion method}}

\begin{fulllineitems}
\phantomsection\label{\detokenize{birdears.questions:birdears.questions.melodicdictation.MelodicDictationQuestion.play_resolution}}\pysiglinewithargsret{\sphinxbfcode{\sphinxupquote{play\_resolution}}}{\emph{\DUrole{n}{callback}\DUrole{o}{=}\DUrole{default_value}{None}}, \emph{\DUrole{n}{end\_callback}\DUrole{o}{=}\DUrole{default_value}{None}}, \emph{\DUrole{o}{*}\DUrole{n}{args}}, \emph{\DUrole{o}{**}\DUrole{n}{kwargs}}}{}
\end{fulllineitems}


\end{fulllineitems}



\subsubsection{birdears.questions.melodicinterval module}
\label{\detokenize{birdears.questions:module-birdears.questions.melodicinterval}}\label{\detokenize{birdears.questions:birdears-questions-melodicinterval-module}}\index{module@\spxentry{module}!birdears.questions.melodicinterval@\spxentry{birdears.questions.melodicinterval}}\index{birdears.questions.melodicinterval@\spxentry{birdears.questions.melodicinterval}!module@\spxentry{module}}\index{MelodicIntervalQuestion (class in birdears.questions.melodicinterval)@\spxentry{MelodicIntervalQuestion}\spxextra{class in birdears.questions.melodicinterval}}

\begin{fulllineitems}
\phantomsection\label{\detokenize{birdears.questions:birdears.questions.melodicinterval.MelodicIntervalQuestion}}\pysiglinewithargsret{\sphinxbfcode{\sphinxupquote{class }}\sphinxcode{\sphinxupquote{birdears.questions.melodicinterval.}}\sphinxbfcode{\sphinxupquote{MelodicIntervalQuestion}}}{\emph{\DUrole{n}{mode}\DUrole{o}{=}\DUrole{default_value}{\textquotesingle{}major\textquotesingle{}}}, \emph{\DUrole{n}{tonic}\DUrole{o}{=}\DUrole{default_value}{\textquotesingle{}C\textquotesingle{}}}, \emph{\DUrole{n}{octave}\DUrole{o}{=}\DUrole{default_value}{4}}, \emph{\DUrole{n}{descending}\DUrole{o}{=}\DUrole{default_value}{False}}, \emph{\DUrole{n}{chromatic}\DUrole{o}{=}\DUrole{default_value}{False}}, \emph{\DUrole{n}{n\_octaves}\DUrole{o}{=}\DUrole{default_value}{1}}, \emph{\DUrole{n}{valid\_intervals}\DUrole{o}{=}\DUrole{default_value}{(0, 1, 2, 3, 4, 5, 6, 7, 8, 9, 10, 11)}}, \emph{\DUrole{n}{user\_durations}\DUrole{o}{=}\DUrole{default_value}{None}}, \emph{\DUrole{n}{prequestion\_method}\DUrole{o}{=}\DUrole{default_value}{\textquotesingle{}tonic\_only\textquotesingle{}}}, \emph{\DUrole{n}{resolution\_method}\DUrole{o}{=}\DUrole{default_value}{\textquotesingle{}nearest\_tonic\textquotesingle{}}}, \emph{\DUrole{o}{*}\DUrole{n}{args}}, \emph{\DUrole{o}{**}\DUrole{n}{kwargs}}}{}
\sphinxAtStartPar
Bases: {\hyperref[\detokenize{index:birdears.questionbase.QuestionBase}]{\sphinxcrossref{\sphinxcode{\sphinxupquote{birdears.questionbase.QuestionBase}}}}}

\sphinxAtStartPar
Implements a Melodic Interval test.
\index{\_\_init\_\_() (birdears.questions.melodicinterval.MelodicIntervalQuestion method)@\spxentry{\_\_init\_\_()}\spxextra{birdears.questions.melodicinterval.MelodicIntervalQuestion method}}

\begin{fulllineitems}
\phantomsection\label{\detokenize{birdears.questions:birdears.questions.melodicinterval.MelodicIntervalQuestion.__init__}}\pysiglinewithargsret{\sphinxbfcode{\sphinxupquote{\_\_init\_\_}}}{\emph{\DUrole{n}{mode}\DUrole{o}{=}\DUrole{default_value}{\textquotesingle{}major\textquotesingle{}}}, \emph{\DUrole{n}{tonic}\DUrole{o}{=}\DUrole{default_value}{\textquotesingle{}C\textquotesingle{}}}, \emph{\DUrole{n}{octave}\DUrole{o}{=}\DUrole{default_value}{4}}, \emph{\DUrole{n}{descending}\DUrole{o}{=}\DUrole{default_value}{False}}, \emph{\DUrole{n}{chromatic}\DUrole{o}{=}\DUrole{default_value}{False}}, \emph{\DUrole{n}{n\_octaves}\DUrole{o}{=}\DUrole{default_value}{1}}, \emph{\DUrole{n}{valid\_intervals}\DUrole{o}{=}\DUrole{default_value}{(0, 1, 2, 3, 4, 5, 6, 7, 8, 9, 10, 11)}}, \emph{\DUrole{n}{user\_durations}\DUrole{o}{=}\DUrole{default_value}{None}}, \emph{\DUrole{n}{prequestion\_method}\DUrole{o}{=}\DUrole{default_value}{\textquotesingle{}tonic\_only\textquotesingle{}}}, \emph{\DUrole{n}{resolution\_method}\DUrole{o}{=}\DUrole{default_value}{\textquotesingle{}nearest\_tonic\textquotesingle{}}}, \emph{\DUrole{o}{*}\DUrole{n}{args}}, \emph{\DUrole{o}{**}\DUrole{n}{kwargs}}}{}
\sphinxAtStartPar
Inits the class.
\begin{quote}\begin{description}
\item[{Parameters}] \leavevmode\begin{itemize}
\item {} 
\sphinxAtStartPar
\sphinxstyleliteralstrong{\sphinxupquote{mode}} (\sphinxstyleliteralemphasis{\sphinxupquote{str}}) \textendash{} A string representing the mode of the question.
Eg., ‘major’ or ‘minor’

\item {} 
\sphinxAtStartPar
\sphinxstyleliteralstrong{\sphinxupquote{tonic}} (\sphinxstyleliteralemphasis{\sphinxupquote{str}}) \textendash{} A string representing the tonic of the question,
eg.: ‘C’; if omitted, it will be selected randomly.

\item {} 
\sphinxAtStartPar
\sphinxstyleliteralstrong{\sphinxupquote{octave}} (\sphinxstyleliteralemphasis{\sphinxupquote{int}}) \textendash{} A scienfic octave notation, for example, 4 for ‘C4’;
if not present, it will be randomly chosen.

\item {} 
\sphinxAtStartPar
\sphinxstyleliteralstrong{\sphinxupquote{descending}} (\sphinxstyleliteralemphasis{\sphinxupquote{bool}}) \textendash{} Is the question direction in descending, ie.,
intervals have lower pitch than the tonic.

\item {} 
\sphinxAtStartPar
\sphinxstyleliteralstrong{\sphinxupquote{chromatic}} (\sphinxstyleliteralemphasis{\sphinxupquote{bool}}) \textendash{} If the question can have (True) or not (False)
chromatic intervals, ie., intervals not in the diatonic scale
of tonic/mode.

\item {} 
\sphinxAtStartPar
\sphinxstyleliteralstrong{\sphinxupquote{n\_octaves}} (\sphinxstyleliteralemphasis{\sphinxupquote{int}}) \textendash{} Maximum number of octaves of the question.

\item {} 
\sphinxAtStartPar
\sphinxstyleliteralstrong{\sphinxupquote{valid\_intervals}} (\sphinxstyleliteralemphasis{\sphinxupquote{list}}) \textendash{} A list with intervals (int) valid for
random choice, 1 is 1st, 2 is second etc. Eg. {[}1, 4, 5{]} to
allow only tonics, fourths and fifths.

\item {} 
\sphinxAtStartPar
\sphinxstyleliteralstrong{\sphinxupquote{user\_durations}} (\sphinxstyleliteralemphasis{\sphinxupquote{str}}) \textendash{} 
\sphinxAtStartPar
A string with 9 comma\sphinxhyphen{}separated \sphinxtitleref{int} or
\sphinxtitleref{float\textasciigrave{}s to set the default duration for the notes played. The
values are respectively for: pre\sphinxhyphen{}question duration (1st),
pre\sphinxhyphen{}question delay (2nd), and pre\sphinxhyphen{}question pos\sphinxhyphen{}delay (3rd);
question duration (4th), question delay (5th), and question
pos\sphinxhyphen{}delay (6th); resolution duration (7th), resolution
delay (8th), and resolution pos\sphinxhyphen{}delay (9th).
duration is the duration in of the note in seconds; delay is
the time to wait before playing the next note, and pos\_delay is
the time to wait after all the notes of the respective sequence
have been played. If any of the user durations is \textasciigrave{}n}, the
default duration for the type of question will be used instead.
Example:

\begin{sphinxVerbatim}[commandchars=\\\{\}]
\PYGZdq{}2,0.5,1,2,n,0,2.5,n,1\PYGZdq{}
\end{sphinxVerbatim}


\item {} 
\sphinxAtStartPar
\sphinxstyleliteralstrong{\sphinxupquote{prequestion\_method}} (\sphinxstyleliteralemphasis{\sphinxupquote{str}}) \textendash{} Method of playing a cadence or the
exercise tonic before the question so to affirm the question
musical tonic key to the ear. Valid ones are registered in the
\sphinxtitleref{birdears.prequestion.PREQUESION\_METHODS} global dict.

\item {} 
\sphinxAtStartPar
\sphinxstyleliteralstrong{\sphinxupquote{resolution\_method}} (\sphinxstyleliteralemphasis{\sphinxupquote{str}}) \textendash{} Method of playing the resolution of an
exercise. Valid ones are registered in the
\sphinxtitleref{birdears.resolution.RESOLUTION\_METHODS} global dict.

\end{itemize}

\end{description}\end{quote}

\end{fulllineitems}

\index{check\_question() (birdears.questions.melodicinterval.MelodicIntervalQuestion method)@\spxentry{check\_question()}\spxextra{birdears.questions.melodicinterval.MelodicIntervalQuestion method}}

\begin{fulllineitems}
\phantomsection\label{\detokenize{birdears.questions:birdears.questions.melodicinterval.MelodicIntervalQuestion.check_question}}\pysiglinewithargsret{\sphinxbfcode{\sphinxupquote{check\_question}}}{\emph{\DUrole{n}{user\_input\_char}}}{}
\sphinxAtStartPar
Checks whether the given answer is correct.

\end{fulllineitems}

\index{make\_pre\_question() (birdears.questions.melodicinterval.MelodicIntervalQuestion method)@\spxentry{make\_pre\_question()}\spxextra{birdears.questions.melodicinterval.MelodicIntervalQuestion method}}

\begin{fulllineitems}
\phantomsection\label{\detokenize{birdears.questions:birdears.questions.melodicinterval.MelodicIntervalQuestion.make_pre_question}}\pysiglinewithargsret{\sphinxbfcode{\sphinxupquote{make\_pre\_question}}}{\emph{\DUrole{n}{method}}}{}
\end{fulllineitems}

\index{make\_question() (birdears.questions.melodicinterval.MelodicIntervalQuestion method)@\spxentry{make\_question()}\spxextra{birdears.questions.melodicinterval.MelodicIntervalQuestion method}}

\begin{fulllineitems}
\phantomsection\label{\detokenize{birdears.questions:birdears.questions.melodicinterval.MelodicIntervalQuestion.make_question}}\pysiglinewithargsret{\sphinxbfcode{\sphinxupquote{make\_question}}}{}{}
\sphinxAtStartPar
This method should be overwritten by the question subclasses.

\end{fulllineitems}

\index{make\_resolution() (birdears.questions.melodicinterval.MelodicIntervalQuestion method)@\spxentry{make\_resolution()}\spxextra{birdears.questions.melodicinterval.MelodicIntervalQuestion method}}

\begin{fulllineitems}
\phantomsection\label{\detokenize{birdears.questions:birdears.questions.melodicinterval.MelodicIntervalQuestion.make_resolution}}\pysiglinewithargsret{\sphinxbfcode{\sphinxupquote{make\_resolution}}}{\emph{\DUrole{n}{method}}}{}
\sphinxAtStartPar
This method should be overwritten by the question subclasses.

\end{fulllineitems}

\index{name (birdears.questions.melodicinterval.MelodicIntervalQuestion attribute)@\spxentry{name}\spxextra{birdears.questions.melodicinterval.MelodicIntervalQuestion attribute}}

\begin{fulllineitems}
\phantomsection\label{\detokenize{birdears.questions:birdears.questions.melodicinterval.MelodicIntervalQuestion.name}}\pysigline{\sphinxbfcode{\sphinxupquote{name}}\sphinxbfcode{\sphinxupquote{ = \textquotesingle{}melodic\textquotesingle{}}}}
\end{fulllineitems}

\index{play\_question() (birdears.questions.melodicinterval.MelodicIntervalQuestion method)@\spxentry{play\_question()}\spxextra{birdears.questions.melodicinterval.MelodicIntervalQuestion method}}

\begin{fulllineitems}
\phantomsection\label{\detokenize{birdears.questions:birdears.questions.melodicinterval.MelodicIntervalQuestion.play_question}}\pysiglinewithargsret{\sphinxbfcode{\sphinxupquote{play\_question}}}{\emph{\DUrole{n}{callback}\DUrole{o}{=}\DUrole{default_value}{None}}, \emph{\DUrole{n}{end\_callback}\DUrole{o}{=}\DUrole{default_value}{None}}, \emph{\DUrole{o}{*}\DUrole{n}{args}}, \emph{\DUrole{o}{**}\DUrole{n}{kwargs}}}{}
\sphinxAtStartPar
This method should be overwritten by the question subclasses.

\end{fulllineitems}

\index{play\_resolution() (birdears.questions.melodicinterval.MelodicIntervalQuestion method)@\spxentry{play\_resolution()}\spxextra{birdears.questions.melodicinterval.MelodicIntervalQuestion method}}

\begin{fulllineitems}
\phantomsection\label{\detokenize{birdears.questions:birdears.questions.melodicinterval.MelodicIntervalQuestion.play_resolution}}\pysiglinewithargsret{\sphinxbfcode{\sphinxupquote{play\_resolution}}}{\emph{\DUrole{n}{callback}\DUrole{o}{=}\DUrole{default_value}{None}}, \emph{\DUrole{n}{end\_callback}\DUrole{o}{=}\DUrole{default_value}{None}}, \emph{\DUrole{o}{*}\DUrole{n}{args}}, \emph{\DUrole{o}{**}\DUrole{n}{kwargs}}}{}
\end{fulllineitems}


\end{fulllineitems}



\section{Submodules}
\label{\detokenize{birdears:submodules}}

\section{birdears.interval module}
\label{\detokenize{birdears:module-birdears.interval}}\label{\detokenize{birdears:birdears-interval-module}}\index{module@\spxentry{module}!birdears.interval@\spxentry{birdears.interval}}\index{birdears.interval@\spxentry{birdears.interval}!module@\spxentry{module}}\index{Interval (class in birdears.interval)@\spxentry{Interval}\spxextra{class in birdears.interval}}

\begin{fulllineitems}
\phantomsection\label{\detokenize{birdears:birdears.interval.Interval}}\pysiglinewithargsret{\sphinxbfcode{\sphinxupquote{class }}\sphinxcode{\sphinxupquote{birdears.interval.}}\sphinxbfcode{\sphinxupquote{Interval}}}{\emph{\DUrole{n}{pitch\_a}}, \emph{\DUrole{n}{pitch\_b}}}{}
\sphinxAtStartPar
Bases: \sphinxcode{\sphinxupquote{dict}}

\sphinxAtStartPar
This class represents the interval between two pitches..
\index{tonic\_octave (birdears.interval.Interval attribute)@\spxentry{tonic\_octave}\spxextra{birdears.interval.Interval attribute}}

\begin{fulllineitems}
\phantomsection\label{\detokenize{birdears:birdears.interval.Interval.tonic_octave}}\pysigline{\sphinxbfcode{\sphinxupquote{tonic\_octave}}}
\sphinxAtStartPar
Scientific octave for the tonic. For example, if
the tonic is a ‘C4’ then \sphinxtitleref{tonic\_octave} is 4.
\begin{quote}\begin{description}
\item[{Type}] \leavevmode
\sphinxAtStartPar
int

\end{description}\end{quote}

\end{fulllineitems}



\begin{fulllineitems}
\pysigline{\sphinxbfcode{\sphinxupquote{interval~octave}}}
\sphinxAtStartPar
Scientific octave for the interval. For example,
if the interval is a ‘G5’ then \sphinxtitleref{tonic\_octave} is 5.
\begin{quote}\begin{description}
\item[{Type}] \leavevmode
\sphinxAtStartPar
int

\end{description}\end{quote}

\end{fulllineitems}

\index{chromatic\_offset (birdears.interval.Interval attribute)@\spxentry{chromatic\_offset}\spxextra{birdears.interval.Interval attribute}}

\begin{fulllineitems}
\phantomsection\label{\detokenize{birdears:birdears.interval.Interval.chromatic_offset}}\pysigline{\sphinxbfcode{\sphinxupquote{chromatic\_offset}}}
\sphinxAtStartPar
The offset in semitones inside one octave.
Relative semitones to tonic.
\begin{quote}\begin{description}
\item[{Type}] \leavevmode
\sphinxAtStartPar
int

\end{description}\end{quote}

\end{fulllineitems}

\index{note\_and\_octave (birdears.interval.Interval attribute)@\spxentry{note\_and\_octave}\spxextra{birdears.interval.Interval attribute}}

\begin{fulllineitems}
\phantomsection\label{\detokenize{birdears:birdears.interval.Interval.note_and_octave}}\pysigline{\sphinxbfcode{\sphinxupquote{note\_and\_octave}}}
\sphinxAtStartPar
Note and octave of the interval, for example, if
the interval is G5 the note name is ‘G5’.
\begin{quote}\begin{description}
\item[{Type}] \leavevmode
\sphinxAtStartPar
str

\end{description}\end{quote}

\end{fulllineitems}

\index{note\_name (birdears.interval.Interval attribute)@\spxentry{note\_name}\spxextra{birdears.interval.Interval attribute}}

\begin{fulllineitems}
\phantomsection\label{\detokenize{birdears:birdears.interval.Interval.note_name}}\pysigline{\sphinxbfcode{\sphinxupquote{note\_name}}}
\sphinxAtStartPar
The note name of the interval, for example, if the
interval is G5 then the name is ‘G’.
\begin{quote}\begin{description}
\item[{Type}] \leavevmode
\sphinxAtStartPar
str

\end{description}\end{quote}

\end{fulllineitems}

\index{semitones (birdears.interval.Interval attribute)@\spxentry{semitones}\spxextra{birdears.interval.Interval attribute}}

\begin{fulllineitems}
\phantomsection\label{\detokenize{birdears:birdears.interval.Interval.semitones}}\pysigline{\sphinxbfcode{\sphinxupquote{semitones}}}
\sphinxAtStartPar
Semitones from tonic to octave. If tonic is C4 and
interval is G5 the number of semitones is 19.
\begin{quote}\begin{description}
\item[{Type}] \leavevmode
\sphinxAtStartPar
int

\end{description}\end{quote}

\end{fulllineitems}

\index{is\_chromatic (birdears.interval.Interval attribute)@\spxentry{is\_chromatic}\spxextra{birdears.interval.Interval attribute}}

\begin{fulllineitems}
\phantomsection\label{\detokenize{birdears:birdears.interval.Interval.is_chromatic}}\pysigline{\sphinxbfcode{\sphinxupquote{is\_chromatic}}}
\sphinxAtStartPar
If the current interval is chromatic (True) or if
it exists in the diatonic scale which key is tonic.
\begin{quote}\begin{description}
\item[{Type}] \leavevmode
\sphinxAtStartPar
bool

\end{description}\end{quote}

\end{fulllineitems}

\index{is\_descending (birdears.interval.Interval attribute)@\spxentry{is\_descending}\spxextra{birdears.interval.Interval attribute}}

\begin{fulllineitems}
\phantomsection\label{\detokenize{birdears:birdears.interval.Interval.is_descending}}\pysigline{\sphinxbfcode{\sphinxupquote{is\_descending}}}
\sphinxAtStartPar
If the interval has a descending direction, ie.,
has a lower pitch than the tonic.
\begin{quote}\begin{description}
\item[{Type}] \leavevmode
\sphinxAtStartPar
bool

\end{description}\end{quote}

\end{fulllineitems}

\index{diatonic\_index (birdears.interval.Interval attribute)@\spxentry{diatonic\_index}\spxextra{birdears.interval.Interval attribute}}

\begin{fulllineitems}
\phantomsection\label{\detokenize{birdears:birdears.interval.Interval.diatonic_index}}\pysigline{\sphinxbfcode{\sphinxupquote{diatonic\_index}}}
\sphinxAtStartPar
If the interval is chromatic, this will be the
nearest diatonic interval in the direction of the resolution
(closest tonic.) From II to IV degrees, it is the ditonic interval
before; from V to VII it is the diatonic interval after.
\begin{quote}\begin{description}
\item[{Type}] \leavevmode
\sphinxAtStartPar
int

\end{description}\end{quote}

\end{fulllineitems}

\index{distance (birdears.interval.Interval attribute)@\spxentry{distance}\spxextra{birdears.interval.Interval attribute}}

\begin{fulllineitems}
\phantomsection\label{\detokenize{birdears:birdears.interval.Interval.distance}}\pysigline{\sphinxbfcode{\sphinxupquote{distance}}}
\sphinxAtStartPar
A dictionary which the distance from tonic to
interval, for example, if tonic is C4 and interval is G5:

\begin{sphinxVerbatim}[commandchars=\\\{\}]
\PYGZob{}
    \PYGZsq{}octaves\PYGZsq{}: 1,
    \PYGZsq{}semitones\PYGZsq{}: 7
\PYGZcb{}
\end{sphinxVerbatim}
\begin{quote}\begin{description}
\item[{Type}] \leavevmode
\sphinxAtStartPar
dict

\end{description}\end{quote}

\end{fulllineitems}

\index{data (birdears.interval.Interval attribute)@\spxentry{data}\spxextra{birdears.interval.Interval attribute}}

\begin{fulllineitems}
\phantomsection\label{\detokenize{birdears:birdears.interval.Interval.data}}\pysigline{\sphinxbfcode{\sphinxupquote{data}}}
\sphinxAtStartPar
A tuple representing the interval data in the form of
(semitones, short\_name, long\_name), for example:

\begin{sphinxVerbatim}[commandchars=\\\{\}]
(19, \PYGZsq{}P12\PYGZsq{}, \PYGZsq{}Perfect Twelfth\PYGZsq{})
\end{sphinxVerbatim}
\begin{quote}\begin{description}
\item[{Type}] \leavevmode
\sphinxAtStartPar
tuple

\end{description}\end{quote}

\end{fulllineitems}

\index{\_\_init\_\_() (birdears.interval.Interval method)@\spxentry{\_\_init\_\_()}\spxextra{birdears.interval.Interval method}}

\begin{fulllineitems}
\phantomsection\label{\detokenize{birdears:birdears.interval.Interval.__init__}}\pysiglinewithargsret{\sphinxbfcode{\sphinxupquote{\_\_init\_\_}}}{\emph{\DUrole{n}{pitch\_a}}, \emph{\DUrole{n}{pitch\_b}}}{}
\sphinxAtStartPar
Measures the musical interval from pitch\_a to pitch\_b.
\begin{quote}\begin{description}
\item[{Parameters}] \leavevmode\begin{itemize}
\item {} 
\sphinxAtStartPar
\sphinxstyleliteralstrong{\sphinxupquote{pitch\_a}} (\sphinxstyleliteralemphasis{\sphinxupquote{str}}) \textendash{} First \sphinxtitleref{Pitch} object to be measured.

\item {} 
\sphinxAtStartPar
\sphinxstyleliteralstrong{\sphinxupquote{pitch\_b}} (\sphinxstyleliteralemphasis{\sphinxupquote{str}}) \textendash{} Second \sphinxtitleref{Pitch} object to be measured.

\end{itemize}

\end{description}\end{quote}

\end{fulllineitems}


\end{fulllineitems}

\index{get\_interval\_by\_semitones() (in module birdears.interval)@\spxentry{get\_interval\_by\_semitones()}\spxextra{in module birdears.interval}}

\begin{fulllineitems}
\phantomsection\label{\detokenize{birdears:birdears.interval.get_interval_by_semitones}}\pysiglinewithargsret{\sphinxcode{\sphinxupquote{birdears.interval.}}\sphinxbfcode{\sphinxupquote{get\_interval\_by\_semitones}}}{\emph{\DUrole{n}{semitones}}}{}
\end{fulllineitems}



\section{birdears.logger module}
\label{\detokenize{birdears:module-birdears.logger}}\label{\detokenize{birdears:birdears-logger-module}}\index{module@\spxentry{module}!birdears.logger@\spxentry{birdears.logger}}\index{birdears.logger@\spxentry{birdears.logger}!module@\spxentry{module}}
\sphinxAtStartPar
This submodule exports \sphinxtitleref{logger} to log events.

\sphinxAtStartPar
Logging messages which are less severe than \sphinxtitleref{lvl} will be ignored:

\begin{sphinxVerbatim}[commandchars=\\\{\}]
Level       Numeric value
\PYGZhy{}\PYGZhy{}\PYGZhy{}\PYGZhy{}\PYGZhy{}       \PYGZhy{}\PYGZhy{}\PYGZhy{}\PYGZhy{}\PYGZhy{}\PYGZhy{}\PYGZhy{}\PYGZhy{}\PYGZhy{}\PYGZhy{}\PYGZhy{}\PYGZhy{}\PYGZhy{}
CRITICAL    50
ERROR       40
WARNING     30
INFO        20
DEBUG       10
NOTSET      0

Level       When it’s used
\PYGZhy{}\PYGZhy{}\PYGZhy{}\PYGZhy{}\PYGZhy{}       \PYGZhy{}\PYGZhy{}\PYGZhy{}\PYGZhy{}\PYGZhy{}\PYGZhy{}\PYGZhy{}\PYGZhy{}\PYGZhy{}\PYGZhy{}\PYGZhy{}\PYGZhy{}\PYGZhy{}\PYGZhy{}
DEBUG       Detailed information, typically of interest only when
                diagnosing problems.
INFO        Confirmation that things are working as expected.
WARNING     An indication that something unexpected happened, or indicative
                of some problem in the near future (e.g. ‘disk space low’).
                The software is still working as expected.
ERROR       Due to a more serious problem, the software has not been able
                to perform some function.
CRITICAL    A serious error, indicating that the program itself may be
                unable to continue running.
\end{sphinxVerbatim}
\index{log\_event() (in module birdears.logger)@\spxentry{log\_event()}\spxextra{in module birdears.logger}}

\begin{fulllineitems}
\phantomsection\label{\detokenize{birdears:birdears.logger.log_event}}\pysiglinewithargsret{\sphinxcode{\sphinxupquote{birdears.logger.}}\sphinxbfcode{\sphinxupquote{log\_event}}}{\emph{\DUrole{n}{f}}, \emph{\DUrole{o}{*}\DUrole{n}{args}}, \emph{\DUrole{o}{**}\DUrole{n}{kwargs}}}{}
\sphinxAtStartPar
Decorator. Functions and method decorated with this decorator will have
their signature logged when birdears is executed with \sphinxtitleref{\textendash{}debug} mode. Both
function signature with their call values and their return will be logged.

\end{fulllineitems}



\section{birdears.prequestion module}
\label{\detokenize{birdears:module-birdears.prequestion}}\label{\detokenize{birdears:birdears-prequestion-module}}\index{module@\spxentry{module}!birdears.prequestion@\spxentry{birdears.prequestion}}\index{birdears.prequestion@\spxentry{birdears.prequestion}!module@\spxentry{module}}
\sphinxAtStartPar
This module implements pre\sphinxhyphen{}questions’ progressions.

\sphinxAtStartPar
Pre questions are chord progressions or notes played before the question is
played, so to affirmate the sound of the question’s key.

\sphinxAtStartPar
For example a common cadence is chords I\sphinxhyphen{}IV\sphinxhyphen{}V\sphinxhyphen{}I from the diatonic scale, which
in a key of \sphinxtitleref{C} is \sphinxtitleref{CM\sphinxhyphen{}FM\sphinxhyphen{}GM\sphinxhyphen{}CM} and in a key of \sphinxtitleref{A} is \sphinxtitleref{AM\sphinxhyphen{}DM\sphinxhyphen{}EM\sphinxhyphen{}AM}.

\sphinxAtStartPar
Pre\sphinxhyphen{}question methods should be decorated with \sphinxtitleref{register\_prequestion\_method}
decorator, so that they will be registered as a valid pre\sphinxhyphen{}question method.
\index{PreQuestion (class in birdears.prequestion)@\spxentry{PreQuestion}\spxextra{class in birdears.prequestion}}

\begin{fulllineitems}
\phantomsection\label{\detokenize{birdears:birdears.prequestion.PreQuestion}}\pysiglinewithargsret{\sphinxbfcode{\sphinxupquote{class }}\sphinxcode{\sphinxupquote{birdears.prequestion.}}\sphinxbfcode{\sphinxupquote{PreQuestion}}}{\emph{\DUrole{n}{method}}, \emph{\DUrole{n}{question}}}{}
\sphinxAtStartPar
Bases: \sphinxcode{\sphinxupquote{object}}
\index{\_\_call\_\_() (birdears.prequestion.PreQuestion method)@\spxentry{\_\_call\_\_()}\spxextra{birdears.prequestion.PreQuestion method}}

\begin{fulllineitems}
\phantomsection\label{\detokenize{birdears:birdears.prequestion.PreQuestion.__call__}}\pysiglinewithargsret{\sphinxbfcode{\sphinxupquote{\_\_call\_\_}}}{\emph{\DUrole{o}{*}\DUrole{n}{args}}, \emph{\DUrole{o}{**}\DUrole{n}{kwargs}}}{}
\sphinxAtStartPar
Calls the resolution method and pass arguments to it.

\sphinxAtStartPar
Returns a \sphinxtitleref{birdears.Sequence} object with the pre\sphinxhyphen{}question generated by
the method.

\end{fulllineitems}

\index{\_\_init\_\_() (birdears.prequestion.PreQuestion method)@\spxentry{\_\_init\_\_()}\spxextra{birdears.prequestion.PreQuestion method}}

\begin{fulllineitems}
\phantomsection\label{\detokenize{birdears:birdears.prequestion.PreQuestion.__init__}}\pysiglinewithargsret{\sphinxbfcode{\sphinxupquote{\_\_init\_\_}}}{\emph{\DUrole{n}{method}}, \emph{\DUrole{n}{question}}}{}
\sphinxAtStartPar
This class implements methods for different types of pre\sphinxhyphen{}question
progressions.
\begin{quote}\begin{description}
\item[{Parameters}] \leavevmode\begin{itemize}
\item {} 
\sphinxAtStartPar
\sphinxstyleliteralstrong{\sphinxupquote{method}} (\sphinxstyleliteralemphasis{\sphinxupquote{str}}) \textendash{} The method used in the pre question.

\item {} 
\sphinxAtStartPar
\sphinxstyleliteralstrong{\sphinxupquote{question}} (\sphinxstyleliteralemphasis{\sphinxupquote{obj}}) \textendash{} Question object from which to generate the

\item {} 
\sphinxAtStartPar
\sphinxstyleliteralstrong{\sphinxupquote{sequence.}} (\sphinxstyleliteralemphasis{\sphinxupquote{pre\sphinxhyphen{}question}}) \textendash{} 

\end{itemize}

\end{description}\end{quote}

\end{fulllineitems}


\end{fulllineitems}

\index{none() (in module birdears.prequestion)@\spxentry{none()}\spxextra{in module birdears.prequestion}}

\begin{fulllineitems}
\phantomsection\label{\detokenize{birdears:birdears.prequestion.none}}\pysiglinewithargsret{\sphinxcode{\sphinxupquote{birdears.prequestion.}}\sphinxbfcode{\sphinxupquote{none}}}{\emph{\DUrole{n}{question}}, \emph{\DUrole{o}{*}\DUrole{n}{args}}, \emph{\DUrole{o}{**}\DUrole{n}{kwargs}}}{}
\sphinxAtStartPar
Pre\sphinxhyphen{}question method that return an empty sequence with no delay.
:param question: Question object from which to generate the
\begin{quote}

\sphinxAtStartPar
pre\sphinxhyphen{}question sequence. (this is provided by the \sphinxtitleref{Resolution} class
when it is {\color{red}\bfseries{}\textasciigrave{}}\_\_call\_\_\textasciigrave{}ed)
\end{quote}
\begin{quote}\begin{description}
\end{description}\end{quote}

\end{fulllineitems}

\index{progression\_i\_iv\_v\_i() (in module birdears.prequestion)@\spxentry{progression\_i\_iv\_v\_i()}\spxextra{in module birdears.prequestion}}

\begin{fulllineitems}
\phantomsection\label{\detokenize{birdears:birdears.prequestion.progression_i_iv_v_i}}\pysiglinewithargsret{\sphinxcode{\sphinxupquote{birdears.prequestion.}}\sphinxbfcode{\sphinxupquote{progression\_i\_iv\_v\_i}}}{\emph{\DUrole{n}{question}}, \emph{\DUrole{o}{*}\DUrole{n}{args}}, \emph{\DUrole{o}{**}\DUrole{n}{kwargs}}}{}
\sphinxAtStartPar
Pre\sphinxhyphen{}question method that play’s a chord progression with triad chords
built on the grades I, IV, V the I of the question key.
\begin{quote}\begin{description}
\item[{Parameters}] \leavevmode
\sphinxAtStartPar
\sphinxstyleliteralstrong{\sphinxupquote{question}} (\sphinxstyleliteralemphasis{\sphinxupquote{obj}}) \textendash{} Question object from which to generate the
pre\sphinxhyphen{}question sequence. (this is provided by the \sphinxtitleref{Resolution} class
when it is {\color{red}\bfseries{}\textasciigrave{}}\_\_call\_\_\textasciigrave{}ed)

\end{description}\end{quote}

\end{fulllineitems}

\index{register\_prequestion\_method() (in module birdears.prequestion)@\spxentry{register\_prequestion\_method()}\spxextra{in module birdears.prequestion}}

\begin{fulllineitems}
\phantomsection\label{\detokenize{birdears:birdears.prequestion.register_prequestion_method}}\pysiglinewithargsret{\sphinxcode{\sphinxupquote{birdears.prequestion.}}\sphinxbfcode{\sphinxupquote{register\_prequestion\_method}}}{\emph{\DUrole{n}{f}}, \emph{\DUrole{o}{*}\DUrole{n}{args}}, \emph{\DUrole{o}{**}\DUrole{n}{kwargs}}}{}
\sphinxAtStartPar
Decorator for prequestion method functions.

\sphinxAtStartPar
Functions decorated with this decorator will be registered in the
\sphinxtitleref{PREQUESTION\_METHODS} global dict.

\end{fulllineitems}

\index{tonic\_only() (in module birdears.prequestion)@\spxentry{tonic\_only()}\spxextra{in module birdears.prequestion}}

\begin{fulllineitems}
\phantomsection\label{\detokenize{birdears:birdears.prequestion.tonic_only}}\pysiglinewithargsret{\sphinxcode{\sphinxupquote{birdears.prequestion.}}\sphinxbfcode{\sphinxupquote{tonic\_only}}}{\emph{\DUrole{n}{question}}, \emph{\DUrole{o}{*}\DUrole{n}{args}}, \emph{\DUrole{o}{**}\DUrole{n}{kwargs}}}{}
\sphinxAtStartPar
Pre\sphinxhyphen{}question method that only play’s the question tonic note before the
question.
\begin{quote}\begin{description}
\item[{Parameters}] \leavevmode
\sphinxAtStartPar
\sphinxstyleliteralstrong{\sphinxupquote{question}} (\sphinxstyleliteralemphasis{\sphinxupquote{object}}) \textendash{} Question object from which to generate the
pre\sphinxhyphen{}question sequence. (this is provided by the \sphinxtitleref{Resolution} class
when it is {\color{red}\bfseries{}\textasciigrave{}}\_\_call\_\_\textasciigrave{}ed)

\end{description}\end{quote}

\end{fulllineitems}



\section{birdears.questionbase module}
\label{\detokenize{birdears:module-birdears.questionbase}}\label{\detokenize{birdears:birdears-questionbase-module}}\index{module@\spxentry{module}!birdears.questionbase@\spxentry{birdears.questionbase}}\index{birdears.questionbase@\spxentry{birdears.questionbase}!module@\spxentry{module}}\index{QuestionBase (class in birdears.questionbase)@\spxentry{QuestionBase}\spxextra{class in birdears.questionbase}}

\begin{fulllineitems}
\phantomsection\label{\detokenize{birdears:birdears.questionbase.QuestionBase}}\pysiglinewithargsret{\sphinxbfcode{\sphinxupquote{class }}\sphinxcode{\sphinxupquote{birdears.questionbase.}}\sphinxbfcode{\sphinxupquote{QuestionBase}}}{\emph{\DUrole{n}{mode}\DUrole{o}{=}\DUrole{default_value}{\textquotesingle{}major\textquotesingle{}}}, \emph{\DUrole{n}{tonic}\DUrole{o}{=}\DUrole{default_value}{\textquotesingle{}C\textquotesingle{}}}, \emph{\DUrole{n}{octave}\DUrole{o}{=}\DUrole{default_value}{4}}, \emph{\DUrole{n}{descending}\DUrole{o}{=}\DUrole{default_value}{False}}, \emph{\DUrole{n}{chromatic}\DUrole{o}{=}\DUrole{default_value}{False}}, \emph{\DUrole{n}{n\_octaves}\DUrole{o}{=}\DUrole{default_value}{1}}, \emph{\DUrole{n}{valid\_intervals}\DUrole{o}{=}\DUrole{default_value}{(0, 1, 2, 3, 4, 5, 6, 7, 8, 9, 10, 11)}}, \emph{\DUrole{n}{user\_durations}\DUrole{o}{=}\DUrole{default_value}{None}}, \emph{\DUrole{n}{prequestion\_method}\DUrole{o}{=}\DUrole{default_value}{None}}, \emph{\DUrole{n}{resolution\_method}\DUrole{o}{=}\DUrole{default_value}{None}}, \emph{\DUrole{n}{default\_durations}\DUrole{o}{=}\DUrole{default_value}{None}}, \emph{\DUrole{o}{*}\DUrole{n}{args}}, \emph{\DUrole{o}{**}\DUrole{n}{kwargs}}}{}
\sphinxAtStartPar
Bases: \sphinxcode{\sphinxupquote{object}}

\sphinxAtStartPar
Base Class to be subclassed for Question classes.

\sphinxAtStartPar
This class implements attributes and routines to be used in Question
subclasses.
\index{\_\_init\_\_() (birdears.questionbase.QuestionBase method)@\spxentry{\_\_init\_\_()}\spxextra{birdears.questionbase.QuestionBase method}}

\begin{fulllineitems}
\phantomsection\label{\detokenize{birdears:birdears.questionbase.QuestionBase.__init__}}\pysiglinewithargsret{\sphinxbfcode{\sphinxupquote{\_\_init\_\_}}}{\emph{\DUrole{n}{mode}\DUrole{o}{=}\DUrole{default_value}{\textquotesingle{}major\textquotesingle{}}}, \emph{\DUrole{n}{tonic}\DUrole{o}{=}\DUrole{default_value}{\textquotesingle{}C\textquotesingle{}}}, \emph{\DUrole{n}{octave}\DUrole{o}{=}\DUrole{default_value}{4}}, \emph{\DUrole{n}{descending}\DUrole{o}{=}\DUrole{default_value}{False}}, \emph{\DUrole{n}{chromatic}\DUrole{o}{=}\DUrole{default_value}{False}}, \emph{\DUrole{n}{n\_octaves}\DUrole{o}{=}\DUrole{default_value}{1}}, \emph{\DUrole{n}{valid\_intervals}\DUrole{o}{=}\DUrole{default_value}{(0, 1, 2, 3, 4, 5, 6, 7, 8, 9, 10, 11)}}, \emph{\DUrole{n}{user\_durations}\DUrole{o}{=}\DUrole{default_value}{None}}, \emph{\DUrole{n}{prequestion\_method}\DUrole{o}{=}\DUrole{default_value}{None}}, \emph{\DUrole{n}{resolution\_method}\DUrole{o}{=}\DUrole{default_value}{None}}, \emph{\DUrole{n}{default\_durations}\DUrole{o}{=}\DUrole{default_value}{None}}, \emph{\DUrole{o}{*}\DUrole{n}{args}}, \emph{\DUrole{o}{**}\DUrole{n}{kwargs}}}{}
\sphinxAtStartPar
Inits the class.
\begin{quote}\begin{description}
\item[{Parameters}] \leavevmode\begin{itemize}
\item {} 
\sphinxAtStartPar
\sphinxstyleliteralstrong{\sphinxupquote{mode}} (\sphinxstyleliteralemphasis{\sphinxupquote{str}}) \textendash{} A string represnting the mode of the question.
Eg., ‘major’ or ‘minor’

\item {} 
\sphinxAtStartPar
\sphinxstyleliteralstrong{\sphinxupquote{tonic}} (\sphinxstyleliteralemphasis{\sphinxupquote{str}}) \textendash{} A string representing the tonic of the
question, eg.: ‘C’; if omitted, it will be selected
randomly.

\item {} 
\sphinxAtStartPar
\sphinxstyleliteralstrong{\sphinxupquote{octave}} (\sphinxstyleliteralemphasis{\sphinxupquote{int}}) \textendash{} A scienfic octave notation, for example,
4 for ‘C4’; if not present, it will be randomly chosen.

\item {} 
\sphinxAtStartPar
\sphinxstyleliteralstrong{\sphinxupquote{descending}} (\sphinxstyleliteralemphasis{\sphinxupquote{bool}}) \textendash{} Is the question direction in descending,
ie., intervals have lower pitch than the tonic.

\item {} 
\sphinxAtStartPar
\sphinxstyleliteralstrong{\sphinxupquote{chromatic}} (\sphinxstyleliteralemphasis{\sphinxupquote{bool}}) \textendash{} If the question can have (True) or not
(False) chromatic intervals, ie., intervals not in the
diatonic scale of tonic/mode.

\item {} 
\sphinxAtStartPar
\sphinxstyleliteralstrong{\sphinxupquote{n\_octaves}} (\sphinxstyleliteralemphasis{\sphinxupquote{int}}) \textendash{} Maximum numbr of octaves of the question.

\item {} 
\sphinxAtStartPar
\sphinxstyleliteralstrong{\sphinxupquote{valid\_intervals}} (\sphinxstyleliteralemphasis{\sphinxupquote{list}}) \textendash{} A list with intervals (int) valid for
random choice, 1 is 1st, 2 is second etc. Eg. {[}1, 4, 5{]} to
allow only tonics, fourths and fifths.

\item {} 
\sphinxAtStartPar
\sphinxstyleliteralstrong{\sphinxupquote{user\_durations}} (\sphinxstyleliteralemphasis{\sphinxupquote{dict}}) \textendash{} 
\sphinxAtStartPar
A string with 9 comma\sphinxhyphen{}separated \sphinxtitleref{int} or
\sphinxtitleref{float\textasciigrave{}s to set the default duration for the notes played. The
values are respectively for: pre\sphinxhyphen{}question duration (1st),
pre\sphinxhyphen{}question delay (2nd), and pre\sphinxhyphen{}question pos\sphinxhyphen{}delay (3rd);
question duration (4th), question delay (5th), and question
pos\sphinxhyphen{}delay (6th); resolution duration (7th), resolution
delay (8th), and resolution pos\sphinxhyphen{}delay (9th).
duration is the duration in of the note in seconds; delay is
the time to wait before playing the next note, and pos\_delay is
the time to wait after all the notes of the respective sequence
have been played. If any of the user durations is \textasciigrave{}n}, the
default duration for the type of question will be used instead.
Example:

\begin{sphinxVerbatim}[commandchars=\\\{\}]
\PYGZdq{}2,0.5,1,2,n,0,2.5,n,1\PYGZdq{}
\end{sphinxVerbatim}


\item {} 
\sphinxAtStartPar
\sphinxstyleliteralstrong{\sphinxupquote{prequestion\_method}} (\sphinxstyleliteralemphasis{\sphinxupquote{str}}) \textendash{} Method of playing a cadence or the
exercise tonic before the question so to affirm the question
musical tonic key to the ear. Valid ones are registered in the
\sphinxtitleref{birdears.prequestion.PREQUESION\_METHODS} global dict.

\item {} 
\sphinxAtStartPar
\sphinxstyleliteralstrong{\sphinxupquote{resolution\_method}} (\sphinxstyleliteralemphasis{\sphinxupquote{str}}) \textendash{} Method of playing the resolution of an
exercise Valid ones are registered in the
\sphinxtitleref{birdears.resolution.RESOLUTION\_METHODS} global dict.

\item {} 
\sphinxAtStartPar
\sphinxstyleliteralstrong{\sphinxupquote{user\_durations}} \textendash{} Dictionary with the default durations for
each type of sequence. This is provided by the subclasses.

\end{itemize}

\end{description}\end{quote}

\end{fulllineitems}

\index{check\_question() (birdears.questionbase.QuestionBase method)@\spxentry{check\_question()}\spxextra{birdears.questionbase.QuestionBase method}}

\begin{fulllineitems}
\phantomsection\label{\detokenize{birdears:birdears.questionbase.QuestionBase.check_question}}\pysiglinewithargsret{\sphinxbfcode{\sphinxupquote{check\_question}}}{}{}
\sphinxAtStartPar
This method should be overwritten by the question subclasses.

\end{fulllineitems}

\index{make\_question() (birdears.questionbase.QuestionBase method)@\spxentry{make\_question()}\spxextra{birdears.questionbase.QuestionBase method}}

\begin{fulllineitems}
\phantomsection\label{\detokenize{birdears:birdears.questionbase.QuestionBase.make_question}}\pysiglinewithargsret{\sphinxbfcode{\sphinxupquote{make\_question}}}{}{}
\sphinxAtStartPar
This method should be overwritten by the question subclasses.

\end{fulllineitems}

\index{make\_resolution() (birdears.questionbase.QuestionBase method)@\spxentry{make\_resolution()}\spxextra{birdears.questionbase.QuestionBase method}}

\begin{fulllineitems}
\phantomsection\label{\detokenize{birdears:birdears.questionbase.QuestionBase.make_resolution}}\pysiglinewithargsret{\sphinxbfcode{\sphinxupquote{make\_resolution}}}{}{}
\sphinxAtStartPar
This method should be overwritten by the question subclasses.

\end{fulllineitems}

\index{play\_question() (birdears.questionbase.QuestionBase method)@\spxentry{play\_question()}\spxextra{birdears.questionbase.QuestionBase method}}

\begin{fulllineitems}
\phantomsection\label{\detokenize{birdears:birdears.questionbase.QuestionBase.play_question}}\pysiglinewithargsret{\sphinxbfcode{\sphinxupquote{play\_question}}}{}{}
\sphinxAtStartPar
This method should be overwritten by the question subclasses.

\end{fulllineitems}


\end{fulllineitems}

\index{get\_valid\_pitches() (in module birdears.questionbase)@\spxentry{get\_valid\_pitches()}\spxextra{in module birdears.questionbase}}

\begin{fulllineitems}
\phantomsection\label{\detokenize{birdears:birdears.questionbase.get_valid_pitches}}\pysiglinewithargsret{\sphinxcode{\sphinxupquote{birdears.questionbase.}}\sphinxbfcode{\sphinxupquote{get\_valid\_pitches}}}{\emph{\DUrole{n}{scale}}, \emph{\DUrole{n}{valid\_intervals}\DUrole{o}{=}\DUrole{default_value}{(0, 1, 2, 3, 4, 5, 6, 7, 8, 9, 10, 11)}}}{}
\end{fulllineitems}

\index{register\_question\_class() (in module birdears.questionbase)@\spxentry{register\_question\_class()}\spxextra{in module birdears.questionbase}}

\begin{fulllineitems}
\phantomsection\label{\detokenize{birdears:birdears.questionbase.register_question_class}}\pysiglinewithargsret{\sphinxcode{\sphinxupquote{birdears.questionbase.}}\sphinxbfcode{\sphinxupquote{register\_question\_class}}}{\emph{\DUrole{n}{cls}}, \emph{\DUrole{o}{*}\DUrole{n}{args}}, \emph{\DUrole{o}{**}\DUrole{n}{kwargs}}}{}
\sphinxAtStartPar
Decorator for question classes.

\sphinxAtStartPar
Classes decorated with this decorator will be registered in the
\sphinxtitleref{QUESTION\_CLASSES} global.

\end{fulllineitems}



\section{birdears.resolution module}
\label{\detokenize{birdears:module-birdears.resolution}}\label{\detokenize{birdears:birdears-resolution-module}}\index{module@\spxentry{module}!birdears.resolution@\spxentry{birdears.resolution}}\index{birdears.resolution@\spxentry{birdears.resolution}!module@\spxentry{module}}\index{Resolution (class in birdears.resolution)@\spxentry{Resolution}\spxextra{class in birdears.resolution}}

\begin{fulllineitems}
\phantomsection\label{\detokenize{birdears:birdears.resolution.Resolution}}\pysiglinewithargsret{\sphinxbfcode{\sphinxupquote{class }}\sphinxcode{\sphinxupquote{birdears.resolution.}}\sphinxbfcode{\sphinxupquote{Resolution}}}{\emph{\DUrole{n}{method}}, \emph{\DUrole{n}{question}}}{}
\sphinxAtStartPar
Bases: \sphinxcode{\sphinxupquote{object}}

\sphinxAtStartPar
This class implements methods for different types of question
resolutions.

\sphinxAtStartPar
A resolution is an answer to a question. It aims to create a mnemonic on
how the inverval resvolver to the tonic.
\index{\_\_call\_\_() (birdears.resolution.Resolution method)@\spxentry{\_\_call\_\_()}\spxextra{birdears.resolution.Resolution method}}

\begin{fulllineitems}
\phantomsection\label{\detokenize{birdears:birdears.resolution.Resolution.__call__}}\pysiglinewithargsret{\sphinxbfcode{\sphinxupquote{\_\_call\_\_}}}{\emph{\DUrole{o}{*}\DUrole{n}{args}}, \emph{\DUrole{o}{**}\DUrole{n}{kwargs}}}{}
\sphinxAtStartPar
Calls the resolution method and pass arguments to it.

\sphinxAtStartPar
Returns a \sphinxtitleref{birdears.Sequence} object with the resolution generated by
the.method.

\end{fulllineitems}

\index{\_\_init\_\_() (birdears.resolution.Resolution method)@\spxentry{\_\_init\_\_()}\spxextra{birdears.resolution.Resolution method}}

\begin{fulllineitems}
\phantomsection\label{\detokenize{birdears:birdears.resolution.Resolution.__init__}}\pysiglinewithargsret{\sphinxbfcode{\sphinxupquote{\_\_init\_\_}}}{\emph{\DUrole{n}{method}}, \emph{\DUrole{n}{question}}}{}
\sphinxAtStartPar
Inits the resolution class.
\begin{quote}\begin{description}
\item[{Parameters}] \leavevmode\begin{itemize}
\item {} 
\sphinxAtStartPar
\sphinxstyleliteralstrong{\sphinxupquote{method}} (\sphinxstyleliteralemphasis{\sphinxupquote{str}}) \textendash{} The method used in the resolution.

\item {} 
\sphinxAtStartPar
\sphinxstyleliteralstrong{\sphinxupquote{question}} (\sphinxstyleliteralemphasis{\sphinxupquote{obj}}) \textendash{} Question object from which to generate the

\item {} 
\sphinxAtStartPar
\sphinxstyleliteralstrong{\sphinxupquote{sequence.}} (\sphinxstyleliteralemphasis{\sphinxupquote{resolution}}) \textendash{} 

\end{itemize}

\end{description}\end{quote}

\end{fulllineitems}


\end{fulllineitems}

\index{nearest\_tonic() (in module birdears.resolution)@\spxentry{nearest\_tonic()}\spxextra{in module birdears.resolution}}

\begin{fulllineitems}
\phantomsection\label{\detokenize{birdears:birdears.resolution.nearest_tonic}}\pysiglinewithargsret{\sphinxcode{\sphinxupquote{birdears.resolution.}}\sphinxbfcode{\sphinxupquote{nearest\_tonic}}}{\emph{\DUrole{n}{question}}}{}
\sphinxAtStartPar
Resolution method that resolve the intervals to their nearest tonics.
\begin{quote}\begin{description}
\item[{Parameters}] \leavevmode
\sphinxAtStartPar
\sphinxstyleliteralstrong{\sphinxupquote{question}} (\sphinxstyleliteralemphasis{\sphinxupquote{obj}}) \textendash{} Question object from which to generate the
resolution sequence. (this is provided by the \sphinxtitleref{Prequestion} class
when it is {\color{red}\bfseries{}\textasciigrave{}}\_\_call\_\_\textasciigrave{}ed)

\end{description}\end{quote}

\end{fulllineitems}

\index{register\_resolution\_method() (in module birdears.resolution)@\spxentry{register\_resolution\_method()}\spxextra{in module birdears.resolution}}

\begin{fulllineitems}
\phantomsection\label{\detokenize{birdears:birdears.resolution.register_resolution_method}}\pysiglinewithargsret{\sphinxcode{\sphinxupquote{birdears.resolution.}}\sphinxbfcode{\sphinxupquote{register\_resolution\_method}}}{\emph{\DUrole{n}{f}}, \emph{\DUrole{o}{*}\DUrole{n}{args}}, \emph{\DUrole{o}{**}\DUrole{n}{kwargs}}}{}
\sphinxAtStartPar
Decorator for resolution method functions.

\sphinxAtStartPar
Functions decorated with this decorator will be registered in the
\sphinxtitleref{RESOLUTION\_METHODS} global dict.

\end{fulllineitems}

\index{repeat\_only() (in module birdears.resolution)@\spxentry{repeat\_only()}\spxextra{in module birdears.resolution}}

\begin{fulllineitems}
\phantomsection\label{\detokenize{birdears:birdears.resolution.repeat_only}}\pysiglinewithargsret{\sphinxcode{\sphinxupquote{birdears.resolution.}}\sphinxbfcode{\sphinxupquote{repeat\_only}}}{\emph{\DUrole{n}{question}}}{}
\sphinxAtStartPar
Resolution method that only repeats the sequence elements with given
durations.
\begin{quote}\begin{description}
\item[{Parameters}] \leavevmode
\sphinxAtStartPar
\sphinxstyleliteralstrong{\sphinxupquote{question}} (\sphinxstyleliteralemphasis{\sphinxupquote{obj}}) \textendash{} Question object from which to generate the
resolution sequence. (this is provided by the \sphinxtitleref{Prequestion} class
when it is {\color{red}\bfseries{}\textasciigrave{}}\_\_call\_\_\textasciigrave{}ed)

\end{description}\end{quote}

\end{fulllineitems}



\section{birdears.scale module}
\label{\detokenize{birdears:module-birdears.scale}}\label{\detokenize{birdears:birdears-scale-module}}\index{module@\spxentry{module}!birdears.scale@\spxentry{birdears.scale}}\index{birdears.scale@\spxentry{birdears.scale}!module@\spxentry{module}}\index{ChromaticScale (class in birdears.scale)@\spxentry{ChromaticScale}\spxextra{class in birdears.scale}}

\begin{fulllineitems}
\phantomsection\label{\detokenize{birdears:birdears.scale.ChromaticScale}}\pysiglinewithargsret{\sphinxbfcode{\sphinxupquote{class }}\sphinxcode{\sphinxupquote{birdears.scale.}}\sphinxbfcode{\sphinxupquote{ChromaticScale}}}{\emph{\DUrole{n}{tonic}\DUrole{o}{=}\DUrole{default_value}{\textquotesingle{}C\textquotesingle{}}}, \emph{\DUrole{n}{octave}\DUrole{o}{=}\DUrole{default_value}{4}}, \emph{\DUrole{n}{n\_octaves}\DUrole{o}{=}\DUrole{default_value}{1}}, \emph{\DUrole{n}{descending}\DUrole{o}{=}\DUrole{default_value}{False}}, \emph{\DUrole{n}{dont\_repeat\_tonic}\DUrole{o}{=}\DUrole{default_value}{False}}}{}
\sphinxAtStartPar
Bases: {\hyperref[\detokenize{index:birdears.scale.ScaleBase}]{\sphinxcrossref{\sphinxcode{\sphinxupquote{birdears.scale.ScaleBase}}}}}

\sphinxAtStartPar
Builds a musical chromatic scale.
\index{scale (birdears.scale.ChromaticScale attribute)@\spxentry{scale}\spxextra{birdears.scale.ChromaticScale attribute}}

\begin{fulllineitems}
\phantomsection\label{\detokenize{birdears:birdears.scale.ChromaticScale.scale}}\pysigline{\sphinxbfcode{\sphinxupquote{scale}}}
\sphinxAtStartPar
The array of notes representing the scale.
\begin{quote}\begin{description}
\item[{Type}] \leavevmode
\sphinxAtStartPar
array\_type

\end{description}\end{quote}

\end{fulllineitems}

\index{\_\_init\_\_() (birdears.scale.ChromaticScale method)@\spxentry{\_\_init\_\_()}\spxextra{birdears.scale.ChromaticScale method}}

\begin{fulllineitems}
\phantomsection\label{\detokenize{birdears:birdears.scale.ChromaticScale.__init__}}\pysiglinewithargsret{\sphinxbfcode{\sphinxupquote{\_\_init\_\_}}}{\emph{\DUrole{n}{tonic}\DUrole{o}{=}\DUrole{default_value}{\textquotesingle{}C\textquotesingle{}}}, \emph{\DUrole{n}{octave}\DUrole{o}{=}\DUrole{default_value}{4}}, \emph{\DUrole{n}{n\_octaves}\DUrole{o}{=}\DUrole{default_value}{1}}, \emph{\DUrole{n}{descending}\DUrole{o}{=}\DUrole{default_value}{False}}, \emph{\DUrole{n}{dont\_repeat\_tonic}\DUrole{o}{=}\DUrole{default_value}{False}}}{}
\sphinxAtStartPar
Returns a chromatic scale from tonic.
\begin{quote}\begin{description}
\item[{Parameters}] \leavevmode\begin{itemize}
\item {} 
\sphinxAtStartPar
\sphinxstyleliteralstrong{\sphinxupquote{tonic}} (\sphinxstyleliteralemphasis{\sphinxupquote{str}}) \textendash{} The note which the scale will be built upon.

\item {} 
\sphinxAtStartPar
\sphinxstyleliteralstrong{\sphinxupquote{octave}} (\sphinxstyleliteralemphasis{\sphinxupquote{int}}) \textendash{} The scientific octave the scale will be built upon.

\item {} 
\sphinxAtStartPar
\sphinxstyleliteralstrong{\sphinxupquote{n\_octaves}} (\sphinxstyleliteralemphasis{\sphinxupquote{int}}) \textendash{} The number of octaves the scale will contain.

\item {} 
\sphinxAtStartPar
\sphinxstyleliteralstrong{\sphinxupquote{descending}} (\sphinxstyleliteralemphasis{\sphinxupquote{bool}}) \textendash{} Whether the scale is descending.

\item {} 
\sphinxAtStartPar
\sphinxstyleliteralstrong{\sphinxupquote{dont\_repeat\_tonic}} (\sphinxstyleliteralemphasis{\sphinxupquote{bool}}) \textendash{} Whether to skip appending the last
note (octave) to the scale.

\end{itemize}

\end{description}\end{quote}

\end{fulllineitems}

\index{get\_triad() (birdears.scale.ChromaticScale method)@\spxentry{get\_triad()}\spxextra{birdears.scale.ChromaticScale method}}

\begin{fulllineitems}
\phantomsection\label{\detokenize{birdears:birdears.scale.ChromaticScale.get_triad}}\pysiglinewithargsret{\sphinxbfcode{\sphinxupquote{get\_triad}}}{\emph{\DUrole{n}{mode}}, \emph{\DUrole{n}{index}\DUrole{o}{=}\DUrole{default_value}{0}}, \emph{\DUrole{n}{degree}\DUrole{o}{=}\DUrole{default_value}{None}}}{}
\sphinxAtStartPar
Returns an array with notes from a scale’s triad.
\begin{quote}\begin{description}
\item[{Parameters}] \leavevmode\begin{itemize}
\item {} 
\sphinxAtStartPar
\sphinxstyleliteralstrong{\sphinxupquote{mode}} (\sphinxstyleliteralemphasis{\sphinxupquote{str}}) \textendash{} Mode of the scale (eg. ‘major’ or ‘minor’)

\item {} 
\sphinxAtStartPar
\sphinxstyleliteralstrong{\sphinxupquote{index}} (\sphinxstyleliteralemphasis{\sphinxupquote{int}}) \textendash{} Triad index (eg.: 0 for 1st degree triad.)

\item {} 
\sphinxAtStartPar
\sphinxstyleliteralstrong{\sphinxupquote{degree}} (\sphinxstyleliteralemphasis{\sphinxupquote{int}}) \textendash{} Degree of the scale. If provided, overrides the
\sphinxtitleref{index} argument. (eg.: \sphinxtitleref{1} for the 1st degree triad.)

\end{itemize}

\item[{Returns}] \leavevmode
\sphinxAtStartPar
A list with three pitches (str), one for each note of the triad.

\end{description}\end{quote}

\end{fulllineitems}


\end{fulllineitems}

\index{DiatonicScale (class in birdears.scale)@\spxentry{DiatonicScale}\spxextra{class in birdears.scale}}

\begin{fulllineitems}
\phantomsection\label{\detokenize{birdears:birdears.scale.DiatonicScale}}\pysiglinewithargsret{\sphinxbfcode{\sphinxupquote{class }}\sphinxcode{\sphinxupquote{birdears.scale.}}\sphinxbfcode{\sphinxupquote{DiatonicScale}}}{\emph{\DUrole{n}{tonic}\DUrole{o}{=}\DUrole{default_value}{\textquotesingle{}C\textquotesingle{}}}, \emph{\DUrole{n}{mode}\DUrole{o}{=}\DUrole{default_value}{\textquotesingle{}major\textquotesingle{}}}, \emph{\DUrole{n}{octave}\DUrole{o}{=}\DUrole{default_value}{4}}, \emph{\DUrole{n}{n\_octaves}\DUrole{o}{=}\DUrole{default_value}{1}}, \emph{\DUrole{n}{descending}\DUrole{o}{=}\DUrole{default_value}{False}}, \emph{\DUrole{n}{dont\_repeat\_tonic}\DUrole{o}{=}\DUrole{default_value}{False}}}{}
\sphinxAtStartPar
Bases: {\hyperref[\detokenize{index:birdears.scale.ScaleBase}]{\sphinxcrossref{\sphinxcode{\sphinxupquote{birdears.scale.ScaleBase}}}}}

\sphinxAtStartPar
Builds a musical diatonic scale.
\index{scale (birdears.scale.DiatonicScale attribute)@\spxentry{scale}\spxextra{birdears.scale.DiatonicScale attribute}}

\begin{fulllineitems}
\phantomsection\label{\detokenize{birdears:birdears.scale.DiatonicScale.scale}}\pysigline{\sphinxbfcode{\sphinxupquote{scale}}}
\sphinxAtStartPar
The array of notes representing the scale.
\begin{quote}\begin{description}
\item[{Type}] \leavevmode
\sphinxAtStartPar
array\_type

\end{description}\end{quote}

\end{fulllineitems}

\index{\_\_init\_\_() (birdears.scale.DiatonicScale method)@\spxentry{\_\_init\_\_()}\spxextra{birdears.scale.DiatonicScale method}}

\begin{fulllineitems}
\phantomsection\label{\detokenize{birdears:birdears.scale.DiatonicScale.__init__}}\pysiglinewithargsret{\sphinxbfcode{\sphinxupquote{\_\_init\_\_}}}{\emph{\DUrole{n}{tonic}\DUrole{o}{=}\DUrole{default_value}{\textquotesingle{}C\textquotesingle{}}}, \emph{\DUrole{n}{mode}\DUrole{o}{=}\DUrole{default_value}{\textquotesingle{}major\textquotesingle{}}}, \emph{\DUrole{n}{octave}\DUrole{o}{=}\DUrole{default_value}{4}}, \emph{\DUrole{n}{n\_octaves}\DUrole{o}{=}\DUrole{default_value}{1}}, \emph{\DUrole{n}{descending}\DUrole{o}{=}\DUrole{default_value}{False}}, \emph{\DUrole{n}{dont\_repeat\_tonic}\DUrole{o}{=}\DUrole{default_value}{False}}}{}
\sphinxAtStartPar
Returns a diatonic scale from tonic and mode.
\begin{quote}\begin{description}
\item[{Parameters}] \leavevmode\begin{itemize}
\item {} 
\sphinxAtStartPar
\sphinxstyleliteralstrong{\sphinxupquote{tonic}} (\sphinxstyleliteralemphasis{\sphinxupquote{str}}) \textendash{} The note which the scale will be built upon.

\item {} 
\sphinxAtStartPar
\sphinxstyleliteralstrong{\sphinxupquote{mode}} (\sphinxstyleliteralemphasis{\sphinxupquote{str}}) \textendash{} The mode the scale will be built upon.
(‘major’ or ‘minor’)

\item {} 
\sphinxAtStartPar
\sphinxstyleliteralstrong{\sphinxupquote{octave}} (\sphinxstyleliteralemphasis{\sphinxupquote{int}}) \textendash{} The scientific octave the scale will be built upon.

\item {} 
\sphinxAtStartPar
\sphinxstyleliteralstrong{\sphinxupquote{n\_octaves}} (\sphinxstyleliteralemphasis{\sphinxupquote{int}}) \textendash{} The number of octaves the scale will contain.

\item {} 
\sphinxAtStartPar
\sphinxstyleliteralstrong{\sphinxupquote{descending}} (\sphinxstyleliteralemphasis{\sphinxupquote{bool}}) \textendash{} Whether the scale is descending.

\item {} 
\sphinxAtStartPar
\sphinxstyleliteralstrong{\sphinxupquote{dont\_repeat\_tonic}} (\sphinxstyleliteralemphasis{\sphinxupquote{bool}}) \textendash{} Whether to skip appending the last
note (octave) to the scale.

\end{itemize}

\end{description}\end{quote}

\end{fulllineitems}

\index{get\_triad() (birdears.scale.DiatonicScale method)@\spxentry{get\_triad()}\spxextra{birdears.scale.DiatonicScale method}}

\begin{fulllineitems}
\phantomsection\label{\detokenize{birdears:birdears.scale.DiatonicScale.get_triad}}\pysiglinewithargsret{\sphinxbfcode{\sphinxupquote{get\_triad}}}{\emph{\DUrole{n}{index}\DUrole{o}{=}\DUrole{default_value}{0}}, \emph{\DUrole{n}{degree}\DUrole{o}{=}\DUrole{default_value}{None}}}{}
\sphinxAtStartPar
Returns an array with notes from a scale’s triad.
\begin{quote}\begin{description}
\item[{Parameters}] \leavevmode\begin{itemize}
\item {} 
\sphinxAtStartPar
\sphinxstyleliteralstrong{\sphinxupquote{index}} (\sphinxstyleliteralemphasis{\sphinxupquote{int}}) \textendash{} triad index (eg.: 0 for 1st degree triad.)

\item {} 
\sphinxAtStartPar
\sphinxstyleliteralstrong{\sphinxupquote{degree}} (\sphinxstyleliteralemphasis{\sphinxupquote{int}}) \textendash{} Degree of the scale. If provided, overrides the
\sphinxtitleref{index} argument. (eg.: \sphinxtitleref{1} for the 1st degree triad.)

\end{itemize}

\item[{Returns}] \leavevmode
\sphinxAtStartPar
An array with three pitches, one for each note of the triad.

\end{description}\end{quote}

\end{fulllineitems}


\end{fulllineitems}

\index{ScaleBase (class in birdears.scale)@\spxentry{ScaleBase}\spxextra{class in birdears.scale}}

\begin{fulllineitems}
\phantomsection\label{\detokenize{birdears:birdears.scale.ScaleBase}}\pysigline{\sphinxbfcode{\sphinxupquote{class }}\sphinxcode{\sphinxupquote{birdears.scale.}}\sphinxbfcode{\sphinxupquote{ScaleBase}}}
\sphinxAtStartPar
Bases: \sphinxcode{\sphinxupquote{list}}

\end{fulllineitems}



\section{birdears.sequence module}
\label{\detokenize{birdears:module-birdears.sequence}}\label{\detokenize{birdears:birdears-sequence-module}}\index{module@\spxentry{module}!birdears.sequence@\spxentry{birdears.sequence}}\index{birdears.sequence@\spxentry{birdears.sequence}!module@\spxentry{module}}\index{Sequence (class in birdears.sequence)@\spxentry{Sequence}\spxextra{class in birdears.sequence}}

\begin{fulllineitems}
\phantomsection\label{\detokenize{birdears:birdears.sequence.Sequence}}\pysiglinewithargsret{\sphinxbfcode{\sphinxupquote{class }}\sphinxcode{\sphinxupquote{birdears.sequence.}}\sphinxbfcode{\sphinxupquote{Sequence}}}{\emph{\DUrole{n}{elements}\DUrole{o}{=}\DUrole{default_value}{{[}{]}}}, \emph{\DUrole{n}{duration}\DUrole{o}{=}\DUrole{default_value}{2}}, \emph{\DUrole{n}{delay}\DUrole{o}{=}\DUrole{default_value}{1.5}}, \emph{\DUrole{n}{pos\_delay}\DUrole{o}{=}\DUrole{default_value}{1}}}{}
\sphinxAtStartPar
Bases: \sphinxcode{\sphinxupquote{list}}

\sphinxAtStartPar
Register a Sequence of notes and/or chords.
\index{elements (birdears.sequence.Sequence attribute)@\spxentry{elements}\spxextra{birdears.sequence.Sequence attribute}}

\begin{fulllineitems}
\phantomsection\label{\detokenize{birdears:birdears.sequence.Sequence.elements}}\pysigline{\sphinxbfcode{\sphinxupquote{elements}}}
\sphinxAtStartPar
List of notes (strings) ou chords (list of
strings) in this Sequence.
\begin{quote}\begin{description}
\item[{Type}] \leavevmode
\sphinxAtStartPar
array\_type

\end{description}\end{quote}

\end{fulllineitems}

\index{\_\_init\_\_() (birdears.sequence.Sequence method)@\spxentry{\_\_init\_\_()}\spxextra{birdears.sequence.Sequence method}}

\begin{fulllineitems}
\phantomsection\label{\detokenize{birdears:birdears.sequence.Sequence.__init__}}\pysiglinewithargsret{\sphinxbfcode{\sphinxupquote{\_\_init\_\_}}}{\emph{\DUrole{n}{elements}\DUrole{o}{=}\DUrole{default_value}{{[}{]}}}, \emph{\DUrole{n}{duration}\DUrole{o}{=}\DUrole{default_value}{2}}, \emph{\DUrole{n}{delay}\DUrole{o}{=}\DUrole{default_value}{1.5}}, \emph{\DUrole{n}{pos\_delay}\DUrole{o}{=}\DUrole{default_value}{1}}}{}~\begin{description}
\item[{Inits the Sequence with an array and sets the default times for}] \leavevmode
\sphinxAtStartPar
playing / pausing the elements.

\end{description}
\begin{quote}\begin{description}
\item[{Parameters}] \leavevmode\begin{itemize}
\item {} 
\sphinxAtStartPar
\sphinxstyleliteralstrong{\sphinxupquote{elements}} (\sphinxstyleliteralemphasis{\sphinxupquote{array\_type}}) \textendash{} List of elements in this sequence.
(Pitch’es and/or Chord’s)

\item {} 
\sphinxAtStartPar
\sphinxstyleliteralstrong{\sphinxupquote{duration}} (\sphinxstyleliteralemphasis{\sphinxupquote{float}}) \textendash{} Default playing time for each element in the
sequence.

\item {} 
\sphinxAtStartPar
\sphinxstyleliteralstrong{\sphinxupquote{delay}} (\sphinxstyleliteralemphasis{\sphinxupquote{float}}) \textendash{} Default waiting time to play the next element
in the sequence.

\item {} 
\sphinxAtStartPar
\sphinxstyleliteralstrong{\sphinxupquote{pos\_delay}} (\sphinxstyleliteralemphasis{\sphinxupquote{float}}) \textendash{} Waiting time after playing the last element
in the sequence.

\end{itemize}

\end{description}\end{quote}

\end{fulllineitems}

\index{async\_play() (birdears.sequence.Sequence method)@\spxentry{async\_play()}\spxextra{birdears.sequence.Sequence method}}

\begin{fulllineitems}
\phantomsection\label{\detokenize{birdears:birdears.sequence.Sequence.async_play}}\pysiglinewithargsret{\sphinxbfcode{\sphinxupquote{async\_play}}}{\emph{\DUrole{n}{callback}}, \emph{\DUrole{n}{end\_callback}}, \emph{\DUrole{n}{args}}, \emph{\DUrole{n}{kwargs}}}{}
\sphinxAtStartPar
Plays the Sequence elements of notes and/or chords and wait for
\sphinxtitleref{Sequence.pos\_delay} seconds.

\end{fulllineitems}

\index{make\_chord\_progression() (birdears.sequence.Sequence method)@\spxentry{make\_chord\_progression()}\spxextra{birdears.sequence.Sequence method}}

\begin{fulllineitems}
\phantomsection\label{\detokenize{birdears:birdears.sequence.Sequence.make_chord_progression}}\pysiglinewithargsret{\sphinxbfcode{\sphinxupquote{make\_chord\_progression}}}{\emph{\DUrole{n}{tonic\_pitch}}, \emph{\DUrole{n}{mode}}, \emph{\DUrole{n}{degrees}}}{}
\sphinxAtStartPar
Appends triad chord(s) to the Sequence.
\begin{quote}\begin{description}
\item[{Parameters}] \leavevmode\begin{itemize}
\item {} 
\sphinxAtStartPar
\sphinxstyleliteralstrong{\sphinxupquote{tonic}} (\sphinxstyleliteralemphasis{\sphinxupquote{str}}) \textendash{} Tonic note of the scale.

\item {} 
\sphinxAtStartPar
\sphinxstyleliteralstrong{\sphinxupquote{mode}} (\sphinxstyleliteralemphasis{\sphinxupquote{str}}) \textendash{} Mode of the scale from which build the triads upon.

\item {} 
\sphinxAtStartPar
\sphinxstyleliteralstrong{\sphinxupquote{degrees}} (\sphinxstyleliteralemphasis{\sphinxupquote{array\_type}}) \textendash{} List with integers represending the degrees
of each triad.

\end{itemize}

\end{description}\end{quote}

\end{fulllineitems}

\index{play() (birdears.sequence.Sequence method)@\spxentry{play()}\spxextra{birdears.sequence.Sequence method}}

\begin{fulllineitems}
\phantomsection\label{\detokenize{birdears:birdears.sequence.Sequence.play}}\pysiglinewithargsret{\sphinxbfcode{\sphinxupquote{play}}}{\emph{\DUrole{n}{callback}\DUrole{o}{=}\DUrole{default_value}{None}}, \emph{\DUrole{n}{end\_callback}\DUrole{o}{=}\DUrole{default_value}{None}}, \emph{\DUrole{o}{*}\DUrole{n}{args}}, \emph{\DUrole{o}{**}\DUrole{n}{kwargs}}}{}
\end{fulllineitems}


\end{fulllineitems}



\chapter{Support}
\label{\detokenize{index:support}}
\sphinxAtStartPar
If you need help you can get in touch via IRC or file an issue on any matter regarding birdears at Github.


\begin{savenotes}\sphinxattablestart
\centering
\begin{tabulary}{\linewidth}[t]{|T|T|}
\hline
\sphinxstyletheadfamily 
\sphinxAtStartPar
Media
&\sphinxstyletheadfamily 
\sphinxAtStartPar
Channel
\\
\hline
\sphinxAtStartPar
IRC
&
\sphinxAtStartPar
\sphinxhref{https://webchat.freenode.net/?randomnick=1\&channels=\%23birdears\&uio=MTY9dHJ1ZSYxMT0yNDY57}{\#birdears} at irc.freenode.org/6697 \sphinxhyphen{}ssl
\\
\hline
\sphinxAtStartPar
GitHub
&
\sphinxAtStartPar
\sphinxurl{https://github.com/iacchus/birdears}
\\
\hline
\sphinxAtStartPar
GH issues
&
\sphinxAtStartPar
\sphinxurl{https://github.com/iacchus/birdears/issues}
\\
\hline
\sphinxAtStartPar
ReadTheDocs
&
\sphinxAtStartPar
\sphinxurl{https://birdears.readthedocs.io}
\\
\hline
\sphinxAtStartPar
PyPI
&
\sphinxAtStartPar
\sphinxurl{https://pypi.python.org/pypi/birdears}
\\
\hline
\sphinxAtStartPar
TravisCI
&
\sphinxAtStartPar
\sphinxurl{https://travis-ci.org/iacchus/birdears}
\\
\hline
\sphinxAtStartPar
Coveralls
&
\sphinxAtStartPar
\sphinxurl{https://coveralls.io/github/iacchus/birdears}
\\
\hline
\end{tabulary}
\par
\sphinxattableend\end{savenotes}


\chapter{Features}
\label{\detokenize{index:features}}\begin{itemize}
\item {} 
\sphinxAtStartPar
questions

\item {} 
\sphinxAtStartPar
pretty much configurable

\item {} 
\sphinxAtStartPar
load from config file

\item {} 
\sphinxAtStartPar
you can make your own presets

\item {} 
\sphinxAtStartPar
can be used interactively \sphinxstyleemphasis{(docs needed)}

\item {} 
\sphinxAtStartPar
can be used as a library \sphinxstyleemphasis{(docs needed)}

\end{itemize}


\chapter{Installing birdears}
\label{\detokenize{index:installing-birdears}}

\section{Installing the dependencies}
\label{\detokenize{index:installing-the-dependencies}}

\subsection{Arch Linux}
\label{\detokenize{index:arch-linux}}
\begin{sphinxVerbatim}[commandchars=\\\{\}]
sudo pacman \PYGZhy{}Syu sox python python\PYGZhy{}pip
\end{sphinxVerbatim}


\section{Installing birdears}
\label{\detokenize{index:id1}}
\sphinxAtStartPar
To install,simple do this command with pip3

\begin{sphinxVerbatim}[commandchars=\\\{\}]
pip3 install \PYGZhy{}\PYGZhy{}user \PYGZhy{}\PYGZhy{}upgrade \PYGZhy{}\PYGZhy{}no\PYGZhy{}cache\PYGZhy{}dir birdears
\end{sphinxVerbatim}


\subsection{In\sphinxhyphen{}depth installation}
\label{\detokenize{index:in-depth-installation}}
\sphinxAtStartPar
You can choose to use a virtualenv to use birdears; this should give you
an idea on how to setup one virtualenv.

\sphinxAtStartPar
You should first install virtualenv (for python3) using your
distribution’s package (supposing you’re on linux), then issue on terminal:

\begin{sphinxVerbatim}[commandchars=\\\{\}]
virtualenv \PYGZhy{}p python3 \PYGZti{}/.venv \PYGZsh{} use the directory \PYGZti{}/.venv/ for the virtualenv

source \PYGZti{}/.venv/bin/activate   \PYGZsh{} activate the virtualenv; this should be done
                              \PYGZsh{} every time you may want to run the software
                              \PYGZsh{} installed here.

pip3 install birdears         \PYGZsh{} this will install the software

birdears \PYGZhy{}\PYGZhy{}help               \PYGZsh{} and this will run it
\end{sphinxVerbatim}


\chapter{Using birdears}
\label{\detokenize{index:using-birdears}}

\section{What is Functional Ear Training}
\label{\detokenize{index:what-is-functional-ear-training}}
\sphinxAtStartPar
\sphinxstyleemphasis{write me!}


\section{The method}
\label{\detokenize{index:the-method}}
\sphinxAtStartPar
We can use abc language to notate music within the documentation, ok

\begin{sphinxVerbatim}[commandchars=\\\{\}]
X: 1
T: Banish Misfortune
R: jig
M: 6/8
L: 1/8
K: Dmix
fed cAG| A2d cAG| F2D DED| FEF GFG|
AGA cAG| AGA cde|fed cAG| Ad\PYGZca{}c d3:|
f2d d\PYGZca{}cd| f2g agf| e2c cBc|e2f gfe|
f2g agf| e2f gfe|fed cAG|Ad\PYGZca{}c d3:|
f2g e2f| d2e c2d|ABA GAG| F2F GED|
c3 cAG| AGA cde| fed cAG| Ad\PYGZca{}c d3:|
\end{sphinxVerbatim}


\section{birdears modes and basic usage}
\label{\detokenize{index:birdears-modes-and-basic-usage}}
\sphinxAtStartPar
birdears actually has four modes:
\begin{itemize}
\item {} 
\sphinxAtStartPar
melodic interval question

\item {} 
\sphinxAtStartPar
harmonic interval question

\item {} 
\sphinxAtStartPar
melodic dictation question

\item {} 
\sphinxAtStartPar
instrumental dictation question

\end{itemize}

\sphinxAtStartPar
To see the commands avaliable just invoke the command without any arguments:

\begin{sphinxVerbatim}[commandchars=\\\{\}]
birdears
\end{sphinxVerbatim}

\begin{sphinxVerbatim}[commandchars=\\\{\}]
Usage: birdears  \PYGZlt{}command\PYGZgt{} [options]

  birdears ─ Functional Ear Training for Musicians!

Options:
  \PYGZhy{}\PYGZhy{}debug / \PYGZhy{}\PYGZhy{}no\PYGZhy{}debug  Turns on debugging; instead you can set DEBUG=1.
  \PYGZhy{}h, \PYGZhy{}\PYGZhy{}help            Show this message and exit.

Commands:
  dictation     Melodic dictation
  harmonic      Harmonic interval recognition
  instrumental  Instrumental melodic time\PYGZhy{}based dictation
  load          Loads exercise from .toml config file...
  melodic       Melodic interval recognition

  You can use \PYGZsq{}birdears \PYGZlt{}command\PYGZgt{} \PYGZhy{}\PYGZhy{}help\PYGZsq{} to show options for a specific
  command.

  More info at https://github.com/iacchus/birdears
\end{sphinxVerbatim}

\begin{sphinxVerbatim}[commandchars=\\\{\}]
birdears \PYGZlt{}command\PYGZgt{} \PYGZhy{}\PYGZhy{}help
\end{sphinxVerbatim}


\subsection{melodic}
\label{\detokenize{index:melodic}}
\sphinxAtStartPar
In this exercise birdears will play two notes, the tonic and the interval
melodically, ie., one after the other and you should reply which is the
correct distance between the two.

\begin{sphinxVerbatim}[commandchars=\\\{\}]
birdears melodic \PYGZhy{}\PYGZhy{}help
\end{sphinxVerbatim}

\begin{sphinxVerbatim}[commandchars=\\\{\}]
Usage: birdears melodic [options]

  Melodic interval recognition

Options:
  \PYGZhy{}m, \PYGZhy{}\PYGZhy{}mode \PYGZlt{}mode\PYGZgt{}               Mode of the question.
  \PYGZhy{}t, \PYGZhy{}\PYGZhy{}tonic \PYGZlt{}tonic\PYGZgt{}             Tonic of the question.
  \PYGZhy{}o, \PYGZhy{}\PYGZhy{}octave \PYGZlt{}octave\PYGZgt{}           Octave of the question.
  \PYGZhy{}d, \PYGZhy{}\PYGZhy{}descending                Whether the question interval is descending.
  \PYGZhy{}c, \PYGZhy{}\PYGZhy{}chromatic                 If chosen, question has chromatic notes.
  \PYGZhy{}n, \PYGZhy{}\PYGZhy{}n\PYGZus{}octaves \PYGZlt{}n max\PYGZgt{}         Maximum number of octaves.
  \PYGZhy{}v, \PYGZhy{}\PYGZhy{}valid\PYGZus{}intervals \PYGZlt{}1,2,..\PYGZgt{}  A comma\PYGZhy{}separated list without spaces
                                  of valid scale degrees to be chosen for the
                                  question.
  \PYGZhy{}q, \PYGZhy{}\PYGZhy{}user\PYGZus{}durations \PYGZlt{}1,0.5,n..\PYGZgt{}
                                  A comma\PYGZhy{}separated list without
                                  spaces with PRECISLY 9 floating values. Or
                                  \PYGZsq{}n\PYGZsq{} for default              duration.
  \PYGZhy{}p, \PYGZhy{}\PYGZhy{}prequestion\PYGZus{}method \PYGZlt{}prequestion\PYGZus{}method\PYGZgt{}
                                  The name of a pre\PYGZhy{}question method.
  \PYGZhy{}r, \PYGZhy{}\PYGZhy{}resolution\PYGZus{}method \PYGZlt{}resolution\PYGZus{}method\PYGZgt{}
                                  The name of a resolution method.
  \PYGZhy{}h, \PYGZhy{}\PYGZhy{}help                      Show this message and exit.

  In this exercise birdears will play two notes, the tonic and the interval
  melodically, ie., one after the other and you should reply which is the
  correct distance between the two.

  Valid values are as follows:

  \PYGZhy{}m \PYGZlt{}mode\PYGZgt{} is one of: major, dorian, phrygian, lydian, mixolydian, minor,
  locrian

  \PYGZhy{}t \PYGZlt{}tonic\PYGZgt{} is one of: A, A\PYGZsh{}, Ab, B, Bb, C, C\PYGZsh{}, D, D\PYGZsh{}, Db, E, Eb, F, F\PYGZsh{}, G,
  G\PYGZsh{}, Gb

  \PYGZhy{}p \PYGZlt{}prequestion\PYGZus{}method\PYGZgt{} is one of: none, tonic\PYGZus{}only, progression\PYGZus{}i\PYGZus{}iv\PYGZus{}v\PYGZus{}i

  \PYGZhy{}r \PYGZlt{}resolution\PYGZus{}method\PYGZgt{} is one of: nearest\PYGZus{}tonic, repeat\PYGZus{}only
\end{sphinxVerbatim}


\subsection{harmonic}
\label{\detokenize{index:harmonic}}
\sphinxAtStartPar
In this exercise birdears will play two notes, the tonic and the interval
harmonically, ie., both on the same time and you should reply which is the
correct distance between the two.

\begin{sphinxVerbatim}[commandchars=\\\{\}]
birdears harmonic \PYGZhy{}\PYGZhy{}help
\end{sphinxVerbatim}

\begin{sphinxVerbatim}[commandchars=\\\{\}]
Usage: birdears harmonic [options]

  Harmonic interval recognition

Options:
  \PYGZhy{}m, \PYGZhy{}\PYGZhy{}mode \PYGZlt{}mode\PYGZgt{}               Mode of the question.
  \PYGZhy{}t, \PYGZhy{}\PYGZhy{}tonic \PYGZlt{}note\PYGZgt{}              Tonic of the question.
  \PYGZhy{}o, \PYGZhy{}\PYGZhy{}octave \PYGZlt{}octave\PYGZgt{}           Octave of the question.
  \PYGZhy{}d, \PYGZhy{}\PYGZhy{}descending                Whether the question interval is descending.
  \PYGZhy{}c, \PYGZhy{}\PYGZhy{}chromatic                 If chosen, question has chromatic notes.
  \PYGZhy{}n, \PYGZhy{}\PYGZhy{}n\PYGZus{}octaves \PYGZlt{}n max\PYGZgt{}         Maximum number of octaves.
  \PYGZhy{}v, \PYGZhy{}\PYGZhy{}valid\PYGZus{}intervals \PYGZlt{}1,2,..\PYGZgt{}  A comma\PYGZhy{}separated list without spaces
                                  of valid scale degrees to be chosen for the
                                  question.
  \PYGZhy{}q, \PYGZhy{}\PYGZhy{}user\PYGZus{}durations \PYGZlt{}1,0.5,n..\PYGZgt{}
                                  A comma\PYGZhy{}separated list without
                                  spaces with PRECISLY 9 floating values. Or
                                  \PYGZsq{}n\PYGZsq{} for default              duration.
  \PYGZhy{}p, \PYGZhy{}\PYGZhy{}prequestion\PYGZus{}method \PYGZlt{}prequestion\PYGZus{}method\PYGZgt{}
                                  The name of a pre\PYGZhy{}question method.
  \PYGZhy{}r, \PYGZhy{}\PYGZhy{}resolution\PYGZus{}method \PYGZlt{}resolution\PYGZus{}method\PYGZgt{}
                                  The name of a resolution method.
  \PYGZhy{}h, \PYGZhy{}\PYGZhy{}help                      Show this message and exit.

  In this exercise birdears will play two notes, the tonic and the interval
  harmonically, ie., both on the same time and you should reply which is the
  correct distance between the two.

  Valid values are as follows:

  \PYGZhy{}m \PYGZlt{}mode\PYGZgt{} is one of: major, dorian, phrygian, lydian, mixolydian, minor,
  locrian

  \PYGZhy{}t \PYGZlt{}tonic\PYGZgt{} is one of: A, A\PYGZsh{}, Ab, B, Bb, C, C\PYGZsh{}, D, D\PYGZsh{}, Db, E, Eb, F, F\PYGZsh{}, G,
  G\PYGZsh{}, Gb

  \PYGZhy{}p \PYGZlt{}prequestion\PYGZus{}method\PYGZgt{} is one of: none, tonic\PYGZus{}only, progression\PYGZus{}i\PYGZus{}iv\PYGZus{}v\PYGZus{}i

  \PYGZhy{}r \PYGZlt{}resolution\PYGZus{}method\PYGZgt{} is one of: nearest\PYGZus{}tonic, repeat\PYGZus{}only
\end{sphinxVerbatim}


\subsection{dictation}
\label{\detokenize{index:dictation}}
\sphinxAtStartPar
In this exercise birdears will choose some random intervals and create a
melodic dictation with them. You should reply the correct intervals of the
melodic dictation.

\begin{sphinxVerbatim}[commandchars=\\\{\}]
birdears dictation \PYGZhy{}\PYGZhy{}help
\end{sphinxVerbatim}

\begin{sphinxVerbatim}[commandchars=\\\{\}]
Usage: birdears dictation [options]

  Melodic dictation

Options:
  \PYGZhy{}m, \PYGZhy{}\PYGZhy{}mode \PYGZlt{}mode\PYGZgt{}               Mode of the question.
  \PYGZhy{}i, \PYGZhy{}\PYGZhy{}max\PYGZus{}intervals \PYGZlt{}n max\PYGZgt{}     Max random intervals for the dictation.
  \PYGZhy{}x, \PYGZhy{}\PYGZhy{}n\PYGZus{}notes \PYGZlt{}n notes\PYGZgt{}         Number of notes for the dictation.
  \PYGZhy{}t, \PYGZhy{}\PYGZhy{}tonic \PYGZlt{}note\PYGZgt{}              Tonic of the question.
  \PYGZhy{}o, \PYGZhy{}\PYGZhy{}octave \PYGZlt{}octave\PYGZgt{}           Octave of the question.
  \PYGZhy{}d, \PYGZhy{}\PYGZhy{}descending                Wether the question interval is descending.
  \PYGZhy{}c, \PYGZhy{}\PYGZhy{}chromatic                 If chosen, question has chromatic notes.
  \PYGZhy{}n, \PYGZhy{}\PYGZhy{}n\PYGZus{}octaves \PYGZlt{}n max\PYGZgt{}         Maximum number of octaves.
  \PYGZhy{}v, \PYGZhy{}\PYGZhy{}valid\PYGZus{}intervals \PYGZlt{}1,2,..\PYGZgt{}  A comma\PYGZhy{}separated list without spaces
                                  of valid scale degrees to be chosen for the
                                  question.
  \PYGZhy{}q, \PYGZhy{}\PYGZhy{}user\PYGZus{}durations \PYGZlt{}1,0.5,n..\PYGZgt{}
                                  A comma\PYGZhy{}separated list without
                                  spaces with PRECISLY 9 floating values. Or
                                  \PYGZsq{}n\PYGZsq{} for default              duration.
  \PYGZhy{}p, \PYGZhy{}\PYGZhy{}prequestion\PYGZus{}method \PYGZlt{}prequestion\PYGZus{}method\PYGZgt{}
                                  The name of a pre\PYGZhy{}question method.
  \PYGZhy{}r, \PYGZhy{}\PYGZhy{}resolution\PYGZus{}method \PYGZlt{}resolution\PYGZus{}method\PYGZgt{}
                                  The name of a resolution method.
  \PYGZhy{}h, \PYGZhy{}\PYGZhy{}help                      Show this message and exit.

  In this exercise birdears will choose some random intervals and create a
  melodic dictation with them. You should reply the correct intervals of the
  melodic dictation.

  Valid values are as follows:

  \PYGZhy{}m \PYGZlt{}mode\PYGZgt{} is one of: major, dorian, phrygian, lydian, mixolydian, minor,
  locrian

  \PYGZhy{}t \PYGZlt{}tonic\PYGZgt{} is one of: A, A\PYGZsh{}, Ab, B, Bb, C, C\PYGZsh{}, D, D\PYGZsh{}, Db, E, Eb, F, F\PYGZsh{}, G,
  G\PYGZsh{}, Gb

  \PYGZhy{}p \PYGZlt{}prequestion\PYGZus{}method\PYGZgt{} is one of: none, tonic\PYGZus{}only, progression\PYGZus{}i\PYGZus{}iv\PYGZus{}v\PYGZus{}i

  \PYGZhy{}r \PYGZlt{}resolution\PYGZus{}method\PYGZgt{} is one of: nearest\PYGZus{}tonic, repeat\PYGZus{}only
\end{sphinxVerbatim}


\subsection{instrumental}
\label{\detokenize{index:instrumental}}
\sphinxAtStartPar
In this exercise birdears will choose some random intervals and create a
melodic dictation with them. You should play the correct melody in you
musical instrument.

\begin{sphinxVerbatim}[commandchars=\\\{\}]
birdears instrumental \PYGZhy{}\PYGZhy{}help
\end{sphinxVerbatim}

\begin{sphinxVerbatim}[commandchars=\\\{\}]
Usage: birdears instrumental [options]

  Instrumental melodic time\PYGZhy{}based dictation

Options:
  \PYGZhy{}m, \PYGZhy{}\PYGZhy{}mode \PYGZlt{}mode\PYGZgt{}               Mode of the question.
  \PYGZhy{}w, \PYGZhy{}\PYGZhy{}wait\PYGZus{}time \PYGZlt{}seconds\PYGZgt{}       Time in seconds for next question/repeat.
  \PYGZhy{}u, \PYGZhy{}\PYGZhy{}n\PYGZus{}repeats \PYGZlt{}times\PYGZgt{}         Times to repeat question.
  \PYGZhy{}i, \PYGZhy{}\PYGZhy{}max\PYGZus{}intervals \PYGZlt{}n max\PYGZgt{}     Max random intervals for the dictation.
  \PYGZhy{}x, \PYGZhy{}\PYGZhy{}n\PYGZus{}notes \PYGZlt{}n notes\PYGZgt{}         Number of notes for the dictation.
  \PYGZhy{}t, \PYGZhy{}\PYGZhy{}tonic \PYGZlt{}note\PYGZgt{}              Tonic of the question.
  \PYGZhy{}o, \PYGZhy{}\PYGZhy{}octave \PYGZlt{}octave\PYGZgt{}           Octave of the question.
  \PYGZhy{}d, \PYGZhy{}\PYGZhy{}descending                Wether the question interval is descending.
  \PYGZhy{}c, \PYGZhy{}\PYGZhy{}chromatic                 If chosen, question has chromatic notes.
  \PYGZhy{}n, \PYGZhy{}\PYGZhy{}n\PYGZus{}octaves \PYGZlt{}n max\PYGZgt{}         Maximum number of octaves.
  \PYGZhy{}v, \PYGZhy{}\PYGZhy{}valid\PYGZus{}intervals \PYGZlt{}1,2,..\PYGZgt{}  A comma\PYGZhy{}separated list without spaces
                                  of valid scale degrees to be chosen for the
                                  question.
  \PYGZhy{}q, \PYGZhy{}\PYGZhy{}user\PYGZus{}durations \PYGZlt{}1,0.5,n..\PYGZgt{}
                                  A comma\PYGZhy{}separated list without
                                  spaces with PRECISLY 9 floating values. Or
                                  \PYGZsq{}n\PYGZsq{} for default              duration.
  \PYGZhy{}p, \PYGZhy{}\PYGZhy{}prequestion\PYGZus{}method \PYGZlt{}prequestion\PYGZus{}method\PYGZgt{}
                                  The name of a pre\PYGZhy{}question method.
  \PYGZhy{}r, \PYGZhy{}\PYGZhy{}resolution\PYGZus{}method \PYGZlt{}resolution\PYGZus{}method\PYGZgt{}
                                  The name of a resolution method.
  \PYGZhy{}h, \PYGZhy{}\PYGZhy{}help                      Show this message and exit.

  In this exercise birdears will choose some random intervals and create a
  melodic dictation with them. You should play the correct melody in you
  musical instrument.

  Valid values are as follows:

  \PYGZhy{}m \PYGZlt{}mode\PYGZgt{} is one of: major, dorian, phrygian, lydian, mixolydian, minor,
  locrian

  \PYGZhy{}t \PYGZlt{}tonic\PYGZgt{} is one of: A, A\PYGZsh{}, Ab, B, Bb, C, C\PYGZsh{}, D, D\PYGZsh{}, Db, E, Eb, F, F\PYGZsh{}, G,
  G\PYGZsh{}, Gb

  \PYGZhy{}p \PYGZlt{}prequestion\PYGZus{}method\PYGZgt{} is one of: none, tonic\PYGZus{}only, progression\PYGZus{}i\PYGZus{}iv\PYGZus{}v\PYGZus{}i

  \PYGZhy{}r \PYGZlt{}resolution\PYGZus{}method\PYGZgt{} is one of: nearest\PYGZus{}tonic, repeat\PYGZus{}only
\end{sphinxVerbatim}


\section{Loading from config/preset files}
\label{\detokenize{index:loading-from-config-preset-files}}

\subsection{Pre\sphinxhyphen{}made presets}
\label{\detokenize{index:pre-made-presets}}
\sphinxAtStartPar
\sphinxcode{\sphinxupquote{birdears}} cointains some pre\sphinxhyphen{}made presets in it’s \sphinxcode{\sphinxupquote{presets/}}
subdirectory.

\sphinxAtStartPar
The study for beginners is recommended by following the numeric order of
those files (000, 001, then 002 etc.)


\subsubsection{Pre\sphinxhyphen{}made presets description}
\label{\detokenize{index:pre-made-presets-description}}
\sphinxAtStartPar
\sphinxstyleemphasis{write me}


\subsection{Creating new preset files}
\label{\detokenize{index:creating-new-preset-files}}
\sphinxAtStartPar
You can open the files cointained in birdears premade \sphinxcode{\sphinxupquote{presets/}}
folder to have an ideia on how config files are made; it is simply the
command line options written in a form \sphinxcode{\sphinxupquote{toml}} standard.


\section{Keybindings}
\label{\detokenize{index:keybindings}}

\subsection{On the keybindings}
\label{\detokenize{index:on-the-keybindings}}
\sphinxAtStartPar
The following keyboard diagrams should give you an idea on how the
keybindings work. Please note how the keys on the line from \sphinxcode{\sphinxupquote{z}}
(\sphinxstyleemphasis{unison}) to \sphinxcode{\sphinxupquote{,}} (comma, \sphinxstyleemphasis{octave}) represent the notes that are
\sphinxstyleemphasis{natural} to the mode, and the line above represent the chromatics.

\sphinxAtStartPar
Also, for exercises with two octaves, the \sphinxstylestrong{uppercased keys represent
the second octave}. For example, \sphinxcode{\sphinxupquote{z}} is \sphinxstyleemphasis{unison}, \sphinxcode{\sphinxupquote{,}} is the
\sphinxstyleemphasis{octave}, \sphinxcode{\sphinxupquote{Z}} (uppercased) is the \sphinxstyleemphasis{double octave}. The same for all the other
intervals.


\subsection{Major (Ionian)}
\label{\detokenize{index:major-ionian}}
\begin{figure}[htbp]
\centering
\capstart

\noindent\sphinxincludegraphics[scale=1.0]{{ionian}.png}
\caption{Keyboard diagram for the \sphinxcode{\sphinxupquote{\sphinxhyphen{}\sphinxhyphen{}mode major}} (default).}\label{\detokenize{index:id14}}\end{figure}


\subsection{Dorian}
\label{\detokenize{index:dorian}}
\begin{figure}[htbp]
\centering
\capstart

\noindent\sphinxincludegraphics[scale=1.0]{{dorian}.png}
\caption{Keyboard diagram for the \sphinxcode{\sphinxupquote{\sphinxhyphen{}\sphinxhyphen{}mode dorian}}.}\label{\detokenize{index:id15}}\end{figure}


\subsection{Phrygian}
\label{\detokenize{index:phrygian}}
\begin{figure}[htbp]
\centering
\capstart

\noindent\sphinxincludegraphics[scale=1.0]{{phrygian}.png}
\caption{Keyboard diagram for the \sphinxcode{\sphinxupquote{\sphinxhyphen{}\sphinxhyphen{}mode phrygian}}.}\label{\detokenize{index:id16}}\end{figure}


\subsection{Lydian}
\label{\detokenize{index:lydian}}
\begin{figure}[htbp]
\centering
\capstart

\noindent\sphinxincludegraphics[scale=1.0]{{lydian}.png}
\caption{Keyboard diagram for the \sphinxcode{\sphinxupquote{\sphinxhyphen{}\sphinxhyphen{}mode lydian}}.}\label{\detokenize{index:id17}}\end{figure}


\subsection{Mixolydian}
\label{\detokenize{index:mixolydian}}
\begin{figure}[htbp]
\centering
\capstart

\noindent\sphinxincludegraphics[scale=1.0]{{mixolydian}.png}
\caption{Keyboard diagram for the \sphinxcode{\sphinxupquote{\sphinxhyphen{}\sphinxhyphen{}mode mixolydian}}.}\label{\detokenize{index:id18}}\end{figure}


\subsection{Minor (Aeolian)}
\label{\detokenize{index:minor-aeolian}}
\begin{figure}[htbp]
\centering
\capstart

\noindent\sphinxincludegraphics[scale=1.0]{{minor}.png}
\caption{Keyboard diagram for the \sphinxcode{\sphinxupquote{\sphinxhyphen{}\sphinxhyphen{}mode minor}}.}\label{\detokenize{index:id19}}\end{figure}


\subsection{Locrian}
\label{\detokenize{index:locrian}}
\begin{figure}[htbp]
\centering
\capstart

\noindent\sphinxincludegraphics[scale=1.0]{{locrian}.png}
\caption{Keyboard diagram for the \sphinxcode{\sphinxupquote{\sphinxhyphen{}\sphinxhyphen{}mode locrian}}.}\label{\detokenize{index:id20}}\end{figure}


\chapter{API}
\label{\detokenize{index:api}}

\chapter{birdears package}
\label{\detokenize{index:module-birdears}}\label{\detokenize{index:birdears-package}}\index{module@\spxentry{module}!birdears@\spxentry{birdears}}\index{birdears@\spxentry{birdears}!module@\spxentry{module}}
\sphinxAtStartPar
birdears provides facilities to building musical ear training exercises.
\index{CHROMATIC\_FLAT (in module birdears)@\spxentry{CHROMATIC\_FLAT}\spxextra{in module birdears}}

\begin{fulllineitems}
\phantomsection\label{\detokenize{index:birdears.CHROMATIC_FLAT}}\pysigline{\sphinxcode{\sphinxupquote{birdears.}}\sphinxbfcode{\sphinxupquote{CHROMATIC\_FLAT}}\sphinxbfcode{\sphinxupquote{ = (\textquotesingle{}C\textquotesingle{}, \textquotesingle{}Db\textquotesingle{}, \textquotesingle{}D\textquotesingle{}, \textquotesingle{}Eb\textquotesingle{}, \textquotesingle{}E\textquotesingle{}, \textquotesingle{}F\textquotesingle{}, \textquotesingle{}Gb\textquotesingle{}, \textquotesingle{}G\textquotesingle{}, \textquotesingle{}Ab\textquotesingle{}, \textquotesingle{}A\textquotesingle{}, \textquotesingle{}Bb\textquotesingle{}, \textquotesingle{}B\textquotesingle{})}}}
\sphinxAtStartPar
Chromatic notes names using flats.

\sphinxAtStartPar
A mapping of the chromatic note names using flats.
\begin{quote}\begin{description}
\item[{Type}] \leavevmode
\sphinxAtStartPar
tuple

\end{description}\end{quote}

\end{fulllineitems}

\index{CHROMATIC\_SHARP (in module birdears)@\spxentry{CHROMATIC\_SHARP}\spxextra{in module birdears}}

\begin{fulllineitems}
\phantomsection\label{\detokenize{index:birdears.CHROMATIC_SHARP}}\pysigline{\sphinxcode{\sphinxupquote{birdears.}}\sphinxbfcode{\sphinxupquote{CHROMATIC\_SHARP}}\sphinxbfcode{\sphinxupquote{ = (\textquotesingle{}C\textquotesingle{}, \textquotesingle{}C\#\textquotesingle{}, \textquotesingle{}D\textquotesingle{}, \textquotesingle{}D\#\textquotesingle{}, \textquotesingle{}E\textquotesingle{}, \textquotesingle{}F\textquotesingle{}, \textquotesingle{}F\#\textquotesingle{}, \textquotesingle{}G\textquotesingle{}, \textquotesingle{}G\#\textquotesingle{}, \textquotesingle{}A\textquotesingle{}, \textquotesingle{}A\#\textquotesingle{}, \textquotesingle{}B\textquotesingle{})}}}
\sphinxAtStartPar
Chromatic notes names using sharps.

\sphinxAtStartPar
A mapping of the chromatic note namesu sing sharps
\begin{quote}\begin{description}
\item[{Type}] \leavevmode
\sphinxAtStartPar
tuple

\end{description}\end{quote}

\end{fulllineitems}

\index{CHROMATIC\_TYPE (in module birdears)@\spxentry{CHROMATIC\_TYPE}\spxextra{in module birdears}}

\begin{fulllineitems}
\phantomsection\label{\detokenize{index:birdears.CHROMATIC_TYPE}}\pysigline{\sphinxcode{\sphinxupquote{birdears.}}\sphinxbfcode{\sphinxupquote{CHROMATIC\_TYPE}}\sphinxbfcode{\sphinxupquote{ = (0, 1, 2, 3, 4, 5, 6, 7, 8, 9, 10, 11)}}}
\sphinxAtStartPar
A map of the chromatic scale.

\sphinxAtStartPar
A map of the the semitones which compound the chromatic scale.
\begin{quote}\begin{description}
\item[{Type}] \leavevmode
\sphinxAtStartPar
tuple

\end{description}\end{quote}

\end{fulllineitems}

\index{CIRCLE\_OF\_FIFTHS (in module birdears)@\spxentry{CIRCLE\_OF\_FIFTHS}\spxextra{in module birdears}}

\begin{fulllineitems}
\phantomsection\label{\detokenize{index:birdears.CIRCLE_OF_FIFTHS}}\pysigline{\sphinxcode{\sphinxupquote{birdears.}}\sphinxbfcode{\sphinxupquote{CIRCLE\_OF\_FIFTHS}}\sphinxbfcode{\sphinxupquote{ = {[}(\textquotesingle{}C\textquotesingle{}, \textquotesingle{}G\textquotesingle{}, \textquotesingle{}D\textquotesingle{}, \textquotesingle{}A\textquotesingle{}, \textquotesingle{}E\textquotesingle{}, \textquotesingle{}B\textquotesingle{}, \textquotesingle{}Gb\textquotesingle{}, \textquotesingle{}Db\textquotesingle{}, \textquotesingle{}Ab\textquotesingle{}, \textquotesingle{}Eb\textquotesingle{}, \textquotesingle{}Bb\textquotesingle{}, \textquotesingle{}F\textquotesingle{}), (\textquotesingle{}C\textquotesingle{}, \textquotesingle{}F\textquotesingle{}, \textquotesingle{}Bb\textquotesingle{}, \textquotesingle{}Eb\textquotesingle{}, \textquotesingle{}Ab\textquotesingle{}, \textquotesingle{}C\#\textquotesingle{}, \textquotesingle{}F\#\textquotesingle{}, \textquotesingle{}B\textquotesingle{}, \textquotesingle{}E\textquotesingle{}, \textquotesingle{}A\textquotesingle{}, \textquotesingle{}D\textquotesingle{}, \textquotesingle{}G\textquotesingle{}){]}}}}
\sphinxAtStartPar
Circle of fifths.

\sphinxAtStartPar
These are the circle of fifth in both directions.
\begin{quote}\begin{description}
\item[{Type}] \leavevmode
\sphinxAtStartPar
list of tuples

\end{description}\end{quote}

\end{fulllineitems}

\index{D() (in module birdears)@\spxentry{D()}\spxextra{in module birdears}}

\begin{fulllineitems}
\phantomsection\label{\detokenize{index:birdears.D}}\pysiglinewithargsret{\sphinxcode{\sphinxupquote{birdears.}}\sphinxbfcode{\sphinxupquote{D}}}{\emph{\DUrole{n}{data}}, \emph{\DUrole{n}{nlines}\DUrole{o}{=}\DUrole{default_value}{0}}}{}
\end{fulllineitems}

\index{DEGREE\_INDEX (in module birdears)@\spxentry{DEGREE\_INDEX}\spxextra{in module birdears}}

\begin{fulllineitems}
\phantomsection\label{\detokenize{index:birdears.DEGREE_INDEX}}\pysigline{\sphinxcode{\sphinxupquote{birdears.}}\sphinxbfcode{\sphinxupquote{DEGREE\_INDEX}}\sphinxbfcode{\sphinxupquote{ = \{\textquotesingle{}i\textquotesingle{}: {[}0{]}, \textquotesingle{}ii\textquotesingle{}: {[}1, 2{]}, \textquotesingle{}iii\textquotesingle{}: {[}3, 4{]}, \textquotesingle{}iv\textquotesingle{}: {[}5, 6{]}, \textquotesingle{}v\textquotesingle{}: {[}6, 7{]}, \textquotesingle{}vi\textquotesingle{}: {[}8, 9{]}, \textquotesingle{}vii\textquotesingle{}: {[}10, 11{]}, \textquotesingle{}viii\textquotesingle{}: {[}12{]}\}}}}
\sphinxAtStartPar
A mapping of semitones of each degree.

\sphinxAtStartPar
A mapping of semitones which index to each degree roman numeral, major/minor,
perfect, augmented/diminished
\begin{quote}\begin{description}
\item[{Type}] \leavevmode
\sphinxAtStartPar
dict of lists

\end{description}\end{quote}

\end{fulllineitems}

\index{DIATONIC\_MASK (in module birdears)@\spxentry{DIATONIC\_MASK}\spxextra{in module birdears}}

\begin{fulllineitems}
\phantomsection\label{\detokenize{index:birdears.DIATONIC_MASK}}\pysigline{\sphinxcode{\sphinxupquote{birdears.}}\sphinxbfcode{\sphinxupquote{DIATONIC\_MASK}}\sphinxbfcode{\sphinxupquote{ = \{\textquotesingle{}dorian\textquotesingle{}: (1, 0, 1, 1, 0, 1, 0, 1, 0, 1, 1, 0), \textquotesingle{}locrian\textquotesingle{}: (1, 1, 0, 1, 0, 1, 1, 0, 1, 0, 1, 0), \textquotesingle{}lydian\textquotesingle{}: (1, 0, 1, 0, 1, 0, 1, 1, 0, 1, 0, 1), \textquotesingle{}major\textquotesingle{}: (1, 0, 1, 0, 1, 1, 0, 1, 0, 1, 0, 1), \textquotesingle{}minor\textquotesingle{}: (1, 0, 1, 1, 0, 1, 0, 1, 1, 0, 1, 0), \textquotesingle{}mixolydian\textquotesingle{}: (1, 0, 1, 0, 1, 1, 0, 1, 0, 1, 1, 0), \textquotesingle{}phrygian\textquotesingle{}: (1, 1, 0, 1, 0, 1, 0, 1, 1, 0, 1, 0)\}}}}
\sphinxAtStartPar
A map of the diatonic scale.

\sphinxAtStartPar
A mapping of the semitones which compound each of the greek modes.
\begin{quote}\begin{description}
\item[{Type}] \leavevmode
\sphinxAtStartPar
dict of tuples

\end{description}\end{quote}

\end{fulllineitems}

\index{INTERVALS (in module birdears)@\spxentry{INTERVALS}\spxextra{in module birdears}}

\begin{fulllineitems}
\phantomsection\label{\detokenize{index:birdears.INTERVALS}}\pysigline{\sphinxcode{\sphinxupquote{birdears.}}\sphinxbfcode{\sphinxupquote{INTERVALS}}\sphinxbfcode{\sphinxupquote{ = ((0, \textquotesingle{}P1\textquotesingle{}, \textquotesingle{}Perfect Unison\textquotesingle{}), (1, \textquotesingle{}m2\textquotesingle{}, \textquotesingle{}Minor Second\textquotesingle{}), (2, \textquotesingle{}M2\textquotesingle{}, \textquotesingle{}Major Second\textquotesingle{}), (3, \textquotesingle{}m3\textquotesingle{}, \textquotesingle{}Minor Third\textquotesingle{}), (4, \textquotesingle{}M3\textquotesingle{}, \textquotesingle{}Major Third\textquotesingle{}), (5, \textquotesingle{}P4\textquotesingle{}, \textquotesingle{}Perfect Fourth\textquotesingle{}), (6, \textquotesingle{}A4\textquotesingle{}, \textquotesingle{}Augmented Fourth\textquotesingle{}), (7, \textquotesingle{}P5\textquotesingle{}, \textquotesingle{}Perfect Fifth\textquotesingle{}), (8, \textquotesingle{}m6\textquotesingle{}, \textquotesingle{}Minor Sixth\textquotesingle{}), (9, \textquotesingle{}M6\textquotesingle{}, \textquotesingle{}Major Sixth\textquotesingle{}), (10, \textquotesingle{}m7\textquotesingle{}, \textquotesingle{}Minor Seventh\textquotesingle{}), (11, \textquotesingle{}M7\textquotesingle{}, \textquotesingle{}Major Seventh\textquotesingle{}), (12, \textquotesingle{}P8\textquotesingle{}, \textquotesingle{}Perfect Octave\textquotesingle{}), (13, \textquotesingle{}A8\textquotesingle{}, \textquotesingle{}Minor Ninth\textquotesingle{}), (14, \textquotesingle{}M9\textquotesingle{}, \textquotesingle{}Major Ninth\textquotesingle{}), (15, \textquotesingle{}m10\textquotesingle{}, \textquotesingle{}Minor Tenth\textquotesingle{}), (16, \textquotesingle{}M10\textquotesingle{}, \textquotesingle{}Major Tenth\textquotesingle{}), (17, \textquotesingle{}P11\textquotesingle{}, \textquotesingle{}Perfect Eleventh\textquotesingle{}), (18, \textquotesingle{}A11\textquotesingle{}, \textquotesingle{}Augmented Eleventh\textquotesingle{}), (19, \textquotesingle{}P12\textquotesingle{}, \textquotesingle{}Perfect Twelfth\textquotesingle{}), (20, \textquotesingle{}m13\textquotesingle{}, \textquotesingle{}Minor Thirteenth\textquotesingle{}), (21, \textquotesingle{}M13\textquotesingle{}, \textquotesingle{}Major Thirteenth\textquotesingle{}), (22, \textquotesingle{}m14\textquotesingle{}, \textquotesingle{}Minor Fourteenth\textquotesingle{}), (23, \textquotesingle{}M14\textquotesingle{}, \textquotesingle{}Major Fourteenth\textquotesingle{}), (24, \textquotesingle{}P15\textquotesingle{}, \textquotesingle{}Perfect Double\sphinxhyphen{}octave\textquotesingle{}), (25, \textquotesingle{}A15\textquotesingle{}, \textquotesingle{}Minor Sixteenth\textquotesingle{}), (26, \textquotesingle{}M16\textquotesingle{}, \textquotesingle{}Major Sixteenth\textquotesingle{}), (27, \textquotesingle{}m17\textquotesingle{}, \textquotesingle{}Minor Seventeenth\textquotesingle{}), (28, \textquotesingle{}M17\textquotesingle{}, \textquotesingle{}Major Seventeenth\textquotesingle{}), (29, \textquotesingle{}P18\textquotesingle{}, \textquotesingle{}Perfect Eighteenth\textquotesingle{}), (30, \textquotesingle{}A18\textquotesingle{}, \textquotesingle{}Augmented Eighteenth\textquotesingle{}), (31, \textquotesingle{}P19\textquotesingle{}, \textquotesingle{}Perfect Nineteenth\textquotesingle{}), (32, \textquotesingle{}m20\textquotesingle{}, \textquotesingle{}Minor Twentieth\textquotesingle{}), (33, \textquotesingle{}M20\textquotesingle{}, \textquotesingle{}Major Twentieth\textquotesingle{}), (34, \textquotesingle{}m21\textquotesingle{}, \textquotesingle{}Minor Twenty\sphinxhyphen{}first\textquotesingle{}), (35, \textquotesingle{}M21\textquotesingle{}, \textquotesingle{}Major Twenty\sphinxhyphen{}first\textquotesingle{}), (36, \textquotesingle{}P22\textquotesingle{}, \textquotesingle{}Perfect Triple\sphinxhyphen{}octave\textquotesingle{}))}}}
\sphinxAtStartPar
Data representing intervals.

\sphinxAtStartPar
A tuple of tuples representing data for the intervals with format
(semitones, short name, full name).
\begin{quote}\begin{description}
\item[{Type}] \leavevmode
\sphinxAtStartPar
tuple of tuples

\end{description}\end{quote}

\end{fulllineitems}

\index{INTERVAL\_INDEX (in module birdears)@\spxentry{INTERVAL\_INDEX}\spxextra{in module birdears}}

\begin{fulllineitems}
\phantomsection\label{\detokenize{index:birdears.INTERVAL_INDEX}}\pysigline{\sphinxcode{\sphinxupquote{birdears.}}\sphinxbfcode{\sphinxupquote{INTERVAL\_INDEX}}\sphinxbfcode{\sphinxupquote{ = \{1: {[}0{]}, 2: {[}1, 2{]}, 3: {[}3, 4{]}, 4: {[}5, 6{]}, 5: {[}6, 7{]}, 6: {[}8, 9{]}, 7: {[}10, 11{]}, 8: {[}12{]}\}}}}
\sphinxAtStartPar
A mapping of semitones of each interval.

\sphinxAtStartPar
A mapping of semitones which index to each interval name, major/minor,
perfect, augmented/diminished
\begin{quote}\begin{description}
\item[{Type}] \leavevmode
\sphinxAtStartPar
dict of lists

\end{description}\end{quote}

\end{fulllineitems}

\index{KEYS (in module birdears)@\spxentry{KEYS}\spxextra{in module birdears}}

\begin{fulllineitems}
\phantomsection\label{\detokenize{index:birdears.KEYS}}\pysigline{\sphinxcode{\sphinxupquote{birdears.}}\sphinxbfcode{\sphinxupquote{KEYS}}\sphinxbfcode{\sphinxupquote{ = (\textquotesingle{}C\textquotesingle{}, \textquotesingle{}C\#\textquotesingle{}, \textquotesingle{}Db\textquotesingle{}, \textquotesingle{}D\textquotesingle{}, \textquotesingle{}D\#\textquotesingle{}, \textquotesingle{}Eb\textquotesingle{}, \textquotesingle{}E\textquotesingle{}, \textquotesingle{}F\textquotesingle{}, \textquotesingle{}F\#\textquotesingle{}, \textquotesingle{}Gb\textquotesingle{}, \textquotesingle{}G\textquotesingle{}, \textquotesingle{}G\#\textquotesingle{}, \textquotesingle{}Ab\textquotesingle{}, \textquotesingle{}A\textquotesingle{}, \textquotesingle{}A\#\textquotesingle{}, \textquotesingle{}Bb\textquotesingle{}, \textquotesingle{}B\textquotesingle{})}}}
\sphinxAtStartPar
Allowed keys

\sphinxAtStartPar
These are the allowed keys for exercise as comprehended by birdears.
\begin{quote}\begin{description}
\item[{Type}] \leavevmode
\sphinxAtStartPar
tuple

\end{description}\end{quote}

\end{fulllineitems}



\section{Subpackages}
\label{\detokenize{index:subpackages}}

\section{Submodules}
\label{\detokenize{index:submodules}}

\section{birdears.interval module}
\label{\detokenize{index:module-birdears.interval}}\label{\detokenize{index:birdears-interval-module}}\index{module@\spxentry{module}!birdears.interval@\spxentry{birdears.interval}}\index{birdears.interval@\spxentry{birdears.interval}!module@\spxentry{module}}\index{Interval (class in birdears.interval)@\spxentry{Interval}\spxextra{class in birdears.interval}}

\begin{fulllineitems}
\phantomsection\label{\detokenize{index:birdears.interval.Interval}}\pysiglinewithargsret{\sphinxbfcode{\sphinxupquote{class }}\sphinxcode{\sphinxupquote{birdears.interval.}}\sphinxbfcode{\sphinxupquote{Interval}}}{\emph{\DUrole{n}{pitch\_a}}, \emph{\DUrole{n}{pitch\_b}}}{}
\sphinxAtStartPar
Bases: \sphinxcode{\sphinxupquote{dict}}

\sphinxAtStartPar
This class represents the interval between two pitches..
\index{tonic\_octave (birdears.interval.Interval attribute)@\spxentry{tonic\_octave}\spxextra{birdears.interval.Interval attribute}}

\begin{fulllineitems}
\phantomsection\label{\detokenize{index:birdears.interval.Interval.tonic_octave}}\pysigline{\sphinxbfcode{\sphinxupquote{tonic\_octave}}}
\sphinxAtStartPar
Scientific octave for the tonic. For example, if
the tonic is a ‘C4’ then \sphinxtitleref{tonic\_octave} is 4.
\begin{quote}\begin{description}
\item[{Type}] \leavevmode
\sphinxAtStartPar
int

\end{description}\end{quote}

\end{fulllineitems}



\begin{fulllineitems}
\pysigline{\sphinxbfcode{\sphinxupquote{interval~octave}}}
\sphinxAtStartPar
Scientific octave for the interval. For example,
if the interval is a ‘G5’ then \sphinxtitleref{tonic\_octave} is 5.
\begin{quote}\begin{description}
\item[{Type}] \leavevmode
\sphinxAtStartPar
int

\end{description}\end{quote}

\end{fulllineitems}

\index{chromatic\_offset (birdears.interval.Interval attribute)@\spxentry{chromatic\_offset}\spxextra{birdears.interval.Interval attribute}}

\begin{fulllineitems}
\phantomsection\label{\detokenize{index:birdears.interval.Interval.chromatic_offset}}\pysigline{\sphinxbfcode{\sphinxupquote{chromatic\_offset}}}
\sphinxAtStartPar
The offset in semitones inside one octave.
Relative semitones to tonic.
\begin{quote}\begin{description}
\item[{Type}] \leavevmode
\sphinxAtStartPar
int

\end{description}\end{quote}

\end{fulllineitems}

\index{note\_and\_octave (birdears.interval.Interval attribute)@\spxentry{note\_and\_octave}\spxextra{birdears.interval.Interval attribute}}

\begin{fulllineitems}
\phantomsection\label{\detokenize{index:birdears.interval.Interval.note_and_octave}}\pysigline{\sphinxbfcode{\sphinxupquote{note\_and\_octave}}}
\sphinxAtStartPar
Note and octave of the interval, for example, if
the interval is G5 the note name is ‘G5’.
\begin{quote}\begin{description}
\item[{Type}] \leavevmode
\sphinxAtStartPar
str

\end{description}\end{quote}

\end{fulllineitems}

\index{note\_name (birdears.interval.Interval attribute)@\spxentry{note\_name}\spxextra{birdears.interval.Interval attribute}}

\begin{fulllineitems}
\phantomsection\label{\detokenize{index:birdears.interval.Interval.note_name}}\pysigline{\sphinxbfcode{\sphinxupquote{note\_name}}}
\sphinxAtStartPar
The note name of the interval, for example, if the
interval is G5 then the name is ‘G’.
\begin{quote}\begin{description}
\item[{Type}] \leavevmode
\sphinxAtStartPar
str

\end{description}\end{quote}

\end{fulllineitems}

\index{semitones (birdears.interval.Interval attribute)@\spxentry{semitones}\spxextra{birdears.interval.Interval attribute}}

\begin{fulllineitems}
\phantomsection\label{\detokenize{index:birdears.interval.Interval.semitones}}\pysigline{\sphinxbfcode{\sphinxupquote{semitones}}}
\sphinxAtStartPar
Semitones from tonic to octave. If tonic is C4 and
interval is G5 the number of semitones is 19.
\begin{quote}\begin{description}
\item[{Type}] \leavevmode
\sphinxAtStartPar
int

\end{description}\end{quote}

\end{fulllineitems}

\index{is\_chromatic (birdears.interval.Interval attribute)@\spxentry{is\_chromatic}\spxextra{birdears.interval.Interval attribute}}

\begin{fulllineitems}
\phantomsection\label{\detokenize{index:birdears.interval.Interval.is_chromatic}}\pysigline{\sphinxbfcode{\sphinxupquote{is\_chromatic}}}
\sphinxAtStartPar
If the current interval is chromatic (True) or if
it exists in the diatonic scale which key is tonic.
\begin{quote}\begin{description}
\item[{Type}] \leavevmode
\sphinxAtStartPar
bool

\end{description}\end{quote}

\end{fulllineitems}

\index{is\_descending (birdears.interval.Interval attribute)@\spxentry{is\_descending}\spxextra{birdears.interval.Interval attribute}}

\begin{fulllineitems}
\phantomsection\label{\detokenize{index:birdears.interval.Interval.is_descending}}\pysigline{\sphinxbfcode{\sphinxupquote{is\_descending}}}
\sphinxAtStartPar
If the interval has a descending direction, ie.,
has a lower pitch than the tonic.
\begin{quote}\begin{description}
\item[{Type}] \leavevmode
\sphinxAtStartPar
bool

\end{description}\end{quote}

\end{fulllineitems}

\index{diatonic\_index (birdears.interval.Interval attribute)@\spxentry{diatonic\_index}\spxextra{birdears.interval.Interval attribute}}

\begin{fulllineitems}
\phantomsection\label{\detokenize{index:birdears.interval.Interval.diatonic_index}}\pysigline{\sphinxbfcode{\sphinxupquote{diatonic\_index}}}
\sphinxAtStartPar
If the interval is chromatic, this will be the
nearest diatonic interval in the direction of the resolution
(closest tonic.) From II to IV degrees, it is the ditonic interval
before; from V to VII it is the diatonic interval after.
\begin{quote}\begin{description}
\item[{Type}] \leavevmode
\sphinxAtStartPar
int

\end{description}\end{quote}

\end{fulllineitems}

\index{distance (birdears.interval.Interval attribute)@\spxentry{distance}\spxextra{birdears.interval.Interval attribute}}

\begin{fulllineitems}
\phantomsection\label{\detokenize{index:birdears.interval.Interval.distance}}\pysigline{\sphinxbfcode{\sphinxupquote{distance}}}
\sphinxAtStartPar
A dictionary which the distance from tonic to
interval, for example, if tonic is C4 and interval is G5:

\begin{sphinxVerbatim}[commandchars=\\\{\}]
\PYGZob{}
    \PYGZsq{}octaves\PYGZsq{}: 1,
    \PYGZsq{}semitones\PYGZsq{}: 7
\PYGZcb{}
\end{sphinxVerbatim}
\begin{quote}\begin{description}
\item[{Type}] \leavevmode
\sphinxAtStartPar
dict

\end{description}\end{quote}

\end{fulllineitems}

\index{data (birdears.interval.Interval attribute)@\spxentry{data}\spxextra{birdears.interval.Interval attribute}}

\begin{fulllineitems}
\phantomsection\label{\detokenize{index:birdears.interval.Interval.data}}\pysigline{\sphinxbfcode{\sphinxupquote{data}}}
\sphinxAtStartPar
A tuple representing the interval data in the form of
(semitones, short\_name, long\_name), for example:

\begin{sphinxVerbatim}[commandchars=\\\{\}]
(19, \PYGZsq{}P12\PYGZsq{}, \PYGZsq{}Perfect Twelfth\PYGZsq{})
\end{sphinxVerbatim}
\begin{quote}\begin{description}
\item[{Type}] \leavevmode
\sphinxAtStartPar
tuple

\end{description}\end{quote}

\end{fulllineitems}

\index{\_\_init\_\_() (birdears.interval.Interval method)@\spxentry{\_\_init\_\_()}\spxextra{birdears.interval.Interval method}}

\begin{fulllineitems}
\phantomsection\label{\detokenize{index:birdears.interval.Interval.__init__}}\pysiglinewithargsret{\sphinxbfcode{\sphinxupquote{\_\_init\_\_}}}{\emph{\DUrole{n}{pitch\_a}}, \emph{\DUrole{n}{pitch\_b}}}{}
\sphinxAtStartPar
Measures the musical interval from pitch\_a to pitch\_b.
\begin{quote}\begin{description}
\item[{Parameters}] \leavevmode\begin{itemize}
\item {} 
\sphinxAtStartPar
\sphinxstyleliteralstrong{\sphinxupquote{pitch\_a}} (\sphinxstyleliteralemphasis{\sphinxupquote{str}}) \textendash{} First \sphinxtitleref{Pitch} object to be measured.

\item {} 
\sphinxAtStartPar
\sphinxstyleliteralstrong{\sphinxupquote{pitch\_b}} (\sphinxstyleliteralemphasis{\sphinxupquote{str}}) \textendash{} Second \sphinxtitleref{Pitch} object to be measured.

\end{itemize}

\end{description}\end{quote}

\end{fulllineitems}


\end{fulllineitems}

\index{get\_interval\_by\_semitones() (in module birdears.interval)@\spxentry{get\_interval\_by\_semitones()}\spxextra{in module birdears.interval}}

\begin{fulllineitems}
\phantomsection\label{\detokenize{index:birdears.interval.get_interval_by_semitones}}\pysiglinewithargsret{\sphinxcode{\sphinxupquote{birdears.interval.}}\sphinxbfcode{\sphinxupquote{get\_interval\_by\_semitones}}}{\emph{\DUrole{n}{semitones}}}{}
\end{fulllineitems}



\section{birdears.logger module}
\label{\detokenize{index:module-birdears.logger}}\label{\detokenize{index:birdears-logger-module}}\index{module@\spxentry{module}!birdears.logger@\spxentry{birdears.logger}}\index{birdears.logger@\spxentry{birdears.logger}!module@\spxentry{module}}
\sphinxAtStartPar
This submodule exports \sphinxtitleref{logger} to log events.

\sphinxAtStartPar
Logging messages which are less severe than \sphinxtitleref{lvl} will be ignored:

\begin{sphinxVerbatim}[commandchars=\\\{\}]
Level       Numeric value
\PYGZhy{}\PYGZhy{}\PYGZhy{}\PYGZhy{}\PYGZhy{}       \PYGZhy{}\PYGZhy{}\PYGZhy{}\PYGZhy{}\PYGZhy{}\PYGZhy{}\PYGZhy{}\PYGZhy{}\PYGZhy{}\PYGZhy{}\PYGZhy{}\PYGZhy{}\PYGZhy{}
CRITICAL    50
ERROR       40
WARNING     30
INFO        20
DEBUG       10
NOTSET      0

Level       When it’s used
\PYGZhy{}\PYGZhy{}\PYGZhy{}\PYGZhy{}\PYGZhy{}       \PYGZhy{}\PYGZhy{}\PYGZhy{}\PYGZhy{}\PYGZhy{}\PYGZhy{}\PYGZhy{}\PYGZhy{}\PYGZhy{}\PYGZhy{}\PYGZhy{}\PYGZhy{}\PYGZhy{}\PYGZhy{}
DEBUG       Detailed information, typically of interest only when
                diagnosing problems.
INFO        Confirmation that things are working as expected.
WARNING     An indication that something unexpected happened, or indicative
                of some problem in the near future (e.g. ‘disk space low’).
                The software is still working as expected.
ERROR       Due to a more serious problem, the software has not been able
                to perform some function.
CRITICAL    A serious error, indicating that the program itself may be
                unable to continue running.
\end{sphinxVerbatim}
\index{log\_event() (in module birdears.logger)@\spxentry{log\_event()}\spxextra{in module birdears.logger}}

\begin{fulllineitems}
\phantomsection\label{\detokenize{index:birdears.logger.log_event}}\pysiglinewithargsret{\sphinxcode{\sphinxupquote{birdears.logger.}}\sphinxbfcode{\sphinxupquote{log\_event}}}{\emph{\DUrole{n}{f}}, \emph{\DUrole{o}{*}\DUrole{n}{args}}, \emph{\DUrole{o}{**}\DUrole{n}{kwargs}}}{}
\sphinxAtStartPar
Decorator. Functions and method decorated with this decorator will have
their signature logged when birdears is executed with \sphinxtitleref{\textendash{}debug} mode. Both
function signature with their call values and their return will be logged.

\end{fulllineitems}



\section{birdears.prequestion module}
\label{\detokenize{index:module-birdears.prequestion}}\label{\detokenize{index:birdears-prequestion-module}}\index{module@\spxentry{module}!birdears.prequestion@\spxentry{birdears.prequestion}}\index{birdears.prequestion@\spxentry{birdears.prequestion}!module@\spxentry{module}}
\sphinxAtStartPar
This module implements pre\sphinxhyphen{}questions’ progressions.

\sphinxAtStartPar
Pre questions are chord progressions or notes played before the question is
played, so to affirmate the sound of the question’s key.

\sphinxAtStartPar
For example a common cadence is chords I\sphinxhyphen{}IV\sphinxhyphen{}V\sphinxhyphen{}I from the diatonic scale, which
in a key of \sphinxtitleref{C} is \sphinxtitleref{CM\sphinxhyphen{}FM\sphinxhyphen{}GM\sphinxhyphen{}CM} and in a key of \sphinxtitleref{A} is \sphinxtitleref{AM\sphinxhyphen{}DM\sphinxhyphen{}EM\sphinxhyphen{}AM}.

\sphinxAtStartPar
Pre\sphinxhyphen{}question methods should be decorated with \sphinxtitleref{register\_prequestion\_method}
decorator, so that they will be registered as a valid pre\sphinxhyphen{}question method.
\index{PreQuestion (class in birdears.prequestion)@\spxentry{PreQuestion}\spxextra{class in birdears.prequestion}}

\begin{fulllineitems}
\phantomsection\label{\detokenize{index:birdears.prequestion.PreQuestion}}\pysiglinewithargsret{\sphinxbfcode{\sphinxupquote{class }}\sphinxcode{\sphinxupquote{birdears.prequestion.}}\sphinxbfcode{\sphinxupquote{PreQuestion}}}{\emph{\DUrole{n}{method}}, \emph{\DUrole{n}{question}}}{}
\sphinxAtStartPar
Bases: \sphinxcode{\sphinxupquote{object}}
\index{\_\_call\_\_() (birdears.prequestion.PreQuestion method)@\spxentry{\_\_call\_\_()}\spxextra{birdears.prequestion.PreQuestion method}}

\begin{fulllineitems}
\phantomsection\label{\detokenize{index:birdears.prequestion.PreQuestion.__call__}}\pysiglinewithargsret{\sphinxbfcode{\sphinxupquote{\_\_call\_\_}}}{\emph{\DUrole{o}{*}\DUrole{n}{args}}, \emph{\DUrole{o}{**}\DUrole{n}{kwargs}}}{}
\sphinxAtStartPar
Calls the resolution method and pass arguments to it.

\sphinxAtStartPar
Returns a \sphinxtitleref{birdears.Sequence} object with the pre\sphinxhyphen{}question generated by
the method.

\end{fulllineitems}

\index{\_\_init\_\_() (birdears.prequestion.PreQuestion method)@\spxentry{\_\_init\_\_()}\spxextra{birdears.prequestion.PreQuestion method}}

\begin{fulllineitems}
\phantomsection\label{\detokenize{index:birdears.prequestion.PreQuestion.__init__}}\pysiglinewithargsret{\sphinxbfcode{\sphinxupquote{\_\_init\_\_}}}{\emph{\DUrole{n}{method}}, \emph{\DUrole{n}{question}}}{}
\sphinxAtStartPar
This class implements methods for different types of pre\sphinxhyphen{}question
progressions.
\begin{quote}\begin{description}
\item[{Parameters}] \leavevmode\begin{itemize}
\item {} 
\sphinxAtStartPar
\sphinxstyleliteralstrong{\sphinxupquote{method}} (\sphinxstyleliteralemphasis{\sphinxupquote{str}}) \textendash{} The method used in the pre question.

\item {} 
\sphinxAtStartPar
\sphinxstyleliteralstrong{\sphinxupquote{question}} (\sphinxstyleliteralemphasis{\sphinxupquote{obj}}) \textendash{} Question object from which to generate the

\item {} 
\sphinxAtStartPar
\sphinxstyleliteralstrong{\sphinxupquote{sequence.}} (\sphinxstyleliteralemphasis{\sphinxupquote{pre\sphinxhyphen{}question}}) \textendash{} 

\end{itemize}

\end{description}\end{quote}

\end{fulllineitems}


\end{fulllineitems}

\index{none() (in module birdears.prequestion)@\spxentry{none()}\spxextra{in module birdears.prequestion}}

\begin{fulllineitems}
\phantomsection\label{\detokenize{index:birdears.prequestion.none}}\pysiglinewithargsret{\sphinxcode{\sphinxupquote{birdears.prequestion.}}\sphinxbfcode{\sphinxupquote{none}}}{\emph{\DUrole{n}{question}}, \emph{\DUrole{o}{*}\DUrole{n}{args}}, \emph{\DUrole{o}{**}\DUrole{n}{kwargs}}}{}
\sphinxAtStartPar
Pre\sphinxhyphen{}question method that return an empty sequence with no delay.
:param question: Question object from which to generate the
\begin{quote}

\sphinxAtStartPar
pre\sphinxhyphen{}question sequence. (this is provided by the \sphinxtitleref{Resolution} class
when it is {\color{red}\bfseries{}\textasciigrave{}}\_\_call\_\_\textasciigrave{}ed)
\end{quote}
\begin{quote}\begin{description}
\end{description}\end{quote}

\end{fulllineitems}

\index{progression\_i\_iv\_v\_i() (in module birdears.prequestion)@\spxentry{progression\_i\_iv\_v\_i()}\spxextra{in module birdears.prequestion}}

\begin{fulllineitems}
\phantomsection\label{\detokenize{index:birdears.prequestion.progression_i_iv_v_i}}\pysiglinewithargsret{\sphinxcode{\sphinxupquote{birdears.prequestion.}}\sphinxbfcode{\sphinxupquote{progression\_i\_iv\_v\_i}}}{\emph{\DUrole{n}{question}}, \emph{\DUrole{o}{*}\DUrole{n}{args}}, \emph{\DUrole{o}{**}\DUrole{n}{kwargs}}}{}
\sphinxAtStartPar
Pre\sphinxhyphen{}question method that play’s a chord progression with triad chords
built on the grades I, IV, V the I of the question key.
\begin{quote}\begin{description}
\item[{Parameters}] \leavevmode
\sphinxAtStartPar
\sphinxstyleliteralstrong{\sphinxupquote{question}} (\sphinxstyleliteralemphasis{\sphinxupquote{obj}}) \textendash{} Question object from which to generate the
pre\sphinxhyphen{}question sequence. (this is provided by the \sphinxtitleref{Resolution} class
when it is {\color{red}\bfseries{}\textasciigrave{}}\_\_call\_\_\textasciigrave{}ed)

\end{description}\end{quote}

\end{fulllineitems}

\index{register\_prequestion\_method() (in module birdears.prequestion)@\spxentry{register\_prequestion\_method()}\spxextra{in module birdears.prequestion}}

\begin{fulllineitems}
\phantomsection\label{\detokenize{index:birdears.prequestion.register_prequestion_method}}\pysiglinewithargsret{\sphinxcode{\sphinxupquote{birdears.prequestion.}}\sphinxbfcode{\sphinxupquote{register\_prequestion\_method}}}{\emph{\DUrole{n}{f}}, \emph{\DUrole{o}{*}\DUrole{n}{args}}, \emph{\DUrole{o}{**}\DUrole{n}{kwargs}}}{}
\sphinxAtStartPar
Decorator for prequestion method functions.

\sphinxAtStartPar
Functions decorated with this decorator will be registered in the
\sphinxtitleref{PREQUESTION\_METHODS} global dict.

\end{fulllineitems}

\index{tonic\_only() (in module birdears.prequestion)@\spxentry{tonic\_only()}\spxextra{in module birdears.prequestion}}

\begin{fulllineitems}
\phantomsection\label{\detokenize{index:birdears.prequestion.tonic_only}}\pysiglinewithargsret{\sphinxcode{\sphinxupquote{birdears.prequestion.}}\sphinxbfcode{\sphinxupquote{tonic\_only}}}{\emph{\DUrole{n}{question}}, \emph{\DUrole{o}{*}\DUrole{n}{args}}, \emph{\DUrole{o}{**}\DUrole{n}{kwargs}}}{}
\sphinxAtStartPar
Pre\sphinxhyphen{}question method that only play’s the question tonic note before the
question.
\begin{quote}\begin{description}
\item[{Parameters}] \leavevmode
\sphinxAtStartPar
\sphinxstyleliteralstrong{\sphinxupquote{question}} (\sphinxstyleliteralemphasis{\sphinxupquote{object}}) \textendash{} Question object from which to generate the
pre\sphinxhyphen{}question sequence. (this is provided by the \sphinxtitleref{Resolution} class
when it is {\color{red}\bfseries{}\textasciigrave{}}\_\_call\_\_\textasciigrave{}ed)

\end{description}\end{quote}

\end{fulllineitems}



\section{birdears.questionbase module}
\label{\detokenize{index:module-birdears.questionbase}}\label{\detokenize{index:birdears-questionbase-module}}\index{module@\spxentry{module}!birdears.questionbase@\spxentry{birdears.questionbase}}\index{birdears.questionbase@\spxentry{birdears.questionbase}!module@\spxentry{module}}\index{QuestionBase (class in birdears.questionbase)@\spxentry{QuestionBase}\spxextra{class in birdears.questionbase}}

\begin{fulllineitems}
\phantomsection\label{\detokenize{index:birdears.questionbase.QuestionBase}}\pysiglinewithargsret{\sphinxbfcode{\sphinxupquote{class }}\sphinxcode{\sphinxupquote{birdears.questionbase.}}\sphinxbfcode{\sphinxupquote{QuestionBase}}}{\emph{\DUrole{n}{mode}\DUrole{o}{=}\DUrole{default_value}{\textquotesingle{}major\textquotesingle{}}}, \emph{\DUrole{n}{tonic}\DUrole{o}{=}\DUrole{default_value}{\textquotesingle{}C\textquotesingle{}}}, \emph{\DUrole{n}{octave}\DUrole{o}{=}\DUrole{default_value}{4}}, \emph{\DUrole{n}{descending}\DUrole{o}{=}\DUrole{default_value}{False}}, \emph{\DUrole{n}{chromatic}\DUrole{o}{=}\DUrole{default_value}{False}}, \emph{\DUrole{n}{n\_octaves}\DUrole{o}{=}\DUrole{default_value}{1}}, \emph{\DUrole{n}{valid\_intervals}\DUrole{o}{=}\DUrole{default_value}{(0, 1, 2, 3, 4, 5, 6, 7, 8, 9, 10, 11)}}, \emph{\DUrole{n}{user\_durations}\DUrole{o}{=}\DUrole{default_value}{None}}, \emph{\DUrole{n}{prequestion\_method}\DUrole{o}{=}\DUrole{default_value}{None}}, \emph{\DUrole{n}{resolution\_method}\DUrole{o}{=}\DUrole{default_value}{None}}, \emph{\DUrole{n}{default\_durations}\DUrole{o}{=}\DUrole{default_value}{None}}, \emph{\DUrole{o}{*}\DUrole{n}{args}}, \emph{\DUrole{o}{**}\DUrole{n}{kwargs}}}{}
\sphinxAtStartPar
Bases: \sphinxcode{\sphinxupquote{object}}

\sphinxAtStartPar
Base Class to be subclassed for Question classes.

\sphinxAtStartPar
This class implements attributes and routines to be used in Question
subclasses.
\index{\_\_init\_\_() (birdears.questionbase.QuestionBase method)@\spxentry{\_\_init\_\_()}\spxextra{birdears.questionbase.QuestionBase method}}

\begin{fulllineitems}
\phantomsection\label{\detokenize{index:birdears.questionbase.QuestionBase.__init__}}\pysiglinewithargsret{\sphinxbfcode{\sphinxupquote{\_\_init\_\_}}}{\emph{\DUrole{n}{mode}\DUrole{o}{=}\DUrole{default_value}{\textquotesingle{}major\textquotesingle{}}}, \emph{\DUrole{n}{tonic}\DUrole{o}{=}\DUrole{default_value}{\textquotesingle{}C\textquotesingle{}}}, \emph{\DUrole{n}{octave}\DUrole{o}{=}\DUrole{default_value}{4}}, \emph{\DUrole{n}{descending}\DUrole{o}{=}\DUrole{default_value}{False}}, \emph{\DUrole{n}{chromatic}\DUrole{o}{=}\DUrole{default_value}{False}}, \emph{\DUrole{n}{n\_octaves}\DUrole{o}{=}\DUrole{default_value}{1}}, \emph{\DUrole{n}{valid\_intervals}\DUrole{o}{=}\DUrole{default_value}{(0, 1, 2, 3, 4, 5, 6, 7, 8, 9, 10, 11)}}, \emph{\DUrole{n}{user\_durations}\DUrole{o}{=}\DUrole{default_value}{None}}, \emph{\DUrole{n}{prequestion\_method}\DUrole{o}{=}\DUrole{default_value}{None}}, \emph{\DUrole{n}{resolution\_method}\DUrole{o}{=}\DUrole{default_value}{None}}, \emph{\DUrole{n}{default\_durations}\DUrole{o}{=}\DUrole{default_value}{None}}, \emph{\DUrole{o}{*}\DUrole{n}{args}}, \emph{\DUrole{o}{**}\DUrole{n}{kwargs}}}{}
\sphinxAtStartPar
Inits the class.
\begin{quote}\begin{description}
\item[{Parameters}] \leavevmode\begin{itemize}
\item {} 
\sphinxAtStartPar
\sphinxstyleliteralstrong{\sphinxupquote{mode}} (\sphinxstyleliteralemphasis{\sphinxupquote{str}}) \textendash{} A string represnting the mode of the question.
Eg., ‘major’ or ‘minor’

\item {} 
\sphinxAtStartPar
\sphinxstyleliteralstrong{\sphinxupquote{tonic}} (\sphinxstyleliteralemphasis{\sphinxupquote{str}}) \textendash{} A string representing the tonic of the
question, eg.: ‘C’; if omitted, it will be selected
randomly.

\item {} 
\sphinxAtStartPar
\sphinxstyleliteralstrong{\sphinxupquote{octave}} (\sphinxstyleliteralemphasis{\sphinxupquote{int}}) \textendash{} A scienfic octave notation, for example,
4 for ‘C4’; if not present, it will be randomly chosen.

\item {} 
\sphinxAtStartPar
\sphinxstyleliteralstrong{\sphinxupquote{descending}} (\sphinxstyleliteralemphasis{\sphinxupquote{bool}}) \textendash{} Is the question direction in descending,
ie., intervals have lower pitch than the tonic.

\item {} 
\sphinxAtStartPar
\sphinxstyleliteralstrong{\sphinxupquote{chromatic}} (\sphinxstyleliteralemphasis{\sphinxupquote{bool}}) \textendash{} If the question can have (True) or not
(False) chromatic intervals, ie., intervals not in the
diatonic scale of tonic/mode.

\item {} 
\sphinxAtStartPar
\sphinxstyleliteralstrong{\sphinxupquote{n\_octaves}} (\sphinxstyleliteralemphasis{\sphinxupquote{int}}) \textendash{} Maximum numbr of octaves of the question.

\item {} 
\sphinxAtStartPar
\sphinxstyleliteralstrong{\sphinxupquote{valid\_intervals}} (\sphinxstyleliteralemphasis{\sphinxupquote{list}}) \textendash{} A list with intervals (int) valid for
random choice, 1 is 1st, 2 is second etc. Eg. {[}1, 4, 5{]} to
allow only tonics, fourths and fifths.

\item {} 
\sphinxAtStartPar
\sphinxstyleliteralstrong{\sphinxupquote{user\_durations}} (\sphinxstyleliteralemphasis{\sphinxupquote{dict}}) \textendash{} 
\sphinxAtStartPar
A string with 9 comma\sphinxhyphen{}separated \sphinxtitleref{int} or
\sphinxtitleref{float\textasciigrave{}s to set the default duration for the notes played. The
values are respectively for: pre\sphinxhyphen{}question duration (1st),
pre\sphinxhyphen{}question delay (2nd), and pre\sphinxhyphen{}question pos\sphinxhyphen{}delay (3rd);
question duration (4th), question delay (5th), and question
pos\sphinxhyphen{}delay (6th); resolution duration (7th), resolution
delay (8th), and resolution pos\sphinxhyphen{}delay (9th).
duration is the duration in of the note in seconds; delay is
the time to wait before playing the next note, and pos\_delay is
the time to wait after all the notes of the respective sequence
have been played. If any of the user durations is \textasciigrave{}n}, the
default duration for the type of question will be used instead.
Example:

\begin{sphinxVerbatim}[commandchars=\\\{\}]
\PYGZdq{}2,0.5,1,2,n,0,2.5,n,1\PYGZdq{}
\end{sphinxVerbatim}


\item {} 
\sphinxAtStartPar
\sphinxstyleliteralstrong{\sphinxupquote{prequestion\_method}} (\sphinxstyleliteralemphasis{\sphinxupquote{str}}) \textendash{} Method of playing a cadence or the
exercise tonic before the question so to affirm the question
musical tonic key to the ear. Valid ones are registered in the
\sphinxtitleref{birdears.prequestion.PREQUESION\_METHODS} global dict.

\item {} 
\sphinxAtStartPar
\sphinxstyleliteralstrong{\sphinxupquote{resolution\_method}} (\sphinxstyleliteralemphasis{\sphinxupquote{str}}) \textendash{} Method of playing the resolution of an
exercise Valid ones are registered in the
\sphinxtitleref{birdears.resolution.RESOLUTION\_METHODS} global dict.

\item {} 
\sphinxAtStartPar
\sphinxstyleliteralstrong{\sphinxupquote{user\_durations}} \textendash{} Dictionary with the default durations for
each type of sequence. This is provided by the subclasses.

\end{itemize}

\end{description}\end{quote}

\end{fulllineitems}

\index{check\_question() (birdears.questionbase.QuestionBase method)@\spxentry{check\_question()}\spxextra{birdears.questionbase.QuestionBase method}}

\begin{fulllineitems}
\phantomsection\label{\detokenize{index:birdears.questionbase.QuestionBase.check_question}}\pysiglinewithargsret{\sphinxbfcode{\sphinxupquote{check\_question}}}{}{}
\sphinxAtStartPar
This method should be overwritten by the question subclasses.

\end{fulllineitems}

\index{make\_question() (birdears.questionbase.QuestionBase method)@\spxentry{make\_question()}\spxextra{birdears.questionbase.QuestionBase method}}

\begin{fulllineitems}
\phantomsection\label{\detokenize{index:birdears.questionbase.QuestionBase.make_question}}\pysiglinewithargsret{\sphinxbfcode{\sphinxupquote{make\_question}}}{}{}
\sphinxAtStartPar
This method should be overwritten by the question subclasses.

\end{fulllineitems}

\index{make\_resolution() (birdears.questionbase.QuestionBase method)@\spxentry{make\_resolution()}\spxextra{birdears.questionbase.QuestionBase method}}

\begin{fulllineitems}
\phantomsection\label{\detokenize{index:birdears.questionbase.QuestionBase.make_resolution}}\pysiglinewithargsret{\sphinxbfcode{\sphinxupquote{make\_resolution}}}{}{}
\sphinxAtStartPar
This method should be overwritten by the question subclasses.

\end{fulllineitems}

\index{play\_question() (birdears.questionbase.QuestionBase method)@\spxentry{play\_question()}\spxextra{birdears.questionbase.QuestionBase method}}

\begin{fulllineitems}
\phantomsection\label{\detokenize{index:birdears.questionbase.QuestionBase.play_question}}\pysiglinewithargsret{\sphinxbfcode{\sphinxupquote{play\_question}}}{}{}
\sphinxAtStartPar
This method should be overwritten by the question subclasses.

\end{fulllineitems}


\end{fulllineitems}

\index{get\_valid\_pitches() (in module birdears.questionbase)@\spxentry{get\_valid\_pitches()}\spxextra{in module birdears.questionbase}}

\begin{fulllineitems}
\phantomsection\label{\detokenize{index:birdears.questionbase.get_valid_pitches}}\pysiglinewithargsret{\sphinxcode{\sphinxupquote{birdears.questionbase.}}\sphinxbfcode{\sphinxupquote{get\_valid\_pitches}}}{\emph{\DUrole{n}{scale}}, \emph{\DUrole{n}{valid\_intervals}\DUrole{o}{=}\DUrole{default_value}{(0, 1, 2, 3, 4, 5, 6, 7, 8, 9, 10, 11)}}}{}
\end{fulllineitems}

\index{register\_question\_class() (in module birdears.questionbase)@\spxentry{register\_question\_class()}\spxextra{in module birdears.questionbase}}

\begin{fulllineitems}
\phantomsection\label{\detokenize{index:birdears.questionbase.register_question_class}}\pysiglinewithargsret{\sphinxcode{\sphinxupquote{birdears.questionbase.}}\sphinxbfcode{\sphinxupquote{register\_question\_class}}}{\emph{\DUrole{n}{cls}}, \emph{\DUrole{o}{*}\DUrole{n}{args}}, \emph{\DUrole{o}{**}\DUrole{n}{kwargs}}}{}
\sphinxAtStartPar
Decorator for question classes.

\sphinxAtStartPar
Classes decorated with this decorator will be registered in the
\sphinxtitleref{QUESTION\_CLASSES} global.

\end{fulllineitems}



\section{birdears.resolution module}
\label{\detokenize{index:module-birdears.resolution}}\label{\detokenize{index:birdears-resolution-module}}\index{module@\spxentry{module}!birdears.resolution@\spxentry{birdears.resolution}}\index{birdears.resolution@\spxentry{birdears.resolution}!module@\spxentry{module}}\index{Resolution (class in birdears.resolution)@\spxentry{Resolution}\spxextra{class in birdears.resolution}}

\begin{fulllineitems}
\phantomsection\label{\detokenize{index:birdears.resolution.Resolution}}\pysiglinewithargsret{\sphinxbfcode{\sphinxupquote{class }}\sphinxcode{\sphinxupquote{birdears.resolution.}}\sphinxbfcode{\sphinxupquote{Resolution}}}{\emph{\DUrole{n}{method}}, \emph{\DUrole{n}{question}}}{}
\sphinxAtStartPar
Bases: \sphinxcode{\sphinxupquote{object}}

\sphinxAtStartPar
This class implements methods for different types of question
resolutions.

\sphinxAtStartPar
A resolution is an answer to a question. It aims to create a mnemonic on
how the inverval resvolver to the tonic.
\index{\_\_call\_\_() (birdears.resolution.Resolution method)@\spxentry{\_\_call\_\_()}\spxextra{birdears.resolution.Resolution method}}

\begin{fulllineitems}
\phantomsection\label{\detokenize{index:birdears.resolution.Resolution.__call__}}\pysiglinewithargsret{\sphinxbfcode{\sphinxupquote{\_\_call\_\_}}}{\emph{\DUrole{o}{*}\DUrole{n}{args}}, \emph{\DUrole{o}{**}\DUrole{n}{kwargs}}}{}
\sphinxAtStartPar
Calls the resolution method and pass arguments to it.

\sphinxAtStartPar
Returns a \sphinxtitleref{birdears.Sequence} object with the resolution generated by
the.method.

\end{fulllineitems}

\index{\_\_init\_\_() (birdears.resolution.Resolution method)@\spxentry{\_\_init\_\_()}\spxextra{birdears.resolution.Resolution method}}

\begin{fulllineitems}
\phantomsection\label{\detokenize{index:birdears.resolution.Resolution.__init__}}\pysiglinewithargsret{\sphinxbfcode{\sphinxupquote{\_\_init\_\_}}}{\emph{\DUrole{n}{method}}, \emph{\DUrole{n}{question}}}{}
\sphinxAtStartPar
Inits the resolution class.
\begin{quote}\begin{description}
\item[{Parameters}] \leavevmode\begin{itemize}
\item {} 
\sphinxAtStartPar
\sphinxstyleliteralstrong{\sphinxupquote{method}} (\sphinxstyleliteralemphasis{\sphinxupquote{str}}) \textendash{} The method used in the resolution.

\item {} 
\sphinxAtStartPar
\sphinxstyleliteralstrong{\sphinxupquote{question}} (\sphinxstyleliteralemphasis{\sphinxupquote{obj}}) \textendash{} Question object from which to generate the

\item {} 
\sphinxAtStartPar
\sphinxstyleliteralstrong{\sphinxupquote{sequence.}} (\sphinxstyleliteralemphasis{\sphinxupquote{resolution}}) \textendash{} 

\end{itemize}

\end{description}\end{quote}

\end{fulllineitems}


\end{fulllineitems}

\index{nearest\_tonic() (in module birdears.resolution)@\spxentry{nearest\_tonic()}\spxextra{in module birdears.resolution}}

\begin{fulllineitems}
\phantomsection\label{\detokenize{index:birdears.resolution.nearest_tonic}}\pysiglinewithargsret{\sphinxcode{\sphinxupquote{birdears.resolution.}}\sphinxbfcode{\sphinxupquote{nearest\_tonic}}}{\emph{\DUrole{n}{question}}}{}
\sphinxAtStartPar
Resolution method that resolve the intervals to their nearest tonics.
\begin{quote}\begin{description}
\item[{Parameters}] \leavevmode
\sphinxAtStartPar
\sphinxstyleliteralstrong{\sphinxupquote{question}} (\sphinxstyleliteralemphasis{\sphinxupquote{obj}}) \textendash{} Question object from which to generate the
resolution sequence. (this is provided by the \sphinxtitleref{Prequestion} class
when it is {\color{red}\bfseries{}\textasciigrave{}}\_\_call\_\_\textasciigrave{}ed)

\end{description}\end{quote}

\end{fulllineitems}

\index{register\_resolution\_method() (in module birdears.resolution)@\spxentry{register\_resolution\_method()}\spxextra{in module birdears.resolution}}

\begin{fulllineitems}
\phantomsection\label{\detokenize{index:birdears.resolution.register_resolution_method}}\pysiglinewithargsret{\sphinxcode{\sphinxupquote{birdears.resolution.}}\sphinxbfcode{\sphinxupquote{register\_resolution\_method}}}{\emph{\DUrole{n}{f}}, \emph{\DUrole{o}{*}\DUrole{n}{args}}, \emph{\DUrole{o}{**}\DUrole{n}{kwargs}}}{}
\sphinxAtStartPar
Decorator for resolution method functions.

\sphinxAtStartPar
Functions decorated with this decorator will be registered in the
\sphinxtitleref{RESOLUTION\_METHODS} global dict.

\end{fulllineitems}

\index{repeat\_only() (in module birdears.resolution)@\spxentry{repeat\_only()}\spxextra{in module birdears.resolution}}

\begin{fulllineitems}
\phantomsection\label{\detokenize{index:birdears.resolution.repeat_only}}\pysiglinewithargsret{\sphinxcode{\sphinxupquote{birdears.resolution.}}\sphinxbfcode{\sphinxupquote{repeat\_only}}}{\emph{\DUrole{n}{question}}}{}
\sphinxAtStartPar
Resolution method that only repeats the sequence elements with given
durations.
\begin{quote}\begin{description}
\item[{Parameters}] \leavevmode
\sphinxAtStartPar
\sphinxstyleliteralstrong{\sphinxupquote{question}} (\sphinxstyleliteralemphasis{\sphinxupquote{obj}}) \textendash{} Question object from which to generate the
resolution sequence. (this is provided by the \sphinxtitleref{Prequestion} class
when it is {\color{red}\bfseries{}\textasciigrave{}}\_\_call\_\_\textasciigrave{}ed)

\end{description}\end{quote}

\end{fulllineitems}



\section{birdears.scale module}
\label{\detokenize{index:module-birdears.scale}}\label{\detokenize{index:birdears-scale-module}}\index{module@\spxentry{module}!birdears.scale@\spxentry{birdears.scale}}\index{birdears.scale@\spxentry{birdears.scale}!module@\spxentry{module}}\index{ChromaticScale (class in birdears.scale)@\spxentry{ChromaticScale}\spxextra{class in birdears.scale}}

\begin{fulllineitems}
\phantomsection\label{\detokenize{index:birdears.scale.ChromaticScale}}\pysiglinewithargsret{\sphinxbfcode{\sphinxupquote{class }}\sphinxcode{\sphinxupquote{birdears.scale.}}\sphinxbfcode{\sphinxupquote{ChromaticScale}}}{\emph{\DUrole{n}{tonic}\DUrole{o}{=}\DUrole{default_value}{\textquotesingle{}C\textquotesingle{}}}, \emph{\DUrole{n}{octave}\DUrole{o}{=}\DUrole{default_value}{4}}, \emph{\DUrole{n}{n\_octaves}\DUrole{o}{=}\DUrole{default_value}{1}}, \emph{\DUrole{n}{descending}\DUrole{o}{=}\DUrole{default_value}{False}}, \emph{\DUrole{n}{dont\_repeat\_tonic}\DUrole{o}{=}\DUrole{default_value}{False}}}{}
\sphinxAtStartPar
Bases: {\hyperref[\detokenize{index:birdears.scale.ScaleBase}]{\sphinxcrossref{\sphinxcode{\sphinxupquote{birdears.scale.ScaleBase}}}}}

\sphinxAtStartPar
Builds a musical chromatic scale.
\index{scale (birdears.scale.ChromaticScale attribute)@\spxentry{scale}\spxextra{birdears.scale.ChromaticScale attribute}}

\begin{fulllineitems}
\phantomsection\label{\detokenize{index:birdears.scale.ChromaticScale.scale}}\pysigline{\sphinxbfcode{\sphinxupquote{scale}}}
\sphinxAtStartPar
The array of notes representing the scale.
\begin{quote}\begin{description}
\item[{Type}] \leavevmode
\sphinxAtStartPar
array\_type

\end{description}\end{quote}

\end{fulllineitems}

\index{\_\_init\_\_() (birdears.scale.ChromaticScale method)@\spxentry{\_\_init\_\_()}\spxextra{birdears.scale.ChromaticScale method}}

\begin{fulllineitems}
\phantomsection\label{\detokenize{index:birdears.scale.ChromaticScale.__init__}}\pysiglinewithargsret{\sphinxbfcode{\sphinxupquote{\_\_init\_\_}}}{\emph{\DUrole{n}{tonic}\DUrole{o}{=}\DUrole{default_value}{\textquotesingle{}C\textquotesingle{}}}, \emph{\DUrole{n}{octave}\DUrole{o}{=}\DUrole{default_value}{4}}, \emph{\DUrole{n}{n\_octaves}\DUrole{o}{=}\DUrole{default_value}{1}}, \emph{\DUrole{n}{descending}\DUrole{o}{=}\DUrole{default_value}{False}}, \emph{\DUrole{n}{dont\_repeat\_tonic}\DUrole{o}{=}\DUrole{default_value}{False}}}{}
\sphinxAtStartPar
Returns a chromatic scale from tonic.
\begin{quote}\begin{description}
\item[{Parameters}] \leavevmode\begin{itemize}
\item {} 
\sphinxAtStartPar
\sphinxstyleliteralstrong{\sphinxupquote{tonic}} (\sphinxstyleliteralemphasis{\sphinxupquote{str}}) \textendash{} The note which the scale will be built upon.

\item {} 
\sphinxAtStartPar
\sphinxstyleliteralstrong{\sphinxupquote{octave}} (\sphinxstyleliteralemphasis{\sphinxupquote{int}}) \textendash{} The scientific octave the scale will be built upon.

\item {} 
\sphinxAtStartPar
\sphinxstyleliteralstrong{\sphinxupquote{n\_octaves}} (\sphinxstyleliteralemphasis{\sphinxupquote{int}}) \textendash{} The number of octaves the scale will contain.

\item {} 
\sphinxAtStartPar
\sphinxstyleliteralstrong{\sphinxupquote{descending}} (\sphinxstyleliteralemphasis{\sphinxupquote{bool}}) \textendash{} Whether the scale is descending.

\item {} 
\sphinxAtStartPar
\sphinxstyleliteralstrong{\sphinxupquote{dont\_repeat\_tonic}} (\sphinxstyleliteralemphasis{\sphinxupquote{bool}}) \textendash{} Whether to skip appending the last
note (octave) to the scale.

\end{itemize}

\end{description}\end{quote}

\end{fulllineitems}

\index{get\_triad() (birdears.scale.ChromaticScale method)@\spxentry{get\_triad()}\spxextra{birdears.scale.ChromaticScale method}}

\begin{fulllineitems}
\phantomsection\label{\detokenize{index:birdears.scale.ChromaticScale.get_triad}}\pysiglinewithargsret{\sphinxbfcode{\sphinxupquote{get\_triad}}}{\emph{\DUrole{n}{mode}}, \emph{\DUrole{n}{index}\DUrole{o}{=}\DUrole{default_value}{0}}, \emph{\DUrole{n}{degree}\DUrole{o}{=}\DUrole{default_value}{None}}}{}
\sphinxAtStartPar
Returns an array with notes from a scale’s triad.
\begin{quote}\begin{description}
\item[{Parameters}] \leavevmode\begin{itemize}
\item {} 
\sphinxAtStartPar
\sphinxstyleliteralstrong{\sphinxupquote{mode}} (\sphinxstyleliteralemphasis{\sphinxupquote{str}}) \textendash{} Mode of the scale (eg. ‘major’ or ‘minor’)

\item {} 
\sphinxAtStartPar
\sphinxstyleliteralstrong{\sphinxupquote{index}} (\sphinxstyleliteralemphasis{\sphinxupquote{int}}) \textendash{} Triad index (eg.: 0 for 1st degree triad.)

\item {} 
\sphinxAtStartPar
\sphinxstyleliteralstrong{\sphinxupquote{degree}} (\sphinxstyleliteralemphasis{\sphinxupquote{int}}) \textendash{} Degree of the scale. If provided, overrides the
\sphinxtitleref{index} argument. (eg.: \sphinxtitleref{1} for the 1st degree triad.)

\end{itemize}

\item[{Returns}] \leavevmode
\sphinxAtStartPar
A list with three pitches (str), one for each note of the triad.

\end{description}\end{quote}

\end{fulllineitems}


\end{fulllineitems}

\index{DiatonicScale (class in birdears.scale)@\spxentry{DiatonicScale}\spxextra{class in birdears.scale}}

\begin{fulllineitems}
\phantomsection\label{\detokenize{index:birdears.scale.DiatonicScale}}\pysiglinewithargsret{\sphinxbfcode{\sphinxupquote{class }}\sphinxcode{\sphinxupquote{birdears.scale.}}\sphinxbfcode{\sphinxupquote{DiatonicScale}}}{\emph{\DUrole{n}{tonic}\DUrole{o}{=}\DUrole{default_value}{\textquotesingle{}C\textquotesingle{}}}, \emph{\DUrole{n}{mode}\DUrole{o}{=}\DUrole{default_value}{\textquotesingle{}major\textquotesingle{}}}, \emph{\DUrole{n}{octave}\DUrole{o}{=}\DUrole{default_value}{4}}, \emph{\DUrole{n}{n\_octaves}\DUrole{o}{=}\DUrole{default_value}{1}}, \emph{\DUrole{n}{descending}\DUrole{o}{=}\DUrole{default_value}{False}}, \emph{\DUrole{n}{dont\_repeat\_tonic}\DUrole{o}{=}\DUrole{default_value}{False}}}{}
\sphinxAtStartPar
Bases: {\hyperref[\detokenize{index:birdears.scale.ScaleBase}]{\sphinxcrossref{\sphinxcode{\sphinxupquote{birdears.scale.ScaleBase}}}}}

\sphinxAtStartPar
Builds a musical diatonic scale.
\index{scale (birdears.scale.DiatonicScale attribute)@\spxentry{scale}\spxextra{birdears.scale.DiatonicScale attribute}}

\begin{fulllineitems}
\phantomsection\label{\detokenize{index:birdears.scale.DiatonicScale.scale}}\pysigline{\sphinxbfcode{\sphinxupquote{scale}}}
\sphinxAtStartPar
The array of notes representing the scale.
\begin{quote}\begin{description}
\item[{Type}] \leavevmode
\sphinxAtStartPar
array\_type

\end{description}\end{quote}

\end{fulllineitems}

\index{\_\_init\_\_() (birdears.scale.DiatonicScale method)@\spxentry{\_\_init\_\_()}\spxextra{birdears.scale.DiatonicScale method}}

\begin{fulllineitems}
\phantomsection\label{\detokenize{index:birdears.scale.DiatonicScale.__init__}}\pysiglinewithargsret{\sphinxbfcode{\sphinxupquote{\_\_init\_\_}}}{\emph{\DUrole{n}{tonic}\DUrole{o}{=}\DUrole{default_value}{\textquotesingle{}C\textquotesingle{}}}, \emph{\DUrole{n}{mode}\DUrole{o}{=}\DUrole{default_value}{\textquotesingle{}major\textquotesingle{}}}, \emph{\DUrole{n}{octave}\DUrole{o}{=}\DUrole{default_value}{4}}, \emph{\DUrole{n}{n\_octaves}\DUrole{o}{=}\DUrole{default_value}{1}}, \emph{\DUrole{n}{descending}\DUrole{o}{=}\DUrole{default_value}{False}}, \emph{\DUrole{n}{dont\_repeat\_tonic}\DUrole{o}{=}\DUrole{default_value}{False}}}{}
\sphinxAtStartPar
Returns a diatonic scale from tonic and mode.
\begin{quote}\begin{description}
\item[{Parameters}] \leavevmode\begin{itemize}
\item {} 
\sphinxAtStartPar
\sphinxstyleliteralstrong{\sphinxupquote{tonic}} (\sphinxstyleliteralemphasis{\sphinxupquote{str}}) \textendash{} The note which the scale will be built upon.

\item {} 
\sphinxAtStartPar
\sphinxstyleliteralstrong{\sphinxupquote{mode}} (\sphinxstyleliteralemphasis{\sphinxupquote{str}}) \textendash{} The mode the scale will be built upon.
(‘major’ or ‘minor’)

\item {} 
\sphinxAtStartPar
\sphinxstyleliteralstrong{\sphinxupquote{octave}} (\sphinxstyleliteralemphasis{\sphinxupquote{int}}) \textendash{} The scientific octave the scale will be built upon.

\item {} 
\sphinxAtStartPar
\sphinxstyleliteralstrong{\sphinxupquote{n\_octaves}} (\sphinxstyleliteralemphasis{\sphinxupquote{int}}) \textendash{} The number of octaves the scale will contain.

\item {} 
\sphinxAtStartPar
\sphinxstyleliteralstrong{\sphinxupquote{descending}} (\sphinxstyleliteralemphasis{\sphinxupquote{bool}}) \textendash{} Whether the scale is descending.

\item {} 
\sphinxAtStartPar
\sphinxstyleliteralstrong{\sphinxupquote{dont\_repeat\_tonic}} (\sphinxstyleliteralemphasis{\sphinxupquote{bool}}) \textendash{} Whether to skip appending the last
note (octave) to the scale.

\end{itemize}

\end{description}\end{quote}

\end{fulllineitems}

\index{get\_triad() (birdears.scale.DiatonicScale method)@\spxentry{get\_triad()}\spxextra{birdears.scale.DiatonicScale method}}

\begin{fulllineitems}
\phantomsection\label{\detokenize{index:birdears.scale.DiatonicScale.get_triad}}\pysiglinewithargsret{\sphinxbfcode{\sphinxupquote{get\_triad}}}{\emph{\DUrole{n}{index}\DUrole{o}{=}\DUrole{default_value}{0}}, \emph{\DUrole{n}{degree}\DUrole{o}{=}\DUrole{default_value}{None}}}{}
\sphinxAtStartPar
Returns an array with notes from a scale’s triad.
\begin{quote}\begin{description}
\item[{Parameters}] \leavevmode\begin{itemize}
\item {} 
\sphinxAtStartPar
\sphinxstyleliteralstrong{\sphinxupquote{index}} (\sphinxstyleliteralemphasis{\sphinxupquote{int}}) \textendash{} triad index (eg.: 0 for 1st degree triad.)

\item {} 
\sphinxAtStartPar
\sphinxstyleliteralstrong{\sphinxupquote{degree}} (\sphinxstyleliteralemphasis{\sphinxupquote{int}}) \textendash{} Degree of the scale. If provided, overrides the
\sphinxtitleref{index} argument. (eg.: \sphinxtitleref{1} for the 1st degree triad.)

\end{itemize}

\item[{Returns}] \leavevmode
\sphinxAtStartPar
An array with three pitches, one for each note of the triad.

\end{description}\end{quote}

\end{fulllineitems}


\end{fulllineitems}

\index{ScaleBase (class in birdears.scale)@\spxentry{ScaleBase}\spxextra{class in birdears.scale}}

\begin{fulllineitems}
\phantomsection\label{\detokenize{index:birdears.scale.ScaleBase}}\pysigline{\sphinxbfcode{\sphinxupquote{class }}\sphinxcode{\sphinxupquote{birdears.scale.}}\sphinxbfcode{\sphinxupquote{ScaleBase}}}
\sphinxAtStartPar
Bases: \sphinxcode{\sphinxupquote{list}}

\end{fulllineitems}



\section{birdears.sequence module}
\label{\detokenize{index:module-birdears.sequence}}\label{\detokenize{index:birdears-sequence-module}}\index{module@\spxentry{module}!birdears.sequence@\spxentry{birdears.sequence}}\index{birdears.sequence@\spxentry{birdears.sequence}!module@\spxentry{module}}\index{Sequence (class in birdears.sequence)@\spxentry{Sequence}\spxextra{class in birdears.sequence}}

\begin{fulllineitems}
\phantomsection\label{\detokenize{index:birdears.sequence.Sequence}}\pysiglinewithargsret{\sphinxbfcode{\sphinxupquote{class }}\sphinxcode{\sphinxupquote{birdears.sequence.}}\sphinxbfcode{\sphinxupquote{Sequence}}}{\emph{\DUrole{n}{elements}\DUrole{o}{=}\DUrole{default_value}{{[}{]}}}, \emph{\DUrole{n}{duration}\DUrole{o}{=}\DUrole{default_value}{2}}, \emph{\DUrole{n}{delay}\DUrole{o}{=}\DUrole{default_value}{1.5}}, \emph{\DUrole{n}{pos\_delay}\DUrole{o}{=}\DUrole{default_value}{1}}}{}
\sphinxAtStartPar
Bases: \sphinxcode{\sphinxupquote{list}}

\sphinxAtStartPar
Register a Sequence of notes and/or chords.
\index{elements (birdears.sequence.Sequence attribute)@\spxentry{elements}\spxextra{birdears.sequence.Sequence attribute}}

\begin{fulllineitems}
\phantomsection\label{\detokenize{index:birdears.sequence.Sequence.elements}}\pysigline{\sphinxbfcode{\sphinxupquote{elements}}}
\sphinxAtStartPar
List of notes (strings) ou chords (list of
strings) in this Sequence.
\begin{quote}\begin{description}
\item[{Type}] \leavevmode
\sphinxAtStartPar
array\_type

\end{description}\end{quote}

\end{fulllineitems}

\index{\_\_init\_\_() (birdears.sequence.Sequence method)@\spxentry{\_\_init\_\_()}\spxextra{birdears.sequence.Sequence method}}

\begin{fulllineitems}
\phantomsection\label{\detokenize{index:birdears.sequence.Sequence.__init__}}\pysiglinewithargsret{\sphinxbfcode{\sphinxupquote{\_\_init\_\_}}}{\emph{\DUrole{n}{elements}\DUrole{o}{=}\DUrole{default_value}{{[}{]}}}, \emph{\DUrole{n}{duration}\DUrole{o}{=}\DUrole{default_value}{2}}, \emph{\DUrole{n}{delay}\DUrole{o}{=}\DUrole{default_value}{1.5}}, \emph{\DUrole{n}{pos\_delay}\DUrole{o}{=}\DUrole{default_value}{1}}}{}~\begin{description}
\item[{Inits the Sequence with an array and sets the default times for}] \leavevmode
\sphinxAtStartPar
playing / pausing the elements.

\end{description}
\begin{quote}\begin{description}
\item[{Parameters}] \leavevmode\begin{itemize}
\item {} 
\sphinxAtStartPar
\sphinxstyleliteralstrong{\sphinxupquote{elements}} (\sphinxstyleliteralemphasis{\sphinxupquote{array\_type}}) \textendash{} List of elements in this sequence.
(Pitch’es and/or Chord’s)

\item {} 
\sphinxAtStartPar
\sphinxstyleliteralstrong{\sphinxupquote{duration}} (\sphinxstyleliteralemphasis{\sphinxupquote{float}}) \textendash{} Default playing time for each element in the
sequence.

\item {} 
\sphinxAtStartPar
\sphinxstyleliteralstrong{\sphinxupquote{delay}} (\sphinxstyleliteralemphasis{\sphinxupquote{float}}) \textendash{} Default waiting time to play the next element
in the sequence.

\item {} 
\sphinxAtStartPar
\sphinxstyleliteralstrong{\sphinxupquote{pos\_delay}} (\sphinxstyleliteralemphasis{\sphinxupquote{float}}) \textendash{} Waiting time after playing the last element
in the sequence.

\end{itemize}

\end{description}\end{quote}

\end{fulllineitems}

\index{async\_play() (birdears.sequence.Sequence method)@\spxentry{async\_play()}\spxextra{birdears.sequence.Sequence method}}

\begin{fulllineitems}
\phantomsection\label{\detokenize{index:birdears.sequence.Sequence.async_play}}\pysiglinewithargsret{\sphinxbfcode{\sphinxupquote{async\_play}}}{\emph{\DUrole{n}{callback}}, \emph{\DUrole{n}{end\_callback}}, \emph{\DUrole{n}{args}}, \emph{\DUrole{n}{kwargs}}}{}
\sphinxAtStartPar
Plays the Sequence elements of notes and/or chords and wait for
\sphinxtitleref{Sequence.pos\_delay} seconds.

\end{fulllineitems}

\index{make\_chord\_progression() (birdears.sequence.Sequence method)@\spxentry{make\_chord\_progression()}\spxextra{birdears.sequence.Sequence method}}

\begin{fulllineitems}
\phantomsection\label{\detokenize{index:birdears.sequence.Sequence.make_chord_progression}}\pysiglinewithargsret{\sphinxbfcode{\sphinxupquote{make\_chord\_progression}}}{\emph{\DUrole{n}{tonic\_pitch}}, \emph{\DUrole{n}{mode}}, \emph{\DUrole{n}{degrees}}}{}
\sphinxAtStartPar
Appends triad chord(s) to the Sequence.
\begin{quote}\begin{description}
\item[{Parameters}] \leavevmode\begin{itemize}
\item {} 
\sphinxAtStartPar
\sphinxstyleliteralstrong{\sphinxupquote{tonic}} (\sphinxstyleliteralemphasis{\sphinxupquote{str}}) \textendash{} Tonic note of the scale.

\item {} 
\sphinxAtStartPar
\sphinxstyleliteralstrong{\sphinxupquote{mode}} (\sphinxstyleliteralemphasis{\sphinxupquote{str}}) \textendash{} Mode of the scale from which build the triads upon.

\item {} 
\sphinxAtStartPar
\sphinxstyleliteralstrong{\sphinxupquote{degrees}} (\sphinxstyleliteralemphasis{\sphinxupquote{array\_type}}) \textendash{} List with integers represending the degrees
of each triad.

\end{itemize}

\end{description}\end{quote}

\end{fulllineitems}

\index{play() (birdears.sequence.Sequence method)@\spxentry{play()}\spxextra{birdears.sequence.Sequence method}}

\begin{fulllineitems}
\phantomsection\label{\detokenize{index:birdears.sequence.Sequence.play}}\pysiglinewithargsret{\sphinxbfcode{\sphinxupquote{play}}}{\emph{\DUrole{n}{callback}\DUrole{o}{=}\DUrole{default_value}{None}}, \emph{\DUrole{n}{end\_callback}\DUrole{o}{=}\DUrole{default_value}{None}}, \emph{\DUrole{o}{*}\DUrole{n}{args}}, \emph{\DUrole{o}{**}\DUrole{n}{kwargs}}}{}
\end{fulllineitems}


\end{fulllineitems}



\chapter{birdears.questions package}
\label{\detokenize{index:module-birdears.questions}}\label{\detokenize{index:birdears-questions-package}}\index{module@\spxentry{module}!birdears.questions@\spxentry{birdears.questions}}\index{birdears.questions@\spxentry{birdears.questions}!module@\spxentry{module}}

\section{Submodules}
\label{\detokenize{index:id12}}

\section{birdears.questions.harmonicinterval module}
\label{\detokenize{index:module-birdears.questions.harmonicinterval}}\label{\detokenize{index:birdears-questions-harmonicinterval-module}}\index{module@\spxentry{module}!birdears.questions.harmonicinterval@\spxentry{birdears.questions.harmonicinterval}}\index{birdears.questions.harmonicinterval@\spxentry{birdears.questions.harmonicinterval}!module@\spxentry{module}}\index{HarmonicIntervalQuestion (class in birdears.questions.harmonicinterval)@\spxentry{HarmonicIntervalQuestion}\spxextra{class in birdears.questions.harmonicinterval}}

\begin{fulllineitems}
\phantomsection\label{\detokenize{index:birdears.questions.harmonicinterval.HarmonicIntervalQuestion}}\pysiglinewithargsret{\sphinxbfcode{\sphinxupquote{class }}\sphinxcode{\sphinxupquote{birdears.questions.harmonicinterval.}}\sphinxbfcode{\sphinxupquote{HarmonicIntervalQuestion}}}{\emph{\DUrole{n}{mode}\DUrole{o}{=}\DUrole{default_value}{\textquotesingle{}major\textquotesingle{}}}, \emph{\DUrole{n}{tonic}\DUrole{o}{=}\DUrole{default_value}{\textquotesingle{}C\textquotesingle{}}}, \emph{\DUrole{n}{octave}\DUrole{o}{=}\DUrole{default_value}{4}}, \emph{\DUrole{n}{descending}\DUrole{o}{=}\DUrole{default_value}{False}}, \emph{\DUrole{n}{chromatic}\DUrole{o}{=}\DUrole{default_value}{False}}, \emph{\DUrole{n}{n\_octaves}\DUrole{o}{=}\DUrole{default_value}{1}}, \emph{\DUrole{n}{valid\_intervals}\DUrole{o}{=}\DUrole{default_value}{(0, 1, 2, 3, 4, 5, 6, 7, 8, 9, 10, 11)}}, \emph{\DUrole{n}{user\_durations}\DUrole{o}{=}\DUrole{default_value}{None}}, \emph{\DUrole{n}{prequestion\_method}\DUrole{o}{=}\DUrole{default_value}{\textquotesingle{}none\textquotesingle{}}}, \emph{\DUrole{n}{resolution\_method}\DUrole{o}{=}\DUrole{default_value}{\textquotesingle{}nearest\_tonic\textquotesingle{}}}, \emph{\DUrole{o}{*}\DUrole{n}{args}}, \emph{\DUrole{o}{**}\DUrole{n}{kwargs}}}{}
\sphinxAtStartPar
Bases: {\hyperref[\detokenize{index:birdears.questionbase.QuestionBase}]{\sphinxcrossref{\sphinxcode{\sphinxupquote{birdears.questionbase.QuestionBase}}}}}

\sphinxAtStartPar
Implements a Harmonic Interval test.
\index{\_\_init\_\_() (birdears.questions.harmonicinterval.HarmonicIntervalQuestion method)@\spxentry{\_\_init\_\_()}\spxextra{birdears.questions.harmonicinterval.HarmonicIntervalQuestion method}}

\begin{fulllineitems}
\phantomsection\label{\detokenize{index:birdears.questions.harmonicinterval.HarmonicIntervalQuestion.__init__}}\pysiglinewithargsret{\sphinxbfcode{\sphinxupquote{\_\_init\_\_}}}{\emph{\DUrole{n}{mode}\DUrole{o}{=}\DUrole{default_value}{\textquotesingle{}major\textquotesingle{}}}, \emph{\DUrole{n}{tonic}\DUrole{o}{=}\DUrole{default_value}{\textquotesingle{}C\textquotesingle{}}}, \emph{\DUrole{n}{octave}\DUrole{o}{=}\DUrole{default_value}{4}}, \emph{\DUrole{n}{descending}\DUrole{o}{=}\DUrole{default_value}{False}}, \emph{\DUrole{n}{chromatic}\DUrole{o}{=}\DUrole{default_value}{False}}, \emph{\DUrole{n}{n\_octaves}\DUrole{o}{=}\DUrole{default_value}{1}}, \emph{\DUrole{n}{valid\_intervals}\DUrole{o}{=}\DUrole{default_value}{(0, 1, 2, 3, 4, 5, 6, 7, 8, 9, 10, 11)}}, \emph{\DUrole{n}{user\_durations}\DUrole{o}{=}\DUrole{default_value}{None}}, \emph{\DUrole{n}{prequestion\_method}\DUrole{o}{=}\DUrole{default_value}{\textquotesingle{}none\textquotesingle{}}}, \emph{\DUrole{n}{resolution\_method}\DUrole{o}{=}\DUrole{default_value}{\textquotesingle{}nearest\_tonic\textquotesingle{}}}, \emph{\DUrole{o}{*}\DUrole{n}{args}}, \emph{\DUrole{o}{**}\DUrole{n}{kwargs}}}{}
\sphinxAtStartPar
Inits the class.
\begin{quote}\begin{description}
\item[{Parameters}] \leavevmode\begin{itemize}
\item {} 
\sphinxAtStartPar
\sphinxstyleliteralstrong{\sphinxupquote{mode}} (\sphinxstyleliteralemphasis{\sphinxupquote{str}}) \textendash{} A string representing the mode of the question.
Eg., ‘major’ or ‘minor’

\item {} 
\sphinxAtStartPar
\sphinxstyleliteralstrong{\sphinxupquote{tonic}} (\sphinxstyleliteralemphasis{\sphinxupquote{str}}) \textendash{} A string representing the tonic of the question,
eg.: ‘C’; if omitted, it will be selected randomly.

\item {} 
\sphinxAtStartPar
\sphinxstyleliteralstrong{\sphinxupquote{octave}} (\sphinxstyleliteralemphasis{\sphinxupquote{int}}) \textendash{} A scienfic octave notation, for example, 4 for ‘C4’;
if not present, it will be randomly chosen.

\item {} 
\sphinxAtStartPar
\sphinxstyleliteralstrong{\sphinxupquote{descending}} (\sphinxstyleliteralemphasis{\sphinxupquote{bool}}) \textendash{} Is the question direction in descending, ie.,
intervals have lower pitch than the tonic.

\item {} 
\sphinxAtStartPar
\sphinxstyleliteralstrong{\sphinxupquote{chromatic}} (\sphinxstyleliteralemphasis{\sphinxupquote{bool}}) \textendash{} If the question can have (True) or not (False)
chromatic intervals, ie., intervals not in the diatonic scale
of tonic/mode.

\item {} 
\sphinxAtStartPar
\sphinxstyleliteralstrong{\sphinxupquote{n\_octaves}} (\sphinxstyleliteralemphasis{\sphinxupquote{int}}) \textendash{} Maximum number of octaves of the question.

\item {} 
\sphinxAtStartPar
\sphinxstyleliteralstrong{\sphinxupquote{valid\_intervals}} (\sphinxstyleliteralemphasis{\sphinxupquote{list}}) \textendash{} A list with intervals (int) valid for
random choice, 1 is 1st, 2 is second etc. Eg. {[}1, 4, 5{]} to
allow only tonics, fourths and fifths.

\item {} 
\sphinxAtStartPar
\sphinxstyleliteralstrong{\sphinxupquote{user\_durations}} (\sphinxstyleliteralemphasis{\sphinxupquote{str}}) \textendash{} 
\sphinxAtStartPar
A string with 9 comma\sphinxhyphen{}separated \sphinxtitleref{int} or
\sphinxtitleref{float\textasciigrave{}s to set the default duration for the notes played. The
values are respectively for: pre\sphinxhyphen{}question duration (1st),
pre\sphinxhyphen{}question delay (2nd), and pre\sphinxhyphen{}question pos\sphinxhyphen{}delay (3rd);
question duration (4th), question delay (5th), and question
pos\sphinxhyphen{}delay (6th); resolution duration (7th), resolution
delay (8th), and resolution pos\sphinxhyphen{}delay (9th).
duration is the duration in of the note in seconds; delay is
the time to wait before playing the next note, and pos\_delay is
the time to wait after all the notes of the respective sequence
have been played. If any of the user durations is \textasciigrave{}n}, the
default duration for the type of question will be used instead.
Example:

\begin{sphinxVerbatim}[commandchars=\\\{\}]
\PYGZdq{}2,0.5,1,2,n,0,2.5,n,1\PYGZdq{}
\end{sphinxVerbatim}


\item {} 
\sphinxAtStartPar
\sphinxstyleliteralstrong{\sphinxupquote{prequestion\_method}} (\sphinxstyleliteralemphasis{\sphinxupquote{str}}) \textendash{} Method of playing a cadence or the
exercise tonic before the question so to affirm the question
musical tonic key to the ear. Valid ones are registered in the
\sphinxtitleref{birdears.prequestion.PREQUESION\_METHODS} global dict.

\item {} 
\sphinxAtStartPar
\sphinxstyleliteralstrong{\sphinxupquote{resolution\_method}} (\sphinxstyleliteralemphasis{\sphinxupquote{str}}) \textendash{} Method of playing the resolution of an
exercise. Valid ones are registered in the
\sphinxtitleref{birdears.resolution.RESOLUTION\_METHODS} global dict.

\end{itemize}

\end{description}\end{quote}

\end{fulllineitems}

\index{check\_question() (birdears.questions.harmonicinterval.HarmonicIntervalQuestion method)@\spxentry{check\_question()}\spxextra{birdears.questions.harmonicinterval.HarmonicIntervalQuestion method}}

\begin{fulllineitems}
\phantomsection\label{\detokenize{index:birdears.questions.harmonicinterval.HarmonicIntervalQuestion.check_question}}\pysiglinewithargsret{\sphinxbfcode{\sphinxupquote{check\_question}}}{\emph{\DUrole{n}{user\_input\_char}}}{}
\sphinxAtStartPar
Checks whether the given answer is correct.

\end{fulllineitems}

\index{make\_pre\_question() (birdears.questions.harmonicinterval.HarmonicIntervalQuestion method)@\spxentry{make\_pre\_question()}\spxextra{birdears.questions.harmonicinterval.HarmonicIntervalQuestion method}}

\begin{fulllineitems}
\phantomsection\label{\detokenize{index:birdears.questions.harmonicinterval.HarmonicIntervalQuestion.make_pre_question}}\pysiglinewithargsret{\sphinxbfcode{\sphinxupquote{make\_pre\_question}}}{\emph{\DUrole{n}{method}}}{}
\end{fulllineitems}

\index{make\_question() (birdears.questions.harmonicinterval.HarmonicIntervalQuestion method)@\spxentry{make\_question()}\spxextra{birdears.questions.harmonicinterval.HarmonicIntervalQuestion method}}

\begin{fulllineitems}
\phantomsection\label{\detokenize{index:birdears.questions.harmonicinterval.HarmonicIntervalQuestion.make_question}}\pysiglinewithargsret{\sphinxbfcode{\sphinxupquote{make\_question}}}{}{}
\sphinxAtStartPar
This method should be overwritten by the question subclasses.

\end{fulllineitems}

\index{make\_resolution() (birdears.questions.harmonicinterval.HarmonicIntervalQuestion method)@\spxentry{make\_resolution()}\spxextra{birdears.questions.harmonicinterval.HarmonicIntervalQuestion method}}

\begin{fulllineitems}
\phantomsection\label{\detokenize{index:birdears.questions.harmonicinterval.HarmonicIntervalQuestion.make_resolution}}\pysiglinewithargsret{\sphinxbfcode{\sphinxupquote{make\_resolution}}}{\emph{\DUrole{n}{method}}}{}
\sphinxAtStartPar
This method should be overwritten by the question subclasses.

\end{fulllineitems}

\index{name (birdears.questions.harmonicinterval.HarmonicIntervalQuestion attribute)@\spxentry{name}\spxextra{birdears.questions.harmonicinterval.HarmonicIntervalQuestion attribute}}

\begin{fulllineitems}
\phantomsection\label{\detokenize{index:birdears.questions.harmonicinterval.HarmonicIntervalQuestion.name}}\pysigline{\sphinxbfcode{\sphinxupquote{name}}\sphinxbfcode{\sphinxupquote{ = \textquotesingle{}harmonic\textquotesingle{}}}}
\end{fulllineitems}

\index{play\_question() (birdears.questions.harmonicinterval.HarmonicIntervalQuestion method)@\spxentry{play\_question()}\spxextra{birdears.questions.harmonicinterval.HarmonicIntervalQuestion method}}

\begin{fulllineitems}
\phantomsection\label{\detokenize{index:birdears.questions.harmonicinterval.HarmonicIntervalQuestion.play_question}}\pysiglinewithargsret{\sphinxbfcode{\sphinxupquote{play\_question}}}{\emph{\DUrole{n}{callback}\DUrole{o}{=}\DUrole{default_value}{None}}, \emph{\DUrole{n}{end\_callback}\DUrole{o}{=}\DUrole{default_value}{None}}, \emph{\DUrole{o}{*}\DUrole{n}{args}}, \emph{\DUrole{o}{**}\DUrole{n}{kwargs}}}{}
\sphinxAtStartPar
This method should be overwritten by the question subclasses.

\end{fulllineitems}

\index{play\_resolution() (birdears.questions.harmonicinterval.HarmonicIntervalQuestion method)@\spxentry{play\_resolution()}\spxextra{birdears.questions.harmonicinterval.HarmonicIntervalQuestion method}}

\begin{fulllineitems}
\phantomsection\label{\detokenize{index:birdears.questions.harmonicinterval.HarmonicIntervalQuestion.play_resolution}}\pysiglinewithargsret{\sphinxbfcode{\sphinxupquote{play\_resolution}}}{\emph{\DUrole{n}{callback}\DUrole{o}{=}\DUrole{default_value}{None}}, \emph{\DUrole{n}{end\_callback}\DUrole{o}{=}\DUrole{default_value}{None}}, \emph{\DUrole{o}{*}\DUrole{n}{args}}, \emph{\DUrole{o}{**}\DUrole{n}{kwargs}}}{}
\end{fulllineitems}


\end{fulllineitems}



\section{birdears.questions.instrumentaldictation module}
\label{\detokenize{index:module-birdears.questions.instrumentaldictation}}\label{\detokenize{index:birdears-questions-instrumentaldictation-module}}\index{module@\spxentry{module}!birdears.questions.instrumentaldictation@\spxentry{birdears.questions.instrumentaldictation}}\index{birdears.questions.instrumentaldictation@\spxentry{birdears.questions.instrumentaldictation}!module@\spxentry{module}}\index{InstrumentalDictationQuestion (class in birdears.questions.instrumentaldictation)@\spxentry{InstrumentalDictationQuestion}\spxextra{class in birdears.questions.instrumentaldictation}}

\begin{fulllineitems}
\phantomsection\label{\detokenize{index:birdears.questions.instrumentaldictation.InstrumentalDictationQuestion}}\pysiglinewithargsret{\sphinxbfcode{\sphinxupquote{class }}\sphinxcode{\sphinxupquote{birdears.questions.instrumentaldictation.}}\sphinxbfcode{\sphinxupquote{InstrumentalDictationQuestion}}}{\emph{\DUrole{n}{mode}\DUrole{o}{=}\DUrole{default_value}{\textquotesingle{}major\textquotesingle{}}}, \emph{\DUrole{n}{wait\_time}\DUrole{o}{=}\DUrole{default_value}{11}}, \emph{\DUrole{n}{n\_repeats}\DUrole{o}{=}\DUrole{default_value}{1}}, \emph{\DUrole{n}{max\_intervals}\DUrole{o}{=}\DUrole{default_value}{3}}, \emph{\DUrole{n}{n\_notes}\DUrole{o}{=}\DUrole{default_value}{4}}, \emph{\DUrole{n}{tonic}\DUrole{o}{=}\DUrole{default_value}{\textquotesingle{}C\textquotesingle{}}}, \emph{\DUrole{n}{octave}\DUrole{o}{=}\DUrole{default_value}{4}}, \emph{\DUrole{n}{descending}\DUrole{o}{=}\DUrole{default_value}{False}}, \emph{\DUrole{n}{chromatic}\DUrole{o}{=}\DUrole{default_value}{False}}, \emph{\DUrole{n}{n\_octaves}\DUrole{o}{=}\DUrole{default_value}{1}}, \emph{\DUrole{n}{valid\_intervals}\DUrole{o}{=}\DUrole{default_value}{(0, 1, 2, 3, 4, 5, 6, 7, 8, 9, 10, 11)}}, \emph{\DUrole{n}{user\_durations}\DUrole{o}{=}\DUrole{default_value}{None}}, \emph{\DUrole{n}{prequestion\_method}\DUrole{o}{=}\DUrole{default_value}{\textquotesingle{}progression\_i\_iv\_v\_i\textquotesingle{}}}, \emph{\DUrole{n}{resolution\_method}\DUrole{o}{=}\DUrole{default_value}{\textquotesingle{}repeat\_only\textquotesingle{}}}, \emph{\DUrole{o}{*}\DUrole{n}{args}}, \emph{\DUrole{o}{**}\DUrole{n}{kwargs}}}{}
\sphinxAtStartPar
Bases: {\hyperref[\detokenize{index:birdears.questionbase.QuestionBase}]{\sphinxcrossref{\sphinxcode{\sphinxupquote{birdears.questionbase.QuestionBase}}}}}

\sphinxAtStartPar
Implements an instrumental dictation test.
\index{\_\_init\_\_() (birdears.questions.instrumentaldictation.InstrumentalDictationQuestion method)@\spxentry{\_\_init\_\_()}\spxextra{birdears.questions.instrumentaldictation.InstrumentalDictationQuestion method}}

\begin{fulllineitems}
\phantomsection\label{\detokenize{index:birdears.questions.instrumentaldictation.InstrumentalDictationQuestion.__init__}}\pysiglinewithargsret{\sphinxbfcode{\sphinxupquote{\_\_init\_\_}}}{\emph{\DUrole{n}{mode}\DUrole{o}{=}\DUrole{default_value}{\textquotesingle{}major\textquotesingle{}}}, \emph{\DUrole{n}{wait\_time}\DUrole{o}{=}\DUrole{default_value}{11}}, \emph{\DUrole{n}{n\_repeats}\DUrole{o}{=}\DUrole{default_value}{1}}, \emph{\DUrole{n}{max\_intervals}\DUrole{o}{=}\DUrole{default_value}{3}}, \emph{\DUrole{n}{n\_notes}\DUrole{o}{=}\DUrole{default_value}{4}}, \emph{\DUrole{n}{tonic}\DUrole{o}{=}\DUrole{default_value}{\textquotesingle{}C\textquotesingle{}}}, \emph{\DUrole{n}{octave}\DUrole{o}{=}\DUrole{default_value}{4}}, \emph{\DUrole{n}{descending}\DUrole{o}{=}\DUrole{default_value}{False}}, \emph{\DUrole{n}{chromatic}\DUrole{o}{=}\DUrole{default_value}{False}}, \emph{\DUrole{n}{n\_octaves}\DUrole{o}{=}\DUrole{default_value}{1}}, \emph{\DUrole{n}{valid\_intervals}\DUrole{o}{=}\DUrole{default_value}{(0, 1, 2, 3, 4, 5, 6, 7, 8, 9, 10, 11)}}, \emph{\DUrole{n}{user\_durations}\DUrole{o}{=}\DUrole{default_value}{None}}, \emph{\DUrole{n}{prequestion\_method}\DUrole{o}{=}\DUrole{default_value}{\textquotesingle{}progression\_i\_iv\_v\_i\textquotesingle{}}}, \emph{\DUrole{n}{resolution\_method}\DUrole{o}{=}\DUrole{default_value}{\textquotesingle{}repeat\_only\textquotesingle{}}}, \emph{\DUrole{o}{*}\DUrole{n}{args}}, \emph{\DUrole{o}{**}\DUrole{n}{kwargs}}}{}
\sphinxAtStartPar
Inits the class.
\begin{quote}\begin{description}
\item[{Parameters}] \leavevmode\begin{itemize}
\item {} 
\sphinxAtStartPar
\sphinxstyleliteralstrong{\sphinxupquote{mode}} (\sphinxstyleliteralemphasis{\sphinxupquote{str}}) \textendash{} A string representing the mode of the question.
Eg., ‘major’ or ‘minor’.

\item {} 
\sphinxAtStartPar
\sphinxstyleliteralstrong{\sphinxupquote{wait\_time}} (\sphinxstyleliteralemphasis{\sphinxupquote{float}}) \textendash{} Wait time in seconds for the next question or
repeat.

\item {} 
\sphinxAtStartPar
\sphinxstyleliteralstrong{\sphinxupquote{n\_repeats}} (\sphinxstyleliteralemphasis{\sphinxupquote{int}}) \textendash{} Number of times the same dictation will be
repeated before the end of the exercise.

\item {} 
\sphinxAtStartPar
\sphinxstyleliteralstrong{\sphinxupquote{max\_intervals}} (\sphinxstyleliteralemphasis{\sphinxupquote{int}}) \textendash{} The maximum number of random intervals the
question will have.

\item {} 
\sphinxAtStartPar
\sphinxstyleliteralstrong{\sphinxupquote{n\_notes}} (\sphinxstyleliteralemphasis{\sphinxupquote{int}}) \textendash{} The number of notes the melodic dictation will have.

\item {} 
\sphinxAtStartPar
\sphinxstyleliteralstrong{\sphinxupquote{tonic}} (\sphinxstyleliteralemphasis{\sphinxupquote{str}}) \textendash{} A string representing the tonic of the question,
eg.: ‘C’; if omitted, it will be selected randomly.

\item {} 
\sphinxAtStartPar
\sphinxstyleliteralstrong{\sphinxupquote{octave}} (\sphinxstyleliteralemphasis{\sphinxupquote{int}}) \textendash{} A scienfic octave notation, for example, 4 for ‘C4’;
if not present, it will be randomly chosen.

\item {} 
\sphinxAtStartPar
\sphinxstyleliteralstrong{\sphinxupquote{descending}} (\sphinxstyleliteralemphasis{\sphinxupquote{bool}}) \textendash{} Is the question direction in descending, ie.,
intervals have lower pitch than the tonic.

\item {} 
\sphinxAtStartPar
\sphinxstyleliteralstrong{\sphinxupquote{chromatic}} (\sphinxstyleliteralemphasis{\sphinxupquote{bool}}) \textendash{} If the question can have (True) or not (False)
chromatic intervals, ie., intervals not in the diatonic scale
of tonic/mode.

\item {} 
\sphinxAtStartPar
\sphinxstyleliteralstrong{\sphinxupquote{n\_octaves}} (\sphinxstyleliteralemphasis{\sphinxupquote{int}}) \textendash{} Maximum number of octaves of the question.

\item {} 
\sphinxAtStartPar
\sphinxstyleliteralstrong{\sphinxupquote{valid\_intervals}} (\sphinxstyleliteralemphasis{\sphinxupquote{list}}) \textendash{} A list with intervals (int) valid for
random choice, 1 is 1st, 2 is second etc. Eg. {[}1, 4, 5{]} to
allow only tonics, fourths and fifths.

\item {} 
\sphinxAtStartPar
\sphinxstyleliteralstrong{\sphinxupquote{user\_durations}} (\sphinxstyleliteralemphasis{\sphinxupquote{str}}) \textendash{} 
\sphinxAtStartPar
A string with 9 comma\sphinxhyphen{}separated \sphinxtitleref{int} or
\sphinxtitleref{float\textasciigrave{}s to set the default duration for the notes played. The
values are respectively for: pre\sphinxhyphen{}question duration (1st),
pre\sphinxhyphen{}question delay (2nd), and pre\sphinxhyphen{}question pos\sphinxhyphen{}delay (3rd);
question duration (4th), question delay (5th), and question
pos\sphinxhyphen{}delay (6th); resolution duration (7th), resolution
delay (8th), and resolution pos\sphinxhyphen{}delay (9th).
duration is the duration in of the note in seconds; delay is
the time to wait before playing the next note, and pos\_delay is
the time to wait after all the notes of the respective sequence
have been played. If any of the user durations is \textasciigrave{}n}, the
default duration for the type of question will be used instead.
Example:

\begin{sphinxVerbatim}[commandchars=\\\{\}]
\PYGZdq{}2,0.5,1,2,n,0,2.5,n,1\PYGZdq{}
\end{sphinxVerbatim}


\item {} 
\sphinxAtStartPar
\sphinxstyleliteralstrong{\sphinxupquote{prequestion\_method}} (\sphinxstyleliteralemphasis{\sphinxupquote{str}}) \textendash{} Method of playing a cadence or the
exercise tonic before the question so to affirm the question
musical tonic key to the ear. Valid ones are registered in the
\sphinxtitleref{birdears.prequestion.PREQUESION\_METHODS} global dict.

\item {} 
\sphinxAtStartPar
\sphinxstyleliteralstrong{\sphinxupquote{resolution\_method}} (\sphinxstyleliteralemphasis{\sphinxupquote{str}}) \textendash{} Method of playing the resolution of an
exercise. Valid ones are registered in the
\sphinxtitleref{birdears.resolution.RESOLUTION\_METHODS} global dict.

\end{itemize}

\end{description}\end{quote}

\end{fulllineitems}

\index{check\_question() (birdears.questions.instrumentaldictation.InstrumentalDictationQuestion method)@\spxentry{check\_question()}\spxextra{birdears.questions.instrumentaldictation.InstrumentalDictationQuestion method}}

\begin{fulllineitems}
\phantomsection\label{\detokenize{index:birdears.questions.instrumentaldictation.InstrumentalDictationQuestion.check_question}}\pysiglinewithargsret{\sphinxbfcode{\sphinxupquote{check\_question}}}{}{}
\sphinxAtStartPar
Checks whether the given answer is correct.

\sphinxAtStartPar
This currently doesn’t applies to instrumental dictation questions.

\end{fulllineitems}

\index{make\_pre\_question() (birdears.questions.instrumentaldictation.InstrumentalDictationQuestion method)@\spxentry{make\_pre\_question()}\spxextra{birdears.questions.instrumentaldictation.InstrumentalDictationQuestion method}}

\begin{fulllineitems}
\phantomsection\label{\detokenize{index:birdears.questions.instrumentaldictation.InstrumentalDictationQuestion.make_pre_question}}\pysiglinewithargsret{\sphinxbfcode{\sphinxupquote{make\_pre\_question}}}{\emph{\DUrole{n}{method}}}{}
\end{fulllineitems}

\index{make\_question() (birdears.questions.instrumentaldictation.InstrumentalDictationQuestion method)@\spxentry{make\_question()}\spxextra{birdears.questions.instrumentaldictation.InstrumentalDictationQuestion method}}

\begin{fulllineitems}
\phantomsection\label{\detokenize{index:birdears.questions.instrumentaldictation.InstrumentalDictationQuestion.make_question}}\pysiglinewithargsret{\sphinxbfcode{\sphinxupquote{make\_question}}}{}{}
\sphinxAtStartPar
This method should be overwritten by the question subclasses.

\end{fulllineitems}

\index{make\_resolution() (birdears.questions.instrumentaldictation.InstrumentalDictationQuestion method)@\spxentry{make\_resolution()}\spxextra{birdears.questions.instrumentaldictation.InstrumentalDictationQuestion method}}

\begin{fulllineitems}
\phantomsection\label{\detokenize{index:birdears.questions.instrumentaldictation.InstrumentalDictationQuestion.make_resolution}}\pysiglinewithargsret{\sphinxbfcode{\sphinxupquote{make\_resolution}}}{\emph{\DUrole{n}{method}}}{}
\sphinxAtStartPar
This method should be overwritten by the question subclasses.

\end{fulllineitems}

\index{name (birdears.questions.instrumentaldictation.InstrumentalDictationQuestion attribute)@\spxentry{name}\spxextra{birdears.questions.instrumentaldictation.InstrumentalDictationQuestion attribute}}

\begin{fulllineitems}
\phantomsection\label{\detokenize{index:birdears.questions.instrumentaldictation.InstrumentalDictationQuestion.name}}\pysigline{\sphinxbfcode{\sphinxupquote{name}}\sphinxbfcode{\sphinxupquote{ = \textquotesingle{}instrumental\textquotesingle{}}}}
\end{fulllineitems}

\index{play\_question() (birdears.questions.instrumentaldictation.InstrumentalDictationQuestion method)@\spxentry{play\_question()}\spxextra{birdears.questions.instrumentaldictation.InstrumentalDictationQuestion method}}

\begin{fulllineitems}
\phantomsection\label{\detokenize{index:birdears.questions.instrumentaldictation.InstrumentalDictationQuestion.play_question}}\pysiglinewithargsret{\sphinxbfcode{\sphinxupquote{play\_question}}}{\emph{\DUrole{n}{callback}\DUrole{o}{=}\DUrole{default_value}{None}}, \emph{\DUrole{n}{end\_callback}\DUrole{o}{=}\DUrole{default_value}{None}}, \emph{\DUrole{o}{*}\DUrole{n}{args}}, \emph{\DUrole{o}{**}\DUrole{n}{kwargs}}}{}
\sphinxAtStartPar
This method should be overwritten by the question subclasses.

\end{fulllineitems}


\end{fulllineitems}



\section{birdears.questions.melodicdictation module}
\label{\detokenize{index:module-birdears.questions.melodicdictation}}\label{\detokenize{index:birdears-questions-melodicdictation-module}}\index{module@\spxentry{module}!birdears.questions.melodicdictation@\spxentry{birdears.questions.melodicdictation}}\index{birdears.questions.melodicdictation@\spxentry{birdears.questions.melodicdictation}!module@\spxentry{module}}\index{MelodicDictationQuestion (class in birdears.questions.melodicdictation)@\spxentry{MelodicDictationQuestion}\spxextra{class in birdears.questions.melodicdictation}}

\begin{fulllineitems}
\phantomsection\label{\detokenize{index:birdears.questions.melodicdictation.MelodicDictationQuestion}}\pysiglinewithargsret{\sphinxbfcode{\sphinxupquote{class }}\sphinxcode{\sphinxupquote{birdears.questions.melodicdictation.}}\sphinxbfcode{\sphinxupquote{MelodicDictationQuestion}}}{\emph{\DUrole{n}{mode}\DUrole{o}{=}\DUrole{default_value}{\textquotesingle{}major\textquotesingle{}}}, \emph{\DUrole{n}{max\_intervals}\DUrole{o}{=}\DUrole{default_value}{3}}, \emph{\DUrole{n}{n\_notes}\DUrole{o}{=}\DUrole{default_value}{4}}, \emph{\DUrole{n}{tonic}\DUrole{o}{=}\DUrole{default_value}{\textquotesingle{}C\textquotesingle{}}}, \emph{\DUrole{n}{octave}\DUrole{o}{=}\DUrole{default_value}{4}}, \emph{\DUrole{n}{descending}\DUrole{o}{=}\DUrole{default_value}{False}}, \emph{\DUrole{n}{chromatic}\DUrole{o}{=}\DUrole{default_value}{False}}, \emph{\DUrole{n}{n\_octaves}\DUrole{o}{=}\DUrole{default_value}{1}}, \emph{\DUrole{n}{valid\_intervals}\DUrole{o}{=}\DUrole{default_value}{(0, 1, 2, 3, 4, 5, 6, 7, 8, 9, 10, 11)}}, \emph{\DUrole{n}{user\_durations}\DUrole{o}{=}\DUrole{default_value}{None}}, \emph{\DUrole{n}{prequestion\_method}\DUrole{o}{=}\DUrole{default_value}{\textquotesingle{}progression\_i\_iv\_v\_i\textquotesingle{}}}, \emph{\DUrole{n}{resolution\_method}\DUrole{o}{=}\DUrole{default_value}{\textquotesingle{}repeat\_only\textquotesingle{}}}, \emph{\DUrole{o}{*}\DUrole{n}{args}}, \emph{\DUrole{o}{**}\DUrole{n}{kwargs}}}{}
\sphinxAtStartPar
Bases: {\hyperref[\detokenize{index:birdears.questionbase.QuestionBase}]{\sphinxcrossref{\sphinxcode{\sphinxupquote{birdears.questionbase.QuestionBase}}}}}

\sphinxAtStartPar
Implements a melodic dictation test.
\index{\_\_init\_\_() (birdears.questions.melodicdictation.MelodicDictationQuestion method)@\spxentry{\_\_init\_\_()}\spxextra{birdears.questions.melodicdictation.MelodicDictationQuestion method}}

\begin{fulllineitems}
\phantomsection\label{\detokenize{index:birdears.questions.melodicdictation.MelodicDictationQuestion.__init__}}\pysiglinewithargsret{\sphinxbfcode{\sphinxupquote{\_\_init\_\_}}}{\emph{\DUrole{n}{mode}\DUrole{o}{=}\DUrole{default_value}{\textquotesingle{}major\textquotesingle{}}}, \emph{\DUrole{n}{max\_intervals}\DUrole{o}{=}\DUrole{default_value}{3}}, \emph{\DUrole{n}{n\_notes}\DUrole{o}{=}\DUrole{default_value}{4}}, \emph{\DUrole{n}{tonic}\DUrole{o}{=}\DUrole{default_value}{\textquotesingle{}C\textquotesingle{}}}, \emph{\DUrole{n}{octave}\DUrole{o}{=}\DUrole{default_value}{4}}, \emph{\DUrole{n}{descending}\DUrole{o}{=}\DUrole{default_value}{False}}, \emph{\DUrole{n}{chromatic}\DUrole{o}{=}\DUrole{default_value}{False}}, \emph{\DUrole{n}{n\_octaves}\DUrole{o}{=}\DUrole{default_value}{1}}, \emph{\DUrole{n}{valid\_intervals}\DUrole{o}{=}\DUrole{default_value}{(0, 1, 2, 3, 4, 5, 6, 7, 8, 9, 10, 11)}}, \emph{\DUrole{n}{user\_durations}\DUrole{o}{=}\DUrole{default_value}{None}}, \emph{\DUrole{n}{prequestion\_method}\DUrole{o}{=}\DUrole{default_value}{\textquotesingle{}progression\_i\_iv\_v\_i\textquotesingle{}}}, \emph{\DUrole{n}{resolution\_method}\DUrole{o}{=}\DUrole{default_value}{\textquotesingle{}repeat\_only\textquotesingle{}}}, \emph{\DUrole{o}{*}\DUrole{n}{args}}, \emph{\DUrole{o}{**}\DUrole{n}{kwargs}}}{}
\sphinxAtStartPar
Inits the class.
\begin{quote}\begin{description}
\item[{Parameters}] \leavevmode\begin{itemize}
\item {} 
\sphinxAtStartPar
\sphinxstyleliteralstrong{\sphinxupquote{mode}} (\sphinxstyleliteralemphasis{\sphinxupquote{str}}) \textendash{} A string representing the mode of the question.
Eg., ‘major’ or ‘minor’.

\item {} 
\sphinxAtStartPar
\sphinxstyleliteralstrong{\sphinxupquote{max\_intervals}} (\sphinxstyleliteralemphasis{\sphinxupquote{int}}) \textendash{} The maximum number of random intervals
the question will have.

\item {} 
\sphinxAtStartPar
\sphinxstyleliteralstrong{\sphinxupquote{n\_notes}} (\sphinxstyleliteralemphasis{\sphinxupquote{int}}) \textendash{} The number of notes the melodic dictation will have.

\item {} 
\sphinxAtStartPar
\sphinxstyleliteralstrong{\sphinxupquote{tonic}} (\sphinxstyleliteralemphasis{\sphinxupquote{str}}) \textendash{} A string representing the tonic of the question,
eg.: ‘C’; if omitted, it will be selected randomly.

\item {} 
\sphinxAtStartPar
\sphinxstyleliteralstrong{\sphinxupquote{octave}} (\sphinxstyleliteralemphasis{\sphinxupquote{int}}) \textendash{} A scienfic octave notation, for example, 4 for ‘C4’;
if not present, it will be randomly chosen.

\item {} 
\sphinxAtStartPar
\sphinxstyleliteralstrong{\sphinxupquote{descending}} (\sphinxstyleliteralemphasis{\sphinxupquote{bool}}) \textendash{} Is the question direction in descending, ie.,
intervals have lower pitch than the tonic.

\item {} 
\sphinxAtStartPar
\sphinxstyleliteralstrong{\sphinxupquote{chromatic}} (\sphinxstyleliteralemphasis{\sphinxupquote{bool}}) \textendash{} If the question can have (True) or not (False)
chromatic intervals, ie., intervals not in the diatonic scale
of tonic/mode.

\item {} 
\sphinxAtStartPar
\sphinxstyleliteralstrong{\sphinxupquote{n\_octaves}} (\sphinxstyleliteralemphasis{\sphinxupquote{int}}) \textendash{} Maximum number of octaves of the question.

\item {} 
\sphinxAtStartPar
\sphinxstyleliteralstrong{\sphinxupquote{valid\_intervals}} (\sphinxstyleliteralemphasis{\sphinxupquote{list}}) \textendash{} A list with intervals (int) valid for
random choice, 1 is 1st, 2 is second etc. Eg. {[}1, 4, 5{]} to
allow only tonics, fourths and fifths.

\item {} 
\sphinxAtStartPar
\sphinxstyleliteralstrong{\sphinxupquote{user\_durations}} (\sphinxstyleliteralemphasis{\sphinxupquote{str}}) \textendash{} 
\sphinxAtStartPar
A string with 9 comma\sphinxhyphen{}separated \sphinxtitleref{int} or
\sphinxtitleref{float\textasciigrave{}s to set the default duration for the notes played. The
values are respectively for: pre\sphinxhyphen{}question duration (1st),
pre\sphinxhyphen{}question delay (2nd), and pre\sphinxhyphen{}question pos\sphinxhyphen{}delay (3rd);
question duration (4th), question delay (5th), and question
pos\sphinxhyphen{}delay (6th); resolution duration (7th), resolution
delay (8th), and resolution pos\sphinxhyphen{}delay (9th).
duration is the duration in of the note in seconds; delay is
the time to wait before playing the next note, and pos\_delay is
the time to wait after all the notes of the respective sequence
have been played. If any of the user durations is \textasciigrave{}n}, the
default duration for the type of question will be used instead.
Example:

\begin{sphinxVerbatim}[commandchars=\\\{\}]
\PYGZdq{}2,0.5,1,2,n,0,2.5,n,1\PYGZdq{}
\end{sphinxVerbatim}


\item {} 
\sphinxAtStartPar
\sphinxstyleliteralstrong{\sphinxupquote{prequestion\_method}} (\sphinxstyleliteralemphasis{\sphinxupquote{str}}) \textendash{} Method of playing a cadence or the
exercise tonic before the question so to affirm the question
musical tonic key to the ear. Valid ones are registered in the
\sphinxtitleref{birdears.prequestion.PREQUESION\_METHODS} global dict.

\item {} 
\sphinxAtStartPar
\sphinxstyleliteralstrong{\sphinxupquote{resolution\_method}} (\sphinxstyleliteralemphasis{\sphinxupquote{str}}) \textendash{} Method of playing the resolution of an
exercise. Valid ones are registered in the
\sphinxtitleref{birdears.resolution.RESOLUTION\_METHODS} global dict.

\end{itemize}

\end{description}\end{quote}

\end{fulllineitems}

\index{check\_question() (birdears.questions.melodicdictation.MelodicDictationQuestion method)@\spxentry{check\_question()}\spxextra{birdears.questions.melodicdictation.MelodicDictationQuestion method}}

\begin{fulllineitems}
\phantomsection\label{\detokenize{index:birdears.questions.melodicdictation.MelodicDictationQuestion.check_question}}\pysiglinewithargsret{\sphinxbfcode{\sphinxupquote{check\_question}}}{\emph{\DUrole{n}{user\_input\_keys}}}{}
\sphinxAtStartPar
Checks whether the given answer is correct.

\end{fulllineitems}

\index{make\_pre\_question() (birdears.questions.melodicdictation.MelodicDictationQuestion method)@\spxentry{make\_pre\_question()}\spxextra{birdears.questions.melodicdictation.MelodicDictationQuestion method}}

\begin{fulllineitems}
\phantomsection\label{\detokenize{index:birdears.questions.melodicdictation.MelodicDictationQuestion.make_pre_question}}\pysiglinewithargsret{\sphinxbfcode{\sphinxupquote{make\_pre\_question}}}{\emph{\DUrole{n}{method}}}{}
\end{fulllineitems}

\index{make\_question() (birdears.questions.melodicdictation.MelodicDictationQuestion method)@\spxentry{make\_question()}\spxextra{birdears.questions.melodicdictation.MelodicDictationQuestion method}}

\begin{fulllineitems}
\phantomsection\label{\detokenize{index:birdears.questions.melodicdictation.MelodicDictationQuestion.make_question}}\pysiglinewithargsret{\sphinxbfcode{\sphinxupquote{make\_question}}}{}{}
\sphinxAtStartPar
This method should be overwritten by the question subclasses.

\end{fulllineitems}

\index{make\_resolution() (birdears.questions.melodicdictation.MelodicDictationQuestion method)@\spxentry{make\_resolution()}\spxextra{birdears.questions.melodicdictation.MelodicDictationQuestion method}}

\begin{fulllineitems}
\phantomsection\label{\detokenize{index:birdears.questions.melodicdictation.MelodicDictationQuestion.make_resolution}}\pysiglinewithargsret{\sphinxbfcode{\sphinxupquote{make\_resolution}}}{\emph{\DUrole{n}{method}}}{}
\sphinxAtStartPar
This method should be overwritten by the question subclasses.

\end{fulllineitems}

\index{name (birdears.questions.melodicdictation.MelodicDictationQuestion attribute)@\spxentry{name}\spxextra{birdears.questions.melodicdictation.MelodicDictationQuestion attribute}}

\begin{fulllineitems}
\phantomsection\label{\detokenize{index:birdears.questions.melodicdictation.MelodicDictationQuestion.name}}\pysigline{\sphinxbfcode{\sphinxupquote{name}}\sphinxbfcode{\sphinxupquote{ = \textquotesingle{}dictation\textquotesingle{}}}}
\end{fulllineitems}

\index{play\_question() (birdears.questions.melodicdictation.MelodicDictationQuestion method)@\spxentry{play\_question()}\spxextra{birdears.questions.melodicdictation.MelodicDictationQuestion method}}

\begin{fulllineitems}
\phantomsection\label{\detokenize{index:birdears.questions.melodicdictation.MelodicDictationQuestion.play_question}}\pysiglinewithargsret{\sphinxbfcode{\sphinxupquote{play\_question}}}{\emph{\DUrole{n}{callback}\DUrole{o}{=}\DUrole{default_value}{None}}, \emph{\DUrole{n}{end\_callback}\DUrole{o}{=}\DUrole{default_value}{None}}, \emph{\DUrole{o}{*}\DUrole{n}{args}}, \emph{\DUrole{o}{**}\DUrole{n}{kwargs}}}{}
\sphinxAtStartPar
This method should be overwritten by the question subclasses.

\end{fulllineitems}

\index{play\_resolution() (birdears.questions.melodicdictation.MelodicDictationQuestion method)@\spxentry{play\_resolution()}\spxextra{birdears.questions.melodicdictation.MelodicDictationQuestion method}}

\begin{fulllineitems}
\phantomsection\label{\detokenize{index:birdears.questions.melodicdictation.MelodicDictationQuestion.play_resolution}}\pysiglinewithargsret{\sphinxbfcode{\sphinxupquote{play\_resolution}}}{\emph{\DUrole{n}{callback}\DUrole{o}{=}\DUrole{default_value}{None}}, \emph{\DUrole{n}{end\_callback}\DUrole{o}{=}\DUrole{default_value}{None}}, \emph{\DUrole{o}{*}\DUrole{n}{args}}, \emph{\DUrole{o}{**}\DUrole{n}{kwargs}}}{}
\end{fulllineitems}


\end{fulllineitems}



\section{birdears.questions.melodicinterval module}
\label{\detokenize{index:module-birdears.questions.melodicinterval}}\label{\detokenize{index:birdears-questions-melodicinterval-module}}\index{module@\spxentry{module}!birdears.questions.melodicinterval@\spxentry{birdears.questions.melodicinterval}}\index{birdears.questions.melodicinterval@\spxentry{birdears.questions.melodicinterval}!module@\spxentry{module}}\index{MelodicIntervalQuestion (class in birdears.questions.melodicinterval)@\spxentry{MelodicIntervalQuestion}\spxextra{class in birdears.questions.melodicinterval}}

\begin{fulllineitems}
\phantomsection\label{\detokenize{index:birdears.questions.melodicinterval.MelodicIntervalQuestion}}\pysiglinewithargsret{\sphinxbfcode{\sphinxupquote{class }}\sphinxcode{\sphinxupquote{birdears.questions.melodicinterval.}}\sphinxbfcode{\sphinxupquote{MelodicIntervalQuestion}}}{\emph{\DUrole{n}{mode}\DUrole{o}{=}\DUrole{default_value}{\textquotesingle{}major\textquotesingle{}}}, \emph{\DUrole{n}{tonic}\DUrole{o}{=}\DUrole{default_value}{\textquotesingle{}C\textquotesingle{}}}, \emph{\DUrole{n}{octave}\DUrole{o}{=}\DUrole{default_value}{4}}, \emph{\DUrole{n}{descending}\DUrole{o}{=}\DUrole{default_value}{False}}, \emph{\DUrole{n}{chromatic}\DUrole{o}{=}\DUrole{default_value}{False}}, \emph{\DUrole{n}{n\_octaves}\DUrole{o}{=}\DUrole{default_value}{1}}, \emph{\DUrole{n}{valid\_intervals}\DUrole{o}{=}\DUrole{default_value}{(0, 1, 2, 3, 4, 5, 6, 7, 8, 9, 10, 11)}}, \emph{\DUrole{n}{user\_durations}\DUrole{o}{=}\DUrole{default_value}{None}}, \emph{\DUrole{n}{prequestion\_method}\DUrole{o}{=}\DUrole{default_value}{\textquotesingle{}tonic\_only\textquotesingle{}}}, \emph{\DUrole{n}{resolution\_method}\DUrole{o}{=}\DUrole{default_value}{\textquotesingle{}nearest\_tonic\textquotesingle{}}}, \emph{\DUrole{o}{*}\DUrole{n}{args}}, \emph{\DUrole{o}{**}\DUrole{n}{kwargs}}}{}
\sphinxAtStartPar
Bases: {\hyperref[\detokenize{index:birdears.questionbase.QuestionBase}]{\sphinxcrossref{\sphinxcode{\sphinxupquote{birdears.questionbase.QuestionBase}}}}}

\sphinxAtStartPar
Implements a Melodic Interval test.
\index{\_\_init\_\_() (birdears.questions.melodicinterval.MelodicIntervalQuestion method)@\spxentry{\_\_init\_\_()}\spxextra{birdears.questions.melodicinterval.MelodicIntervalQuestion method}}

\begin{fulllineitems}
\phantomsection\label{\detokenize{index:birdears.questions.melodicinterval.MelodicIntervalQuestion.__init__}}\pysiglinewithargsret{\sphinxbfcode{\sphinxupquote{\_\_init\_\_}}}{\emph{\DUrole{n}{mode}\DUrole{o}{=}\DUrole{default_value}{\textquotesingle{}major\textquotesingle{}}}, \emph{\DUrole{n}{tonic}\DUrole{o}{=}\DUrole{default_value}{\textquotesingle{}C\textquotesingle{}}}, \emph{\DUrole{n}{octave}\DUrole{o}{=}\DUrole{default_value}{4}}, \emph{\DUrole{n}{descending}\DUrole{o}{=}\DUrole{default_value}{False}}, \emph{\DUrole{n}{chromatic}\DUrole{o}{=}\DUrole{default_value}{False}}, \emph{\DUrole{n}{n\_octaves}\DUrole{o}{=}\DUrole{default_value}{1}}, \emph{\DUrole{n}{valid\_intervals}\DUrole{o}{=}\DUrole{default_value}{(0, 1, 2, 3, 4, 5, 6, 7, 8, 9, 10, 11)}}, \emph{\DUrole{n}{user\_durations}\DUrole{o}{=}\DUrole{default_value}{None}}, \emph{\DUrole{n}{prequestion\_method}\DUrole{o}{=}\DUrole{default_value}{\textquotesingle{}tonic\_only\textquotesingle{}}}, \emph{\DUrole{n}{resolution\_method}\DUrole{o}{=}\DUrole{default_value}{\textquotesingle{}nearest\_tonic\textquotesingle{}}}, \emph{\DUrole{o}{*}\DUrole{n}{args}}, \emph{\DUrole{o}{**}\DUrole{n}{kwargs}}}{}
\sphinxAtStartPar
Inits the class.
\begin{quote}\begin{description}
\item[{Parameters}] \leavevmode\begin{itemize}
\item {} 
\sphinxAtStartPar
\sphinxstyleliteralstrong{\sphinxupquote{mode}} (\sphinxstyleliteralemphasis{\sphinxupquote{str}}) \textendash{} A string representing the mode of the question.
Eg., ‘major’ or ‘minor’

\item {} 
\sphinxAtStartPar
\sphinxstyleliteralstrong{\sphinxupquote{tonic}} (\sphinxstyleliteralemphasis{\sphinxupquote{str}}) \textendash{} A string representing the tonic of the question,
eg.: ‘C’; if omitted, it will be selected randomly.

\item {} 
\sphinxAtStartPar
\sphinxstyleliteralstrong{\sphinxupquote{octave}} (\sphinxstyleliteralemphasis{\sphinxupquote{int}}) \textendash{} A scienfic octave notation, for example, 4 for ‘C4’;
if not present, it will be randomly chosen.

\item {} 
\sphinxAtStartPar
\sphinxstyleliteralstrong{\sphinxupquote{descending}} (\sphinxstyleliteralemphasis{\sphinxupquote{bool}}) \textendash{} Is the question direction in descending, ie.,
intervals have lower pitch than the tonic.

\item {} 
\sphinxAtStartPar
\sphinxstyleliteralstrong{\sphinxupquote{chromatic}} (\sphinxstyleliteralemphasis{\sphinxupquote{bool}}) \textendash{} If the question can have (True) or not (False)
chromatic intervals, ie., intervals not in the diatonic scale
of tonic/mode.

\item {} 
\sphinxAtStartPar
\sphinxstyleliteralstrong{\sphinxupquote{n\_octaves}} (\sphinxstyleliteralemphasis{\sphinxupquote{int}}) \textendash{} Maximum number of octaves of the question.

\item {} 
\sphinxAtStartPar
\sphinxstyleliteralstrong{\sphinxupquote{valid\_intervals}} (\sphinxstyleliteralemphasis{\sphinxupquote{list}}) \textendash{} A list with intervals (int) valid for
random choice, 1 is 1st, 2 is second etc. Eg. {[}1, 4, 5{]} to
allow only tonics, fourths and fifths.

\item {} 
\sphinxAtStartPar
\sphinxstyleliteralstrong{\sphinxupquote{user\_durations}} (\sphinxstyleliteralemphasis{\sphinxupquote{str}}) \textendash{} 
\sphinxAtStartPar
A string with 9 comma\sphinxhyphen{}separated \sphinxtitleref{int} or
\sphinxtitleref{float\textasciigrave{}s to set the default duration for the notes played. The
values are respectively for: pre\sphinxhyphen{}question duration (1st),
pre\sphinxhyphen{}question delay (2nd), and pre\sphinxhyphen{}question pos\sphinxhyphen{}delay (3rd);
question duration (4th), question delay (5th), and question
pos\sphinxhyphen{}delay (6th); resolution duration (7th), resolution
delay (8th), and resolution pos\sphinxhyphen{}delay (9th).
duration is the duration in of the note in seconds; delay is
the time to wait before playing the next note, and pos\_delay is
the time to wait after all the notes of the respective sequence
have been played. If any of the user durations is \textasciigrave{}n}, the
default duration for the type of question will be used instead.
Example:

\begin{sphinxVerbatim}[commandchars=\\\{\}]
\PYGZdq{}2,0.5,1,2,n,0,2.5,n,1\PYGZdq{}
\end{sphinxVerbatim}


\item {} 
\sphinxAtStartPar
\sphinxstyleliteralstrong{\sphinxupquote{prequestion\_method}} (\sphinxstyleliteralemphasis{\sphinxupquote{str}}) \textendash{} Method of playing a cadence or the
exercise tonic before the question so to affirm the question
musical tonic key to the ear. Valid ones are registered in the
\sphinxtitleref{birdears.prequestion.PREQUESION\_METHODS} global dict.

\item {} 
\sphinxAtStartPar
\sphinxstyleliteralstrong{\sphinxupquote{resolution\_method}} (\sphinxstyleliteralemphasis{\sphinxupquote{str}}) \textendash{} Method of playing the resolution of an
exercise. Valid ones are registered in the
\sphinxtitleref{birdears.resolution.RESOLUTION\_METHODS} global dict.

\end{itemize}

\end{description}\end{quote}

\end{fulllineitems}

\index{check\_question() (birdears.questions.melodicinterval.MelodicIntervalQuestion method)@\spxentry{check\_question()}\spxextra{birdears.questions.melodicinterval.MelodicIntervalQuestion method}}

\begin{fulllineitems}
\phantomsection\label{\detokenize{index:birdears.questions.melodicinterval.MelodicIntervalQuestion.check_question}}\pysiglinewithargsret{\sphinxbfcode{\sphinxupquote{check\_question}}}{\emph{\DUrole{n}{user\_input\_char}}}{}
\sphinxAtStartPar
Checks whether the given answer is correct.

\end{fulllineitems}

\index{make\_pre\_question() (birdears.questions.melodicinterval.MelodicIntervalQuestion method)@\spxentry{make\_pre\_question()}\spxextra{birdears.questions.melodicinterval.MelodicIntervalQuestion method}}

\begin{fulllineitems}
\phantomsection\label{\detokenize{index:birdears.questions.melodicinterval.MelodicIntervalQuestion.make_pre_question}}\pysiglinewithargsret{\sphinxbfcode{\sphinxupquote{make\_pre\_question}}}{\emph{\DUrole{n}{method}}}{}
\end{fulllineitems}

\index{make\_question() (birdears.questions.melodicinterval.MelodicIntervalQuestion method)@\spxentry{make\_question()}\spxextra{birdears.questions.melodicinterval.MelodicIntervalQuestion method}}

\begin{fulllineitems}
\phantomsection\label{\detokenize{index:birdears.questions.melodicinterval.MelodicIntervalQuestion.make_question}}\pysiglinewithargsret{\sphinxbfcode{\sphinxupquote{make\_question}}}{}{}
\sphinxAtStartPar
This method should be overwritten by the question subclasses.

\end{fulllineitems}

\index{make\_resolution() (birdears.questions.melodicinterval.MelodicIntervalQuestion method)@\spxentry{make\_resolution()}\spxextra{birdears.questions.melodicinterval.MelodicIntervalQuestion method}}

\begin{fulllineitems}
\phantomsection\label{\detokenize{index:birdears.questions.melodicinterval.MelodicIntervalQuestion.make_resolution}}\pysiglinewithargsret{\sphinxbfcode{\sphinxupquote{make\_resolution}}}{\emph{\DUrole{n}{method}}}{}
\sphinxAtStartPar
This method should be overwritten by the question subclasses.

\end{fulllineitems}

\index{name (birdears.questions.melodicinterval.MelodicIntervalQuestion attribute)@\spxentry{name}\spxextra{birdears.questions.melodicinterval.MelodicIntervalQuestion attribute}}

\begin{fulllineitems}
\phantomsection\label{\detokenize{index:birdears.questions.melodicinterval.MelodicIntervalQuestion.name}}\pysigline{\sphinxbfcode{\sphinxupquote{name}}\sphinxbfcode{\sphinxupquote{ = \textquotesingle{}melodic\textquotesingle{}}}}
\end{fulllineitems}

\index{play\_question() (birdears.questions.melodicinterval.MelodicIntervalQuestion method)@\spxentry{play\_question()}\spxextra{birdears.questions.melodicinterval.MelodicIntervalQuestion method}}

\begin{fulllineitems}
\phantomsection\label{\detokenize{index:birdears.questions.melodicinterval.MelodicIntervalQuestion.play_question}}\pysiglinewithargsret{\sphinxbfcode{\sphinxupquote{play\_question}}}{\emph{\DUrole{n}{callback}\DUrole{o}{=}\DUrole{default_value}{None}}, \emph{\DUrole{n}{end\_callback}\DUrole{o}{=}\DUrole{default_value}{None}}, \emph{\DUrole{o}{*}\DUrole{n}{args}}, \emph{\DUrole{o}{**}\DUrole{n}{kwargs}}}{}
\sphinxAtStartPar
This method should be overwritten by the question subclasses.

\end{fulllineitems}

\index{play\_resolution() (birdears.questions.melodicinterval.MelodicIntervalQuestion method)@\spxentry{play\_resolution()}\spxextra{birdears.questions.melodicinterval.MelodicIntervalQuestion method}}

\begin{fulllineitems}
\phantomsection\label{\detokenize{index:birdears.questions.melodicinterval.MelodicIntervalQuestion.play_resolution}}\pysiglinewithargsret{\sphinxbfcode{\sphinxupquote{play\_resolution}}}{\emph{\DUrole{n}{callback}\DUrole{o}{=}\DUrole{default_value}{None}}, \emph{\DUrole{n}{end\_callback}\DUrole{o}{=}\DUrole{default_value}{None}}, \emph{\DUrole{o}{*}\DUrole{n}{args}}, \emph{\DUrole{o}{**}\DUrole{n}{kwargs}}}{}
\end{fulllineitems}


\end{fulllineitems}



\chapter{birdears.interfaces package}
\label{\detokenize{index:module-birdears.interfaces}}\label{\detokenize{index:birdears-interfaces-package}}\index{module@\spxentry{module}!birdears.interfaces@\spxentry{birdears.interfaces}}\index{birdears.interfaces@\spxentry{birdears.interfaces}!module@\spxentry{module}}

\section{Submodules}
\label{\detokenize{index:id13}}

\section{birdears.interfaces.commandline module}
\label{\detokenize{index:module-birdears.interfaces.commandline}}\label{\detokenize{index:birdears-interfaces-commandline-module}}\index{module@\spxentry{module}!birdears.interfaces.commandline@\spxentry{birdears.interfaces.commandline}}\index{birdears.interfaces.commandline@\spxentry{birdears.interfaces.commandline}!module@\spxentry{module}}\index{CommandLine (class in birdears.interfaces.commandline)@\spxentry{CommandLine}\spxextra{class in birdears.interfaces.commandline}}

\begin{fulllineitems}
\phantomsection\label{\detokenize{index:birdears.interfaces.commandline.CommandLine}}\pysiglinewithargsret{\sphinxbfcode{\sphinxupquote{class }}\sphinxcode{\sphinxupquote{birdears.interfaces.commandline.}}\sphinxbfcode{\sphinxupquote{CommandLine}}}{\emph{\DUrole{n}{exercise}\DUrole{o}{=}\DUrole{default_value}{None}}, \emph{\DUrole{o}{*}\DUrole{n}{args}}, \emph{\DUrole{o}{**}\DUrole{n}{kwargs}}}{}
\sphinxAtStartPar
Bases: \sphinxcode{\sphinxupquote{object}}
\index{\_\_init\_\_() (birdears.interfaces.commandline.CommandLine method)@\spxentry{\_\_init\_\_()}\spxextra{birdears.interfaces.commandline.CommandLine method}}

\begin{fulllineitems}
\phantomsection\label{\detokenize{index:birdears.interfaces.commandline.CommandLine.__init__}}\pysiglinewithargsret{\sphinxbfcode{\sphinxupquote{\_\_init\_\_}}}{\emph{\DUrole{n}{exercise}\DUrole{o}{=}\DUrole{default_value}{None}}, \emph{\DUrole{o}{*}\DUrole{n}{args}}, \emph{\DUrole{o}{**}\DUrole{n}{kwargs}}}{}
\sphinxAtStartPar
This function implements the birdears loop for command line.
\begin{quote}\begin{description}
\item[{Parameters}] \leavevmode\begin{itemize}
\item {} 
\sphinxAtStartPar
\sphinxstyleliteralstrong{\sphinxupquote{exercise}} (\sphinxstyleliteralemphasis{\sphinxupquote{str}}) \textendash{} The question name.

\item {} 
\sphinxAtStartPar
\sphinxstyleliteralstrong{\sphinxupquote{**kwargs}} (\sphinxstyleliteralemphasis{\sphinxupquote{kwargs}}) \textendash{} FIXME: The kwargs can contain options for specific
questions.

\end{itemize}

\end{description}\end{quote}

\end{fulllineitems}

\index{process\_key() (birdears.interfaces.commandline.CommandLine method)@\spxentry{process\_key()}\spxextra{birdears.interfaces.commandline.CommandLine method}}

\begin{fulllineitems}
\phantomsection\label{\detokenize{index:birdears.interfaces.commandline.CommandLine.process_key}}\pysiglinewithargsret{\sphinxbfcode{\sphinxupquote{process\_key}}}{\emph{\DUrole{n}{user\_input}}}{}
\end{fulllineitems}


\end{fulllineitems}

\index{center\_text() (in module birdears.interfaces.commandline)@\spxentry{center\_text()}\spxextra{in module birdears.interfaces.commandline}}

\begin{fulllineitems}
\phantomsection\label{\detokenize{index:birdears.interfaces.commandline.center_text}}\pysiglinewithargsret{\sphinxcode{\sphinxupquote{birdears.interfaces.commandline.}}\sphinxbfcode{\sphinxupquote{center\_text}}}{\emph{\DUrole{n}{text}}, \emph{\DUrole{n}{sep}\DUrole{o}{=}\DUrole{default_value}{True}}, \emph{\DUrole{n}{nl}\DUrole{o}{=}\DUrole{default_value}{0}}}{}
\sphinxAtStartPar
This function returns input text centered according to terminal columns.
\begin{quote}\begin{description}
\item[{Parameters}] \leavevmode\begin{itemize}
\item {} 
\sphinxAtStartPar
\sphinxstyleliteralstrong{\sphinxupquote{text}} (\sphinxstyleliteralemphasis{\sphinxupquote{str}}) \textendash{} The string to be centered, it can have multiple lines.

\item {} 
\sphinxAtStartPar
\sphinxstyleliteralstrong{\sphinxupquote{sep}} (\sphinxstyleliteralemphasis{\sphinxupquote{bool}}) \textendash{} Add line separator after centered text (True) or
not (False).

\item {} 
\sphinxAtStartPar
\sphinxstyleliteralstrong{\sphinxupquote{nl}} (\sphinxstyleliteralemphasis{\sphinxupquote{int}}) \textendash{} How many new lines to add after text.

\end{itemize}

\end{description}\end{quote}

\end{fulllineitems}

\index{make\_input\_str() (in module birdears.interfaces.commandline)@\spxentry{make\_input\_str()}\spxextra{in module birdears.interfaces.commandline}}

\begin{fulllineitems}
\phantomsection\label{\detokenize{index:birdears.interfaces.commandline.make_input_str}}\pysiglinewithargsret{\sphinxcode{\sphinxupquote{birdears.interfaces.commandline.}}\sphinxbfcode{\sphinxupquote{make\_input\_str}}}{\emph{\DUrole{n}{user\_input}}, \emph{\DUrole{n}{keyboard\_index}}}{}
\sphinxAtStartPar
Makes a string representing intervals entered by the user.

\sphinxAtStartPar
This function is to be used by questions which takes more than one interval
input as MelodicDictation, and formats the intervals already entered.
\begin{quote}\begin{description}
\item[{Parameters}] \leavevmode\begin{itemize}
\item {} 
\sphinxAtStartPar
\sphinxstyleliteralstrong{\sphinxupquote{user\_input}} (\sphinxstyleliteralemphasis{\sphinxupquote{array\_type}}) \textendash{} The list of keyboard keys entered by user.

\item {} 
\sphinxAtStartPar
\sphinxstyleliteralstrong{\sphinxupquote{keyboard\_index}} (\sphinxstyleliteralemphasis{\sphinxupquote{array\_type}}) \textendash{} The keyboard mapping used by question.

\end{itemize}

\end{description}\end{quote}

\end{fulllineitems}

\index{print\_instrumental() (in module birdears.interfaces.commandline)@\spxentry{print\_instrumental()}\spxextra{in module birdears.interfaces.commandline}}

\begin{fulllineitems}
\phantomsection\label{\detokenize{index:birdears.interfaces.commandline.print_instrumental}}\pysiglinewithargsret{\sphinxcode{\sphinxupquote{birdears.interfaces.commandline.}}\sphinxbfcode{\sphinxupquote{print\_instrumental}}}{\emph{\DUrole{n}{response}}}{}
\sphinxAtStartPar
Prints the formatted response for ‘instrumental’ exercise.
\begin{quote}\begin{description}
\item[{Parameters}] \leavevmode
\sphinxAtStartPar
\sphinxstyleliteralstrong{\sphinxupquote{response}} (\sphinxstyleliteralemphasis{\sphinxupquote{dict}}) \textendash{} A response returned by question’s check\_question()

\end{description}\end{quote}

\end{fulllineitems}

\index{print\_question() (in module birdears.interfaces.commandline)@\spxentry{print\_question()}\spxextra{in module birdears.interfaces.commandline}}

\begin{fulllineitems}
\phantomsection\label{\detokenize{index:birdears.interfaces.commandline.print_question}}\pysiglinewithargsret{\sphinxcode{\sphinxupquote{birdears.interfaces.commandline.}}\sphinxbfcode{\sphinxupquote{print\_question}}}{\emph{\DUrole{n}{question}}}{}
\sphinxAtStartPar
Prints the question to the user.
\begin{quote}\begin{description}
\item[{Parameters}] \leavevmode
\sphinxAtStartPar
\sphinxstyleliteralstrong{\sphinxupquote{question}} (\sphinxstyleliteralemphasis{\sphinxupquote{obj}}) \textendash{} A Question class with the question to be printed.

\end{description}\end{quote}

\end{fulllineitems}

\index{print\_response() (in module birdears.interfaces.commandline)@\spxentry{print\_response()}\spxextra{in module birdears.interfaces.commandline}}

\begin{fulllineitems}
\phantomsection\label{\detokenize{index:birdears.interfaces.commandline.print_response}}\pysiglinewithargsret{\sphinxcode{\sphinxupquote{birdears.interfaces.commandline.}}\sphinxbfcode{\sphinxupquote{print\_response}}}{\emph{\DUrole{n}{response}}}{}
\sphinxAtStartPar
Prints the formatted response.
\begin{quote}\begin{description}
\item[{Parameters}] \leavevmode
\sphinxAtStartPar
\sphinxstyleliteralstrong{\sphinxupquote{response}} (\sphinxstyleliteralemphasis{\sphinxupquote{dict}}) \textendash{} A response returned by question’s check\_question()

\end{description}\end{quote}

\end{fulllineitems}



\renewcommand{\indexname}{Python Module Index}
\begin{sphinxtheindex}
\let\bigletter\sphinxstyleindexlettergroup
\bigletter{b}
\item\relax\sphinxstyleindexentry{birdears}\sphinxstyleindexpageref{index:\detokenize{module-birdears}}
\item\relax\sphinxstyleindexentry{birdears.interfaces}\sphinxstyleindexpageref{index:\detokenize{module-birdears.interfaces}}
\item\relax\sphinxstyleindexentry{birdears.interfaces.commandline}\sphinxstyleindexpageref{index:\detokenize{module-birdears.interfaces.commandline}}
\item\relax\sphinxstyleindexentry{birdears.interval}\sphinxstyleindexpageref{index:\detokenize{module-birdears.interval}}
\item\relax\sphinxstyleindexentry{birdears.logger}\sphinxstyleindexpageref{index:\detokenize{module-birdears.logger}}
\item\relax\sphinxstyleindexentry{birdears.prequestion}\sphinxstyleindexpageref{index:\detokenize{module-birdears.prequestion}}
\item\relax\sphinxstyleindexentry{birdears.questionbase}\sphinxstyleindexpageref{index:\detokenize{module-birdears.questionbase}}
\item\relax\sphinxstyleindexentry{birdears.questions}\sphinxstyleindexpageref{index:\detokenize{module-birdears.questions}}
\item\relax\sphinxstyleindexentry{birdears.questions.harmonicinterval}\sphinxstyleindexpageref{index:\detokenize{module-birdears.questions.harmonicinterval}}
\item\relax\sphinxstyleindexentry{birdears.questions.instrumentaldictation}\sphinxstyleindexpageref{index:\detokenize{module-birdears.questions.instrumentaldictation}}
\item\relax\sphinxstyleindexentry{birdears.questions.melodicdictation}\sphinxstyleindexpageref{index:\detokenize{module-birdears.questions.melodicdictation}}
\item\relax\sphinxstyleindexentry{birdears.questions.melodicinterval}\sphinxstyleindexpageref{index:\detokenize{module-birdears.questions.melodicinterval}}
\item\relax\sphinxstyleindexentry{birdears.resolution}\sphinxstyleindexpageref{index:\detokenize{module-birdears.resolution}}
\item\relax\sphinxstyleindexentry{birdears.scale}\sphinxstyleindexpageref{index:\detokenize{module-birdears.scale}}
\item\relax\sphinxstyleindexentry{birdears.sequence}\sphinxstyleindexpageref{index:\detokenize{module-birdears.sequence}}
\end{sphinxtheindex}

\renewcommand{\indexname}{Index}
\printindex
\end{document}